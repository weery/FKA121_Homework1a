\input{template_files/packages}

\usepackage{tikz}
\usepackage{pgfplots}
\usepackage{tikzscale}
\usepackage{graphicx}
\usepackage{float}
\usepackage{subcaption}

\title{H1a: Classical scattering by a central potential}
\author{Victor Nilsson and Simon Nilsson}
\date{\today}

\begin{document}

\input{template_files/titlepage}

\section*{Introduction}

Molecular dynamics is a simulation of the movement of atoms and molecules. What is of interest in such a simulations is e.g. the trajectories of the atoms given specific surrounding parameters such as temperature, pressure, crystal formation etc. For this homeproblem we study the dynamics of aluminium atoms in a FCC crystal lattice.

\section*{Problem 1}

\begin{figure}[H]
    \centering
    \captionsetup[subfigure]{justification=centering}
    \begin{subfigure}[b]{0.40\textwidth}
        \centering
        \resizebox{\columnwidth}{!}{% This file was created by matlab2tikz.
%
%The latest updates can be retrieved from
%  http://www.mathworks.com/matlabcentral/fileexchange/22022-matlab2tikz-matlab2tikz
%where you can also make suggestions and rate matlab2tikz.
%
\definecolor{mycolor1}{rgb}{0.00000,0.44700,0.74100}%
\definecolor{mycolor2}{rgb}{0.85000,0.32500,0.09800}%
\definecolor{mycolor3}{rgb}{0.92900,0.69400,0.12500}%
%
\begin{tikzpicture}

\begin{axis}[%
width=4.521in,
height=3.566in,
at={(0.758in,0.481in)},
scale only axis,
xmin=0,
xmax=100,
xlabel={Time / [ASU]},
ymin=-500000000,
ymax=3000000000,
ylabel={Energy / [ASU]},
axis background/.style={fill=white},
title style={font=\bfseries},
title={Awesome title},
legend style={legend cell align=left,align=left,draw=white!15!black}
]
\addplot [color=mycolor1,solid]
  table[row sep=crcr]{%
0	-857.0284\\
0.1	24095.5953\\
0.2	368482.8188\\
0.3	1345138.0751\\
0.4	2290709.7434\\
0.5	4028094.6967\\
0.6	5098138.1067\\
0.7	5828281.2084\\
0.8	6567207.687\\
0.9	7858390.8086\\
1	9842270.9262\\
1.1	12588912.0788\\
1.2	13906521.2805\\
1.3	14363649.6421\\
1.4	15127580.1113\\
1.5	16948566.4579\\
1.6	17909643.987\\
1.7	20367468.1138\\
1.8	22869417.5343\\
1.9	25193878.986\\
2	28342826.4527\\
2.1	28839450.3935\\
2.2	30331130.3247\\
2.3	32527871.7289\\
2.4	34634700.0354\\
2.5	35313003.6665\\
2.6	38613194.3993\\
2.7	48638010.5486\\
2.8	50151638.1548\\
2.9	52594947.3467\\
3	54230397.1507\\
3.1	55551364.0393\\
3.2	56353883.5097\\
3.3	56770473.197\\
3.4	57220501.7233\\
3.5	58584395.3641\\
3.6	60569094.7586\\
3.7	62692067.526\\
3.8	64324208.0889\\
3.9	65762722.5297\\
4	67438763.6942\\
4.1	69552209.7533\\
4.2	70430322.4085\\
4.3	71847176.2897\\
4.4	73325428.205\\
4.5	75173440.891\\
4.6	76403186.4331\\
4.7	78042735.9443\\
4.8	80436309.9774\\
4.9	81020332.2789\\
5	81313509.7819\\
5.1	82973294.789\\
5.2	85802275.0114\\
5.3	88124809.1011\\
5.4	89401665.4643\\
5.5	91378296.6361\\
5.6	94524164.1069\\
5.7	96460820.7844\\
5.8	99448900.6369\\
5.9	101522748.3995\\
6	102529727.8694\\
6.1	103489313.3208\\
6.2	104609660.0967\\
6.3	105365082.4452\\
6.4	105532086.5228\\
6.5	106508943.8725\\
6.6	108344488.3803\\
6.7	110121709.8413\\
6.8	111763183.3774\\
6.9	113233765.0356\\
7	116169073.1049\\
7.1	118104490.6391\\
7.2	119355631.6227\\
7.3	121436893.1461\\
7.4	125414597.4194\\
7.5	126664833.7799\\
7.6	127860903.3605\\
7.7	128494359.1918\\
7.8	131509667.0702\\
7.9	132868415.0923\\
8	135953076.5134\\
8.1	137422666.2762\\
8.2	139361879.1531\\
8.3	141303725.8262\\
8.4	142346542.104\\
8.5	142517633.7039\\
8.6	144270108.2869\\
8.7	145170602.0524\\
8.8	145451627.1261\\
8.9	147643891.664\\
9	151447749.2793\\
9.1	152875540.1467\\
9.2	153359136.4177\\
9.3	156236432.0684\\
9.4	162088141.3522\\
9.5	163921802.1429\\
9.6	165654966.0414\\
9.7	169058975.3003\\
9.8	171540103.7872\\
9.9	172046015.0507\\
10	172736806.8247\\
10.1	173685307.5643\\
10.2	175732714.4342\\
10.3	183295170.1968\\
10.4	200797840.3644\\
10.5	201756614.886\\
10.6	202525726.5692\\
10.7	203422576.5659\\
10.8	203624812.4169\\
10.9	205182760.1219\\
11	207087252.9362\\
11.1	208871185.2308\\
11.2	210006366.1458\\
11.3	210255337.3167\\
11.4	210892225.7987\\
11.5	211806782.2698\\
11.6	212371310.8326\\
11.7	211843505.3813\\
11.8	211416643.4093\\
11.9	212026648.1661\\
12	214093501.2604\\
12.1	216844292.7644\\
12.2	217815810.938\\
12.3	221439033.8285\\
12.4	225262649.954\\
12.5	225909529.3479\\
12.6	236861155.3667\\
12.7	275089737.4125\\
12.8	275934219.2462\\
12.9	277004785.0409\\
13	279092318.4577\\
13.1	280791193.7368\\
13.2	282402272.2673\\
13.3	283471815.5033\\
13.4	283847235.4979\\
13.5	284676749.7207\\
13.6	286699415.3881\\
13.7	289321199.1244\\
13.8	289394353.1631\\
13.9	288548264.4833\\
14	288333036.0154\\
14.1	290495916.6402\\
14.2	293607656.5212\\
14.3	293674332.2067\\
14.4	293411859.06\\
14.5	294030063.2744\\
14.6	295972105.0342\\
14.7	298374422.487\\
14.8	298514850.4058\\
14.9	299809775.5186\\
15	301295545.5831\\
15.1	301679837.5417\\
15.2	301860737.5009\\
15.3	303144766.5294\\
15.4	302825554.9364\\
15.5	304036624.4536\\
15.6	306718791.1735\\
15.7	308355795.3564\\
15.8	309576938.3408\\
15.9	310345230.2321\\
16	311016053.1378\\
16.1	312132047.8506\\
16.2	312627741.0367\\
16.3	311755558.0128\\
16.4	320523105.8083\\
16.5	320510118.3786\\
16.6	323311961.732\\
16.7	330599491.7075\\
16.8	332636466.8764\\
16.9	339348531.9854\\
17	341513488.8474\\
17.1	348059683.4214\\
17.2	358403752.4416\\
17.3	359870723.2124\\
17.4	360714889.1072\\
17.5	361480827.9965\\
17.6	363364939.7069\\
17.7	363760352.8321\\
17.8	364060902.6576\\
17.9	369031431.9849\\
18	381228445.3508\\
18.1	383351806.6154\\
18.2	388212224.3701\\
18.3	388988339.4032\\
18.4	391771464.5376\\
18.5	394362065.1893\\
18.6	395646513.8304\\
18.7	397945075.0721\\
18.8	398553259.5309\\
18.9	398321093.9276\\
19	400935192.1199\\
19.1	402946871.2045\\
19.2	403766205.1315\\
19.3	405453707.4409\\
19.4	407301724.8853\\
19.5	409720817.2219\\
19.6	416568717.5284\\
19.7	424105779.7432\\
19.8	425676240.9535\\
19.9	427654736.783\\
20	429356679.8849\\
20.1	431715546.8356\\
20.2	434805900.0159\\
20.3	437406645.8773\\
20.4	437919276.065\\
20.5	438375024.8556\\
20.6	438841242.6693\\
20.7	442556269.4132\\
20.8	443977557.2853\\
20.9	446017862.1335\\
21	451845473.2756\\
21.1	456739353.0559\\
21.2	456658388.8079\\
21.3	457840365.3658\\
21.4	459058450.6133\\
21.5	458423564.0334\\
21.6	457393070.5846\\
21.7	456942850.7216\\
21.8	461463687.0176\\
21.9	471061340.7374\\
22	477202160.4461\\
22.1	478705123.4672\\
22.2	488435615.8222\\
22.3	521134981.0696\\
22.4	525049019.8306\\
22.5	525474981.6925\\
22.6	525181723.6958\\
22.7	526417843.8027\\
22.8	527317635.2691\\
22.9	525803314.7795\\
23	525764500.8714\\
23.1	527336477.7655\\
23.2	529916180.0102\\
23.3	534508518.1266\\
23.4	537383594.8652\\
23.5	537774918.8335\\
23.6	539073024.5557\\
23.7	542718243.4654\\
23.8	544188970.7669\\
23.9	545942054.7526\\
24	548359291.4185\\
24.1	554984302.5747\\
24.2	570380538.2067\\
24.3	574466807.6143\\
24.4	577988845.9245\\
24.5	578011702.5423\\
24.6	578411579.2878\\
24.7	584428220.5503\\
24.8	590404924.9629\\
24.9	592073848.1824\\
25	596574298.5346\\
25.1	598317854.493\\
25.2	599923718.9772\\
25.3	602991148.2013\\
25.4	609029281.9493\\
25.5	610031740.62\\
25.6	611783496.9415\\
25.7	614356030.7938\\
25.8	619762472.5001\\
25.9	621313001.1378\\
26	626824961.0616\\
26.1	631028428.1747\\
26.2	632975536.4658\\
26.3	637317503.4272\\
26.4	636857881.1961\\
26.5	640885312.7721\\
26.6	655547155.7677\\
26.7	656108625.5827\\
26.8	660550256.6189\\
26.9	670945302.9953\\
27	672082068.7472\\
27.1	672337396.2293\\
27.2	671588972.3386\\
27.3	674636400.8144\\
27.4	689615127.006\\
27.5	690628514.6705\\
27.6	691288078.0049\\
27.7	694490135.2191\\
27.8	696457350.5842\\
27.9	698949610.3248\\
28	701081983.0886\\
28.1	701110476.7211\\
28.2	701327930.5326\\
28.3	706913668.0548\\
28.4	712393100.5709\\
28.5	713844809.6817\\
28.6	716711771.8375\\
28.7	733503045.282\\
28.8	772833419.3031\\
28.9	771801807.2403\\
29	773249530.5513\\
29.1	775772936.9942\\
29.2	775875972.0667\\
29.3	776131906.6816\\
29.4	776892489.8246\\
29.5	778718693.8248\\
29.6	782384381.4187\\
29.7	783736132.0822\\
29.8	784526552.9683\\
29.9	797055720.9875\\
30	828863502.7653\\
30.1	832206763.1577\\
30.2	836010140.576\\
30.3	839251587.7727\\
30.4	840305339.1744\\
30.5	840682339.2226\\
30.6	843107040.3349\\
30.7	843385574.657\\
30.8	850170535.9814\\
30.9	860510453.4279\\
31	864409603.9424\\
31.1	865876117.8453\\
31.2	867386709.6148\\
31.3	869203327.748\\
31.4	871660151.9729\\
31.5	872530483.7138\\
31.6	871793126.0492\\
31.7	872529825.1074\\
31.8	877423351.038\\
31.9	879501291.4968\\
32	884453700.0664\\
32.1	890333747.4335\\
32.2	892744598.2947\\
32.3	892568006.1555\\
32.4	892965184.5267\\
32.5	895343971.8646\\
32.6	897992503.3279\\
32.7	908114588.1794\\
32.8	925772853.2226\\
32.9	933854707.6963\\
33	949296104.0365\\
33.1	955990115.2917\\
33.2	980614635.9956\\
33.3	982956699.6002\\
33.4	983319565.2633\\
33.5	986299351.8147\\
33.6	992092337.6188\\
33.7	991302294.2488\\
33.8	991762740.7527\\
33.9	993054676.6995\\
34	993696786.4112\\
34.1	1001540458.5562\\
34.2	1015023931.4902\\
34.3	1015780728.1515\\
34.4	1016751732.9899\\
34.5	1019171647.4592\\
34.6	1022140667.5181\\
34.7	1022564790.1363\\
34.8	1023121233.6162\\
34.9	1024806163.2327\\
35	1028785406.0861\\
35.1	1032874669.3947\\
35.2	1037902989.3754\\
35.3	1046999064.6373\\
35.4	1064123329.8069\\
35.5	1071087316.5087\\
35.6	1075257695.2999\\
35.7	1081634225.8557\\
35.8	1081537325.4367\\
35.9	1081490625.2059\\
36	1082733629.2658\\
36.1	1085170346.9647\\
36.2	1085906372.6815\\
36.3	1086549615.8606\\
36.4	1088138541.4471\\
36.5	1088233421.2993\\
36.6	1088943423.4032\\
36.7	1089245190.3235\\
36.8	1089039622.9145\\
36.9	1091880696.7371\\
37	1095410000.5298\\
37.1	1098402579.7476\\
37.2	1104354511.817\\
37.3	1117948195.279\\
37.4	1116979530.3162\\
37.5	1123491614.956\\
37.6	1129826667.4575\\
37.7	1127242691.2529\\
37.8	1126277576.0817\\
37.9	1127917017.3167\\
38	1129778620.1831\\
38.1	1132234809.5226\\
38.2	1130428043.9069\\
38.3	1130496227.3817\\
38.4	1131174434.0119\\
38.5	1131264677.5964\\
38.6	1130486941.4946\\
38.7	1127698748.6321\\
38.8	1128346731.9458\\
38.9	1131207661.5838\\
39	1141213078.9854\\
39.1	1144948798.7074\\
39.2	1148742653.1097\\
39.3	1156374927.7691\\
39.4	1167012101.9759\\
39.5	1169185376.6521\\
39.6	1168393917.0665\\
39.7	1171038776.8607\\
39.8	1181978944.4663\\
39.9	1186827991.0393\\
40	1185727429.7074\\
40.1	1189182735.0074\\
40.2	1195278690.2982\\
40.3	1196886524.4249\\
40.4	1198648024.67\\
40.5	1192796864.775\\
40.6	1194533946.9225\\
40.7	1197305763.4108\\
40.8	1199186418.4767\\
40.9	1198366068.7345\\
41	1197631024.4088\\
41.1	1198619882.67\\
41.2	1200638965.4908\\
41.3	1200729070.0593\\
41.4	1201487189.6697\\
41.5	1203977473.9469\\
41.6	1206406568.1025\\
41.7	1210202362.2478\\
41.8	1213299184.6702\\
41.9	1212344590.2116\\
42	1215113236.2315\\
42.1	1215191825.4085\\
42.2	1216120725.3947\\
42.3	1218305612.3715\\
42.4	1219545917.3518\\
42.5	1219974041.1196\\
42.6	1225819202.712\\
42.7	1238988956.5514\\
42.8	1241912797.8186\\
42.9	1244002905.1901\\
43	1243897355.8998\\
43.1	1243362306.2283\\
43.2	1242351055.3611\\
43.3	1244611985.3436\\
43.4	1248910405.6713\\
43.5	1251826949.749\\
43.6	1256472603.5017\\
43.7	1262531021.5702\\
43.8	1266729073.6775\\
43.9	1269107209.485\\
44	1271268898.1432\\
44.1	1274033405.3749\\
44.2	1275725984.3999\\
44.3	1274935965.3726\\
44.4	1277750788.7505\\
44.5	1279049902.8241\\
44.6	1279731645.9033\\
44.7	1282409681.6605\\
44.8	1288971626.2074\\
44.9	1296112957.6328\\
45	1306311802.2639\\
45.1	1306897043.9519\\
45.2	1308305152.5801\\
45.3	1308825768.5881\\
45.4	1309664897.3785\\
45.5	1310650990.2206\\
45.6	1312144672.6092\\
45.7	1311907227.3521\\
45.8	1311878248.857\\
45.9	1312294749.7377\\
46	1314506323.6148\\
46.1	1319556658.1277\\
46.2	1316967424.7274\\
46.3	1316987108.8967\\
46.4	1315940881.7247\\
46.5	1314444362.637\\
46.6	1315499997.7259\\
46.7	1315409004.103\\
46.8	1314508259.6215\\
46.9	1312090102.5915\\
47	1311962730.6328\\
47.1	1313805624.704\\
47.2	1316683052.7597\\
47.3	1312551715.4708\\
47.4	1313508016.3639\\
47.5	1311186891.996\\
47.6	1311092266.0061\\
47.7	1311549870.5431\\
47.8	1310248024.0236\\
47.9	1312142260.9733\\
48	1314713527.7055\\
48.1	1313999488.5716\\
48.2	1314652622.6918\\
48.3	1319787940.6103\\
48.4	1321506241.1643\\
48.5	1322547862.3009\\
48.6	1325398673.2799\\
48.7	1325871192.1294\\
48.8	1326012716.1465\\
48.9	1327923115.2179\\
49	1328454457.0001\\
49.1	1331951614.497\\
49.2	1337515103.1719\\
49.3	1337928896.5122\\
49.4	1340476160.5973\\
49.5	1344285019.8106\\
49.6	1346677004.0363\\
49.7	1350677681.9634\\
49.8	1352534760.33\\
49.9	1355737396.0907\\
50	1358506912.3453\\
50.1	1361089515.5233\\
50.2	1367286016.499\\
50.3	1370310724.5288\\
50.4	1370444119.9211\\
50.5	1372131811.6979\\
50.6	1374299860.6575\\
50.7	1376566877.0535\\
50.8	1376057202.834\\
50.9	1374983351.6992\\
51	1372916186.0439\\
51.1	1372282905.2523\\
51.2	1373979585.2589\\
51.3	1376785609.1131\\
51.4	1380074831.4843\\
51.5	1381372875.1168\\
51.6	1379776062.2553\\
51.7	1380415047.6136\\
51.8	1388955414.5982\\
51.9	1393510674.645\\
52	1397852387.5043\\
52.1	1396184032.1842\\
52.2	1396140507.7996\\
52.3	1400861719.1963\\
52.4	1402922048.3966\\
52.5	1405428127.3913\\
52.6	1408371775.7299\\
52.7	1411840610.0701\\
52.8	1413871343.9289\\
52.9	1414359854.588\\
53	1414450117.1929\\
53.1	1417937604.4933\\
53.2	1420662194.9537\\
53.3	1421263226.0935\\
53.4	1423513029.7296\\
53.5	1428353715.775\\
53.6	1434627586.5977\\
53.7	1437403939.1629\\
53.8	1439344967.0444\\
53.9	1438601600.2968\\
54	1437946555.9644\\
54.1	1438516371.9432\\
54.2	1444208033.5085\\
54.3	1450451803.7732\\
54.4	1450900280.3903\\
54.5	1456747364.0809\\
54.6	1469253633.6896\\
54.7	1475079729.293\\
54.8	1476689827.7538\\
54.9	1479544382.5388\\
55	1477743310.4021\\
55.1	1476858897.7627\\
55.2	1478991578.2506\\
55.3	1480889509.7309\\
55.4	1483293000.1857\\
55.5	1486630151.5665\\
55.6	1487906523.5372\\
55.7	1490092851.8804\\
55.8	1492800231.5284\\
55.9	1496897967.4567\\
56	1503447022.234\\
56.1	1510886324.5957\\
56.2	1515183534.6352\\
56.3	1515684323.2082\\
56.4	1530421868.5792\\
56.5	1526182007.037\\
56.6	1534658709.8211\\
56.7	1535475889.8426\\
56.8	1541542492.7064\\
56.9	1547079000.1211\\
57	1548295552.814\\
57.1	1548977612.2583\\
57.2	1550517853.3667\\
57.3	1551567178.0123\\
57.4	1544415794.2953\\
57.5	1537995008.0477\\
57.6	1536248653.3148\\
57.7	1537554615.6633\\
57.8	1541162988.5994\\
57.9	1541257172.7403\\
58	1539860591.511\\
58.1	1541696881.3501\\
58.2	1540714835.7386\\
58.3	1543763790.9764\\
58.4	1546414356.4385\\
58.5	1548996991.9442\\
58.6	1548506763.7937\\
58.7	1548671008.6344\\
58.8	1551001122.2898\\
58.9	1554247429.9202\\
59	1556755475.2099\\
59.1	1558813759.8915\\
59.2	1558719716.0389\\
59.3	1557569666.0419\\
59.4	1560291380.2985\\
59.5	1563815177.649\\
59.6	1561257314.3897\\
59.7	1561727147.6682\\
59.8	1564100843.9501\\
59.9	1564587019.8964\\
60	1567084133.0583\\
60.1	1570006783.2394\\
60.2	1571853985.0856\\
60.3	1574024327.6358\\
60.4	1579612594.9361\\
60.5	1584022132.4276\\
60.6	1583149544.9859\\
60.7	1591504915.7737\\
60.8	1603202057.7673\\
60.9	1602037555.3889\\
61	1604509829.757\\
61.1	1605798508.587\\
61.2	1604865004.826\\
61.3	1604001667.8196\\
61.4	1604406063.5145\\
61.5	1606526755.6253\\
61.6	1599173580.0047\\
61.7	1608972379.3194\\
61.8	1607949752.9882\\
61.9	1609001590.2422\\
62	1611049024.0355\\
62.1	1610769991.5533\\
62.2	1618388204.7204\\
62.3	1630058140.5457\\
62.4	1634027790.8576\\
62.5	1637746198.1754\\
62.6	1640729532.5749\\
62.7	1643588402.1267\\
62.8	1648205186.1575\\
62.9	1650716427.2161\\
63	1649484764.3946\\
63.1	1647460595.6766\\
63.2	1649475669.4122\\
63.3	1650836844.2628\\
63.4	1656676435.5714\\
63.5	1652085303.9649\\
63.6	1645068372.3408\\
63.7	1641750946.1122\\
63.8	1642898536.6467\\
63.9	1644777481.4389\\
64	1643831061.1389\\
64.1	1637509100.3363\\
64.2	1639954745.7417\\
64.3	1642545817.3164\\
64.4	1643777305.6185\\
64.5	1649052047.8071\\
64.6	1648758501.8872\\
64.7	1647202533.0142\\
64.8	1646086494.2279\\
64.9	1645301065.1324\\
65	1643518394.7243\\
65.1	1646983044.4075\\
65.2	1644522953.7125\\
65.3	1640736757.7926\\
65.4	1640789416.2161\\
65.5	1651678603.7729\\
65.6	1665133849.2297\\
65.7	1665875072.9505\\
65.8	1664834175.6125\\
65.9	1666196728.8044\\
66	1673395631.6711\\
66.1	1682481219.9649\\
66.2	1685683257.3731\\
66.3	1687475995.58\\
66.4	1689294625.5502\\
66.5	1690480936.1731\\
66.6	1689785241.8723\\
66.7	1692005246.3614\\
66.8	1697801921.4298\\
66.9	1700409752.3069\\
67	1707822413.3719\\
67.1	1720961738.8101\\
67.2	1725543641.7828\\
67.3	1728367246.4533\\
67.4	1729825154.5918\\
67.5	1731921791.8515\\
67.6	1739397315.4896\\
67.7	1746004680.2339\\
67.8	1753909409.6571\\
67.9	1767317968.1157\\
68	1774404754.5251\\
68.1	1773425377.3414\\
68.2	1774115264.6113\\
68.3	1777219017.164\\
68.4	1784591530.6052\\
68.5	1792052895.1885\\
68.6	1793513549.8149\\
68.7	1795373826.1081\\
68.8	1798869428.2401\\
68.9	1804217881.6912\\
69	1805114086.6037\\
69.1	1808517665.3892\\
69.2	1810823114.0657\\
69.3	1810891588.445\\
69.4	1809285962.3625\\
69.5	1810562232.5757\\
69.6	1814092611.5244\\
69.7	1814147450.4025\\
69.8	1818905920.875\\
69.9	1823922463.6926\\
70	1830286157.5843\\
70.1	1834876575.1751\\
70.2	1833906903.7975\\
70.3	1836045605.9891\\
70.4	1840951420.4946\\
70.5	1843905307.5169\\
70.6	1841454580.6111\\
70.7	1840562427.6949\\
70.8	1841658293.4687\\
70.9	1846466521.8144\\
71	1845780592.5896\\
71.1	1844106467.8787\\
71.2	1844117895.7581\\
71.3	1846042458.6235\\
71.4	1846226978.1287\\
71.5	1845858767.6191\\
71.6	1848015224.1116\\
71.7	1848937448.1653\\
71.8	1850183424.5598\\
71.9	1850945907.3649\\
72	1852761174.5429\\
72.1	1856033024.1775\\
72.2	1854944088.7283\\
72.3	1855192689.4453\\
72.4	1855415407.363\\
72.5	1855520608.5626\\
72.6	1855334487.5862\\
72.7	1857856031.4503\\
72.8	1861959653.0525\\
72.9	1865181942.2578\\
73	1863292849.768\\
73.1	1870472196.9098\\
73.2	1885442582.0068\\
73.3	1886427694.4611\\
73.4	1889659075.0368\\
73.5	1892850018.1758\\
73.6	1888945659.4234\\
73.7	1888740997.5047\\
73.8	1891715634.3958\\
73.9	1893194068.7043\\
74	1893856538.8091\\
74.1	1892293977.8561\\
74.2	1889609832.9644\\
74.3	1890922919.3732\\
74.4	1892544500.9575\\
74.5	1897331996.7435\\
74.6	1906119244.2767\\
74.7	1908247083.6038\\
74.8	1908458736.6136\\
74.9	1910545419.3166\\
75	1917496251.0133\\
75.1	1926696380.9182\\
75.2	1930920933.6591\\
75.3	1936547014.0836\\
75.4	1940007644.0649\\
75.5	1944059548.8681\\
75.6	1943986203.3649\\
75.7	1944134788.2432\\
75.8	1946803298.5569\\
75.9	1945479553.0443\\
76	1942811686.9705\\
76.1	1943213454.2021\\
76.2	1942576773.5806\\
76.3	1942816517.6378\\
76.4	1945954979.8581\\
76.5	1950048285.5243\\
76.6	1958580937.2508\\
76.7	1966087290.7549\\
76.8	1969489753.8198\\
76.9	1969848494.7049\\
77	1966988521.8671\\
77.1	1965679292.747\\
77.2	1968865294.9065\\
77.3	1970871744.9747\\
77.4	1977831886.0646\\
77.5	1982025659.3659\\
77.6	1981616171.3528\\
77.7	1979169189.5795\\
77.8	1980806881.2203\\
77.9	1981896094.4707\\
78	1981150322.4186\\
78.1	1981800765.7552\\
78.2	1982735524.4444\\
78.3	1982754407.2451\\
78.4	1982470816.9742\\
78.5	1985451673.7237\\
78.6	1988495947.7035\\
78.7	1989918424.7671\\
78.8	1989188477.7972\\
78.9	1987948173.6701\\
79	1992454027.4466\\
79.1	1999294664.9269\\
79.2	2006035052.8997\\
79.3	2020081690.9604\\
79.4	2021551875.7604\\
79.5	2022935809.5872\\
79.6	2024699096.5097\\
79.7	2025001472.889\\
79.8	2023076182.1583\\
79.9	2030568785.8913\\
80	2041752520.8657\\
80.1	2040351509.5251\\
80.2	2042983086.797\\
80.3	2046898440.9229\\
80.4	2044500054.9103\\
80.5	2041141109.99\\
80.6	2041698806.0314\\
80.7	2044850644.2062\\
80.8	2042698908.355\\
80.9	2039896243.5281\\
81	2037658153.6513\\
81.1	2036465289.0906\\
81.2	2039541252.3019\\
81.3	2043696710.0186\\
81.4	2043276699.1607\\
81.5	2044725575.592\\
81.6	2048606675.2547\\
81.7	2049071006.3786\\
81.8	2048137872.9761\\
81.9	2043923412.1441\\
82	2042831194.9546\\
82.1	2041065895.3898\\
82.2	2040774504.834\\
82.3	2043756921.8946\\
82.4	2042570979.5684\\
82.5	2043948713.9059\\
82.6	2045755375.8652\\
82.7	2045923198.0926\\
82.8	2047161665.9772\\
82.9	2045211874.0124\\
83	2040081213.5504\\
83.1	2040636768.0139\\
83.2	2043099401.639\\
83.3	2045478915.5034\\
83.4	2048461416.1856\\
83.5	2050879071.4479\\
83.6	2052658484.967\\
83.7	2054476910.3055\\
83.8	2056008560.416\\
83.9	2058044639.0081\\
84	2057283279.7193\\
84.1	2055288367.9109\\
84.2	2055251597.2935\\
84.3	2056893600.1729\\
84.4	2058031736.4433\\
84.5	2060305776.9766\\
84.6	2060992945.7848\\
84.7	2059191712.7549\\
84.8	2060091308.7848\\
84.9	2064847179.9934\\
85	2065164733.1657\\
85.1	2066931059.1557\\
85.2	2073450170.0216\\
85.3	2075757910.9707\\
85.4	2078991518.614\\
85.5	2086538643.3015\\
85.6	2088810276.1318\\
85.7	2090695152.2586\\
85.8	2088377410.6648\\
85.9	2090870684.1265\\
86	2096158853.9069\\
86.1	2102326796.9347\\
86.2	2104127203.6353\\
86.3	2099208391.0837\\
86.4	2103703327.422\\
86.5	2105398137.6703\\
86.6	2107216620.3532\\
86.7	2108018305.8731\\
86.8	2104903671.5649\\
86.9	2103605989.7349\\
87	2108213898.1034\\
87.1	2113475070.1552\\
87.2	2114108843.513\\
87.3	2120425823.7671\\
87.4	2126059141.5792\\
87.5	2122526207.9867\\
87.6	2121213543.0533\\
87.7	2122162819.5219\\
87.8	2124868208.0462\\
87.9	2135504152.2251\\
88	2140714743.45\\
88.1	2138161348.9777\\
88.2	2138860851.7837\\
88.3	2142290040.7915\\
88.4	2138548140.5887\\
88.5	2137930170.4841\\
88.6	2142263610.7945\\
88.7	2148706922.2026\\
88.8	2153717290.4605\\
88.9	2155688957.3963\\
89	2186426103.3062\\
89.1	2255848590.8376\\
89.2	2256569837.8611\\
89.3	2248656774.0002\\
89.4	2242207301.4437\\
89.5	2244830384.9439\\
89.6	2243653057.9268\\
89.7	2239417211.3556\\
89.8	2237876498.0279\\
89.9	2237668600.4328\\
90	2238172071.8273\\
90.1	2239568659.5222\\
90.2	2241671616.4708\\
90.3	2239866302.7541\\
90.4	2239084892.589\\
90.5	2238944186.2675\\
90.6	2237865891.6053\\
90.7	2240624645.0122\\
90.8	2245822016.0347\\
90.9	2247445402.8937\\
91	2247046090.1738\\
91.1	2251381843.5307\\
91.2	2264213109.6546\\
91.3	2268958912.215\\
91.4	2274240909.022\\
91.5	2277585308.4409\\
91.6	2281812282.3187\\
91.7	2286357363.4844\\
91.8	2288588337.7288\\
91.9	2290228116.8962\\
92	2301631154.6009\\
92.1	2337272629.9255\\
92.2	2332599650.7904\\
92.3	2330639672.5383\\
92.4	2330546795.6919\\
92.5	2330820718.5566\\
92.6	2327525609.9252\\
92.7	2326339155.2578\\
92.8	2334933482.7147\\
92.9	2355444458.8729\\
93	2359491849.2238\\
93.1	2359366681.9672\\
93.2	2362181059.0864\\
93.3	2365303004.539\\
93.4	2372819520.6125\\
93.5	2379927810.3876\\
93.6	2381780271.9991\\
93.7	2385715110.9694\\
93.8	2390900603.8042\\
93.9	2392892988.2466\\
94	2389090656.8036\\
94.1	2387275756.6748\\
94.2	2385726900.281\\
94.3	2382616713.4412\\
94.4	2379724076.2682\\
94.5	2380589573.0291\\
94.6	2384809351.4148\\
94.7	2386513804.8236\\
94.8	2387316574.8983\\
94.9	2392097186.4756\\
95	2394342158.0036\\
95.1	2395661921.6573\\
95.2	2396195773.3245\\
95.3	2396488469.5997\\
95.4	2396079182.0188\\
95.5	2398295100.6444\\
95.6	2403261442.12\\
95.7	2406278970.0452\\
95.8	2410223745.2658\\
95.9	2413997949.8463\\
96	2417192988.0392\\
96.1	2420331432.9727\\
96.2	2429484546.4172\\
96.3	2438413698.6497\\
96.4	2439547707.6858\\
96.5	2438692207.4484\\
96.6	2439318171.1559\\
96.7	2442481475.1183\\
96.8	2443917127.7291\\
96.9	2447237096.3054\\
97	2450379055.586\\
97.1	2455752880.7046\\
97.2	2463421725.768\\
97.3	2468716584.0427\\
97.4	2473394218.3513\\
97.5	2475644416.1858\\
97.6	2475421301.0923\\
97.7	2479615930.9289\\
97.8	2480707742.5339\\
97.9	2479824636.866\\
98	2479124407.4117\\
98.1	2476014830.6727\\
98.2	2474358998.3288\\
98.3	2476628489.6727\\
98.4	2479423235.1316\\
98.5	2483319638.3091\\
98.6	2484557073.1462\\
98.7	2494763505.7914\\
98.8	2512202235.9907\\
98.9	2511083925.2665\\
99	2512385001.1453\\
99.1	2515175978.7071\\
99.2	2519896516.6484\\
99.3	2526145894.8657\\
99.4	2528827832.1583\\
99.5	2531273789.3124\\
99.6	2531410654.3204\\
99.7	2535016865.2887\\
99.8	2522580921.2148\\
99.9	2592106061.8661\\
};
\addlegendentry{Total energy};

\addplot [color=mycolor2,solid]
  table[row sep=crcr]{%
0	-857.0284\\
0.1	-133.2825\\
0.2	2576.5563\\
0.3	2281.3926\\
0.4	2752.7477\\
0.5	2594.4902\\
0.6	1989.902\\
0.7	1856.2758\\
0.8	2670.3686\\
0.9	2521.1948\\
1	2711.3269\\
1.1	2702.3887\\
1.2	2222.4094\\
1.3	1895.6478\\
1.4	2760.4033\\
1.5	1977.6586\\
1.6	2936.1026\\
1.7	2636.902\\
1.8	2542.5431\\
1.9	2568.4542\\
2	2642.5013\\
2.1	2101.5411\\
2.2	2482.1399\\
2.3	2690.3318\\
2.4	1887.1138\\
2.5	2160.667\\
2.6	3435.6702\\
2.7	2517.3828\\
2.8	2840.8781\\
2.9	2508.4154\\
3	2549.2923\\
3.1	1954.4631\\
3.2	1976.2074\\
3.3	2058.327\\
3.4	2419.3428\\
3.5	2583.7818\\
3.6	2497.2806\\
3.7	2727.5677\\
3.8	2018.0071\\
3.9	2338.7138\\
4	2590.1965\\
4.1	2669.299\\
4.2	2603.2727\\
4.3	2228.7105\\
4.4	2925.0473\\
4.5	2226.8022\\
4.6	2720.5287\\
4.7	2734.0307\\
4.8	2344.0141\\
4.9	2498.6158\\
5	2319.0283\\
5.1	2757.7532\\
5.2	2436.3019\\
5.3	2440.45\\
5.4	2153.1431\\
5.5	2449.4314\\
5.6	2750.0143\\
5.7	3178.0237\\
5.8	2619.5416\\
5.9	2409.1643\\
6	2168.5\\
6.1	2218.7701\\
6.2	2344.9023\\
6.3	1823.1828\\
6.4	2506.5739\\
6.5	2334.5655\\
6.6	2802.2436\\
6.7	2597.145\\
6.8	2055.2902\\
6.9	2583.3201\\
7	2877.0557\\
7.1	2302.1299\\
7.2	2337.6708\\
7.3	2812.7387\\
7.4	2670.388\\
7.5	2575.486\\
7.6	2385.416\\
7.7	2773.7285\\
7.8	1827.87\\
7.9	3008.7087\\
8	2348.938\\
8.1	2104.5198\\
8.2	2508.0973\\
8.3	2638.2275\\
8.4	2333.7847\\
8.5	2618.1319\\
8.6	2150.948\\
8.7	2612.5651\\
8.8	2398.8566\\
8.9	2567.571\\
9	2633.1612\\
9.1	2641.1032\\
9.2	2636.8076\\
9.3	3004.0463\\
9.4	2585.6883\\
9.5	2474.264\\
9.6	2527.927\\
9.7	2216.9186\\
9.8	2293.8317\\
9.9	2469.3886\\
10	2224.5888\\
10.1	2931.4828\\
10.2	2617.3429\\
10.3	2959.4956\\
10.4	2759.5183\\
10.5	2514.7843\\
10.6	2706.6276\\
10.7	2249.1643\\
10.8	2573.3541\\
10.9	2226.9087\\
11	2287.5868\\
11.1	2395.121\\
11.2	2039.3587\\
11.3	1970.8763\\
11.4	2262.551\\
11.5	1723.1027\\
11.6	2240.1882\\
11.7	2446.0537\\
11.8	1934.57\\
11.9	2481.1682\\
12	2614.7286\\
12.1	2648.0994\\
12.2	1999.6695\\
12.3	2629.0705\\
12.4	2487.4821\\
12.5	2107.9636\\
12.6	3975.0432\\
12.7	2371.9913\\
12.8	2297.2355\\
12.9	2865.0719\\
13	2540.3668\\
13.1	2292.1583\\
13.2	1926.4242\\
13.3	2061.3916\\
13.4	3025.6736\\
13.5	2762.3162\\
13.6	3001.3668\\
13.7	2578.3007\\
13.8	2296.453\\
13.9	2657.2112\\
14	2617.4945\\
14.1	2738.1555\\
14.2	2041.1622\\
14.3	2584.7228\\
14.4	3036.1019\\
14.5	2533.7327\\
14.6	2603.1436\\
14.7	2620.3555\\
14.8	2614.9338\\
14.9	2163.5346\\
15	2542.981\\
15.1	2327.7708\\
15.2	2778.9173\\
15.3	1919.5084\\
15.4	2422.4056\\
15.5	2734.8021\\
15.6	2609.2352\\
15.7	2299.46\\
15.8	2579.9673\\
15.9	2056.6533\\
16	1995.6017\\
16.1	2359.5028\\
16.2	2024.2216\\
16.3	3366.2327\\
16.4	2049.6949\\
16.5	2549.3654\\
16.6	3193.9206\\
16.7	2709.5838\\
16.8	3113.7107\\
16.9	2450.9856\\
17	2758.3732\\
17.1	3199.8095\\
17.2	2328.7952\\
17.3	2211.8868\\
17.4	2424.046\\
17.5	2641.6388\\
17.6	2420.9898\\
17.7	2622.9501\\
17.8	2370.7863\\
17.9	2708.5073\\
18	2457.8207\\
18.1	3210.1768\\
18.2	2076.5251\\
18.3	2192.1093\\
18.4	2407.7809\\
18.5	2294.0654\\
18.6	2266.4117\\
18.7	2026.6481\\
18.8	2835.6632\\
18.9	2624.0071\\
19	2551.6331\\
19.1	3050.4186\\
19.2	2393.8602\\
19.3	2655.1456\\
19.4	2300.8517\\
19.5	2314.1108\\
19.6	3012.8755\\
19.7	2285.4557\\
19.8	2460.8719\\
19.9	2307.8748\\
20	2944.2872\\
20.1	2576.067\\
20.2	2807.78\\
20.3	2473.284\\
20.4	2052.1989\\
20.5	2537.0949\\
20.6	3453.8635\\
20.7	2461.376\\
20.8	2167.3255\\
20.9	2554.4274\\
21	2610.5935\\
21.1	2131.2183\\
21.2	2325.2927\\
21.3	2434.2203\\
21.4	2548.448\\
21.5	1881.9144\\
21.6	2209.9735\\
21.7	2854.3342\\
21.8	2753.0658\\
21.9	3331.822\\
22	2530.408\\
22.1	2443.7413\\
22.2	3843.7529\\
22.3	3218.5336\\
22.4	2630.0758\\
22.5	2178.9362\\
22.6	2152.6628\\
22.7	2381.9493\\
22.8	2283.0295\\
22.9	2096.9607\\
23	2250.1247\\
23.1	2641.9042\\
23.2	3058.6302\\
23.3	2238.1462\\
23.4	2261.9824\\
23.5	2913.7582\\
23.6	2644.5575\\
23.7	2577.4434\\
23.8	2670.6911\\
23.9	2890.5759\\
24	1699.9347\\
24.1	3758.5186\\
24.2	2196.5972\\
24.3	2868.4904\\
24.4	2366.3985\\
24.5	1896.7284\\
24.6	2539.7214\\
24.7	3179.7968\\
24.8	2009.2106\\
24.9	3098.8898\\
25	2650.1076\\
25.1	2098.6224\\
25.2	2579.5121\\
25.3	2807.9432\\
25.4	2168.9133\\
25.5	2169.2697\\
25.6	2980.5486\\
25.7	2787.5689\\
25.8	1780.5103\\
25.9	3281.5761\\
26	2539.7689\\
26.1	2000.9159\\
26.2	3101.9406\\
26.3	2367.3129\\
26.4	2309.3584\\
26.5	3582.597\\
26.6	2507.676\\
26.7	2826.9261\\
26.8	3398.602\\
26.9	2220.7881\\
27	2213.8377\\
27.1	2413.2348\\
27.2	2362.8716\\
27.3	3315.7763\\
27.4	2979.1235\\
27.5	2525.5892\\
27.6	3160.6176\\
27.7	2396.0558\\
27.8	2614.7922\\
27.9	2531.9669\\
28	1854.8989\\
28.1	2166.7784\\
28.2	2079.8877\\
28.3	2781.1753\\
28.4	2543.1752\\
28.5	3097.6558\\
28.6	2195.7115\\
28.7	3668.0379\\
28.8	2645.0743\\
28.9	2330.5885\\
29	2661.8528\\
29.1	2280.7531\\
29.2	2595.5264\\
29.3	2300.8238\\
29.4	2081.2465\\
29.5	2562.2499\\
29.6	2050.1701\\
29.7	2491.0785\\
29.8	1786.5721\\
29.9	3202.259\\
30	2331.1036\\
30.1	2528.889\\
30.2	2637.9541\\
30.3	2328.2885\\
30.4	1832.7392\\
30.5	2313.6308\\
30.6	2107.2776\\
30.7	2733.0278\\
30.8	2861.8578\\
30.9	2277.4522\\
31	2510.3305\\
31.1	1996.3267\\
31.2	2331.3744\\
31.3	2464.748\\
31.4	2507.8015\\
31.5	2266.9371\\
31.6	3334.3624\\
31.7	3038.6189\\
31.8	2645.3265\\
31.9	2535.3271\\
32	2908.9916\\
32.1	2462.1889\\
32.2	2460.4297\\
32.3	1980.8614\\
32.4	2441.9125\\
32.5	2181.2269\\
32.6	2921.7252\\
32.7	3242.7542\\
32.8	1994.8382\\
32.9	3262.1558\\
33	2739.3475\\
33.1	3629.5389\\
33.2	2174.4726\\
33.3	2341.1536\\
33.4	2666.8943\\
33.5	2715.6804\\
33.6	2512.9207\\
33.7	2900.6259\\
33.8	2259.3263\\
33.9	2327.6115\\
34	2283.1563\\
34.1	2789.5721\\
34.2	2662.1887\\
34.3	2637.6767\\
34.4	2449.0674\\
34.5	2055.0811\\
34.6	2087.8138\\
34.7	3030.1046\\
34.8	2966.0863\\
34.9	2886.8971\\
35	2456.1212\\
35.1	2094.3165\\
35.2	3143.9017\\
35.3	3166.5932\\
35.4	2751.3242\\
35.5	2277.958\\
35.6	2700.7382\\
35.7	2009.9651\\
35.8	2623.8659\\
35.9	2355.1408\\
36	2104.0688\\
36.1	2817.4705\\
36.2	2211.3884\\
36.3	2519.7951\\
36.4	2522.0382\\
36.5	2138.5962\\
36.6	2348.6256\\
36.7	2200.5045\\
36.8	2630.558\\
36.9	1973.4607\\
37	2678.2087\\
37.1	2467.9751\\
37.2	2768.6769\\
37.3	2415.257\\
37.4	2780.4158\\
37.5	3243.9821\\
37.6	2437.0262\\
37.7	2387.0437\\
37.8	2810.8749\\
37.9	2020.057\\
38	2578.655\\
38.1	2125.5747\\
38.2	2495.2278\\
38.3	2413.9006\\
38.4	2607.4702\\
38.5	2287.5415\\
38.6	1975.506\\
38.7	2680.2338\\
38.8	2263.0992\\
38.9	2655.0753\\
39	2279.6457\\
39.1	2789.2465\\
39.2	3421.2422\\
39.3	3074.3496\\
39.4	2430.182\\
39.5	2592.6556\\
39.6	2429.713\\
39.7	2895.8417\\
39.8	2737.2421\\
39.9	1589.2373\\
40	2482.5476\\
40.1	2895.2259\\
40.2	2464.0945\\
40.3	2201.7066\\
40.4	2150.9709\\
40.5	3537.7584\\
40.6	2815.6817\\
40.7	2103.5072\\
40.8	2400.5215\\
40.9	2465.6406\\
41	2416.4203\\
41.1	2344.7888\\
41.2	2382.7903\\
41.3	2215.7876\\
41.4	2765.9183\\
41.5	2688.8625\\
41.6	2413.697\\
41.7	2355.7802\\
41.8	2678.673\\
41.9	2200.7516\\
42	2554.1944\\
42.1	2523.5657\\
42.2	2020.9564\\
42.3	2592.522\\
42.4	2251.1181\\
42.5	2612.1299\\
42.6	3105.3396\\
42.7	2611.3589\\
42.8	2052.1451\\
42.9	2540.1117\\
43	2763.7475\\
43.1	2288.9629\\
43.2	2851.5808\\
43.3	2520.7487\\
43.4	1961.1845\\
43.5	2380.1866\\
43.6	2214.5901\\
43.7	2325.7981\\
43.8	2367.0751\\
43.9	2231.1557\\
44	2376.0235\\
44.1	2671.7417\\
44.2	2518.1735\\
44.3	2926.4697\\
44.4	2576.2257\\
44.5	1993.9101\\
44.6	1968.2661\\
44.7	2708.8203\\
44.8	3138.2336\\
44.9	3941.3331\\
45	2465.771\\
45.1	2687.167\\
45.2	2602.7883\\
45.3	2679.0511\\
45.4	2487.8144\\
45.5	1817.3732\\
45.6	2052.4777\\
45.7	2491.3613\\
45.8	2505.8286\\
45.9	2191.8782\\
46	2897.2656\\
46.1	2300.7183\\
46.2	3183.8157\\
46.3	2468.6303\\
46.4	2148.2166\\
46.5	1817.7402\\
46.6	3294.4993\\
46.7	2372.8293\\
46.8	2440.6957\\
46.9	2935.7665\\
47	2310.9111\\
47.1	2534.6578\\
47.2	1949.2985\\
47.3	3240.4871\\
47.4	3031.8422\\
47.5	2170.4483\\
47.6	2930.5387\\
47.7	2420.3813\\
47.8	2830.6622\\
47.9	2385.4327\\
48	2821.2394\\
48.1	2383.333\\
48.2	2591.54\\
48.3	2342.5742\\
48.4	2985.714\\
48.5	3058.1146\\
48.6	2294.2754\\
48.7	1810.996\\
48.8	2210.2184\\
48.9	2121.4795\\
49	2192.8395\\
49.1	2584.4996\\
49.2	2558.0822\\
49.3	2387.7422\\
49.4	2454.5525\\
49.5	2631.102\\
49.6	2543.2736\\
49.7	2389.8003\\
49.8	2655.0379\\
49.9	2612.024\\
50	2478.518\\
50.1	2462.9177\\
50.2	2445.3504\\
50.3	2364.4666\\
50.4	2377.2133\\
50.5	2410.1174\\
50.6	2357.038\\
50.7	2723.1809\\
50.8	2893.4324\\
50.9	2722.3699\\
51	2146.3349\\
51.1	2435.4897\\
51.2	2604.2993\\
51.3	2878.3523\\
51.4	2325.4999\\
51.5	2334.7038\\
51.6	2260.2874\\
51.7	3202.3521\\
51.8	1785.9158\\
51.9	2524.8514\\
52	2291.307\\
52.1	2218.8778\\
52.2	2877.3565\\
52.3	2490.7283\\
52.4	2957.1496\\
52.5	2502.509\\
52.6	2149.8101\\
52.7	2386.0009\\
52.8	2935.464\\
52.9	2559.3515\\
53	2456.1696\\
53.1	2828.6588\\
53.2	2551.8906\\
53.3	2405.4075\\
53.4	2652.5786\\
53.5	2590.2757\\
53.6	2487.7755\\
53.7	2679.4275\\
53.8	2350.6985\\
53.9	2110.741\\
54	2257.5828\\
54.1	2080.9061\\
54.2	3297.3998\\
54.3	2691.7067\\
54.4	2388.9615\\
54.5	3174.2475\\
54.6	2731.3686\\
54.7	2516.5259\\
54.8	3289.4767\\
54.9	1924.9695\\
55	2766.1375\\
55.1	2131.2849\\
55.2	2254.7825\\
55.3	2547.8567\\
55.4	2563.9244\\
55.5	2761.6474\\
55.6	2205.034\\
55.7	2019.7878\\
55.8	2448.9277\\
55.9	2410.4524\\
56	2930.6713\\
56.1	2402.1295\\
56.2	2603.007\\
56.3	3683.3472\\
56.4	2742.6557\\
56.5	3293.4504\\
56.6	2107.8229\\
56.7	2418.8238\\
56.8	2568.8438\\
56.9	2389.0724\\
57	2699.8523\\
57.1	2194.8103\\
57.2	2258.3374\\
57.3	1757.7264\\
57.4	2649.1758\\
57.5	2197.9996\\
57.6	2466.6136\\
57.7	2402.3114\\
57.8	2554.692\\
57.9	2399.8411\\
58	2570.0922\\
58.1	2043.914\\
58.2	2631.2293\\
58.3	2437.7992\\
58.4	3088.4263\\
58.5	2197.4703\\
58.6	1928.0488\\
58.7	2975.3599\\
58.8	2825.9231\\
58.9	3053.4451\\
59	3021.8612\\
59.1	2180.2691\\
59.2	2155.0538\\
59.3	2595.9099\\
59.4	2782.1033\\
59.5	2481.921\\
59.6	2827.9401\\
59.7	3059.0398\\
59.8	2318.9975\\
59.9	2049.8757\\
60	2724.8635\\
60.1	2438.8575\\
60.2	2510.2091\\
60.3	1828.3558\\
60.4	2640.5143\\
60.5	2444.0667\\
60.6	2382.4147\\
60.7	3321.8875\\
60.8	2306.2345\\
60.9	2628.4167\\
61	2388.0738\\
61.1	2394.0151\\
61.2	2600.5308\\
61.3	2080.2166\\
61.4	2965.4227\\
61.5	1697.5533\\
61.6	3374.9548\\
61.7	2456.3509\\
61.8	2448.2461\\
61.9	2051.8876\\
62	2608.3455\\
62.1	2721.5791\\
62.2	2695.3866\\
62.3	2180.7499\\
62.4	2224.5091\\
62.5	2401.6036\\
62.6	2028.4444\\
62.7	2582.7977\\
62.8	2278.8929\\
62.9	2406.5087\\
63	2291.317\\
63.1	2353.7265\\
63.2	2764.4142\\
63.3	3137.0303\\
63.4	2826.8054\\
63.5	2811.7778\\
63.6	2148.0678\\
63.7	1991.1776\\
63.8	1988.2709\\
63.9	2690.0668\\
64	2763.9714\\
64.1	3238.8348\\
64.2	2321.2339\\
64.3	2410.4853\\
64.4	3462.1643\\
64.5	2670.5681\\
64.6	3061.072\\
64.7	2321.4765\\
64.8	1889.9755\\
64.9	2736.8043\\
65	3093.3643\\
65.1	2310.7775\\
65.2	2622.0662\\
65.3	2847.3149\\
65.4	1963.753\\
65.5	2602.6132\\
65.6	1995.036\\
65.7	2582.7904\\
65.8	2724.8618\\
65.9	2190.0175\\
66	2546.007\\
66.1	2203.8672\\
66.2	1951.2329\\
66.3	2873.1848\\
66.4	2303.3274\\
66.5	2249.6535\\
66.6	2893.7605\\
66.7	2518.1425\\
66.8	2535.134\\
66.9	2607.0622\\
67	2502.1677\\
67.1	2571.0766\\
67.2	2167.2162\\
67.3	2350.862\\
67.4	2683.8918\\
67.5	2352.9227\\
67.6	3083.3793\\
67.7	2635.6641\\
67.8	2948.4259\\
67.9	3395.0968\\
68	2602.7572\\
68.1	2772.983\\
68.2	2634.0047\\
68.3	2344.0692\\
68.4	2517.6485\\
68.5	2300.7152\\
68.6	2222.8429\\
68.7	2525.0562\\
68.8	2372.2091\\
68.9	2411.2377\\
69	2120.7751\\
69.1	2660.858\\
69.2	2776.3099\\
69.3	2295.9831\\
69.4	2255.7594\\
69.5	2583.6528\\
69.6	2194.2958\\
69.7	3174.8569\\
69.8	2648.5834\\
69.9	3066.6867\\
70	3153.5028\\
70.1	2478.3617\\
70.2	2808.953\\
70.3	2604.3209\\
70.4	2519.4357\\
70.5	2733.6514\\
70.6	2707.3863\\
70.7	2452.7867\\
70.8	2912.3457\\
70.9	2727.0136\\
71	2784.6839\\
71.1	2103.6782\\
71.2	2759.893\\
71.3	2350.2806\\
71.4	2452.8131\\
71.5	2355.8326\\
71.6	2653.1457\\
71.7	2475.7623\\
71.8	2281.2045\\
71.9	2247.8534\\
72	2120.7296\\
72.1	2230.4123\\
72.2	2771.8493\\
72.3	2564.6939\\
72.4	2610.1254\\
72.5	2355.8365\\
72.6	2596.6458\\
72.7	2196.026\\
72.8	3309.9729\\
72.9	2544.3677\\
73	2569.7705\\
73.1	3500.1095\\
73.2	2449.279\\
73.3	2374.4726\\
73.4	3180.3319\\
73.5	1952.5744\\
73.6	2240.5548\\
73.7	2465.6403\\
73.8	2494.9027\\
73.9	2992.8309\\
74	2459.5598\\
74.1	2673.7318\\
74.2	2591.1902\\
74.3	2231.9743\\
74.4	2742.8586\\
74.5	2664.2062\\
74.6	2556.705\\
74.7	2000.2335\\
74.8	2388.2466\\
74.9	2123.7475\\
75	2860.0508\\
75.1	2523.5709\\
75.2	2462.1591\\
75.3	2261.5001\\
75.4	2311.9977\\
75.5	2530.8879\\
75.6	2465.1132\\
75.7	1960.6278\\
75.8	2274.555\\
75.9	2843.467\\
76	2296.0021\\
76.1	2511.6882\\
76.2	2971.1393\\
76.3	2397.3642\\
76.4	2796.5365\\
76.5	2745.287\\
76.6	2646.0156\\
76.7	2389.9862\\
76.8	2396.617\\
76.9	2666.6877\\
77	2127.6916\\
77.1	2374.2071\\
77.2	2383.4823\\
77.3	3060.4473\\
77.4	2491.1263\\
77.5	2511.5227\\
77.6	1885.2009\\
77.7	2350.8787\\
77.8	2131.5917\\
77.9	2227.6084\\
78	2305.0244\\
78.1	2222.7597\\
78.2	2083.1019\\
78.3	2150.4954\\
78.4	2431.0946\\
78.5	2207.3195\\
78.6	2167.1117\\
78.7	2141.3549\\
78.8	2269.1265\\
78.9	2079.5591\\
79	2795.3816\\
79.1	2015.3326\\
79.2	3037.252\\
79.3	2500.6269\\
79.4	2327.2603\\
79.5	2057.9611\\
79.6	1904.4214\\
79.7	2720.8584\\
79.8	2628.9312\\
79.9	2843.3942\\
80	2175.5364\\
80.1	2285.1624\\
80.2	2836.9847\\
80.3	2556.2064\\
80.4	2454.2943\\
80.5	2235.4612\\
80.6	2799.2866\\
80.7	2230.3975\\
80.8	2561.04\\
80.9	2009.2339\\
81	2400.7348\\
81.1	2612.7831\\
81.2	2465.8285\\
81.3	2820.7192\\
81.4	2163.9828\\
81.5	2504.4184\\
81.6	2684.1497\\
81.7	2335.6844\\
81.8	2138.8463\\
81.9	3099.7564\\
82	2331.3671\\
82.1	2642.9306\\
82.2	2403.6135\\
82.3	2436.0978\\
82.4	2579.0581\\
82.5	2368.2935\\
82.6	2617.8532\\
82.7	1961.6099\\
82.8	2604.2963\\
82.9	2319.5\\
83	2403.643\\
83.1	2278.7727\\
83.2	2284.5214\\
83.3	2536.8937\\
83.4	2873.678\\
83.5	2207.1463\\
83.6	2640.7316\\
83.7	2067.7051\\
83.8	2582.9467\\
83.9	2608.5348\\
84	2532.0562\\
84.1	2654.8971\\
84.2	2788.7378\\
84.3	2544.2547\\
84.4	2295.7431\\
84.5	2793.7958\\
84.6	2560.8413\\
84.7	2582.3933\\
84.8	2934.3363\\
84.9	2469.2885\\
85	2858.9707\\
85.1	2611.7695\\
85.2	2090.1464\\
85.3	2186.5731\\
85.4	2687.1572\\
85.5	2364.6639\\
85.6	2338.8672\\
85.7	2346.337\\
85.8	2890.1017\\
85.9	2387.4615\\
86	2532.2254\\
86.1	2476.4176\\
86.2	2737.0368\\
86.3	3255.4977\\
86.4	2388.8475\\
86.5	2203.3506\\
86.6	2397.7799\\
86.7	2769.6747\\
86.8	2406.4247\\
86.9	2284.4584\\
87	2538.0418\\
87.1	2586.4214\\
87.2	2739.8083\\
87.3	3152.2386\\
87.4	3407.5024\\
87.5	2957.0868\\
87.6	2576.6446\\
87.7	2387.4997\\
87.8	3171.5802\\
87.9	2557.0033\\
88	2023.8203\\
88.1	2113.5004\\
88.2	2009.4529\\
88.3	2358.6432\\
88.4	2352.8598\\
88.5	2383.0657\\
88.6	2001.8521\\
88.7	2291.4812\\
88.8	2389.3471\\
88.9	2444.6248\\
89	3959.8108\\
89.1	2409.1892\\
89.2	2571.8174\\
89.3	2728.402\\
89.4	1892.2453\\
89.5	2702.3309\\
89.6	2261.1952\\
89.7	2498.0326\\
89.8	2140.732\\
89.9	2122.5929\\
90	2204.8992\\
90.1	2821.4051\\
90.2	2526.152\\
90.3	2894.44\\
90.4	2105.0313\\
90.5	2670.7248\\
90.6	1965.7262\\
90.7	2238.0228\\
90.8	2398.5499\\
90.9	2180.0784\\
91	2126.4253\\
91.1	3285.8705\\
91.2	2562.5308\\
91.3	2472.9794\\
91.4	2199.2799\\
91.5	2526.8236\\
91.6	2375.1838\\
91.7	2350.1536\\
91.8	2556.9812\\
91.9	1980.2704\\
92	4719.1901\\
92.1	2933.4546\\
92.2	2165.3249\\
92.3	2788.6201\\
92.4	2704.2438\\
92.5	2280.663\\
92.6	1884.1424\\
92.7	2191.6387\\
92.8	3039.0155\\
92.9	2258.0906\\
93	1946.6326\\
93.1	3049.0742\\
93.2	2505.2489\\
93.3	3072.9934\\
93.4	2741.8917\\
93.5	2350.8586\\
93.6	2315.5244\\
93.7	2129.332\\
93.8	3099.5152\\
93.9	2505.4613\\
94	2358.0177\\
94.1	2331.7999\\
94.2	2608.5162\\
94.3	2631.5408\\
94.4	2340.998\\
94.5	2243.2255\\
94.6	2442.521\\
94.7	2240.8592\\
94.8	2562.0628\\
94.9	2362.1148\\
95	2185.9575\\
95.1	2273.7198\\
95.2	2248.013\\
95.3	1989.6915\\
95.4	2666.9476\\
95.5	2942.4125\\
95.6	2401.8932\\
95.7	2218.7525\\
95.8	2131.2067\\
95.9	2705.0061\\
96	1675.5385\\
96.1	2143.0542\\
96.2	2477.2473\\
96.3	2230.6587\\
96.4	2611.1197\\
96.5	2234.3516\\
96.6	2578.2167\\
96.7	2560.2412\\
96.8	2582.6303\\
96.9	2509.2163\\
97	2391.6994\\
97.1	2574.0616\\
97.2	2078.7788\\
97.3	2740.162\\
97.4	2387.4407\\
97.5	1916.9423\\
97.6	2312.0046\\
97.7	2604.3082\\
97.8	2548.0345\\
97.9	2606.3608\\
98	2369.6807\\
98.1	2548.9752\\
98.2	2101.6394\\
98.3	2181.5208\\
98.4	2434.3797\\
98.5	2169.6727\\
98.6	2211.3396\\
98.7	3196.6253\\
98.8	2836.2252\\
98.9	2623.4341\\
99	2377.8663\\
99.1	2576.0013\\
99.2	2881.9668\\
99.3	2475.5186\\
99.4	2484.8205\\
99.5	2424.5316\\
99.6	3089.3291\\
99.7	2418.2766\\
99.8	4694.522\\
99.9	3373.073\\
};
\addlegendentry{Potential energy};

\addplot [color=mycolor3,solid]
  table[row sep=crcr]{%
0	0\\
0.1	24228.8778\\
0.2	365906.2625\\
0.3	1342856.6825\\
0.4	2287956.9957\\
0.5	4025500.2065\\
0.6	5096148.2047\\
0.7	5826424.9326\\
0.8	6564537.3184\\
0.9	7855869.6138\\
1	9839559.5993\\
1.1	12586209.6901\\
1.2	13904298.8711\\
1.3	14361753.9943\\
1.4	15124819.708\\
1.5	16946588.7993\\
1.6	17906707.8844\\
1.7	20364831.2118\\
1.8	22866874.9912\\
1.9	25191310.5318\\
2	28340183.9514\\
2.1	28837348.8524\\
2.2	30328648.1848\\
2.3	32525181.3971\\
2.4	34632812.9216\\
2.5	35310842.9995\\
2.6	38609758.7291\\
2.7	48635493.1658\\
2.8	50148797.2767\\
2.9	52592438.9313\\
3	54227847.8584\\
3.1	55549409.5762\\
3.2	56351907.3023\\
3.3	56768414.87\\
3.4	57218082.3805\\
3.5	58581811.5823\\
3.6	60566597.478\\
3.7	62689339.9583\\
3.8	64322190.0818\\
3.9	65760383.8159\\
4	67436173.4977\\
4.1	69549540.4543\\
4.2	70427719.1358\\
4.3	71844947.5792\\
4.4	73322503.1577\\
4.5	75171214.0888\\
4.6	76400465.9044\\
4.7	78040001.9136\\
4.8	80433965.9633\\
4.9	81017833.6631\\
5	81311190.7536\\
5.1	82970537.0358\\
5.2	85799838.7095\\
5.3	88122368.6511\\
5.4	89399512.3212\\
5.5	91375847.2047\\
5.6	94521414.0926\\
5.7	96457642.7607\\
5.8	99446281.0953\\
5.9	101520339.2352\\
6	102527559.3694\\
6.1	103487094.5507\\
6.2	104607315.1944\\
6.3	105363259.2624\\
6.4	105529579.9489\\
6.5	106506609.307\\
6.6	108341686.1367\\
6.7	110119112.6963\\
6.8	111761128.0872\\
6.9	113231181.7155\\
7	116166196.0492\\
7.1	118102188.5092\\
7.2	119353293.9519\\
7.3	121434080.4074\\
7.4	125411927.0314\\
7.5	126662258.2939\\
7.6	127858517.9445\\
7.7	128491585.4633\\
7.8	131507839.2002\\
7.9	132865406.3836\\
8	135950727.5754\\
8.1	137420561.7564\\
8.2	139359371.0558\\
8.3	141301087.5987\\
8.4	142344208.3193\\
8.5	142515015.572\\
8.6	144267957.3389\\
8.7	145167989.4873\\
8.8	145449228.2695\\
8.9	147641324.093\\
9	151445116.1181\\
9.1	152872899.0435\\
9.2	153356499.6101\\
9.3	156233428.0221\\
9.4	162085555.6639\\
9.5	163919327.8789\\
9.6	165652438.1144\\
9.7	169056758.3817\\
9.8	171537809.9555\\
9.9	172043545.6621\\
10	172734582.2359\\
10.1	173682376.0815\\
10.2	175730097.0913\\
10.3	183292210.7012\\
10.4	200795080.8461\\
10.5	201754100.1017\\
10.6	202523019.9416\\
10.7	203420327.4016\\
10.8	203622239.0628\\
10.9	205180533.2132\\
11	207084965.3494\\
11.1	208868790.1098\\
11.2	210004326.7871\\
11.3	210253366.4404\\
11.4	210889963.2477\\
11.5	211805059.1671\\
11.6	212369070.6444\\
11.7	211841059.3276\\
11.8	211414708.8393\\
11.9	212024166.9979\\
12	214090886.5318\\
12.1	216841644.665\\
12.2	217813811.2685\\
12.3	221436404.758\\
12.4	225260162.4719\\
12.5	225907421.3843\\
12.6	236857180.3235\\
12.7	275087365.4212\\
12.8	275931922.0107\\
12.9	277001919.969\\
13	279089778.0909\\
13.1	280788901.5785\\
13.2	282400345.8431\\
13.3	283469754.1117\\
13.4	283844209.8243\\
13.5	284673987.4045\\
13.6	286696414.0213\\
13.7	289318620.8237\\
13.8	289392056.7101\\
13.9	288545607.2721\\
14	288330418.5209\\
14.1	290493178.4847\\
14.2	293605615.359\\
14.3	293671747.4839\\
14.4	293408822.9581\\
14.5	294027529.5417\\
14.6	295969501.8906\\
14.7	298371802.1315\\
14.8	298512235.472\\
14.9	299807611.984\\
15	301293002.6021\\
15.1	301677509.7709\\
15.2	301857958.5836\\
15.3	303142847.021\\
15.4	302823132.5308\\
15.5	304033889.6515\\
15.6	306716181.9383\\
15.7	308353495.8964\\
15.8	309574358.3735\\
15.9	310343173.5788\\
16	311014057.5361\\
16.1	312129688.3478\\
16.2	312625716.8151\\
16.3	311752191.7801\\
16.4	320521056.1134\\
16.5	320507569.0132\\
16.6	323308767.8114\\
16.7	330596782.1237\\
16.8	332633353.1657\\
16.9	339346080.9998\\
17	341510730.4742\\
17.1	348056483.6119\\
17.2	358401423.6464\\
17.3	359868511.3256\\
17.4	360712465.0612\\
17.5	361478186.3577\\
17.6	363362518.7171\\
17.7	363757729.882\\
17.8	364058531.8713\\
17.9	369028723.4776\\
18	381225987.5301\\
18.1	383348596.4386\\
18.2	388210147.845\\
18.3	388986147.2939\\
18.4	391769056.7567\\
18.5	394359771.1239\\
18.6	395644247.4187\\
18.7	397943048.424\\
18.8	398550423.8677\\
18.9	398318469.9205\\
19	400932640.4868\\
19.1	402943820.7859\\
19.2	403763811.2713\\
19.3	405451052.2953\\
19.4	407299424.0336\\
19.5	409718503.1111\\
19.6	416565704.6529\\
19.7	424103494.2875\\
19.8	425673780.0816\\
19.9	427652428.9082\\
20	429353735.5977\\
20.1	431712970.7686\\
20.2	434803092.2359\\
20.3	437404172.5933\\
20.4	437917223.8661\\
20.5	438372487.7607\\
20.6	438837788.8058\\
20.7	442553808.0372\\
20.8	443975389.9598\\
20.9	446015307.7061\\
21	451842862.6821\\
21.1	456737221.8376\\
21.2	456656063.5152\\
21.3	457837931.1455\\
21.4	459055902.1653\\
21.5	458421682.119\\
21.6	457390860.6111\\
21.7	456939996.3874\\
21.8	461460933.9518\\
21.9	471058008.9154\\
22	477199630.0381\\
22.1	478702679.7259\\
22.2	488431772.0693\\
22.3	521131762.536\\
22.4	525046389.7548\\
22.5	525472802.7563\\
22.6	525179571.033\\
22.7	526415461.8534\\
22.8	527315352.2396\\
22.9	525801217.8188\\
23	525762250.7467\\
23.1	527333835.8613\\
23.2	529913121.38\\
23.3	534506279.9804\\
23.4	537381332.8828\\
23.5	537772005.0753\\
23.6	539070379.9982\\
23.7	542715666.022\\
23.8	544186300.0758\\
23.9	545939164.1767\\
24	548357591.4838\\
24.1	554980544.0561\\
24.2	570378341.6095\\
24.3	574463939.1239\\
24.4	577986479.526\\
24.5	578009805.8139\\
24.6	578409039.5664\\
24.7	584425040.7535\\
24.8	590402915.7523\\
24.9	592070749.2926\\
25	596571648.427\\
25.1	598315755.8706\\
25.2	599921139.4651\\
25.3	602988340.2581\\
25.4	609027113.036\\
25.5	610029571.3503\\
25.6	611780516.3929\\
25.7	614353243.2249\\
25.8	619760691.9898\\
25.9	621309719.5617\\
26	626822421.2927\\
26.1	631026427.2588\\
26.2	632972434.5252\\
26.3	637315136.1143\\
26.4	636855571.8377\\
26.5	640881730.1751\\
26.6	655544648.0917\\
26.7	656105798.6566\\
26.8	660546858.0169\\
26.9	670943082.2072\\
27	672079854.9095\\
27.1	672334982.9945\\
27.2	671586609.467\\
27.3	674633085.0381\\
27.4	689612147.8825\\
27.5	690625989.0813\\
27.6	691284917.3873\\
27.7	694487739.1633\\
27.8	696454735.792\\
27.9	698947078.3579\\
28	701080128.1897\\
28.1	701108309.9427\\
28.2	701325850.6449\\
28.3	706910886.8795\\
28.4	712390557.3957\\
28.5	713841712.0259\\
28.6	716709576.126\\
28.7	733499377.2441\\
28.8	772830774.2288\\
28.9	771799476.6518\\
29	773246868.6985\\
29.1	775770656.2411\\
29.2	775873376.5403\\
29.3	776129605.8578\\
29.4	776890408.5781\\
29.5	778716131.5749\\
29.6	782382331.2486\\
29.7	783733641.0037\\
29.8	784524766.3962\\
29.9	797052518.7285\\
30	828861171.6617\\
30.1	832204234.2687\\
30.2	836007502.6219\\
30.3	839249259.4842\\
30.4	840303506.4352\\
30.5	840680025.5918\\
30.6	843104933.0573\\
30.7	843382841.6292\\
30.8	850167674.1236\\
30.9	860508175.9757\\
31	864407093.6119\\
31.1	865874121.5186\\
31.2	867384378.2404\\
31.3	869200863\\
31.4	871657644.1714\\
31.5	872528216.7767\\
31.6	871789791.6868\\
31.7	872526786.4885\\
31.8	877420705.7115\\
31.9	879498756.1697\\
32	884450791.0748\\
32.1	890331285.2446\\
32.2	892742137.865\\
32.3	892566025.2941\\
32.4	892962742.6142\\
32.5	895341790.6377\\
32.6	897989581.6027\\
32.7	908111345.4252\\
32.8	925770858.3844\\
32.9	933851445.5405\\
33	949293364.689\\
33.1	955986485.7528\\
33.2	980612461.523\\
33.3	982954358.4466\\
33.4	983316898.369\\
33.5	986296636.1343\\
33.6	992089824.6981\\
33.7	991299393.6229\\
33.8	991760481.4264\\
33.9	993052349.088\\
34	993694503.2549\\
34.1	1001537668.9841\\
34.2	1015021269.3015\\
34.3	1015778090.4748\\
34.4	1016749283.9225\\
34.5	1019169592.3781\\
34.6	1022138579.7043\\
34.7	1022561760.0317\\
34.8	1023118267.5299\\
34.9	1024803276.3356\\
35	1028782949.9649\\
35.1	1032872575.0782\\
35.2	1037899845.4737\\
35.3	1046995898.0441\\
35.4	1064120578.4827\\
35.5	1071085038.5507\\
35.6	1075254994.5617\\
35.7	1081632215.8906\\
35.8	1081534701.5708\\
35.9	1081488270.0651\\
36	1082731525.197\\
36.1	1085167529.4942\\
36.2	1085904161.2931\\
36.3	1086547096.0655\\
36.4	1088136019.4089\\
36.5	1088231282.7031\\
36.6	1088941074.7776\\
36.7	1089242989.819\\
36.8	1089036992.3565\\
36.9	1091878723.2764\\
37	1095407322.3211\\
37.1	1098400111.7725\\
37.2	1104351743.1401\\
37.3	1117945780.022\\
37.4	1116976749.9004\\
37.5	1123488370.9739\\
37.6	1129824230.4313\\
37.7	1127240304.2092\\
37.8	1126274765.2068\\
37.9	1127914997.2597\\
38	1129776041.5281\\
38.1	1132232683.9479\\
38.2	1130425548.6791\\
38.3	1130493813.4811\\
38.4	1131171826.5417\\
38.5	1131262390.0549\\
38.6	1130484965.9886\\
38.7	1127696068.3983\\
38.8	1128344468.8466\\
38.9	1131205006.5085\\
39	1141210799.3397\\
39.1	1144946009.4609\\
39.2	1148739231.8675\\
39.3	1156371853.4195\\
39.4	1167009671.7939\\
39.5	1169182783.9965\\
39.6	1168391487.3535\\
39.7	1171035881.019\\
39.8	1181976207.2242\\
39.9	1186826401.802\\
40	1185724947.1598\\
40.1	1189179839.7815\\
40.2	1195276226.2037\\
40.3	1196884322.7183\\
40.4	1198645873.6991\\
40.5	1192793327.0166\\
40.6	1194531131.2408\\
40.7	1197303659.9036\\
40.8	1199184017.9552\\
40.9	1198363603.0939\\
41	1197628607.9885\\
41.1	1198617537.8812\\
41.2	1200636582.7005\\
41.3	1200726854.2717\\
41.4	1201484423.7514\\
41.5	1203974785.0844\\
41.6	1206404154.4055\\
41.7	1210200006.4676\\
41.8	1213296505.9972\\
41.9	1212342389.46\\
42	1215110682.0371\\
42.1	1215189301.8428\\
42.2	1216118704.4383\\
42.3	1218303019.8495\\
42.4	1219543666.2337\\
42.5	1219971428.9897\\
42.6	1225816097.3724\\
42.7	1238986345.1925\\
42.8	1241910745.6735\\
42.9	1244000365.0784\\
43	1243894592.1523\\
43.1	1243360017.2654\\
43.2	1242348203.7803\\
43.3	1244609464.5949\\
43.4	1248908444.4868\\
43.5	1251824569.5624\\
43.6	1256470388.9116\\
43.7	1262528695.7721\\
43.8	1266726706.6024\\
43.9	1269104978.3293\\
44	1271266522.1197\\
44.1	1274030733.6332\\
44.2	1275723466.2264\\
44.3	1274933038.9029\\
44.4	1277748212.5248\\
44.5	1279047908.914\\
44.6	1279729677.6372\\
44.7	1282406972.8402\\
44.8	1288968487.9738\\
44.9	1296109016.2997\\
45	1306309336.4929\\
45.1	1306894356.7849\\
45.2	1308302549.7918\\
45.3	1308823089.537\\
45.4	1309662409.5641\\
45.5	1310649172.8474\\
45.6	1312142620.1315\\
45.7	1311904735.9908\\
45.8	1311875743.0284\\
45.9	1312292557.8595\\
46	1314503426.3492\\
46.1	1319554357.4094\\
46.2	1316964240.9117\\
46.3	1316984640.2664\\
46.4	1315938733.5081\\
46.5	1314442544.8968\\
46.6	1315496703.2266\\
46.7	1315406631.2737\\
46.8	1314505818.9258\\
46.9	1312087166.825\\
47	1311960419.7217\\
47.1	1313803090.0462\\
47.2	1316681103.4612\\
47.3	1312548474.9837\\
47.4	1313504984.5217\\
47.5	1311184721.5477\\
47.6	1311089335.4674\\
47.7	1311547450.1618\\
47.8	1310245193.3614\\
47.9	1312139875.5406\\
48	1314710706.4661\\
48.1	1313997105.2386\\
48.2	1314650031.1518\\
48.3	1319785598.0361\\
48.4	1321503255.4503\\
48.5	1322544804.1863\\
48.6	1325396379.0045\\
48.7	1325869381.1334\\
48.8	1326010505.9281\\
48.9	1327920993.7384\\
49	1328452264.1606\\
49.1	1331949029.9974\\
49.2	1337512545.0897\\
49.3	1337926508.77\\
49.4	1340473706.0448\\
49.5	1344282388.7086\\
49.6	1346674460.7627\\
49.7	1350675292.1631\\
49.8	1352532105.2921\\
49.9	1355734784.0667\\
50	1358504433.8273\\
50.1	1361087052.6056\\
50.2	1367283571.1486\\
50.3	1370308360.0622\\
50.4	1370441742.7078\\
50.5	1372129401.5805\\
50.6	1374297503.6195\\
50.7	1376564153.8726\\
50.8	1376054309.4016\\
50.9	1374980629.3293\\
51	1372914039.709\\
51.1	1372280469.7626\\
51.2	1373976980.9596\\
51.3	1376782730.7608\\
51.4	1380072505.9844\\
51.5	1381370540.413\\
51.6	1379773801.9679\\
51.7	1380411845.2615\\
51.8	1388953628.6824\\
51.9	1393508149.7936\\
52	1397850096.1973\\
52.1	1396181813.3064\\
52.2	1396137630.4431\\
52.3	1400859228.468\\
52.4	1402919091.247\\
52.5	1405425624.8823\\
52.6	1408369625.9198\\
52.7	1411838224.0692\\
52.8	1413868408.4649\\
52.9	1414357295.2365\\
53	1414447661.0233\\
53.1	1417934775.8345\\
53.2	1420659643.0631\\
53.3	1421260820.686\\
53.4	1423510377.151\\
53.5	1428351125.4993\\
53.6	1434625098.8222\\
53.7	1437401259.7354\\
53.8	1439342616.3459\\
53.9	1438599489.5558\\
54	1437944298.3816\\
54.1	1438514291.0371\\
54.2	1444204736.1087\\
54.3	1450449112.0665\\
54.4	1450897891.4288\\
54.5	1456744189.8334\\
54.6	1469250902.321\\
54.7	1475077212.7671\\
54.8	1476686538.2771\\
54.9	1479542457.5693\\
55	1477740544.2646\\
55.1	1476856766.4778\\
55.2	1478989323.4681\\
55.3	1480886961.8742\\
55.4	1483290436.2613\\
55.5	1486627389.9191\\
55.6	1487904318.5032\\
55.7	1490090832.0926\\
55.8	1492797782.6007\\
55.9	1496895557.0043\\
56	1503444091.5627\\
56.1	1510883922.4662\\
56.2	1515180931.6282\\
56.3	1515680639.861\\
56.4	1530419125.9235\\
56.5	1526178713.5866\\
56.6	1534656601.9982\\
56.7	1535473471.0188\\
56.8	1541539923.8626\\
56.9	1547076611.0487\\
57	1548292852.9617\\
57.1	1548975417.448\\
57.2	1550515595.0293\\
57.3	1551565420.2859\\
57.4	1544413145.1195\\
57.5	1537992810.0481\\
57.6	1536246186.7012\\
57.7	1537552213.3519\\
57.8	1541160433.9074\\
57.9	1541254772.8992\\
58	1539858021.4188\\
58.1	1541694837.4361\\
58.2	1540712204.5093\\
58.3	1543761353.1772\\
58.4	1546411268.0122\\
58.5	1548994794.4739\\
58.6	1548504835.7449\\
58.7	1548668033.2745\\
58.8	1550998296.3667\\
58.9	1554244376.4751\\
59	1556752453.3487\\
59.1	1558811579.6224\\
59.2	1558717560.9851\\
59.3	1557567070.132\\
59.4	1560288598.1952\\
59.5	1563812695.728\\
59.6	1561254486.4496\\
59.7	1561724088.6284\\
59.8	1564098524.9526\\
59.9	1564584970.0207\\
60	1567081408.1948\\
60.1	1570004344.3819\\
60.2	1571851474.8765\\
60.3	1574022499.28\\
60.4	1579609954.4218\\
60.5	1584019688.3609\\
60.6	1583147162.5712\\
60.7	1591501593.8862\\
60.8	1603199751.5328\\
60.9	1602034926.9722\\
61	1604507441.6832\\
61.1	1605796114.5719\\
61.2	1604862404.2952\\
61.3	1603999587.603\\
61.4	1604403098.0918\\
61.5	1606525058.072\\
61.6	1599170205.0499\\
61.7	1608969922.9685\\
61.8	1607947304.7421\\
61.9	1608999538.3546\\
62	1611046415.69\\
62.1	1610767269.9742\\
62.2	1618385509.3338\\
62.3	1630055959.7958\\
62.4	1634025566.3485\\
62.5	1637743796.5718\\
62.6	1640727504.1305\\
62.7	1643585819.329\\
62.8	1648202907.2646\\
62.9	1650714020.7074\\
63	1649482473.0776\\
63.1	1647458241.9501\\
63.2	1649472904.998\\
63.3	1650833707.2325\\
63.4	1656673608.766\\
63.5	1652082492.1871\\
63.6	1645066224.273\\
63.7	1641748954.9346\\
63.8	1642896548.3758\\
63.9	1644774791.3721\\
64	1643828297.1675\\
64.1	1637505861.5015\\
64.2	1639952424.5078\\
64.3	1642543406.8311\\
64.4	1643773843.4542\\
64.5	1649049377.239\\
64.6	1648755440.8152\\
64.7	1647200211.5377\\
64.8	1646084604.2524\\
64.9	1645298328.3281\\
65	1643515301.36\\
65.1	1646980733.63\\
65.2	1644520331.6463\\
65.3	1640733910.4777\\
65.4	1640787452.4631\\
65.5	1651676001.1597\\
65.6	1665131854.1937\\
65.7	1665872490.1601\\
65.8	1664831450.7507\\
65.9	1666194538.7869\\
66	1673393085.6641\\
66.1	1682479016.0977\\
66.2	1685681306.1402\\
66.3	1687473122.3952\\
66.4	1689292322.2228\\
66.5	1690478686.5196\\
66.6	1689782348.1118\\
66.7	1692002728.2189\\
66.8	1697799386.2958\\
66.9	1700407145.2447\\
67	1707819911.2042\\
67.1	1720959167.7335\\
67.2	1725541474.5666\\
67.3	1728364895.5913\\
67.4	1729822470.7\\
67.5	1731919438.9288\\
67.6	1739394232.1103\\
67.7	1746002044.5698\\
67.8	1753906461.2312\\
67.9	1767314573.0189\\
68	1774402151.7679\\
68.1	1773422604.3584\\
68.2	1774112630.6066\\
68.3	1777216673.0948\\
68.4	1784589012.9567\\
68.5	1792050594.4733\\
68.6	1793511326.972\\
68.7	1795371301.0519\\
68.8	1798867056.031\\
68.9	1804215470.4535\\
69	1805111965.8286\\
69.1	1808515004.5312\\
69.2	1810820337.7558\\
69.3	1810889292.4619\\
69.4	1809283706.6031\\
69.5	1810559648.9229\\
69.6	1814090417.2286\\
69.7	1814144275.5456\\
69.8	1818903272.2916\\
69.9	1823919397.0059\\
70	1830283004.0815\\
70.1	1834874096.8134\\
70.2	1833904094.8445\\
70.3	1836043001.6682\\
70.4	1840948901.0589\\
70.5	1843902573.8655\\
70.6	1841451873.2248\\
70.7	1840559974.9082\\
70.8	1841655381.123\\
70.9	1846463794.8008\\
71	1845777807.9057\\
71.1	1844104364.2005\\
71.2	1844115135.8651\\
71.3	1846040108.3429\\
71.4	1846224525.3156\\
71.5	1845856411.7865\\
71.6	1848012570.9659\\
71.7	1848934972.403\\
71.8	1850181143.3553\\
71.9	1850943659.5115\\
72	1852759053.8133\\
72.1	1856030793.7652\\
72.2	1854941316.879\\
72.3	1855190124.7514\\
72.4	1855412797.2376\\
72.5	1855518252.7261\\
72.6	1855331890.9404\\
72.7	1857853835.4243\\
72.8	1861956343.0796\\
72.9	1865179397.8901\\
73	1863290279.9975\\
73.1	1870468696.8003\\
73.2	1885440132.7278\\
73.3	1886425319.9885\\
73.4	1889655894.7049\\
73.5	1892848065.6014\\
73.6	1888943418.8686\\
73.7	1888738531.8644\\
73.8	1891713139.4931\\
73.9	1893191075.8734\\
74	1893854079.2493\\
74.1	1892291304.1243\\
74.2	1889607241.7742\\
74.3	1890920687.3989\\
74.4	1892541758.0989\\
74.5	1897329332.5373\\
74.6	1906116687.5717\\
74.7	1908245083.3703\\
74.8	1908456348.367\\
74.9	1910543295.5691\\
75	1917493390.9625\\
75.1	1926693857.3473\\
75.2	1930918471.5\\
75.3	1936544752.5835\\
75.4	1940005332.0672\\
75.5	1944057017.9802\\
75.6	1943983738.2517\\
75.7	1944132827.6154\\
75.8	1946801024.0019\\
75.9	1945476709.5773\\
76	1942809390.9684\\
76.1	1943210942.5139\\
76.2	1942573802.4413\\
76.3	1942814120.2736\\
76.4	1945952183.3216\\
76.5	1950045540.2373\\
76.6	1958578291.2352\\
76.7	1966084900.7687\\
76.8	1969487357.2028\\
76.9	1969845828.0172\\
77	1966986394.1755\\
77.1	1965676918.5399\\
77.2	1968862911.4242\\
77.3	1970868684.5274\\
77.4	1977829394.9383\\
77.5	1982023147.8432\\
77.6	1981614286.1519\\
77.7	1979166838.7008\\
77.8	1980804749.6286\\
77.9	1981893866.8623\\
78	1981148017.3942\\
78.1	1981798542.9955\\
78.2	1982733441.3425\\
78.3	1982752256.7497\\
78.4	1982468385.8796\\
78.5	1985449466.4042\\
78.6	1988493780.5918\\
78.7	1989916283.4122\\
78.8	1989186208.6707\\
78.9	1987946094.111\\
79	1992451232.065\\
79.1	1999292649.5943\\
79.2	2006032015.6477\\
79.3	2020079190.3335\\
79.4	2021549548.5001\\
79.5	2022933751.6261\\
79.6	2024697192.0883\\
79.7	2024998752.0306\\
79.8	2023073553.2271\\
79.9	2030565942.4971\\
80	2041750345.3293\\
80.1	2040349224.3627\\
80.2	2042980249.8123\\
80.3	2046895884.7165\\
80.4	2044497600.616\\
80.5	2041138874.5288\\
80.6	2041696006.7448\\
80.7	2044848413.8087\\
80.8	2042696347.315\\
80.9	2039894234.2942\\
81	2037655752.9165\\
81.1	2036462676.3075\\
81.2	2039538786.4734\\
81.3	2043693889.2994\\
81.4	2043274535.1779\\
81.5	2044723071.1736\\
81.6	2048603991.105\\
81.7	2049068670.6942\\
81.8	2048135734.1298\\
81.9	2043920312.3877\\
82	2042828863.5875\\
82.1	2041063252.4592\\
82.2	2040772101.2205\\
82.3	2043754485.7968\\
82.4	2042568400.5103\\
82.5	2043946345.6124\\
82.6	2045752758.012\\
82.7	2045921236.4827\\
82.8	2047159061.6809\\
82.9	2045209554.5124\\
83	2040078809.9074\\
83.1	2040634489.2412\\
83.2	2043097117.1176\\
83.3	2045476378.6097\\
83.4	2048458542.5076\\
83.5	2050876864.3016\\
83.6	2052655844.2354\\
83.7	2054474842.6004\\
83.8	2056005977.4693\\
83.9	2058042030.4733\\
84	2057280747.6631\\
84.1	2055285713.0138\\
84.2	2055248808.5557\\
84.3	2056891055.9182\\
84.4	2058029440.7002\\
84.5	2060302983.1808\\
84.6	2060990384.9435\\
84.7	2059189130.3616\\
84.8	2060088374.4485\\
84.9	2064844710.7049\\
85	2065161874.195\\
85.1	2066928447.3862\\
85.2	2073448079.8752\\
85.3	2075755724.3976\\
85.4	2078988831.4568\\
85.5	2086536278.6376\\
85.6	2088807937.2646\\
85.7	2090692805.9216\\
85.8	2088374520.5631\\
85.9	2090868296.665\\
86	2096156321.6815\\
86.1	2102324320.5171\\
86.2	2104124466.5985\\
86.3	2099205135.586\\
86.4	2103700938.5745\\
86.5	2105395934.3197\\
86.6	2107214222.5733\\
86.7	2108015536.1984\\
86.8	2104901265.1402\\
86.9	2103603705.2765\\
87	2108211360.0616\\
87.1	2113472483.7338\\
87.2	2114106103.7047\\
87.3	2120422671.5285\\
87.4	2126055734.0768\\
87.5	2122523250.8999\\
87.6	2121210966.4087\\
87.7	2122160432.0222\\
87.8	2124865036.466\\
87.9	2135501595.2218\\
88	2140712719.6297\\
88.1	2138159235.4773\\
88.2	2138858842.3308\\
88.3	2142287682.1483\\
88.4	2138545787.7289\\
88.5	2137927787.4184\\
88.6	2142261608.9424\\
88.7	2148704630.7214\\
88.8	2153714901.1134\\
88.9	2155686512.7715\\
89	2186422143.4954\\
89.1	2255846181.6484\\
89.2	2256567266.0437\\
89.3	2248654045.5982\\
89.4	2242205409.1984\\
89.5	2244827682.613\\
89.6	2243650796.7316\\
89.7	2239414713.323\\
89.8	2237874357.2959\\
89.9	2237666477.8399\\
90	2238169866.9281\\
90.1	2239565838.1171\\
90.2	2241669090.3188\\
90.3	2239863408.3141\\
90.4	2239082787.5577\\
90.5	2238941515.5427\\
90.6	2237863925.8791\\
90.7	2240622406.9894\\
90.8	2245819617.4848\\
90.9	2247443222.8153\\
91	2247043963.7485\\
91.1	2251378557.6602\\
91.2	2264210547.1238\\
91.3	2268956439.2356\\
91.4	2274238709.7421\\
91.5	2277582781.6173\\
91.6	2281809907.1349\\
91.7	2286355013.3308\\
91.8	2288585780.7476\\
91.9	2290226136.6258\\
92	2301626435.4108\\
92.1	2337269696.4709\\
92.2	2332597485.4655\\
92.3	2330636883.9182\\
92.4	2330544091.4481\\
92.5	2330818437.8936\\
92.6	2327523725.7828\\
92.7	2326336963.6191\\
92.8	2334930443.6992\\
92.9	2355442200.7823\\
93	2359489902.5912\\
93.1	2359363632.893\\
93.2	2362178553.8375\\
93.3	2365299931.5456\\
93.4	2372816778.7208\\
93.5	2379925459.529\\
93.6	2381777956.4747\\
93.7	2385712981.6374\\
93.8	2390897504.289\\
93.9	2392890482.7853\\
94	2389088298.7859\\
94.1	2387273424.8749\\
94.2	2385724291.7648\\
94.3	2382614081.9004\\
94.4	2379721735.2702\\
94.5	2380587329.8036\\
94.6	2384806908.8938\\
94.7	2386511563.9644\\
94.8	2387314012.8355\\
94.9	2392094824.3608\\
95	2394339972.0461\\
95.1	2395659647.9375\\
95.2	2396193525.3115\\
95.3	2396486479.9082\\
95.4	2396076515.0712\\
95.5	2398292158.2319\\
95.6	2403259040.2268\\
95.7	2406276751.2927\\
95.8	2410221614.0591\\
95.9	2413995244.8402\\
96	2417191312.5007\\
96.1	2420329289.9185\\
96.2	2429482069.1699\\
96.3	2438411467.991\\
96.4	2439545096.5661\\
96.5	2438689973.0968\\
96.6	2439315592.9392\\
96.7	2442478914.8771\\
96.8	2443914545.0988\\
96.9	2447234587.0891\\
97	2450376663.8866\\
97.1	2455750306.643\\
97.2	2463419646.9892\\
97.3	2468713843.8807\\
97.4	2473391830.9106\\
97.5	2475642499.2435\\
97.6	2475418989.0877\\
97.7	2479613326.6207\\
97.8	2480705194.4994\\
97.9	2479822030.5052\\
98	2479122037.731\\
98.1	2476012281.6975\\
98.2	2474356896.6894\\
98.3	2476626308.1519\\
98.4	2479420800.7519\\
98.5	2483317468.6364\\
98.6	2484554861.8066\\
98.7	2494760309.1661\\
98.8	2512199399.7655\\
98.9	2511081301.8324\\
99	2512382623.279\\
99.1	2515173402.7058\\
99.2	2519893634.6816\\
99.3	2526143419.3471\\
99.4	2528825347.3378\\
99.5	2531271364.7808\\
99.6	2531407564.9913\\
99.7	2535014447.0121\\
99.8	2522576226.6928\\
99.9	2592102688.7931\\
};
\addlegendentry{Kinetic energy};

\end{axis}
\end{tikzpicture}%}
        \caption{$\Delta t =\unit[0.1]{ASU}$}
        \label{fig:timestep_a}
    \end{subfigure}
    \begin{subfigure}[b]{0.40\textwidth}
        \centering
        \resizebox{\columnwidth}{!}{% This file was created by matlab2tikz.
%
%The latest updates can be retrieved from
%  http://www.mathworks.com/matlabcentral/fileexchange/22022-matlab2tikz-matlab2tikz
%where you can also make suggestions and rate matlab2tikz.
%
\definecolor{mycolor1}{rgb}{0.00000,0.44700,0.74100}%
\definecolor{mycolor2}{rgb}{0.85000,0.32500,0.09800}%
\definecolor{mycolor3}{rgb}{0.92900,0.69400,0.12500}%
%
\begin{tikzpicture}

\begin{axis}[%
width=4.521in,
height=3.566in,
at={(0.758in,0.481in)},
scale only axis,
xmin=0,
xmax=10,
xlabel={Time / [$\unit{Å}$]},
ymin=-900,
ymax=100,
ylabel={Energy / [$\unit{eV}$]},
label style ={font=\Large},
axis background/.style={fill=white},
title style={font=\bfseries\Huge},
title={Time evolution of energy},
legend style={draw=white!15!black},
legend pos=north west
]
\addplot [color=mycolor1,solid]
  table[row sep=crcr]{%
0	-857.2121\\
0.01	-857.2514\\
0.02	-857.3192\\
0.03	-857.336\\
0.04	-857.2966\\
0.05	-857.2526\\
0.06	-857.243\\
0.07	-857.2734\\
0.08	-857.3111\\
0.09	-857.3104\\
0.1	-857.2777\\
0.11	-857.2559\\
0.12	-857.2698\\
0.13	-857.3015\\
0.14	-857.3088\\
0.15	-857.2852\\
0.16	-857.263\\
0.17	-857.267\\
0.18	-857.2876\\
0.19	-857.2991\\
0.2	-857.2927\\
0.21	-857.2783\\
0.22	-857.271\\
0.23	-857.2789\\
0.24	-857.2916\\
0.25	-857.2945\\
0.26	-857.2843\\
0.27	-857.274\\
0.28	-857.2753\\
0.29	-857.2865\\
0.3	-857.292\\
0.31	-857.2859\\
0.32	-857.2739\\
0.33	-857.274\\
0.34	-857.2841\\
0.35	-857.2936\\
0.36	-857.289\\
0.37	-857.2825\\
0.38	-857.277\\
0.39	-857.2784\\
0.4	-857.2809\\
0.41	-857.2826\\
0.42	-857.2851\\
0.43	-857.2852\\
0.44	-857.2854\\
0.45	-857.2793\\
0.46	-857.2754\\
0.47	-857.2766\\
0.48	-857.2819\\
0.49	-857.2867\\
0.5	-857.2896\\
0.51	-857.2861\\
0.52	-857.28\\
0.53	-857.2761\\
0.54	-857.2777\\
0.55	-857.2858\\
0.56	-857.2898\\
0.57	-857.2865\\
0.58	-857.2798\\
0.59	-857.276\\
0.6	-857.28\\
0.61	-857.2853\\
0.62	-857.2876\\
0.63	-857.2835\\
0.64	-857.2792\\
0.65	-857.2819\\
0.66	-857.2861\\
0.67	-857.2877\\
0.68	-857.2816\\
0.69	-857.2758\\
0.7	-857.2781\\
0.71	-857.2851\\
0.72	-857.2866\\
0.73	-857.2819\\
0.74	-857.2783\\
0.75	-857.2805\\
0.76	-857.2863\\
0.77	-857.2914\\
0.78	-857.2877\\
0.79	-857.2767\\
0.8	-857.2718\\
0.81	-857.2775\\
0.82	-857.2892\\
0.83	-857.2923\\
0.84	-857.2863\\
0.85	-857.2793\\
0.86	-857.2772\\
0.87	-857.2805\\
0.88	-857.2808\\
0.89	-857.2807\\
0.9	-857.2845\\
0.91	-857.2887\\
0.92	-857.2863\\
0.93	-857.2802\\
0.94	-857.2753\\
0.95	-857.2773\\
0.96	-857.284\\
0.97	-857.2911\\
0.98	-857.2912\\
0.99	-857.2829\\
1	-857.2717\\
1.01	-857.2715\\
1.02	-857.2819\\
1.03	-857.2925\\
1.04	-857.2924\\
1.05	-857.283\\
1.06	-857.2721\\
1.07	-857.2696\\
1.08	-857.2772\\
1.09	-857.2903\\
1.1	-857.2934\\
1.11	-857.2856\\
1.12	-857.2777\\
1.13	-857.2759\\
1.14	-857.2806\\
1.15	-857.2859\\
1.16	-857.2862\\
1.17	-857.2831\\
1.18	-857.2799\\
1.19	-857.2784\\
1.2	-857.2794\\
1.21	-857.2807\\
1.22	-857.284\\
1.23	-857.2892\\
1.24	-857.2866\\
1.25	-857.2757\\
1.26	-857.2702\\
1.27	-857.2764\\
1.28	-857.2893\\
1.29	-857.297\\
1.3	-857.2868\\
1.31	-857.2729\\
1.32	-857.2674\\
1.33	-857.2752\\
1.34	-857.2887\\
1.35	-857.2943\\
1.36	-857.2873\\
1.37	-857.2744\\
1.38	-857.2721\\
1.39	-857.2803\\
1.4	-857.2867\\
1.41	-857.2863\\
1.42	-857.2794\\
1.43	-857.2759\\
1.44	-857.2812\\
1.45	-857.2876\\
1.46	-857.2866\\
1.47	-857.2779\\
1.48	-857.2725\\
1.49	-857.2788\\
1.5	-857.2884\\
1.51	-857.2931\\
1.52	-857.2826\\
1.53	-857.2716\\
1.54	-857.2734\\
1.55	-857.2861\\
1.56	-857.2963\\
1.57	-857.2899\\
1.58	-857.2757\\
1.59	-857.2666\\
1.6	-857.2742\\
1.61	-857.2899\\
1.62	-857.2962\\
1.63	-857.2904\\
1.64	-857.2785\\
1.65	-857.2707\\
1.66	-857.2723\\
1.67	-857.2805\\
1.68	-857.287\\
1.69	-857.2875\\
1.7	-857.2834\\
1.71	-857.2815\\
1.72	-857.284\\
1.73	-857.2851\\
1.74	-857.279\\
1.75	-857.274\\
1.76	-857.2775\\
1.77	-857.2873\\
1.78	-857.2912\\
1.79	-857.2865\\
1.8	-857.2776\\
1.81	-857.2734\\
1.82	-857.2809\\
1.83	-857.288\\
1.84	-857.289\\
1.85	-857.2835\\
1.86	-857.2778\\
1.87	-857.2783\\
1.88	-857.2825\\
1.89	-857.2846\\
1.9	-857.2814\\
1.91	-857.2769\\
1.92	-857.2764\\
1.93	-857.2835\\
1.94	-857.2887\\
1.95	-857.2892\\
1.96	-857.2809\\
1.97	-857.2748\\
1.98	-857.2768\\
1.99	-857.2852\\
2	-857.2874\\
2.01	-857.2836\\
2.02	-857.2781\\
2.03	-857.2797\\
2.04	-857.2828\\
2.05	-857.2831\\
2.06	-857.2768\\
2.07	-857.2743\\
2.08	-857.2804\\
2.09	-857.2897\\
2.1	-857.2931\\
2.11	-857.2865\\
2.12	-857.273\\
2.13	-857.2689\\
2.14	-857.2752\\
2.15	-857.287\\
2.16	-857.2929\\
2.17	-857.2911\\
2.18	-857.2819\\
2.19	-857.2739\\
2.2	-857.2748\\
2.21	-857.2806\\
2.22	-857.2875\\
2.23	-857.2897\\
2.24	-857.2867\\
2.25	-857.2786\\
2.26	-857.2723\\
2.27	-857.274\\
2.28	-857.2828\\
2.29	-857.289\\
2.3	-857.2868\\
2.31	-857.2798\\
2.32	-857.2744\\
2.33	-857.2756\\
2.34	-857.2843\\
2.35	-857.2921\\
2.36	-857.2869\\
2.37	-857.2754\\
2.38	-857.2711\\
2.39	-857.278\\
2.4	-857.2888\\
2.41	-857.2935\\
2.42	-857.2877\\
2.43	-857.2785\\
2.44	-857.2744\\
2.45	-857.2786\\
2.46	-857.2861\\
2.47	-857.2873\\
2.48	-857.2832\\
2.49	-857.2783\\
2.5	-857.2804\\
2.51	-857.2845\\
2.52	-857.2849\\
2.53	-857.2815\\
2.54	-857.2766\\
2.55	-857.279\\
2.56	-857.2855\\
2.57	-857.2901\\
2.58	-857.286\\
2.59	-857.2762\\
2.6	-857.2722\\
2.61	-857.2789\\
2.62	-857.2894\\
2.63	-857.2921\\
2.64	-857.2855\\
2.65	-857.2763\\
2.66	-857.2738\\
2.67	-857.2778\\
2.68	-857.2871\\
2.69	-857.29\\
2.7	-857.2864\\
2.71	-857.2812\\
2.72	-857.2789\\
2.73	-857.2799\\
2.74	-857.2827\\
2.75	-857.2831\\
2.76	-857.284\\
2.77	-857.2822\\
2.78	-857.2787\\
2.79	-857.2799\\
2.8	-857.2862\\
2.81	-857.2879\\
2.82	-857.2844\\
2.83	-857.278\\
2.84	-857.2759\\
2.85	-857.2796\\
2.86	-857.285\\
2.87	-857.2871\\
2.88	-857.285\\
2.89	-857.2788\\
2.9	-857.2754\\
2.91	-857.2785\\
2.92	-857.2863\\
2.93	-857.2871\\
2.94	-857.2865\\
2.95	-857.2811\\
2.96	-857.2793\\
2.97	-857.2804\\
2.98	-857.2835\\
2.99	-857.2825\\
3	-857.2817\\
3.01	-857.2809\\
3.02	-857.2823\\
3.03	-857.2844\\
3.04	-857.2836\\
3.05	-857.2799\\
3.06	-857.2755\\
3.07	-857.2738\\
3.08	-857.2816\\
3.09	-857.2914\\
3.1	-857.2944\\
3.11	-857.286\\
3.12	-857.2723\\
3.13	-857.2689\\
3.14	-857.2764\\
3.15	-857.2863\\
3.16	-857.293\\
3.17	-857.2904\\
3.18	-857.2801\\
3.19	-857.27\\
3.2	-857.2723\\
3.21	-857.2841\\
3.22	-857.2906\\
3.23	-857.2871\\
3.24	-857.279\\
3.25	-857.2753\\
3.26	-857.2792\\
3.27	-857.2841\\
3.28	-857.2867\\
3.29	-857.2862\\
3.3	-857.2817\\
3.31	-857.2786\\
3.32	-857.2775\\
3.33	-857.2805\\
3.34	-857.2835\\
3.35	-857.2847\\
3.36	-857.2846\\
3.37	-857.2816\\
3.38	-857.2776\\
3.39	-857.2782\\
3.4	-857.2838\\
3.41	-857.2899\\
3.42	-857.287\\
3.43	-857.2815\\
3.44	-857.2797\\
3.45	-857.2816\\
3.46	-857.2828\\
3.47	-857.2816\\
3.48	-857.2817\\
3.49	-857.2834\\
3.5	-857.2857\\
3.51	-857.2833\\
3.52	-857.2784\\
3.53	-857.2773\\
3.54	-857.2804\\
3.55	-857.2838\\
3.56	-857.2864\\
3.57	-857.2825\\
3.58	-857.2792\\
3.59	-857.2787\\
3.6	-857.2828\\
3.61	-857.2884\\
3.62	-857.2869\\
3.63	-857.2804\\
3.64	-857.2771\\
3.65	-857.2779\\
3.66	-857.2827\\
3.67	-857.2859\\
3.68	-857.2851\\
3.69	-857.2808\\
3.7	-857.2768\\
3.71	-857.2774\\
3.72	-857.2811\\
3.73	-857.2834\\
3.74	-857.2826\\
3.75	-857.2802\\
3.76	-857.2807\\
3.77	-857.2834\\
3.78	-857.284\\
3.79	-857.281\\
3.8	-857.2792\\
3.81	-857.2803\\
3.82	-857.2821\\
3.83	-857.2858\\
3.84	-857.2846\\
3.85	-857.2806\\
3.86	-857.278\\
3.87	-857.2783\\
3.88	-857.2823\\
3.89	-857.2859\\
3.9	-857.285\\
3.91	-857.2797\\
3.92	-857.2754\\
3.93	-857.2767\\
3.94	-857.2831\\
3.95	-857.2851\\
3.96	-857.285\\
3.97	-857.283\\
3.98	-857.2821\\
3.99	-857.2832\\
4	-857.2805\\
4.01	-857.2793\\
4.02	-857.2815\\
4.03	-857.283\\
4.04	-857.2835\\
4.05	-857.2836\\
4.06	-857.2844\\
4.07	-857.2825\\
4.08	-857.279\\
4.09	-857.2807\\
4.1	-857.2847\\
4.11	-857.2843\\
4.12	-857.2794\\
4.13	-857.2769\\
4.14	-857.2786\\
4.15	-857.2853\\
4.16	-857.2877\\
4.17	-857.2843\\
4.18	-857.2787\\
4.19	-857.2737\\
4.2	-857.2767\\
4.21	-857.2836\\
4.22	-857.2893\\
4.23	-857.2875\\
4.24	-857.2822\\
4.25	-857.2763\\
4.26	-857.2784\\
4.27	-857.2816\\
4.28	-857.2855\\
4.29	-857.2857\\
4.3	-857.2834\\
4.31	-857.2805\\
4.32	-857.2809\\
4.33	-857.2797\\
4.34	-857.2773\\
4.35	-857.28\\
4.36	-857.285\\
4.37	-857.2855\\
4.38	-857.2832\\
4.39	-857.28\\
4.4	-857.2786\\
4.41	-857.2803\\
4.42	-857.2831\\
4.43	-857.286\\
4.44	-857.2829\\
4.45	-857.2774\\
4.46	-857.2764\\
4.47	-857.281\\
4.48	-857.2881\\
4.49	-857.2888\\
4.5	-857.281\\
4.51	-857.2739\\
4.52	-857.2748\\
4.53	-857.2816\\
4.54	-857.2893\\
4.55	-857.2882\\
4.56	-857.2798\\
4.57	-857.2746\\
4.58	-857.2761\\
4.59	-857.2846\\
4.6	-857.291\\
4.61	-857.2881\\
4.62	-857.2784\\
4.63	-857.2735\\
4.64	-857.2747\\
4.65	-857.2809\\
4.66	-857.2868\\
4.67	-857.2873\\
4.68	-857.284\\
4.69	-857.2778\\
4.7	-857.2766\\
4.71	-857.2795\\
4.72	-857.2826\\
4.73	-857.2847\\
4.74	-857.2848\\
4.75	-857.2835\\
4.76	-857.2813\\
4.77	-857.2806\\
4.78	-857.281\\
4.79	-857.2773\\
4.8	-857.2761\\
4.81	-857.2805\\
4.82	-857.2858\\
4.83	-857.2888\\
4.84	-857.2863\\
4.85	-857.2796\\
4.86	-857.2783\\
4.87	-857.2793\\
4.88	-857.2846\\
4.89	-857.2865\\
4.9	-857.2825\\
4.91	-857.2788\\
4.92	-857.2783\\
4.93	-857.2811\\
4.94	-857.2836\\
4.95	-857.2863\\
4.96	-857.2816\\
4.97	-857.278\\
4.98	-857.2787\\
4.99	-857.2843\\
5	-857.2884\\
5.01	-857.284\\
5.02	-857.2765\\
5.03	-857.2772\\
5.04	-857.2829\\
5.05	-857.2875\\
5.06	-857.2842\\
5.07	-857.2808\\
5.08	-857.2809\\
5.09	-857.2826\\
5.1	-857.2822\\
5.11	-857.2816\\
5.12	-857.2813\\
5.13	-857.2831\\
5.14	-857.2857\\
5.15	-857.2866\\
5.16	-857.2845\\
5.17	-857.2824\\
5.18	-857.2802\\
5.19	-857.2808\\
5.2	-857.2825\\
5.21	-857.2844\\
5.22	-857.283\\
5.23	-857.2844\\
5.24	-857.2832\\
5.25	-857.2818\\
5.26	-857.2805\\
5.27	-857.2805\\
5.28	-857.2799\\
5.29	-857.2841\\
5.3	-857.2858\\
5.31	-857.2822\\
5.32	-857.2794\\
5.33	-857.2788\\
5.34	-857.2812\\
5.35	-857.2833\\
5.36	-857.2841\\
5.37	-857.2842\\
5.38	-857.2804\\
5.39	-857.2796\\
5.4	-857.2823\\
5.41	-857.2854\\
5.42	-857.2835\\
5.43	-857.2806\\
5.44	-857.277\\
5.45	-857.2788\\
5.46	-857.2834\\
5.47	-857.2884\\
5.48	-857.2867\\
5.49	-857.2808\\
5.5	-857.278\\
5.51	-857.2775\\
5.52	-857.2823\\
5.53	-857.2882\\
5.54	-857.2895\\
5.55	-857.2862\\
5.56	-857.2802\\
5.57	-857.2768\\
5.58	-857.2783\\
5.59	-857.2813\\
5.6	-857.2827\\
5.61	-857.2841\\
5.62	-857.2836\\
5.63	-857.2818\\
5.64	-857.2825\\
5.65	-857.2827\\
5.66	-857.2834\\
5.67	-857.2814\\
5.68	-857.2832\\
5.69	-857.2835\\
5.7	-857.2841\\
5.71	-857.2825\\
5.72	-857.2816\\
5.73	-857.2822\\
5.74	-857.2809\\
5.75	-857.2816\\
5.76	-857.2813\\
5.77	-857.2819\\
5.78	-857.2816\\
5.79	-857.2818\\
5.8	-857.2843\\
5.81	-857.284\\
5.82	-857.2853\\
5.83	-857.2825\\
5.84	-857.2807\\
5.85	-857.2788\\
5.86	-857.2818\\
5.87	-857.2859\\
5.88	-857.2877\\
5.89	-857.2862\\
5.9	-857.2797\\
5.91	-857.2757\\
5.92	-857.2773\\
5.93	-857.2829\\
5.94	-857.2862\\
5.95	-857.2842\\
5.96	-857.2802\\
5.97	-857.279\\
5.98	-857.2812\\
5.99	-857.2855\\
6	-857.2858\\
6.01	-857.2803\\
6.02	-857.2767\\
6.03	-857.2791\\
6.04	-857.283\\
6.05	-857.2864\\
6.06	-857.2847\\
6.07	-857.2803\\
6.08	-857.2769\\
6.09	-857.2796\\
6.1	-857.2852\\
6.11	-857.2876\\
6.12	-857.2856\\
6.13	-857.2808\\
6.14	-857.2796\\
6.15	-857.2817\\
6.16	-857.2834\\
6.17	-857.2822\\
6.18	-857.2812\\
6.19	-857.2812\\
6.2	-857.2811\\
6.21	-857.2813\\
6.22	-857.2822\\
6.23	-857.2821\\
6.24	-857.2814\\
6.25	-857.2823\\
6.26	-857.285\\
6.27	-857.2865\\
6.28	-857.2836\\
6.29	-857.2789\\
6.3	-857.2785\\
6.31	-857.2796\\
6.32	-857.2828\\
6.33	-857.2842\\
6.34	-857.2841\\
6.35	-857.2845\\
6.36	-857.2844\\
6.37	-857.2794\\
6.38	-857.275\\
6.39	-857.2754\\
6.4	-857.2815\\
6.41	-857.2872\\
6.42	-857.2853\\
6.43	-857.2803\\
6.44	-857.2773\\
6.45	-857.2807\\
6.46	-857.2846\\
6.47	-857.2846\\
6.48	-857.2803\\
6.49	-857.279\\
6.5	-857.283\\
6.51	-857.2862\\
6.52	-857.2842\\
6.53	-857.2823\\
6.54	-857.2785\\
6.55	-857.2798\\
6.56	-857.2852\\
6.57	-857.2868\\
6.58	-857.2819\\
6.59	-857.2783\\
6.6	-857.2798\\
6.61	-857.286\\
6.62	-857.287\\
6.63	-857.2824\\
6.64	-857.2807\\
6.65	-857.2818\\
6.66	-857.2842\\
6.67	-857.2817\\
6.68	-857.2828\\
6.69	-857.2825\\
6.7	-857.283\\
6.71	-857.2827\\
6.72	-857.2832\\
6.73	-857.2845\\
6.74	-857.2833\\
6.75	-857.2818\\
6.76	-857.2802\\
6.77	-857.2795\\
6.78	-857.2811\\
6.79	-857.2852\\
6.8	-857.2907\\
6.81	-857.2867\\
6.82	-857.2813\\
6.83	-857.2779\\
6.84	-857.2784\\
6.85	-857.283\\
6.86	-857.2866\\
6.87	-857.2868\\
6.88	-857.283\\
6.89	-857.2822\\
6.9	-857.2838\\
6.91	-857.2821\\
6.92	-857.2795\\
6.93	-857.2797\\
6.94	-857.2823\\
6.95	-857.2839\\
6.96	-857.2827\\
6.97	-857.281\\
6.98	-857.2818\\
6.99	-857.2831\\
7	-857.2822\\
7.01	-857.2799\\
7.02	-857.2804\\
7.03	-857.2828\\
7.04	-857.2863\\
7.05	-857.2845\\
7.06	-857.2814\\
7.07	-857.2802\\
7.08	-857.2796\\
7.09	-857.2791\\
7.1	-857.2806\\
7.11	-857.2816\\
7.12	-857.2832\\
7.13	-857.2856\\
7.14	-857.2835\\
7.15	-857.2789\\
7.16	-857.2766\\
7.17	-857.2784\\
7.18	-857.2834\\
7.19	-857.2864\\
7.2	-857.2873\\
7.21	-857.282\\
7.22	-857.2759\\
7.23	-857.2756\\
7.24	-857.2817\\
7.25	-857.2878\\
7.26	-857.2872\\
7.27	-857.2831\\
7.28	-857.2784\\
7.29	-857.2823\\
7.3	-857.2854\\
7.31	-857.2846\\
7.32	-857.2791\\
7.33	-857.2781\\
7.34	-857.2802\\
7.35	-857.2828\\
7.36	-857.2853\\
7.37	-857.2843\\
7.38	-857.2787\\
7.39	-857.2759\\
7.4	-857.2771\\
7.41	-857.2841\\
7.42	-857.2878\\
7.43	-857.287\\
7.44	-857.2821\\
7.45	-857.277\\
7.46	-857.2787\\
7.47	-857.2838\\
7.48	-857.2852\\
7.49	-857.2852\\
7.5	-857.2823\\
7.51	-857.2824\\
7.52	-857.2848\\
7.53	-857.2863\\
7.54	-857.2817\\
7.55	-857.2755\\
7.56	-857.2768\\
7.57	-857.2817\\
7.58	-857.2864\\
7.59	-857.2858\\
7.6	-857.282\\
7.61	-857.2803\\
7.62	-857.2816\\
7.63	-857.2828\\
7.64	-857.2824\\
7.65	-857.2822\\
7.66	-857.2845\\
7.67	-857.2857\\
7.68	-857.2826\\
7.69	-857.2794\\
7.7	-857.2802\\
7.71	-857.2804\\
7.72	-857.2819\\
7.73	-857.2828\\
7.74	-857.2828\\
7.75	-857.2818\\
7.76	-857.2815\\
7.77	-857.2795\\
7.78	-857.2806\\
7.79	-857.2835\\
7.8	-857.2867\\
7.81	-857.2838\\
7.82	-857.2806\\
7.83	-857.2813\\
7.84	-857.2824\\
7.85	-857.2838\\
7.86	-857.2833\\
7.87	-857.2818\\
7.88	-857.2808\\
7.89	-857.2829\\
7.9	-857.2856\\
7.91	-857.285\\
7.92	-857.2833\\
7.93	-857.2825\\
7.94	-857.2815\\
7.95	-857.2804\\
7.96	-857.2792\\
7.97	-857.2793\\
7.98	-857.2827\\
7.99	-857.286\\
8	-857.2857\\
8.01	-857.2836\\
8.02	-857.2778\\
8.03	-857.2758\\
8.04	-857.278\\
8.05	-857.2839\\
8.06	-857.2882\\
8.07	-857.2858\\
8.08	-857.2803\\
8.09	-857.277\\
8.1	-857.2789\\
8.11	-857.2844\\
8.12	-857.2862\\
8.13	-857.2842\\
8.14	-857.2808\\
8.15	-857.2773\\
8.16	-857.2782\\
8.17	-857.2839\\
8.18	-857.2849\\
8.19	-857.2849\\
8.2	-857.2812\\
8.21	-857.2787\\
8.22	-857.28\\
8.23	-857.2818\\
8.24	-857.2841\\
8.25	-857.2819\\
8.26	-857.2824\\
8.27	-857.2807\\
8.28	-857.2794\\
8.29	-857.2815\\
8.3	-857.2849\\
8.31	-857.2856\\
8.32	-857.2825\\
8.33	-857.2806\\
8.34	-857.278\\
8.35	-857.2784\\
8.36	-857.2821\\
8.37	-857.2857\\
8.38	-857.286\\
8.39	-857.2806\\
8.4	-857.2768\\
8.41	-857.2783\\
8.42	-857.2841\\
8.43	-857.2848\\
8.44	-857.282\\
8.45	-857.2796\\
8.46	-857.281\\
8.47	-857.2831\\
8.48	-857.2828\\
8.49	-857.2814\\
8.5	-857.2814\\
8.51	-857.2822\\
8.52	-857.2839\\
8.53	-857.2843\\
8.54	-857.2836\\
8.55	-857.2817\\
8.56	-857.2788\\
8.57	-857.2775\\
8.58	-857.2808\\
8.59	-857.2839\\
8.6	-857.2837\\
8.61	-857.2826\\
8.62	-857.2817\\
8.63	-857.2815\\
8.64	-857.2837\\
8.65	-857.2833\\
8.66	-857.2824\\
8.67	-857.2795\\
8.68	-857.2785\\
8.69	-857.2818\\
8.7	-857.2863\\
8.71	-857.2856\\
8.72	-857.2809\\
8.73	-857.2784\\
8.74	-857.279\\
8.75	-857.2842\\
8.76	-857.2859\\
8.77	-857.2843\\
8.78	-857.2812\\
8.79	-857.2796\\
8.8	-857.2815\\
8.81	-857.2819\\
8.82	-857.2821\\
8.83	-857.2816\\
8.84	-857.2825\\
8.85	-857.2818\\
8.86	-857.2837\\
8.87	-857.2813\\
8.88	-857.2825\\
8.89	-857.2803\\
8.9	-857.2772\\
8.91	-857.2794\\
8.92	-857.2847\\
8.93	-857.2883\\
8.94	-857.2861\\
8.95	-857.2792\\
8.96	-857.2744\\
8.97	-857.2757\\
8.98	-857.2815\\
8.99	-857.2866\\
9	-857.2879\\
9.01	-857.2837\\
9.02	-857.2792\\
9.03	-857.2787\\
9.04	-857.2811\\
9.05	-857.2836\\
9.06	-857.2867\\
9.07	-857.2871\\
9.08	-857.2833\\
9.09	-857.2793\\
9.1	-857.2796\\
9.11	-857.2838\\
9.12	-857.2837\\
9.13	-857.2808\\
9.14	-857.2758\\
9.15	-857.2784\\
9.16	-857.2849\\
9.17	-857.2875\\
9.18	-857.2846\\
9.19	-857.2798\\
9.2	-857.2745\\
9.21	-857.2769\\
9.22	-857.2837\\
9.23	-857.2883\\
9.24	-857.287\\
9.25	-857.2834\\
9.26	-857.2794\\
9.27	-857.2793\\
9.28	-857.2812\\
9.29	-857.2807\\
9.3	-857.2809\\
9.31	-857.2826\\
9.32	-857.2856\\
9.33	-857.2878\\
9.34	-857.2835\\
9.35	-857.2778\\
9.36	-857.2767\\
9.37	-857.2805\\
9.38	-857.2855\\
9.39	-857.2869\\
9.4	-857.2838\\
9.41	-857.28\\
9.42	-857.279\\
9.43	-857.2812\\
9.44	-857.2865\\
9.45	-857.2861\\
9.46	-857.2814\\
9.47	-857.2778\\
9.48	-857.2795\\
9.49	-857.2827\\
9.5	-857.2857\\
9.51	-857.2858\\
9.52	-857.2829\\
9.53	-857.2804\\
9.54	-857.2785\\
9.55	-857.2819\\
9.56	-857.2874\\
9.57	-857.2868\\
9.58	-857.2827\\
9.59	-857.2778\\
9.6	-857.2796\\
9.61	-857.2795\\
9.62	-857.2825\\
9.63	-857.2852\\
9.64	-857.2858\\
9.65	-857.2836\\
9.66	-857.2801\\
9.67	-857.2775\\
9.68	-857.2799\\
9.69	-857.2832\\
9.7	-857.2845\\
9.71	-857.2838\\
9.72	-857.2835\\
9.73	-857.2836\\
9.74	-857.2844\\
9.75	-857.2815\\
9.76	-857.2775\\
9.77	-857.2796\\
9.78	-857.2852\\
9.79	-857.2864\\
9.8	-857.2832\\
9.81	-857.2809\\
9.82	-857.2797\\
9.83	-857.2807\\
9.84	-857.2807\\
9.85	-857.2792\\
9.86	-857.2827\\
9.87	-857.2851\\
9.88	-857.2831\\
9.89	-857.2809\\
9.9	-857.2817\\
9.91	-857.2829\\
9.92	-857.2836\\
9.93	-857.283\\
9.94	-857.2821\\
9.95	-857.2834\\
9.96	-857.283\\
9.97	-857.283\\
9.98	-857.2844\\
9.99	-857.2827\\
};
\addlegendentry{Total energy};

\addplot [color=mycolor2,solid]
  table[row sep=crcr]{%
0	-857.2121\\
0.01	-857.8086\\
0.02	-858.9914\\
0.03	-859.6548\\
0.04	-859.3671\\
0.05	-858.6332\\
0.06	-858.2333\\
0.07	-858.44\\
0.08	-858.8268\\
0.09	-858.8425\\
0.1	-858.4806\\
0.11	-858.2232\\
0.12	-858.4106\\
0.13	-858.827\\
0.14	-858.988\\
0.15	-858.754\\
0.16	-858.4568\\
0.17	-858.4451\\
0.18	-858.6845\\
0.19	-858.8633\\
0.2	-858.7848\\
0.21	-858.5525\\
0.22	-858.4176\\
0.23	-858.5034\\
0.24	-858.6858\\
0.25	-858.7577\\
0.26	-858.6673\\
0.27	-858.5632\\
0.28	-858.5946\\
0.29	-858.7264\\
0.3	-858.787\\
0.31	-858.6943\\
0.32	-858.5506\\
0.33	-858.5204\\
0.34	-858.622\\
0.35	-858.7234\\
0.36	-858.6965\\
0.37	-858.5725\\
0.38	-858.4712\\
0.39	-858.4841\\
0.4	-858.5956\\
0.41	-858.7382\\
0.42	-858.8465\\
0.43	-858.8648\\
0.44	-858.7735\\
0.45	-858.6097\\
0.46	-858.4774\\
0.47	-858.454\\
0.48	-858.5352\\
0.49	-858.6448\\
0.5	-858.6986\\
0.51	-858.6582\\
0.52	-858.5743\\
0.53	-858.5494\\
0.54	-858.6414\\
0.55	-858.7917\\
0.56	-858.8609\\
0.57	-858.776\\
0.58	-858.6027\\
0.59	-858.4779\\
0.6	-858.4787\\
0.61	-858.5562\\
0.62	-858.6163\\
0.63	-858.6272\\
0.64	-858.6442\\
0.65	-858.7099\\
0.66	-858.7753\\
0.67	-858.7641\\
0.68	-858.6663\\
0.69	-858.5682\\
0.7	-858.5472\\
0.71	-858.5819\\
0.72	-858.5978\\
0.73	-858.5801\\
0.74	-858.5916\\
0.75	-858.6807\\
0.76	-858.7994\\
0.77	-858.8317\\
0.78	-858.7076\\
0.79	-858.5091\\
0.8	-858.416\\
0.81	-858.5132\\
0.82	-858.7038\\
0.83	-858.8094\\
0.84	-858.7593\\
0.85	-858.6306\\
0.86	-858.5344\\
0.87	-858.515\\
0.88	-858.558\\
0.89	-858.6526\\
0.9	-858.7669\\
0.91	-858.8217\\
0.92	-858.7551\\
0.93	-858.6107\\
0.94	-858.5071\\
0.95	-858.5283\\
0.96	-858.6412\\
0.97	-858.7287\\
0.98	-858.6865\\
0.99	-858.5288\\
1	-858.4025\\
1.01	-858.4711\\
1.02	-858.7254\\
1.03	-858.9673\\
1.04	-858.9874\\
1.05	-858.7634\\
1.06	-858.4774\\
1.07	-858.3531\\
1.08	-858.4567\\
1.09	-858.6553\\
1.1	-858.7537\\
1.11	-858.6983\\
1.12	-858.5988\\
1.13	-858.5783\\
1.14	-858.6521\\
1.15	-858.7357\\
1.16	-858.744\\
1.17	-858.67\\
1.18	-858.5754\\
1.19	-858.5299\\
1.2	-858.5623\\
1.21	-858.6527\\
1.22	-858.7532\\
1.23	-858.7876\\
1.24	-858.693\\
1.25	-858.5261\\
1.26	-858.4562\\
1.27	-858.5785\\
1.28	-858.7828\\
1.29	-858.8547\\
1.3	-858.7035\\
1.31	-858.4778\\
1.32	-858.3982\\
1.33	-858.5434\\
1.34	-858.7735\\
1.35	-858.8673\\
1.36	-858.7454\\
1.37	-858.5448\\
1.38	-858.471\\
1.39	-858.5708\\
1.4	-858.7141\\
1.41	-858.7701\\
1.42	-858.7304\\
1.43	-858.6829\\
1.44	-858.6786\\
1.45	-858.6644\\
1.46	-858.5707\\
1.47	-858.4387\\
1.48	-858.4137\\
1.49	-858.5717\\
1.5	-858.7898\\
1.51	-858.8685\\
1.52	-858.7477\\
1.53	-858.594\\
1.54	-858.5902\\
1.55	-858.7168\\
1.56	-858.7785\\
1.57	-858.6395\\
1.58	-858.406\\
1.59	-858.322\\
1.6	-858.517\\
1.61	-858.8468\\
1.62	-859.0326\\
1.63	-858.9342\\
1.64	-858.6455\\
1.65	-858.3893\\
1.66	-858.3315\\
1.67	-858.4666\\
1.68	-858.6551\\
1.69	-858.7697\\
1.7	-858.793\\
1.71	-858.777\\
1.72	-858.7377\\
1.73	-858.6469\\
1.74	-858.5218\\
1.75	-858.4615\\
1.76	-858.5391\\
1.77	-858.7005\\
1.78	-858.7988\\
1.79	-858.7484\\
1.8	-858.6126\\
1.81	-858.5394\\
1.82	-858.5993\\
1.83	-858.703\\
1.84	-858.7284\\
1.85	-858.6512\\
1.86	-858.5586\\
1.87	-858.5411\\
1.88	-858.5988\\
1.89	-858.6626\\
1.9	-858.6845\\
1.91	-858.6847\\
1.92	-858.707\\
1.93	-858.7539\\
1.94	-858.7608\\
1.95	-858.6709\\
1.96	-858.5085\\
1.97	-858.4022\\
1.98	-858.4521\\
1.99	-858.6212\\
2	-858.7718\\
2.01	-858.8215\\
2.02	-858.7919\\
2.03	-858.7428\\
2.04	-858.6735\\
2.05	-858.5599\\
2.06	-858.4376\\
2.07	-858.421\\
2.08	-858.5682\\
2.09	-858.7805\\
2.1	-858.8746\\
2.11	-858.7645\\
2.12	-858.5541\\
2.13	-858.4531\\
2.14	-858.5633\\
2.15	-858.7858\\
2.16	-858.9111\\
2.17	-858.8208\\
2.18	-858.5769\\
2.19	-858.3691\\
2.2	-858.354\\
2.21	-858.529\\
2.22	-858.7583\\
2.23	-858.8832\\
2.24	-858.8328\\
2.25	-858.6614\\
2.26	-858.5153\\
2.27	-858.5091\\
2.28	-858.6185\\
2.29	-858.7127\\
2.3	-858.6968\\
2.31	-858.6036\\
2.32	-858.5522\\
2.33	-858.6218\\
2.34	-858.7573\\
2.35	-858.8089\\
2.36	-858.6918\\
2.37	-858.5059\\
2.38	-858.4403\\
2.39	-858.5708\\
2.4	-858.7784\\
2.41	-858.8723\\
2.42	-858.7733\\
2.43	-858.5765\\
2.44	-858.4457\\
2.45	-858.4649\\
2.46	-858.5779\\
2.47	-858.6681\\
2.48	-858.6949\\
2.49	-858.708\\
2.5	-858.7539\\
2.51	-858.79\\
2.52	-858.745\\
2.53	-858.6248\\
2.54	-858.5227\\
2.55	-858.5271\\
2.56	-858.611\\
2.57	-858.6674\\
2.58	-858.6256\\
2.59	-858.5414\\
2.6	-858.5459\\
2.61	-858.6896\\
2.62	-858.8629\\
2.63	-858.8954\\
2.64	-858.7341\\
2.65	-858.5015\\
2.66	-858.3849\\
2.67	-858.4599\\
2.68	-858.6397\\
2.69	-858.7657\\
2.7	-858.7713\\
2.71	-858.7045\\
2.72	-858.6445\\
2.73	-858.6293\\
2.74	-858.6494\\
2.75	-858.6657\\
2.76	-858.6453\\
2.77	-858.5894\\
2.78	-858.5549\\
2.79	-858.595\\
2.8	-858.6815\\
2.81	-858.7228\\
2.82	-858.6769\\
2.83	-858.5938\\
2.84	-858.5636\\
2.85	-858.6212\\
2.86	-858.7192\\
2.87	-858.7711\\
2.88	-858.7266\\
2.89	-858.6192\\
2.9	-858.5471\\
2.91	-858.5735\\
2.92	-858.6632\\
2.93	-858.7142\\
2.94	-858.681\\
2.95	-858.5962\\
2.96	-858.5453\\
2.97	-858.5671\\
2.98	-858.6352\\
2.99	-858.6923\\
3	-858.717\\
3.01	-858.7193\\
3.02	-858.7144\\
3.03	-858.6903\\
3.04	-858.6258\\
3.05	-858.535\\
3.06	-858.4819\\
3.07	-858.5395\\
3.08	-858.7133\\
3.09	-858.8806\\
3.1	-858.8826\\
3.11	-858.6831\\
3.12	-858.4314\\
3.13	-858.3402\\
3.14	-858.482\\
3.15	-858.7346\\
3.16	-858.8992\\
3.17	-858.8531\\
3.18	-858.648\\
3.19	-858.4704\\
3.2	-858.4773\\
3.21	-858.6339\\
3.22	-858.7626\\
3.23	-858.7475\\
3.24	-858.6363\\
3.25	-858.5597\\
3.26	-858.5869\\
3.27	-858.674\\
3.28	-858.7357\\
3.29	-858.7193\\
3.3	-858.6348\\
3.31	-858.5486\\
3.32	-858.5251\\
3.33	-858.5819\\
3.34	-858.672\\
3.35	-858.7309\\
3.36	-858.7202\\
3.37	-858.6498\\
3.38	-858.5857\\
3.39	-858.601\\
3.4	-858.6911\\
3.41	-858.7651\\
3.42	-858.7352\\
3.43	-858.6234\\
3.44	-858.5204\\
3.45	-858.4903\\
3.46	-858.5341\\
3.47	-858.6222\\
3.48	-858.7224\\
3.49	-858.7916\\
3.5	-858.7899\\
3.51	-858.7089\\
3.52	-858.6015\\
3.53	-858.5445\\
3.54	-858.5653\\
3.55	-858.6221\\
3.56	-858.6542\\
3.57	-858.6359\\
3.58	-858.6102\\
3.59	-858.6284\\
3.6	-858.6953\\
3.61	-858.749\\
3.62	-858.7169\\
3.63	-858.614\\
3.64	-858.5374\\
3.65	-858.5576\\
3.66	-858.6528\\
3.67	-858.7347\\
3.68	-858.7385\\
3.69	-858.6691\\
3.7	-858.5879\\
3.71	-858.5564\\
3.72	-858.5812\\
3.73	-858.6214\\
3.74	-858.6442\\
3.75	-858.6575\\
3.76	-858.685\\
3.77	-858.7154\\
3.78	-858.7123\\
3.79	-858.6703\\
3.8	-858.6294\\
3.81	-858.6193\\
3.82	-858.6286\\
3.83	-858.6312\\
3.84	-858.607\\
3.85	-858.576\\
3.86	-858.583\\
3.87	-858.6555\\
3.88	-858.7606\\
3.89	-858.8106\\
3.9	-858.7437\\
3.91	-858.5947\\
3.92	-858.4731\\
3.93	-858.4635\\
3.94	-858.5548\\
3.95	-858.6635\\
3.96	-858.7323\\
3.97	-858.7539\\
3.98	-858.7487\\
3.99	-858.7248\\
4	-858.6795\\
4.01	-858.6358\\
4.02	-858.61\\
4.03	-858.5943\\
4.04	-858.5871\\
4.05	-858.595\\
4.06	-858.608\\
4.07	-858.6067\\
4.08	-858.6097\\
4.09	-858.6476\\
4.1	-858.6931\\
4.11	-858.6955\\
4.12	-858.6625\\
4.13	-858.6554\\
4.14	-858.7009\\
4.15	-858.7573\\
4.16	-858.7455\\
4.17	-858.6399\\
4.18	-858.5039\\
4.19	-858.4445\\
4.2	-858.5257\\
4.21	-858.6899\\
4.22	-858.8103\\
4.23	-858.7958\\
4.24	-858.6703\\
4.25	-858.5411\\
4.26	-858.512\\
4.27	-858.5823\\
4.28	-858.68\\
4.29	-858.7312\\
4.3	-858.7201\\
4.31	-858.6714\\
4.32	-858.6184\\
4.33	-858.5816\\
4.34	-858.589\\
4.35	-858.6556\\
4.36	-858.7361\\
4.37	-858.7573\\
4.38	-858.6977\\
4.39	-858.603\\
4.4	-858.545\\
4.41	-858.5597\\
4.42	-858.6166\\
4.43	-858.6503\\
4.44	-858.6212\\
4.45	-858.5761\\
4.46	-858.6053\\
4.47	-858.7283\\
4.48	-858.846\\
4.49	-858.8316\\
4.5	-858.6782\\
4.51	-858.5157\\
4.52	-858.476\\
4.53	-858.5659\\
4.54	-858.6758\\
4.55	-858.6898\\
4.56	-858.6096\\
4.57	-858.5509\\
4.58	-858.6107\\
4.59	-858.7557\\
4.6	-858.8393\\
4.61	-858.7632\\
4.62	-858.5847\\
4.63	-858.4555\\
4.64	-858.471\\
4.65	-858.611\\
4.66	-858.7675\\
4.67	-858.8292\\
4.68	-858.7622\\
4.69	-858.6277\\
4.7	-858.5356\\
4.71	-858.5378\\
4.72	-858.6032\\
4.73	-858.6693\\
4.74	-858.6961\\
4.75	-858.6881\\
4.76	-858.6686\\
4.77	-858.6431\\
4.78	-858.6006\\
4.79	-858.5571\\
4.8	-858.5693\\
4.81	-858.6531\\
4.82	-858.7459\\
4.83	-858.7675\\
4.84	-858.6964\\
4.85	-858.5925\\
4.86	-858.5464\\
4.87	-858.5844\\
4.88	-858.66\\
4.89	-858.6926\\
4.9	-858.6605\\
4.91	-858.6179\\
4.92	-858.6213\\
4.93	-858.6718\\
4.94	-858.7198\\
4.95	-858.7191\\
4.96	-858.6584\\
4.97	-858.5969\\
4.98	-858.5936\\
4.99	-858.6367\\
5	-858.6457\\
5.01	-858.5824\\
5.02	-858.5217\\
5.03	-858.5576\\
5.04	-858.6703\\
5.05	-858.7594\\
5.06	-858.762\\
5.07	-858.7148\\
5.08	-858.6737\\
5.09	-858.6501\\
5.1	-858.625\\
5.11	-858.5991\\
5.12	-858.5933\\
5.13	-858.6241\\
5.14	-858.6738\\
5.15	-858.7003\\
5.16	-858.6758\\
5.17	-858.6169\\
5.18	-858.5673\\
5.19	-858.5644\\
5.2	-858.6042\\
5.21	-858.656\\
5.22	-858.6885\\
5.23	-858.697\\
5.24	-858.6772\\
5.25	-858.6415\\
5.26	-858.6126\\
5.27	-858.6199\\
5.28	-858.6697\\
5.29	-858.734\\
5.3	-858.75\\
5.31	-858.692\\
5.32	-858.6042\\
5.33	-858.5482\\
5.34	-858.5504\\
5.35	-858.5886\\
5.36	-858.6245\\
5.37	-858.6363\\
5.38	-858.6316\\
5.39	-858.6459\\
5.4	-858.6854\\
5.41	-858.7174\\
5.42	-858.706\\
5.43	-858.6619\\
5.44	-858.6288\\
5.45	-858.6499\\
5.46	-858.7053\\
5.47	-858.7271\\
5.48	-858.6622\\
5.49	-858.5423\\
5.5	-858.4598\\
5.51	-858.4836\\
5.52	-858.6131\\
5.53	-858.7628\\
5.54	-858.8279\\
5.55	-858.7699\\
5.56	-858.6422\\
5.57	-858.5413\\
5.58	-858.5265\\
5.59	-858.5859\\
5.6	-858.6654\\
5.61	-858.714\\
5.62	-858.7127\\
5.63	-858.6858\\
5.64	-858.6663\\
5.65	-858.6534\\
5.66	-858.6348\\
5.67	-858.6128\\
5.68	-858.6122\\
5.69	-858.6234\\
5.7	-858.63\\
5.71	-858.628\\
5.72	-858.6317\\
5.73	-858.6432\\
5.74	-858.6508\\
5.75	-858.6543\\
5.76	-858.6516\\
5.77	-858.6473\\
5.78	-858.6478\\
5.79	-858.6636\\
5.8	-858.6901\\
5.81	-858.6972\\
5.82	-858.667\\
5.83	-858.5999\\
5.84	-858.5426\\
5.85	-858.5408\\
5.86	-858.6108\\
5.87	-858.7095\\
5.88	-858.7644\\
5.89	-858.7303\\
5.9	-858.6324\\
5.91	-858.5561\\
5.92	-858.5621\\
5.93	-858.6344\\
5.94	-858.6979\\
5.95	-858.6969\\
5.96	-858.6514\\
5.97	-858.625\\
5.98	-858.6456\\
5.99	-858.6761\\
6	-858.6595\\
6.01	-858.5973\\
6.02	-858.5595\\
6.03	-858.6018\\
6.04	-858.6956\\
6.05	-858.7557\\
6.06	-858.7213\\
6.07	-858.6261\\
6.08	-858.5603\\
6.09	-858.5806\\
6.1	-858.6489\\
6.11	-858.6822\\
6.12	-858.6474\\
6.13	-858.5894\\
6.14	-858.5788\\
6.15	-858.6293\\
6.16	-858.695\\
6.17	-858.7263\\
6.18	-858.7124\\
6.19	-858.6724\\
6.2	-858.6322\\
6.21	-858.6075\\
6.22	-858.6026\\
6.23	-858.616\\
6.24	-858.6428\\
6.25	-858.6736\\
6.26	-858.6923\\
6.27	-858.6776\\
6.28	-858.6212\\
6.29	-858.5518\\
6.3	-858.5245\\
6.31	-858.565\\
6.32	-858.6514\\
6.33	-858.7343\\
6.34	-858.7798\\
6.35	-858.7663\\
6.36	-858.6825\\
6.37	-858.5591\\
6.38	-858.4853\\
6.39	-858.5312\\
6.4	-858.671\\
6.41	-858.7895\\
6.42	-858.7934\\
6.43	-858.7062\\
6.44	-858.6213\\
6.45	-858.5964\\
6.46	-858.5979\\
6.47	-858.5729\\
6.48	-858.5314\\
6.49	-858.5383\\
6.5	-858.6162\\
6.51	-858.7107\\
6.52	-858.7554\\
6.53	-858.7395\\
6.54	-858.6987\\
6.55	-858.6829\\
6.56	-858.6818\\
6.57	-858.6419\\
6.58	-858.5619\\
6.59	-858.5197\\
6.6	-858.5687\\
6.61	-858.6646\\
6.62	-858.7137\\
6.63	-858.6905\\
6.64	-858.6452\\
6.65	-858.617\\
6.66	-858.6096\\
6.67	-858.6135\\
6.68	-858.6396\\
6.69	-858.6736\\
6.7	-858.6987\\
6.71	-858.7044\\
6.72	-858.6945\\
6.73	-858.6685\\
6.74	-858.6242\\
6.75	-858.5785\\
6.76	-858.5549\\
6.77	-858.5738\\
6.78	-858.6406\\
6.79	-858.7253\\
6.8	-858.7621\\
6.81	-858.6975\\
6.82	-858.5796\\
6.83	-858.5061\\
6.84	-858.5358\\
6.85	-858.6386\\
6.86	-858.7292\\
6.87	-858.7512\\
6.88	-858.7177\\
6.89	-858.6807\\
6.9	-858.6561\\
6.91	-858.6254\\
6.92	-858.5932\\
6.93	-858.5907\\
6.94	-858.627\\
6.95	-858.6691\\
6.96	-858.6811\\
6.97	-858.6605\\
6.98	-858.6304\\
6.99	-858.6056\\
7	-858.5887\\
7.01	-858.5904\\
7.02	-858.6263\\
7.03	-858.6827\\
7.04	-858.7201\\
7.05	-858.7032\\
7.06	-858.6485\\
7.07	-858.5976\\
7.08	-858.5766\\
7.09	-858.5913\\
7.1	-858.6425\\
7.11	-858.7142\\
7.12	-858.7701\\
7.13	-858.7625\\
7.14	-858.6713\\
7.15	-858.5488\\
7.16	-858.4852\\
7.17	-858.5264\\
7.18	-858.6392\\
7.19	-858.7413\\
7.2	-858.7636\\
7.21	-858.6911\\
7.22	-858.5926\\
7.23	-858.56\\
7.24	-858.6148\\
7.25	-858.6854\\
7.26	-858.6929\\
7.27	-858.646\\
7.28	-858.6116\\
7.29	-858.6304\\
7.3	-858.6592\\
7.31	-858.6577\\
7.32	-858.638\\
7.33	-858.6475\\
7.34	-858.6918\\
7.35	-858.7323\\
7.36	-858.7214\\
7.37	-858.6378\\
7.38	-858.5204\\
7.39	-858.4616\\
7.4	-858.5215\\
7.41	-858.6674\\
7.42	-858.7829\\
7.43	-858.7865\\
7.44	-858.6962\\
7.45	-858.6005\\
7.46	-858.5709\\
7.47	-858.6009\\
7.48	-858.6405\\
7.49	-858.6657\\
7.5	-858.6822\\
7.51	-858.7061\\
7.52	-858.7165\\
7.53	-858.6755\\
7.54	-858.5831\\
7.55	-858.5116\\
7.56	-858.537\\
7.57	-858.6417\\
7.58	-858.7364\\
7.59	-858.7496\\
7.6	-858.6971\\
7.61	-858.6419\\
7.62	-858.6194\\
7.63	-858.6215\\
7.64	-858.632\\
7.65	-858.6447\\
7.66	-858.6533\\
7.67	-858.6446\\
7.68	-858.6185\\
7.69	-858.5975\\
7.7	-858.6065\\
7.71	-858.6441\\
7.72	-858.6923\\
7.73	-858.7226\\
7.74	-858.7178\\
7.75	-858.6755\\
7.76	-858.6147\\
7.77	-858.5686\\
7.78	-858.5731\\
7.79	-858.6218\\
7.8	-858.6732\\
7.81	-858.6881\\
7.82	-858.6798\\
7.83	-858.6806\\
7.84	-858.6918\\
7.85	-858.6875\\
7.86	-858.6461\\
7.87	-858.5864\\
7.88	-858.5524\\
7.89	-858.5724\\
7.9	-858.6241\\
7.91	-858.6604\\
7.92	-858.6671\\
7.93	-858.6609\\
7.94	-858.6521\\
7.95	-858.6415\\
7.96	-858.6392\\
7.97	-858.6633\\
7.98	-858.7083\\
7.99	-858.7325\\
8	-858.696\\
8.01	-858.605\\
8.02	-858.5115\\
8.03	-858.4939\\
8.04	-858.5827\\
8.05	-858.7251\\
8.06	-858.8116\\
8.07	-858.7773\\
8.08	-858.6661\\
8.09	-858.5774\\
8.1	-858.5696\\
8.11	-858.6164\\
8.12	-858.6453\\
8.13	-858.6211\\
8.14	-858.5726\\
8.15	-858.5581\\
8.16	-858.6115\\
8.17	-858.7026\\
8.18	-858.761\\
8.19	-858.7529\\
8.2	-858.691\\
8.21	-858.6301\\
8.22	-858.6107\\
8.23	-858.6273\\
8.24	-858.6509\\
8.25	-858.6513\\
8.26	-858.6325\\
8.27	-858.6077\\
8.28	-858.6074\\
8.29	-858.6451\\
8.3	-858.6906\\
8.31	-858.6943\\
8.32	-858.6402\\
8.33	-858.5697\\
8.34	-858.5397\\
8.35	-858.5857\\
8.36	-858.6825\\
8.37	-858.7563\\
8.38	-858.7431\\
8.39	-858.6534\\
8.4	-858.5786\\
8.41	-858.5885\\
8.42	-858.659\\
8.43	-858.7083\\
8.44	-858.7027\\
8.45	-858.6683\\
8.46	-858.6407\\
8.47	-858.6186\\
8.48	-858.5874\\
8.49	-858.5605\\
8.5	-858.5703\\
8.51	-858.6258\\
8.52	-858.7015\\
8.53	-858.7524\\
8.54	-858.7446\\
8.55	-858.6767\\
8.56	-858.5905\\
8.57	-858.5484\\
8.58	-858.5777\\
8.59	-858.6371\\
8.6	-858.6695\\
8.61	-858.6643\\
8.62	-858.6526\\
8.63	-858.6585\\
8.64	-858.6691\\
8.65	-858.654\\
8.66	-858.6194\\
8.67	-858.5994\\
8.68	-858.631\\
8.69	-858.705\\
8.7	-858.7545\\
8.71	-858.7172\\
8.72	-858.6155\\
8.73	-858.5366\\
8.74	-858.5408\\
8.75	-858.6153\\
8.76	-858.6872\\
8.77	-858.7092\\
8.78	-858.6879\\
8.79	-858.6543\\
8.8	-858.6262\\
8.81	-858.6055\\
8.82	-858.6027\\
8.83	-858.6205\\
8.84	-858.6509\\
8.85	-858.6748\\
8.86	-858.6851\\
8.87	-858.6689\\
8.88	-858.6342\\
8.89	-858.5918\\
8.9	-858.5824\\
8.91	-858.6379\\
8.92	-858.7254\\
8.93	-858.7657\\
8.94	-858.7097\\
8.95	-858.5929\\
8.96	-858.5089\\
8.97	-858.5303\\
8.98	-858.645\\
8.99	-858.7602\\
9	-858.7823\\
9.01	-858.6967\\
9.02	-858.578\\
9.03	-858.5158\\
9.04	-858.544\\
9.05	-858.6314\\
9.06	-858.713\\
9.07	-858.7279\\
9.08	-858.6716\\
9.09	-858.6059\\
9.1	-858.5934\\
9.11	-858.6258\\
9.12	-858.6411\\
9.13	-858.626\\
9.14	-858.6291\\
9.15	-858.6972\\
9.16	-858.7862\\
9.17	-858.8024\\
9.18	-858.7049\\
9.19	-858.5539\\
9.2	-858.458\\
9.21	-858.489\\
9.22	-858.6101\\
9.23	-858.7189\\
9.24	-858.741\\
9.25	-858.685\\
9.26	-858.6105\\
9.27	-858.573\\
9.28	-858.5823\\
9.29	-858.6196\\
9.3	-858.6742\\
9.31	-858.7343\\
9.32	-858.7712\\
9.33	-858.7446\\
9.34	-858.6454\\
9.35	-858.5403\\
9.36	-858.5168\\
9.37	-858.5895\\
9.38	-858.6873\\
9.39	-858.7285\\
9.4	-858.6909\\
9.41	-858.623\\
9.42	-858.5951\\
9.43	-858.6273\\
9.44	-858.6698\\
9.45	-858.6568\\
9.46	-858.5987\\
9.47	-858.5648\\
9.48	-858.6006\\
9.49	-858.6791\\
9.5	-858.7419\\
9.51	-858.7442\\
9.52	-858.6845\\
9.53	-858.6108\\
9.54	-858.5831\\
9.55	-858.6229\\
9.56	-858.6793\\
9.57	-858.6833\\
9.58	-858.6279\\
9.59	-858.5627\\
9.6	-858.5485\\
9.61	-858.596\\
9.62	-858.6897\\
9.63	-858.7734\\
9.64	-858.7889\\
9.65	-858.7199\\
9.66	-858.6114\\
9.67	-858.5356\\
9.68	-858.5382\\
9.69	-858.5967\\
9.7	-858.6554\\
9.71	-858.6844\\
9.72	-858.6926\\
9.73	-858.685\\
9.74	-858.6524\\
9.75	-858.5997\\
9.76	-858.5746\\
9.77	-858.6182\\
9.78	-858.698\\
9.79	-858.7343\\
9.8	-858.6954\\
9.81	-858.6269\\
9.82	-858.5848\\
9.83	-858.5838\\
9.84	-858.6062\\
9.85	-858.644\\
9.86	-858.6925\\
9.87	-858.7143\\
9.88	-858.6882\\
9.89	-858.6454\\
9.9	-858.6269\\
9.91	-858.6331\\
9.92	-858.6396\\
9.93	-858.633\\
9.94	-858.6228\\
9.95	-858.6258\\
9.96	-858.6413\\
9.97	-858.6578\\
9.98	-858.6533\\
9.99	-858.6194\\
};
\addlegendentry{Potential energy};

\addplot [color=mycolor3,solid]
  table[row sep=crcr]{%
0	0\\
0.01	0.5572\\
0.02	1.6722\\
0.03	2.3188\\
0.04	2.0705\\
0.05	1.3806\\
0.06	0.9903\\
0.07	1.1666\\
0.08	1.5157\\
0.09	1.5321\\
0.1	1.2029\\
0.11	0.9673\\
0.12	1.1408\\
0.13	1.5255\\
0.14	1.6792\\
0.15	1.4688\\
0.16	1.1938\\
0.17	1.1781\\
0.18	1.3969\\
0.19	1.5642\\
0.2	1.4921\\
0.21	1.2742\\
0.22	1.1466\\
0.23	1.2245\\
0.24	1.3942\\
0.25	1.4632\\
0.26	1.383\\
0.27	1.2892\\
0.28	1.3193\\
0.29	1.4399\\
0.3	1.495\\
0.31	1.4084\\
0.32	1.2767\\
0.33	1.2464\\
0.34	1.3379\\
0.35	1.4298\\
0.36	1.4075\\
0.37	1.29\\
0.38	1.1942\\
0.39	1.2057\\
0.4	1.3147\\
0.41	1.4556\\
0.42	1.5614\\
0.43	1.5796\\
0.44	1.4881\\
0.45	1.3304\\
0.46	1.202\\
0.47	1.1774\\
0.48	1.2533\\
0.49	1.3581\\
0.5	1.409\\
0.51	1.3721\\
0.52	1.2943\\
0.53	1.2733\\
0.54	1.3637\\
0.55	1.5059\\
0.56	1.5711\\
0.57	1.4895\\
0.58	1.3229\\
0.59	1.2019\\
0.6	1.1987\\
0.61	1.2709\\
0.62	1.3287\\
0.63	1.3437\\
0.64	1.365\\
0.65	1.428\\
0.66	1.4892\\
0.67	1.4764\\
0.68	1.3847\\
0.69	1.2924\\
0.7	1.2691\\
0.71	1.2968\\
0.72	1.3112\\
0.73	1.2982\\
0.74	1.3133\\
0.75	1.4002\\
0.76	1.5131\\
0.77	1.5403\\
0.78	1.4199\\
0.79	1.2324\\
0.8	1.1442\\
0.81	1.2357\\
0.82	1.4146\\
0.83	1.5171\\
0.84	1.473\\
0.85	1.3513\\
0.86	1.2572\\
0.87	1.2345\\
0.88	1.2772\\
0.89	1.3719\\
0.9	1.4824\\
0.91	1.533\\
0.92	1.4688\\
0.93	1.3305\\
0.94	1.2318\\
0.95	1.251\\
0.96	1.3572\\
0.97	1.4376\\
0.98	1.3953\\
0.99	1.2459\\
1	1.1308\\
1.01	1.1996\\
1.02	1.4435\\
1.03	1.6748\\
1.04	1.695\\
1.05	1.4804\\
1.06	1.2053\\
1.07	1.0835\\
1.08	1.1795\\
1.09	1.365\\
1.1	1.4603\\
1.11	1.4127\\
1.12	1.3211\\
1.13	1.3024\\
1.14	1.3715\\
1.15	1.4498\\
1.16	1.4578\\
1.17	1.3869\\
1.18	1.2955\\
1.19	1.2515\\
1.2	1.2829\\
1.21	1.372\\
1.22	1.4692\\
1.23	1.4984\\
1.24	1.4064\\
1.25	1.2504\\
1.26	1.186\\
1.27	1.3021\\
1.28	1.4935\\
1.29	1.5577\\
1.3	1.4167\\
1.31	1.2049\\
1.32	1.1308\\
1.33	1.2682\\
1.34	1.4848\\
1.35	1.573\\
1.36	1.4581\\
1.37	1.2704\\
1.38	1.1989\\
1.39	1.2905\\
1.4	1.4274\\
1.41	1.4838\\
1.42	1.451\\
1.43	1.407\\
1.44	1.3974\\
1.45	1.3768\\
1.46	1.2841\\
1.47	1.1608\\
1.48	1.1412\\
1.49	1.2929\\
1.5	1.5014\\
1.51	1.5754\\
1.52	1.4651\\
1.53	1.3224\\
1.54	1.3168\\
1.55	1.4307\\
1.56	1.4822\\
1.57	1.3496\\
1.58	1.1303\\
1.59	1.0554\\
1.6	1.2428\\
1.61	1.5569\\
1.62	1.7364\\
1.63	1.6438\\
1.64	1.367\\
1.65	1.1186\\
1.66	1.0592\\
1.67	1.1861\\
1.68	1.3681\\
1.69	1.4822\\
1.7	1.5096\\
1.71	1.4955\\
1.72	1.4537\\
1.73	1.3618\\
1.74	1.2428\\
1.75	1.1875\\
1.76	1.2616\\
1.77	1.4132\\
1.78	1.5076\\
1.79	1.4619\\
1.8	1.335\\
1.81	1.266\\
1.82	1.3184\\
1.83	1.415\\
1.84	1.4394\\
1.85	1.3677\\
1.86	1.2808\\
1.87	1.2628\\
1.88	1.3163\\
1.89	1.378\\
1.9	1.4031\\
1.91	1.4078\\
1.92	1.4306\\
1.93	1.4704\\
1.94	1.4721\\
1.95	1.3817\\
1.96	1.2276\\
1.97	1.1274\\
1.98	1.1753\\
1.99	1.336\\
2	1.4844\\
2.01	1.5379\\
2.02	1.5138\\
2.03	1.4631\\
2.04	1.3907\\
2.05	1.2768\\
2.06	1.1608\\
2.07	1.1467\\
2.08	1.2878\\
2.09	1.4908\\
2.1	1.5815\\
2.11	1.478\\
2.12	1.2811\\
2.13	1.1842\\
2.14	1.2881\\
2.15	1.4988\\
2.16	1.6182\\
2.17	1.5297\\
2.18	1.295\\
2.19	1.0952\\
2.2	1.0792\\
2.21	1.2484\\
2.22	1.4708\\
2.23	1.5935\\
2.24	1.5461\\
2.25	1.3828\\
2.26	1.243\\
2.27	1.2351\\
2.28	1.3357\\
2.29	1.4237\\
2.3	1.41\\
2.31	1.3238\\
2.32	1.2778\\
2.33	1.3462\\
2.34	1.473\\
2.35	1.5168\\
2.36	1.4049\\
2.37	1.2305\\
2.38	1.1692\\
2.39	1.2928\\
2.4	1.4896\\
2.41	1.5788\\
2.42	1.4856\\
2.43	1.298\\
2.44	1.1713\\
2.45	1.1863\\
2.46	1.2918\\
2.47	1.3808\\
2.48	1.4117\\
2.49	1.4297\\
2.5	1.4735\\
2.51	1.5055\\
2.52	1.4601\\
2.53	1.3433\\
2.54	1.2461\\
2.55	1.2481\\
2.56	1.3255\\
2.57	1.3773\\
2.58	1.3396\\
2.59	1.2652\\
2.6	1.2737\\
2.61	1.4107\\
2.62	1.5735\\
2.63	1.6033\\
2.64	1.4486\\
2.65	1.2252\\
2.66	1.1111\\
2.67	1.1821\\
2.68	1.3526\\
2.69	1.4757\\
2.7	1.4849\\
2.71	1.4233\\
2.72	1.3656\\
2.73	1.3494\\
2.74	1.3667\\
2.75	1.3826\\
2.76	1.3613\\
2.77	1.3072\\
2.78	1.2762\\
2.79	1.3151\\
2.8	1.3953\\
2.81	1.4349\\
2.82	1.3925\\
2.83	1.3158\\
2.84	1.2877\\
2.85	1.3416\\
2.86	1.4342\\
2.87	1.484\\
2.88	1.4416\\
2.89	1.3404\\
2.9	1.2717\\
2.91	1.295\\
2.92	1.3769\\
2.93	1.4271\\
2.94	1.3945\\
2.95	1.3151\\
2.96	1.266\\
2.97	1.2867\\
2.98	1.3517\\
2.99	1.4098\\
3	1.4353\\
3.01	1.4384\\
3.02	1.4321\\
3.03	1.4059\\
3.04	1.3422\\
3.05	1.2551\\
3.06	1.2064\\
3.07	1.2657\\
3.08	1.4317\\
3.09	1.5892\\
3.1	1.5882\\
3.11	1.3971\\
3.12	1.1591\\
3.13	1.0713\\
3.14	1.2056\\
3.15	1.4483\\
3.16	1.6062\\
3.17	1.5627\\
3.18	1.3679\\
3.19	1.2004\\
3.2	1.205\\
3.21	1.3498\\
3.22	1.472\\
3.23	1.4604\\
3.24	1.3573\\
3.25	1.2844\\
3.26	1.3077\\
3.27	1.3899\\
3.28	1.449\\
3.29	1.4331\\
3.3	1.3531\\
3.31	1.27\\
3.32	1.2476\\
3.33	1.3014\\
3.34	1.3885\\
3.35	1.4462\\
3.36	1.4356\\
3.37	1.3682\\
3.38	1.3081\\
3.39	1.3228\\
3.4	1.4073\\
3.41	1.4752\\
3.42	1.4482\\
3.43	1.3419\\
3.44	1.2407\\
3.45	1.2087\\
3.46	1.2513\\
3.47	1.3406\\
3.48	1.4407\\
3.49	1.5082\\
3.5	1.5042\\
3.51	1.4256\\
3.52	1.3231\\
3.53	1.2672\\
3.54	1.2849\\
3.55	1.3383\\
3.56	1.3678\\
3.57	1.3534\\
3.58	1.331\\
3.59	1.3497\\
3.6	1.4125\\
3.61	1.4606\\
3.62	1.43\\
3.63	1.3336\\
3.64	1.2603\\
3.65	1.2797\\
3.66	1.3701\\
3.67	1.4488\\
3.68	1.4534\\
3.69	1.3883\\
3.7	1.3111\\
3.71	1.279\\
3.72	1.3001\\
3.73	1.338\\
3.74	1.3616\\
3.75	1.3773\\
3.76	1.4043\\
3.77	1.432\\
3.78	1.4283\\
3.79	1.3893\\
3.8	1.3502\\
3.81	1.339\\
3.82	1.3465\\
3.83	1.3454\\
3.84	1.3224\\
3.85	1.2954\\
3.86	1.305\\
3.87	1.3772\\
3.88	1.4783\\
3.89	1.5247\\
3.9	1.4587\\
3.91	1.315\\
3.92	1.1977\\
3.93	1.1868\\
3.94	1.2717\\
3.95	1.3784\\
3.96	1.4473\\
3.97	1.4709\\
3.98	1.4666\\
3.99	1.4416\\
4	1.399\\
4.01	1.3565\\
4.02	1.3285\\
4.03	1.3113\\
4.04	1.3036\\
4.05	1.3114\\
4.06	1.3236\\
4.07	1.3242\\
4.08	1.3307\\
4.09	1.3669\\
4.1	1.4084\\
4.11	1.4112\\
4.12	1.3831\\
4.13	1.3785\\
4.14	1.4223\\
4.15	1.472\\
4.16	1.4578\\
4.17	1.3556\\
4.18	1.2252\\
4.19	1.1708\\
4.2	1.249\\
4.21	1.4063\\
4.22	1.521\\
4.23	1.5083\\
4.24	1.3881\\
4.25	1.2648\\
4.26	1.2336\\
4.27	1.3007\\
4.28	1.3945\\
4.29	1.4455\\
4.3	1.4367\\
4.31	1.3909\\
4.32	1.3375\\
4.33	1.3019\\
4.34	1.3117\\
4.35	1.3756\\
4.36	1.4511\\
4.37	1.4718\\
4.38	1.4145\\
4.39	1.323\\
4.4	1.2664\\
4.41	1.2794\\
4.42	1.3335\\
4.43	1.3643\\
4.44	1.3383\\
4.45	1.2987\\
4.46	1.3289\\
4.47	1.4473\\
4.48	1.5579\\
4.49	1.5428\\
4.5	1.3972\\
4.51	1.2418\\
4.52	1.2012\\
4.53	1.2843\\
4.54	1.3865\\
4.55	1.4016\\
4.56	1.3298\\
4.57	1.2763\\
4.58	1.3346\\
4.59	1.4711\\
4.6	1.5483\\
4.61	1.4751\\
4.62	1.3063\\
4.63	1.182\\
4.64	1.1963\\
4.65	1.3301\\
4.66	1.4807\\
4.67	1.5419\\
4.68	1.4782\\
4.69	1.3499\\
4.7	1.259\\
4.71	1.2583\\
4.72	1.3206\\
4.73	1.3846\\
4.74	1.4113\\
4.75	1.4046\\
4.76	1.3873\\
4.77	1.3625\\
4.78	1.3196\\
4.79	1.2798\\
4.8	1.2932\\
4.81	1.3726\\
4.82	1.4601\\
4.83	1.4787\\
4.84	1.4101\\
4.85	1.3129\\
4.86	1.2681\\
4.87	1.3051\\
4.88	1.3754\\
4.89	1.4061\\
4.9	1.378\\
4.91	1.3391\\
4.92	1.343\\
4.93	1.3907\\
4.94	1.4362\\
4.95	1.4328\\
4.96	1.3768\\
4.97	1.3189\\
4.98	1.3149\\
4.99	1.3524\\
5	1.3573\\
5.01	1.2984\\
5.02	1.2452\\
5.03	1.2804\\
5.04	1.3874\\
5.05	1.4719\\
5.06	1.4778\\
5.07	1.434\\
5.08	1.3928\\
5.09	1.3675\\
5.1	1.3428\\
5.11	1.3175\\
5.12	1.312\\
5.13	1.341\\
5.14	1.3881\\
5.15	1.4137\\
5.16	1.3913\\
5.17	1.3345\\
5.18	1.2871\\
5.19	1.2836\\
5.2	1.3217\\
5.21	1.3716\\
5.22	1.4055\\
5.23	1.4126\\
5.24	1.394\\
5.25	1.3597\\
5.26	1.3321\\
5.27	1.3394\\
5.28	1.3898\\
5.29	1.4499\\
5.3	1.4642\\
5.31	1.4098\\
5.32	1.3248\\
5.33	1.2694\\
5.34	1.2692\\
5.35	1.3053\\
5.36	1.3404\\
5.37	1.3521\\
5.38	1.3512\\
5.39	1.3663\\
5.4	1.4031\\
5.41	1.432\\
5.42	1.4225\\
5.43	1.3813\\
5.44	1.3518\\
5.45	1.3711\\
5.46	1.4219\\
5.47	1.4387\\
5.48	1.3755\\
5.49	1.2615\\
5.5	1.1818\\
5.51	1.2061\\
5.52	1.3308\\
5.53	1.4746\\
5.54	1.5384\\
5.55	1.4837\\
5.56	1.362\\
5.57	1.2645\\
5.58	1.2482\\
5.59	1.3046\\
5.6	1.3827\\
5.61	1.4299\\
5.62	1.4291\\
5.63	1.404\\
5.64	1.3838\\
5.65	1.3707\\
5.66	1.3514\\
5.67	1.3314\\
5.68	1.329\\
5.69	1.3399\\
5.7	1.3459\\
5.71	1.3455\\
5.72	1.3501\\
5.73	1.361\\
5.74	1.3699\\
5.75	1.3727\\
5.76	1.3703\\
5.77	1.3654\\
5.78	1.3662\\
5.79	1.3818\\
5.8	1.4058\\
5.81	1.4132\\
5.82	1.3817\\
5.83	1.3174\\
5.84	1.2619\\
5.85	1.262\\
5.86	1.329\\
5.87	1.4236\\
5.88	1.4767\\
5.89	1.4441\\
5.9	1.3527\\
5.91	1.2804\\
5.92	1.2848\\
5.93	1.3515\\
5.94	1.4117\\
5.95	1.4127\\
5.96	1.3712\\
5.97	1.346\\
5.98	1.3644\\
5.99	1.3906\\
6	1.3737\\
6.01	1.317\\
6.02	1.2828\\
6.03	1.3227\\
6.04	1.4126\\
6.05	1.4693\\
6.06	1.4366\\
6.07	1.3458\\
6.08	1.2834\\
6.09	1.301\\
6.1	1.3637\\
6.11	1.3946\\
6.12	1.3618\\
6.13	1.3086\\
6.14	1.2992\\
6.15	1.3476\\
6.16	1.4116\\
6.17	1.4441\\
6.18	1.4312\\
6.19	1.3912\\
6.2	1.3511\\
6.21	1.3262\\
6.22	1.3204\\
6.23	1.3339\\
6.24	1.3614\\
6.25	1.3913\\
6.26	1.4073\\
6.27	1.3911\\
6.28	1.3376\\
6.29	1.2729\\
6.3	1.246\\
6.31	1.2854\\
6.32	1.3686\\
6.33	1.4501\\
6.34	1.4957\\
6.35	1.4818\\
6.36	1.3981\\
6.37	1.2797\\
6.38	1.2103\\
6.39	1.2558\\
6.4	1.3895\\
6.41	1.5023\\
6.42	1.5081\\
6.43	1.4259\\
6.44	1.344\\
6.45	1.3157\\
6.46	1.3133\\
6.47	1.2883\\
6.48	1.2511\\
6.49	1.2593\\
6.5	1.3332\\
6.51	1.4245\\
6.52	1.4712\\
6.53	1.4572\\
6.54	1.4202\\
6.55	1.4031\\
6.56	1.3966\\
6.57	1.3551\\
6.58	1.28\\
6.59	1.2414\\
6.6	1.2889\\
6.61	1.3786\\
6.62	1.4267\\
6.63	1.4081\\
6.64	1.3645\\
6.65	1.3352\\
6.66	1.3254\\
6.67	1.3318\\
6.68	1.3568\\
6.69	1.3911\\
6.7	1.4157\\
6.71	1.4217\\
6.72	1.4113\\
6.73	1.384\\
6.74	1.3409\\
6.75	1.2967\\
6.76	1.2747\\
6.77	1.2943\\
6.78	1.3595\\
6.79	1.4401\\
6.8	1.4714\\
6.81	1.4108\\
6.82	1.2983\\
6.83	1.2282\\
6.84	1.2574\\
6.85	1.3556\\
6.86	1.4426\\
6.87	1.4644\\
6.88	1.4347\\
6.89	1.3985\\
6.9	1.3723\\
6.91	1.3433\\
6.92	1.3137\\
6.93	1.311\\
6.94	1.3447\\
6.95	1.3852\\
6.96	1.3984\\
6.97	1.3795\\
6.98	1.3486\\
6.99	1.3225\\
7	1.3065\\
7.01	1.3105\\
7.02	1.3459\\
7.03	1.3999\\
7.04	1.4338\\
7.05	1.4187\\
7.06	1.3671\\
7.07	1.3174\\
7.08	1.297\\
7.09	1.3122\\
7.1	1.3619\\
7.11	1.4326\\
7.12	1.4869\\
7.13	1.4769\\
7.14	1.3878\\
7.15	1.2699\\
7.16	1.2086\\
7.17	1.248\\
7.18	1.3558\\
7.19	1.4549\\
7.2	1.4763\\
7.21	1.4091\\
7.22	1.3167\\
7.23	1.2844\\
7.24	1.3331\\
7.25	1.3976\\
7.26	1.4057\\
7.27	1.3629\\
7.28	1.3332\\
7.29	1.3481\\
7.3	1.3738\\
7.31	1.3731\\
7.32	1.3589\\
7.33	1.3694\\
7.34	1.4116\\
7.35	1.4495\\
7.36	1.4361\\
7.37	1.3535\\
7.38	1.2417\\
7.39	1.1857\\
7.4	1.2444\\
7.41	1.3833\\
7.42	1.4951\\
7.43	1.4995\\
7.44	1.4141\\
7.45	1.3235\\
7.46	1.2922\\
7.47	1.3171\\
7.48	1.3553\\
7.49	1.3805\\
7.5	1.3999\\
7.51	1.4237\\
7.52	1.4317\\
7.53	1.3892\\
7.54	1.3014\\
7.55	1.2361\\
7.56	1.2602\\
7.57	1.36\\
7.58	1.45\\
7.59	1.4638\\
7.6	1.4151\\
7.61	1.3616\\
7.62	1.3378\\
7.63	1.3387\\
7.64	1.3496\\
7.65	1.3625\\
7.66	1.3688\\
7.67	1.3589\\
7.68	1.3359\\
7.69	1.3181\\
7.7	1.3263\\
7.71	1.3637\\
7.72	1.4104\\
7.73	1.4398\\
7.74	1.435\\
7.75	1.3937\\
7.76	1.3332\\
7.77	1.2891\\
7.78	1.2925\\
7.79	1.3383\\
7.8	1.3865\\
7.81	1.4043\\
7.82	1.3992\\
7.83	1.3993\\
7.84	1.4094\\
7.85	1.4037\\
7.86	1.3628\\
7.87	1.3046\\
7.88	1.2716\\
7.89	1.2895\\
7.9	1.3385\\
7.91	1.3754\\
7.92	1.3838\\
7.93	1.3784\\
7.94	1.3706\\
7.95	1.3611\\
7.96	1.36\\
7.97	1.384\\
7.98	1.4256\\
7.99	1.4465\\
8	1.4103\\
8.01	1.3214\\
8.02	1.2337\\
8.03	1.2181\\
8.04	1.3047\\
8.05	1.4412\\
8.06	1.5234\\
8.07	1.4915\\
8.08	1.3858\\
8.09	1.3004\\
8.1	1.2907\\
8.11	1.332\\
8.12	1.3591\\
8.13	1.3369\\
8.14	1.2918\\
8.15	1.2808\\
8.16	1.3333\\
8.17	1.4187\\
8.18	1.4761\\
8.19	1.468\\
8.2	1.4098\\
8.21	1.3514\\
8.22	1.3307\\
8.23	1.3455\\
8.24	1.3668\\
8.25	1.3694\\
8.26	1.3501\\
8.27	1.327\\
8.28	1.328\\
8.29	1.3636\\
8.3	1.4057\\
8.31	1.4087\\
8.32	1.3577\\
8.33	1.2891\\
8.34	1.2617\\
8.35	1.3073\\
8.36	1.4004\\
8.37	1.4706\\
8.38	1.4571\\
8.39	1.3728\\
8.4	1.3018\\
8.41	1.3102\\
8.42	1.3749\\
8.43	1.4235\\
8.44	1.4207\\
8.45	1.3887\\
8.46	1.3597\\
8.47	1.3355\\
8.48	1.3046\\
8.49	1.2791\\
8.5	1.2889\\
8.51	1.3436\\
8.52	1.4176\\
8.53	1.4681\\
8.54	1.461\\
8.55	1.395\\
8.56	1.3117\\
8.57	1.2709\\
8.58	1.2969\\
8.59	1.3532\\
8.6	1.3858\\
8.61	1.3817\\
8.62	1.3709\\
8.63	1.377\\
8.64	1.3854\\
8.65	1.3707\\
8.66	1.337\\
8.67	1.3199\\
8.68	1.3525\\
8.69	1.4232\\
8.7	1.4682\\
8.71	1.4316\\
8.72	1.3346\\
8.73	1.2582\\
8.74	1.2618\\
8.75	1.3311\\
8.76	1.4013\\
8.77	1.4249\\
8.78	1.4067\\
8.79	1.3747\\
8.8	1.3447\\
8.81	1.3236\\
8.82	1.3206\\
8.83	1.3389\\
8.84	1.3684\\
8.85	1.393\\
8.86	1.4014\\
8.87	1.3876\\
8.88	1.3517\\
8.89	1.3115\\
8.9	1.3052\\
8.91	1.3585\\
8.92	1.4407\\
8.93	1.4774\\
8.94	1.4236\\
8.95	1.3137\\
8.96	1.2345\\
8.97	1.2546\\
8.98	1.3635\\
8.99	1.4736\\
9	1.4944\\
9.01	1.413\\
9.02	1.2988\\
9.03	1.2371\\
9.04	1.2629\\
9.05	1.3478\\
9.06	1.4263\\
9.07	1.4408\\
9.08	1.3883\\
9.09	1.3266\\
9.1	1.3138\\
9.11	1.342\\
9.12	1.3574\\
9.13	1.3452\\
9.14	1.3533\\
9.15	1.4188\\
9.16	1.5013\\
9.17	1.5149\\
9.18	1.4203\\
9.19	1.2741\\
9.2	1.1835\\
9.21	1.2121\\
9.22	1.3264\\
9.23	1.4306\\
9.24	1.454\\
9.25	1.4016\\
9.26	1.3311\\
9.27	1.2937\\
9.28	1.3011\\
9.29	1.3389\\
9.3	1.3933\\
9.31	1.4517\\
9.32	1.4856\\
9.33	1.4568\\
9.34	1.3619\\
9.35	1.2625\\
9.36	1.2401\\
9.37	1.309\\
9.38	1.4018\\
9.39	1.4416\\
9.4	1.4071\\
9.41	1.343\\
9.42	1.3161\\
9.43	1.3461\\
9.44	1.3833\\
9.45	1.3707\\
9.46	1.3173\\
9.47	1.287\\
9.48	1.3211\\
9.49	1.3964\\
9.5	1.4562\\
9.51	1.4584\\
9.52	1.4016\\
9.53	1.3304\\
9.54	1.3046\\
9.55	1.341\\
9.56	1.3919\\
9.57	1.3965\\
9.58	1.3452\\
9.59	1.2849\\
9.6	1.2689\\
9.61	1.3165\\
9.62	1.4072\\
9.63	1.4882\\
9.64	1.5031\\
9.65	1.4363\\
9.66	1.3313\\
9.67	1.2581\\
9.68	1.2583\\
9.69	1.3135\\
9.7	1.3709\\
9.71	1.4006\\
9.72	1.4091\\
9.73	1.4014\\
9.74	1.368\\
9.75	1.3182\\
9.76	1.2971\\
9.77	1.3386\\
9.78	1.4128\\
9.79	1.4479\\
9.8	1.4122\\
9.81	1.346\\
9.82	1.3051\\
9.83	1.3031\\
9.84	1.3255\\
9.85	1.3648\\
9.86	1.4098\\
9.87	1.4292\\
9.88	1.4051\\
9.89	1.3645\\
9.9	1.3452\\
9.91	1.3502\\
9.92	1.356\\
9.93	1.35\\
9.94	1.3407\\
9.95	1.3424\\
9.96	1.3583\\
9.97	1.3748\\
9.98	1.3689\\
9.99	1.3367\\
};
\addlegendentry{Kinetic energy};

\end{axis}
\end{tikzpicture}%}
        \caption{$\Delta t =\unit[0.01]{ASU}$}
        \label{fig:timestep_b}
    \end{subfigure}
    \caption{For the different energy simulations, the same number of timesteps was used but the lengths of the different timesteps makes them evolve over different times. As can be seen in \ref{fig:timestep_a}, the energy explodes due to insufficient resolution of the time, something which is not present in \ref{fig:timestep_b}.}
    \label{fig:timestep}
\end{figure}

As we can see in figure \ref{fig:timestep} the required timestep is between $\Delta t = 0.1 \sim \unit[0.01]{ASU}$, so for the rest of the assignment a timestep of $\Delta t = \unit[0.01]{ASU}$ will be used.

\section*{Problem 3}

\begin{figure}[H]
    \centering
    \captionsetup[subfigure]{justification=centering}
    \begin{subfigure}[b]{0.40\textwidth}
        \centering
        \includegraphics[width=\textwidth]{graphics/task3/pressure.png}
    \end{subfigure}
    \begin{subfigure}[b]{0.40\textwidth}
        \centering
        \includegraphics[width=\textwidth]{graphics/task3/pressure_avg.png}
    \end{subfigure}
    \begin{subfigure}[b]{0.40\textwidth}
        \centering
        \includegraphics[width=\textwidth]{graphics/task3/temperature.png}
    \end{subfigure}
    \begin{subfigure}[b]{0.40\textwidth}
        \centering
        \includegraphics[width=\textwidth]{graphics/task3/temperature_avg.png}
    \end{subfigure}
    \begin{subfigure}[b]{0.40\textwidth}
        \centering
        \includegraphics[width=\textwidth]{graphics/task3/traj.png}
    \end{subfigure}
    \caption{After the equilibrium, both the pressure and the temperature needs some time to stabilize after rescaling the velocities, but remains stable for longer time. The equalisation temperature was set to $\unit[500]{C^\circ}$.}
    \label{fig:equilibrium500}
\end{figure}

In order to set the system to a certain temperature, a technique involving scaling all the velocities during a equilibrating state. Since the temperature depends on the kinetic energies which in turn depends on the velocities, the  temperature can thusly be changed by changing velocities. In figure \ref{fig:equilibrium500} we can see the temperature after setting the temperature to $\unit[500]{C^\circ}$. There are some fluctuations in the beginning due to the rescaling of the velocities.
 

\section*{Problem 4}

\begin{figure}[H]
    \centering
    \captionsetup[subfigure]{justification=centering}
    \begin{subfigure}[b]{0.40\textwidth}
        \centering
        \includegraphics[width=\textwidth]{graphics/task4/pressure.png}
    \end{subfigure}
    \begin{subfigure}[b]{0.40\textwidth}
        \centering
        \includegraphics[width=\textwidth]{graphics/task4/pressure_avg.png}
    \end{subfigure}
    \begin{subfigure}[b]{0.40\textwidth}
        \centering
        \includegraphics[width=\textwidth]{graphics/task4/temperature.png}
    \end{subfigure}
    \begin{subfigure}[b]{0.40\textwidth}
        \centering
        \includegraphics[width=\textwidth]{graphics/task4/temperature_avg.png}
    \end{subfigure}
    \begin{subfigure}[b]{0.40\textwidth}
        \centering
        \includegraphics[width=\textwidth]{graphics/task4/traj.png}
    \end{subfigure}
    \caption{After the equilibrium, both the pressure and the temperature needs some time to stabilize after rescaling the velocities, but remains stable for longer time. The equalisation temperature was first set to $\unit[1000]{C^\circ}$ for the smelting and then the temperature was reduced to $\unit[700]{C^\circ}$.}
    \label{fig:equilibrium500}
\end{figure}


\section*{Problem 5}

$\unit[1]{\frac{j}{mol K}}=\unit[1.0366e-05]{\frac{eV}{K}}$ \\
$C_v\left[AL\right]:\unit[24.20]{\frac{j}{mol K}}=\unit[1.0366e-05]{\frac{eV}{K}}$

\noindent From our MD simulations we obtained the following values for $C_V$ when using the potential and kinetic energy fluctuations respectively.
\begin{table}[h!]
	\centering
	\caption{Heat capacity obtained by measuring energy fluctuations.}
	\begin{tabular}{l|cc}
		\hline \textbf{Temperature} & \textbf{500$^\circ$ C} & \textbf{700$^\circ$ C} \\ \hline
		$C_V / (\unit{eV/kg\,K})$ (kinetic) & $5.819521 \cdot 10^{-2}$ & $5.056438 \cdot 10^{-2}$ \\
		$C_V / (\unit{eV/kg\,K})$ (potential) & $5.836188 \cdot 10^{-2}$ & $5.056541 \cdot 10^{-2}$ \\ \hline
	\end{tabular}
	\label{tab:prob5}
\end{table}

\section*{Problem 6}

When instead using the relation
\begin{equation}
	C_V = \left( \frac{\partial E}{\partial T} \right)_{N,V}
\end{equation}
to compute the heat capacity, and approximate it with a difference quota, we obtain the results found in the table below.

\begin{table}[h!]
	\centering	
	\caption{Heat capacity obtained by approximating the partial energy derivative with respect to the temperature.}
	\begin{tabular}{l|cc}
		\hline \textbf{Temperature} & \textbf{500$^\circ$ C} & \textbf{700$^\circ$ C} \\ \hline
		$C_V / (\unit{eV/kg\,K})$ & $6.436 \cdot 10^{-2}$ & $8.131 \cdot 10^{-2}$ \\ \hline
	\end{tabular}
	\label{tab:prob6}
\end{table}

If we compare these results to those from the previous problem we see that they are slightly larger and the result for 700 degrees deviates by a larger margin. It's possible that a longer equilibration time would yield a more stable temperature than was obtained now and therefore a more accurate heat capacity. A longer measuring time would also increase the accuracy of the result, as well as doing more simulation and averaging the different results. A $\Delta T$ of 5$^\circ$ C was used here, but further experiementing with this parameter could yield a better result as well.

\section*{Problem 7}

\begin{figure}[H]
\centering
\captionsetup[subfigure]{justification=centering}
\begin{subfigure}[b]{0.40\textwidth}
	\centering
	\includegraphics[width=\textwidth]{graphics/task7/radial.png}
\end{subfigure}
\caption{The radial function is computed by taking the histogram over all the internal distances between the atoms then divided by the random distribution of the same density.}
\label{fig:radial}
\end{figure}

\section*{Problem 8}


\begin{figure}[H]
    \centering
    \captionsetup[subfigure]{justification=centering}
    \begin{subfigure}[b]{0.40\textwidth}
        \centering
        \includegraphics[width=\textwidth]{graphics/task8/integral.png}
    \end{subfigure}
    \begin{subfigure}[b]{0.40\textwidth}
        \centering
        \includegraphics[width=\textwidth]{graphics/task8/simulated.png}
    \end{subfigure}
    \caption{The static structure function computed both from the radial distribution function in figure \ref{fig:radial} in the left and using bins in Fourier space to the right.}
    \label{fig:StaticStructure}
\end{figure}

  
  \begin{comment}
\section*{Problem 2}

In the following we give an example of how to produce a table.
Use the code for Table~\ref{tab1} as a template.

\begin{table}[!ht]
  \begin{center}
    \caption{A dummy table}
    \begin{tabular}{l|c|c}\hline\hline
      \textbf{Col.~1} & \textbf{Col.~2} & \textbf{Col.~3} \\ \hline
      the & quick & brown \\ 
      fox & jumps & over \\ 
      the & lazy  & dog \\ 
      \hline\hline
    \end{tabular}
    \label{tab1}
  \end{center}
\end{table}

\section*{Problem 3}

If you find some part of the code particularly interesting you may 
include it in the text, otherwise it should be included in the appedix.
If you do want to include code the following commands will print
the text directly, with no \LaTeX~commands executed:

\begin{lstlisting}[language=matlab]
% Hello world ten times in MATLAB
for i = 1 : 10
  fprintf('Hello world %d!\n',i);
end
\end{lstlisting}

\begin{lstlisting}[language=python]
# Hello world ten times in Python
for i in range(10):
  print 'Hello world %d!' % i
\end{lstlisting}

\section*{Problem 4}
At some point it may be appropriate to include equations. It is done in the
following way:

\begin{equation}
  V(r) = 4\epsilon \left[ \left( \frac{\sigma}{r} \right)^{12} - 
    \left(\frac{\sigma}{r} \right)^{6} \right]
\end{equation}

Do number and reference all your equations.

\section*{Concluding discussion}

Use your favourite flavor of \LaTeX{} to compile the file:
\begin{verbatim}
xelatex template.tex
pdflatex template.tex
latex template.tex
\end{verbatim}
should all work.
If you use \verb+pdflatex+ or \verb+xelatex+, included figures need to be in
\verb+pdf+, \verb+jpg+, or \verb+png+ format. If you want to include eps
figures, you can easily convert them to \verb+pdf+ using the command
\begin{verbatim}
ps2pdf -dEPSCrop figure.eps figure.pdf
\end{verbatim}

\begin{thebibliography}{69}
\bibitem{lamport94} Leslie Lamport, \emph{\LaTeX: A Document Preparation
System}. Addison Wesley, Massachusetts, 2nd Edition, 1994.
\end{thebibliography}

\newpage

\appendix

\section{Source Code}

Include all source code here in the appendix. Keep the code formatting clean,
use indentation, and comment your code to make it easy to understand. Also,
break lines that are too long. (Keep them under 80 characters!)

\subsection{Calculating pi using matlab: \texttt{pi.m}}
\lstinputlisting[language=matlab,numbers=left]{template_files/pi.m}

\subsection{Calculating pi using python: \texttt{pi.py}}
\lstinputlisting[language=python,numbers=left]{template_files/pi.py}

\subsection{Calculating pi using C: \texttt{pi.c}}
\lstinputlisting[language=c,numbers=left]{template_files/pi.c}

\end{comment}

\end{document}
