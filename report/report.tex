\input{template_files/packages}

\usepackage{tikz}
\usepackage{pgfplots}
\usepackage{tikzscale}
\usepackage{graphicx}
\usepackage{float}
\usepackage{subcaption}

\title{H1a: Molecular Dynamics simulation - static properties}
\author{Victor Nilsson and Simon Nilsson (930128-1854)}
\date{\today}

\begin{document}

\input{template_files/titlepage}

\section*{Introduction}

Molecular dynamics is the movement of atoms and molecules. Such a system can be simulated with certain properties. The interests of doing these kinds of simulations is studying e.g. the trajectories of the atoms given specific surrounding parameters such as temperature, pressure, crystal formation etc. For this homeproblem we study the dynamics of aluminium atoms in a FCC crystal lattice.

For the homeproblem we simulate the dynamics of the aluminium atoms by approximating their interactions by their pair-wise interactions. At the start of each simulation the atoms are placed on a FCC lattice with lattice parameter $\unit[4.046]{\AA}$ \cite{al_wiki}. These atoms are initially displaced about $\unit[5]{\%}$ of the lattice spacing and their velocities are set to zero.

\section*{Problem 1}

\begin{figure}[H]
    \centering
    \includegraphics[width=\textwidth]{graphics/task1/energy.png}
    \caption{For different time-steps the energy evolves differently over time, in the figure there are four simulations with different time-steps. The time-steps increases with $\unit[0.01]{ps}$ for each simulation and the total energy starts to increase for time-steps of $\unit[0.03]{ps}$.}
    \label{fig:timestep}
\end{figure}

For the simulations we update the positions and velocities of the particles by applying the velocity verlet algorithm, which is described below.

\begin{itemize}
\item $v \leftarrow v +\frac{1}{2}a\Delta t, \;$ update velocity from current acceleration with half a timestep.
\item $q \leftarrow q +v\Delta t, \;$ update displacement from current velocity.
\item $a \leftarrow a, \;$ update acceleration from the applied forces.
\item $v \leftarrow v +\frac{1}{2}a\Delta t, \;$ update velocity from new acceleration with another half-step.
\end{itemize}

For this algorithm to be stable, we need to choose a time-step that conserves the total energy. With the algorithm implemented we can simulate the system and study the time evolution of the total energies. In Fig. \ref{fig:timestep} we can see the implications of different time-steps. We can see that the energy is unstable for $\Delta t = \unit[0.03]{ps}$ and stable for the previous one at $\Delta t = \unit[0.025]{ps}$. To have some safety margin we will, for the rest of the report, use a time-step of $\Delta t = \unit[0.005]{ps}$.

\section*{Problem 3}

\begin{figure}[H]
    \centering
    \captionsetup[subfigure]{justification=centering}
    \begin{subfigure}[b]{0.40\textwidth}
        \centering
        \includegraphics[width=\textwidth]{graphics/task3/pressure.png}
    \end{subfigure}
    \begin{subfigure}[b]{0.40\textwidth}
        \centering
        \includegraphics[width=\textwidth]{graphics/task3/temperature.png}
    \end{subfigure}
    \begin{subfigure}[b]{0.40\textwidth}
        \centering
        \includegraphics[width=\textwidth]{graphics/task3/traj.png}
    \end{subfigure}
    \caption{\textit{Upper left:} The time evolution of the pressure during equilibration. The target pressure was set to $\unit[1]{atm} = \unit[6.3249e^{-7}]{eV/\AA^3}$. \textit{Upper right:} The time evolution of the temperature during equilibration. The target temperature was set to 500$^\circ$ C and we can see that it quickly reaches the target temperature. \textit{Lower:} 3D plot of trajectories for three of the aluminium particles during 100 ps after the equilibration process. As we can see they stay close to the initial position, which indicates that the system is in its solid state. Note that the scales are not the same on all axes.}
    \label{fig:equilibrium500}
\end{figure}

The third problem was about implementing routines for equilibrating the molecular dynamics system to a specified temperature, $T_{eq} = 500^\circ$ C, at pressure $P_{eq} = 1$ atm. The equilibration was implemented using scaling of the velocities and the total volume, and consquently the positions of the molecules as well. The goal was to study the temperature and pressure after the equilibration process through constant energy and volume simulation. We also plot the trajectories of a few particles to show that the system is still in a solid state (Fig. \ref{fig:equilibrium500}, lower subfigure).

The three main parameters for the equilibration are the timestep $\Delta t$ used in the velocity Verlet algorithm and the temperature and pressure relaxation times, $\tau_T$ and $\tau_P$ respectively. The timestep used was 5 fs, as described in the first problem, and the temperature relaxation time was chosen to be $\tau_T = 100 \Delta t$, i.e. choosing the $\Delta t/\tau_T$ quotient in the $\alpha_T$ calculation to be equal to 0.01.

When equilibrating the pressure the isothermic compressibility, $\kappa_T$, is used when computing $\alpha_P(t)$. The isothermic compressibility for aluminium is 0.01385 GPa$^{-1}$ \cite{knowledgedoor}, but $\tau_P$ is instead chosen in such a way that $\kappa_T$ is cancelled out. For these simulation we chose $\tau_P = 100\Delta t \kappa_T$.

In order to set the system to a certain temperature, a technique involving scaling all the velocities during a equilibrating state. Since the temperature depends on the kinetic energies which in turn depends on the velocities, the  temperature can thusly be changed by changing velocities. In figure \ref{fig:equilibrium500} we can see the temperature after setting the temperature to $\unit[500]{C^\circ}$. The average temperature and pressure post-equilibration was found to be $T_{avg}=\unit[772.07]{K}$ and $P_{avg}=1.40\cdot 10^{-4}$.

The previously mentioned scaling of the velocities is given by
\begin{equation}
\mathbf{v_i}^{new} = \alpha_T^{1/2}\mathbf{v_i}^{old},
\end{equation}
where
\begin{equation}
\alpha_T(t) = 1+\frac{\Delta t}{\tau_T}\frac{T_{eq}-T(t)}{T(t)},
\end{equation}
$T_{eq}$ being the target temperature and $T(t)$ the instantaneous temperature.

Similarily, but with small changes we scale the positions using
\begin{equation}
\alpha_P(t)=1-\kappa_T\frac{\Delta t}{\tau_P}\left[P_{eq}-P(t)\right],
\end{equation}
where $P_{eq}$ is the target pressure and $P(t)$ is the instantanoeus pressure.

The calculations of the instantaneous temperature and pressure, as well as the averages, can be found in 5.2 and 5.3 in \cite{lecnotes}.


\section*{Problem 4}

\begin{figure}[H]
    \centering
    \captionsetup[subfigure]{justification=centering}
    \begin{subfigure}[b]{0.40\textwidth}
        \centering
        \includegraphics[width=\textwidth]{graphics/task4/pressure.png}
    \end{subfigure}
    %\begin{subfigure}[b]{0.40\textwidth}
    %    \centering
    %    \includegraphics[width=\textwidth]{graphics/task4/pressure_avg.png}
    %\end{subfigure}
    \begin{subfigure}[b]{0.40\textwidth}
        \centering
        \includegraphics[width=\textwidth]{graphics/task4/temperature.png}
    \end{subfigure}
    %\begin{subfigure}[b]{0.40\textwidth}
    %    \centering
    %    \includegraphics[width=\textwidth]{graphics/task4/temperature_avg.png}
    %\end{subfigure}
    \begin{subfigure}[b]{0.40\textwidth}
        \centering
        \includegraphics[width=\textwidth]{graphics/task4/traj.png}
    \end{subfigure}
    \caption{\textit{Upper left:} The time evolution of the pressure during equilibration. The sudden bump at $\unit[30]{ps}$ is because the target temperature is changed from $\unit[1500]{^\circ C}$ to $\unit[700]{^\circ C}$, causing rapid changes in the pressure. The first equilibration at a higher temperature is to ensure that the system enters its liquid state. \textit{Upper right:} The time evolution of the temperature during equilibration. \textit{Lower:} 3D plot showing the trajectories of three particles during $\unit[100]{ps}$ after the equilibration. As we can see they do not stay close to their initial position, which indicates that the system has melted.}
    \label{fig:equilibrium700}
\end{figure}

The fourth problem was very similar to the third one, with the exception that the system was to be equilibrated to a temperature of $\unit[700]{^\circ C}$, which is higher than the melting point of aluminium ($\unit[660]{^\circ C}$ \cite{al_wiki}). Effectively, what is done is that the system is melted. The target pressure was kept at $\unit[1]{atm}$. Other parameters are also the same as in problem 3.

To ensure that the system would enter its melted state, at first, a target temperature of $\unit[1500]{^\circ C}$ was set which was then lowered to $\unit[700]{^\circ C}$. The switch can clearly be seen in both the pressure and temperature plots (upper left and right in Fig. \ref{fig:equilibrium700}).

Like in the previous problem, a few trajectories have been plotted (Fig. \ref{fig:equilibrium700}, lower). Unlike problem 3, we see that the particles do not remain close to their starting position for longer times, indicating that the system has indeed melted.

After equilibration to 700$^\circ$ C and 1 atm the average temperature obtained were approx. 714$^\circ$ C and $3.82 \cdot 10^{-4}$ eV/\r{A}$^3$. Similar to the third problem, the average pressure is much larger than intended, but we have not found a way of correcting it.

\section*{Problem 5}

\noindent One interesting thermal property of a material is the heat capacity. We can calculate it from fluctuations in either the kinetic or the potential energies, as
\begin{equation}
C_v=\frac{3Nk_B}{2}\left[1-\frac{2}{3Nk_b^2T^2}\left\langle \left(\delta \epsilon_{kin}\right)^2  \right\rangle_{NVE} \right]^{-1}
\label{eq:prob5-1}
\end{equation}
or
\begin{equation}
C_v=\frac{3Nk_B}{2}\left[1-\frac{2}{3Nk_b^2T^2}\left\langle \left(\delta \epsilon_{pot}\right)^2  \right\rangle_{NVE} \right]^{-1}.
\label{eq:prob5-2}
\end{equation}

From our MD simulations we obtained the values found in {Table \ref{tab:prob5}} for $C_V$ when using the potential and kinetic energy fluctuations respectively. The same method and parameters as in problem 3 and 4 were used.

\begin{table}[H]
	\centering
	\caption{Heat capacity obtained by measuring energy fluctuations in Problem 5.}
	\begin{tabular}{l|cc}
		\hline \textbf{Temperature} & \textbf{500$^\circ$ C} & \textbf{700$^\circ$ C} \\ \hline
		$C_V / (\unit{eV/K})$ (kinetic) & $ 6.2949 \cdot 10^{-2}$ & $ 8.2569 \cdot 10^{-2}$ \\
		$C_V / (\unit{eV/K})$ (potential) & $ 6.4307 \cdot 10^{-2}$ & $ 8.4425 \cdot 10^{-2}$ \\ \hline
	\end{tabular}
	\label{tab:prob5}
\end{table}

From experiments the heat capacity for aluminium in a solid is found to be $\unit[0.900]{J/g K}$ \cite{al_heat_solid} and for the liquid $\unit[1.18]{J/g K}$ \cite{al_heat_liquid}. For the chunk of 256 Al atoms in this simulation, this corresponds to approx. $\unit[0.0644]{eV/K}$ and $\unit[0.0844]{eV/K}$. Looking at Table \ref{tab:prob5} above we see that this aligns very well to the values obtained through Eq. \eqref{eq:prob5-1} and \eqref{eq:prob5-2} in the simulation. However, we also notice that using the potential energy fluctuations have given more accurate results. We are not sure why this seems to be the case, but we have also noted that results can fluctuate somewhat between runs which might be an explanation.

\section*{Problem 6}

Instead of using the fluctuations in the energies, the heat capacity can be calculated from 
\begin{equation}
	C_V = \left( \frac{\partial E}{\partial T} \right)_{N,V}
\end{equation}
which we calculate by approximating it with a difference quotient. To do this we first equilibrate the system temperature and pressure to the target values, i.e. 500$^\circ$ C or 700$^\circ$ C for the temperature and 1 atm for the pressure. Once this is done we equilibrate only the temperature to the target with a small difference $\Delta T$. After this second equilibration the system is simulated for some time and averages of the temperature and energy are computed. This is repeated for the same but negative $\Delta T$. Approximating the derivative with the difference quotient we obtain the results found in the table below. The parameter $\Delta T$ was chosen to be 5$^\circ$ C.

\begin{table}[h!]
	\centering	
	\caption{Heat capacity obtained by approximating the partial energy derivative with respect to the temperature.}
	\begin{tabular}{l|cc}
		\hline \textbf{Temperature} & \textbf{500$^\circ$ C} & \textbf{700$^\circ$ C} \\ \hline
		$C_V / (\unit{eV/K})$ & $6.436 \cdot 10^{-2}$ & $8.5400 \cdot 10^{-2}$ \\ \hline
	\end{tabular}
	\label{tab:prob6}
\end{table}

If we compare the result to the ones obtained in experiments and those obtained in the previous problem, we see that they are similar and that this method also seems to give good results.

\section*{Problem 7}

The radial distribution function is a measure of the radial distribution of the atoms in the crystal compared to an ideal spherical distribution. For large radius this quota should converge to one. The radial distribution function obtained can be found in figure~\ref{fig:radial}. Using Matlab we find that the first peak is at the distance 2.85 \r{A}, which corresponds to the shortest distance in a fcc structure with the unit cell length of 4.046 \r{A}. This is the distance between one of the corner atoms and a face centered atom close to that corner, which is expected.

\begin{figure}[H]
\centering
\captionsetup[subfigure]{justification=centering}
\begin{subfigure}[b]{0.40\textwidth}
	\centering
	\includegraphics[width=\textwidth]{graphics/task7/radial.png}
\end{subfigure}
\caption{The radial function is computed by taking the histogram over all the internal distances between the atoms then divided by the random distribution of the same density.}
\label{fig:radial}
\end{figure}

Normally the radial distribution would converge to unity as the radius tends to infinity as over large distances the atoms are almost randomly distributed. This is not something we can see in out figure and is not expected. This is due to that our system is limited in size and we therefore can't find any particles for all distances as we have already counted that particle in the opposite direction.

With Matlab we find that the first local minimum is localized at $r_m = 3.7797$, giving us the coordination number $I(r_m) \approx 12.928$. The expected coordination number for an fcc structure is 12 \cite{al_coordination_nbr}, but our result is slightly higher. This might be due to numerical inaccuracy in the integral summation.

\section*{Problem 8}

One way to see that the crystal is properly melted is by studying the scattering of an incoming beam. This can be studied through the reciprocal vectors, which when coincides with the incoming beam's wavevectors contributes to the structure factor \cite{kittel},

\begin{equation}
S(\mathbf{q})=\frac{1}{N}\left<|\sum_{i=1}^{N}e^{i\mathbf{q}\cdot \mathbf{r_i}(t)}|^2\right>\cite{lecnotes},
\end{equation}
where $\mathbf{q}$ are the reciprocal vectors and $r_i(t)$ is the trajectories of the particles. The reciprocal lattice is initialized on a grid with side-length $2\pi/L$, where $L$ is the cell length, so that the reciprocal lattice can be written as,
\begin{equation}
	\mathbf{q}=\frac{2\pi}{L}(n_x,n_y,n_z),
\end{equation}
where $n_x$, $n_y$ and $n_z$ are some integers. For our simulation we use $n_x, n_y, n_z \in [-40, 40]$, so the lattice points are some combination of $3$ numbers between $-40$ to $40$.

The results of this simulation can be seen in (Fig.~\ref{fig:StaticStructure}) where we only consider the length of the reciprocal vectors, not their direction, i.e. we have plotted the spherical mean.

In a crystalline solid the expression given above will only contain delta-function distribution contributions for the reciprocal vectors. But for a liquid we have a continuous contribution which depend on the radial distribution seen in Problem 7. Using this distribution we can write the form factor as
\begin{equation}
S(q)= 1+ 4\pi n \int r^2\left[g(r)-1\right]\frac{\sin{qr}}{qr}dr.
\end{equation}

The result of direct evaluation of this integral, for the same $q$ values as in the simulation, can be found in Fig.~\ref{fig:radial} (left). Only half of the maximum distance is used in the radial integral to not get boundary condition problems. This gives a very similar result to the $q$-grid evaluation as seen in Fig.~\ref{fig:StaticStructure} (right).

\begin{figure}[H]
    \centering
    \captionsetup[subfigure]{justification=centering}
    \begin{subfigure}[b]{0.40\textwidth}
        \centering
        \includegraphics[width=\textwidth]{graphics/task8/integral.png}
    \end{subfigure}
    \begin{subfigure}[b]{0.40\textwidth}
        \centering
        \includegraphics[width=\textwidth]{graphics/task8/simulated.png}
    \end{subfigure}
    \caption{The static structure function computed both from the radial distribution function in Figure \ref{fig:radial} (left) and using bins in Fourier space (right).}
    \label{fig:StaticStructure}
\end{figure}



\vfill
\bibliography{references}
\bibliographystyle{plain}

\newpage
\appendix
\section{Source code}

\subsection{\texttt{Task1/MD\_main.c}}
\lstinputlisting[language=c, numbers=left]{../code/Task1/MD_main.c}

\subsection{\texttt{Task3/MD\_main.c}}
\lstinputlisting[language=c, numbers=left]{../code/Task3/MD_main.c}

\subsection{\texttt{Task4/MD\_main.c}}
\lstinputlisting[language=c, numbers=left]{../code/Task4/MD_main.c}

\subsection{\texttt{Task5/MD\_main.c}}
\lstinputlisting[language=c, numbers=left]{../code/Task5/MD_main.c}

\subsection{\texttt{Task6/MD\_main.c}}
\lstinputlisting[language=c, numbers=left]{../code/Task6/MD_main.c}

\subsection{\texttt{Task7/MD\_main.c}}
\lstinputlisting[language=c, numbers=left]{../code/Task7/MD_main.c}
\subsection{\texttt{Task7/plot\_histogram.m}}
\lstinputlisting[language=matlab,numbers=left]{../code/Task7/plot_histogram.m}

\subsection{\texttt{Task8/MD\_main.c}}
\lstinputlisting[language=c, numbers=left]{../code/Task8/MD_main.c}
\subsection{\texttt{Task8/plot\_reciprocal.m}}
\lstinputlisting[language=matlab,numbers=left]{../code/Task8/plot_reciprocal.m}
\subsection{\texttt{Task8/plot\_integral.m}}
\lstinputlisting[language=matlab,numbers=left]{../code/Task8/plot_integral.m}
\subsection{\texttt{Task8/alpotential.m}}
\lstinputlisting[language=c, numbers=left]{../code/Task8/alpotential.c}

\end{document}
