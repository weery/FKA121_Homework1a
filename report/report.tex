\input{template_files/packages}

\usepackage{tikz}
\usepackage{pgfplots}
\usepackage{tikzscale}
\usepackage{graphicx}
\usepackage{float}
\usepackage{subcaption}

\title{H1a: Classical scattering by a central potential}
\author{Victor Nilsson and Simon Nilsson}
\date{\today}

\begin{document}

\input{template_files/titlepage}

\section*{Introduction}

Molecular dynamics is a simulation of the movement of atoms and molecules. What is of interest in such a simulations is e.g. the trajectories of the atoms given specific surrounding parameters such as temperature, pressure, crystal formation etc. For this homeproblem we study the dynamics of aluminium atoms in a FCC crystal lattice.

\begin{comment}
Already in antiquity people studied the effect of particles impinging on other
particles. Since then the art has developed \ldots\
(\emph{If you like to do so, you may take the opportunity to put the  methods
in a wider perspective here.}) Here is a random reference.\cite{lamport94}
\end{comment}

\section*{Problem 1}

\begin{figure}[H]
    \centering
    \captionsetup[subfigure]{justification=centering}
    \begin{subfigure}[b]{0.40\textwidth}
        \centering
        \resizebox{\columnwidth}{!}{% This file was created by matlab2tikz.
%
%The latest updates can be retrieved from
%  http://www.mathworks.com/matlabcentral/fileexchange/22022-matlab2tikz-matlab2tikz
%where you can also make suggestions and rate matlab2tikz.
%
\definecolor{mycolor1}{rgb}{0.00000,0.44700,0.74100}%
\definecolor{mycolor2}{rgb}{0.85000,0.32500,0.09800}%
\definecolor{mycolor3}{rgb}{0.92900,0.69400,0.12500}%
%
\begin{tikzpicture}

\begin{axis}[%
width=4.521in,
height=3.566in,
at={(0.758in,0.481in)},
scale only axis,
xmin=0,
xmax=100,
xlabel={Time / [ASU]},
ymin=-500000000,
ymax=3000000000,
ylabel={Energy / [ASU]},
axis background/.style={fill=white},
title style={font=\bfseries},
title={Awesome title},
legend style={legend cell align=left,align=left,draw=white!15!black}
]
\addplot [color=mycolor1,solid]
  table[row sep=crcr]{%
0	-857.0284\\
0.1	24095.5953\\
0.2	368482.8188\\
0.3	1345138.0751\\
0.4	2290709.7434\\
0.5	4028094.6967\\
0.6	5098138.1067\\
0.7	5828281.2084\\
0.8	6567207.687\\
0.9	7858390.8086\\
1	9842270.9262\\
1.1	12588912.0788\\
1.2	13906521.2805\\
1.3	14363649.6421\\
1.4	15127580.1113\\
1.5	16948566.4579\\
1.6	17909643.987\\
1.7	20367468.1138\\
1.8	22869417.5343\\
1.9	25193878.986\\
2	28342826.4527\\
2.1	28839450.3935\\
2.2	30331130.3247\\
2.3	32527871.7289\\
2.4	34634700.0354\\
2.5	35313003.6665\\
2.6	38613194.3993\\
2.7	48638010.5486\\
2.8	50151638.1548\\
2.9	52594947.3467\\
3	54230397.1507\\
3.1	55551364.0393\\
3.2	56353883.5097\\
3.3	56770473.197\\
3.4	57220501.7233\\
3.5	58584395.3641\\
3.6	60569094.7586\\
3.7	62692067.526\\
3.8	64324208.0889\\
3.9	65762722.5297\\
4	67438763.6942\\
4.1	69552209.7533\\
4.2	70430322.4085\\
4.3	71847176.2897\\
4.4	73325428.205\\
4.5	75173440.891\\
4.6	76403186.4331\\
4.7	78042735.9443\\
4.8	80436309.9774\\
4.9	81020332.2789\\
5	81313509.7819\\
5.1	82973294.789\\
5.2	85802275.0114\\
5.3	88124809.1011\\
5.4	89401665.4643\\
5.5	91378296.6361\\
5.6	94524164.1069\\
5.7	96460820.7844\\
5.8	99448900.6369\\
5.9	101522748.3995\\
6	102529727.8694\\
6.1	103489313.3208\\
6.2	104609660.0967\\
6.3	105365082.4452\\
6.4	105532086.5228\\
6.5	106508943.8725\\
6.6	108344488.3803\\
6.7	110121709.8413\\
6.8	111763183.3774\\
6.9	113233765.0356\\
7	116169073.1049\\
7.1	118104490.6391\\
7.2	119355631.6227\\
7.3	121436893.1461\\
7.4	125414597.4194\\
7.5	126664833.7799\\
7.6	127860903.3605\\
7.7	128494359.1918\\
7.8	131509667.0702\\
7.9	132868415.0923\\
8	135953076.5134\\
8.1	137422666.2762\\
8.2	139361879.1531\\
8.3	141303725.8262\\
8.4	142346542.104\\
8.5	142517633.7039\\
8.6	144270108.2869\\
8.7	145170602.0524\\
8.8	145451627.1261\\
8.9	147643891.664\\
9	151447749.2793\\
9.1	152875540.1467\\
9.2	153359136.4177\\
9.3	156236432.0684\\
9.4	162088141.3522\\
9.5	163921802.1429\\
9.6	165654966.0414\\
9.7	169058975.3003\\
9.8	171540103.7872\\
9.9	172046015.0507\\
10	172736806.8247\\
10.1	173685307.5643\\
10.2	175732714.4342\\
10.3	183295170.1968\\
10.4	200797840.3644\\
10.5	201756614.886\\
10.6	202525726.5692\\
10.7	203422576.5659\\
10.8	203624812.4169\\
10.9	205182760.1219\\
11	207087252.9362\\
11.1	208871185.2308\\
11.2	210006366.1458\\
11.3	210255337.3167\\
11.4	210892225.7987\\
11.5	211806782.2698\\
11.6	212371310.8326\\
11.7	211843505.3813\\
11.8	211416643.4093\\
11.9	212026648.1661\\
12	214093501.2604\\
12.1	216844292.7644\\
12.2	217815810.938\\
12.3	221439033.8285\\
12.4	225262649.954\\
12.5	225909529.3479\\
12.6	236861155.3667\\
12.7	275089737.4125\\
12.8	275934219.2462\\
12.9	277004785.0409\\
13	279092318.4577\\
13.1	280791193.7368\\
13.2	282402272.2673\\
13.3	283471815.5033\\
13.4	283847235.4979\\
13.5	284676749.7207\\
13.6	286699415.3881\\
13.7	289321199.1244\\
13.8	289394353.1631\\
13.9	288548264.4833\\
14	288333036.0154\\
14.1	290495916.6402\\
14.2	293607656.5212\\
14.3	293674332.2067\\
14.4	293411859.06\\
14.5	294030063.2744\\
14.6	295972105.0342\\
14.7	298374422.487\\
14.8	298514850.4058\\
14.9	299809775.5186\\
15	301295545.5831\\
15.1	301679837.5417\\
15.2	301860737.5009\\
15.3	303144766.5294\\
15.4	302825554.9364\\
15.5	304036624.4536\\
15.6	306718791.1735\\
15.7	308355795.3564\\
15.8	309576938.3408\\
15.9	310345230.2321\\
16	311016053.1378\\
16.1	312132047.8506\\
16.2	312627741.0367\\
16.3	311755558.0128\\
16.4	320523105.8083\\
16.5	320510118.3786\\
16.6	323311961.732\\
16.7	330599491.7075\\
16.8	332636466.8764\\
16.9	339348531.9854\\
17	341513488.8474\\
17.1	348059683.4214\\
17.2	358403752.4416\\
17.3	359870723.2124\\
17.4	360714889.1072\\
17.5	361480827.9965\\
17.6	363364939.7069\\
17.7	363760352.8321\\
17.8	364060902.6576\\
17.9	369031431.9849\\
18	381228445.3508\\
18.1	383351806.6154\\
18.2	388212224.3701\\
18.3	388988339.4032\\
18.4	391771464.5376\\
18.5	394362065.1893\\
18.6	395646513.8304\\
18.7	397945075.0721\\
18.8	398553259.5309\\
18.9	398321093.9276\\
19	400935192.1199\\
19.1	402946871.2045\\
19.2	403766205.1315\\
19.3	405453707.4409\\
19.4	407301724.8853\\
19.5	409720817.2219\\
19.6	416568717.5284\\
19.7	424105779.7432\\
19.8	425676240.9535\\
19.9	427654736.783\\
20	429356679.8849\\
20.1	431715546.8356\\
20.2	434805900.0159\\
20.3	437406645.8773\\
20.4	437919276.065\\
20.5	438375024.8556\\
20.6	438841242.6693\\
20.7	442556269.4132\\
20.8	443977557.2853\\
20.9	446017862.1335\\
21	451845473.2756\\
21.1	456739353.0559\\
21.2	456658388.8079\\
21.3	457840365.3658\\
21.4	459058450.6133\\
21.5	458423564.0334\\
21.6	457393070.5846\\
21.7	456942850.7216\\
21.8	461463687.0176\\
21.9	471061340.7374\\
22	477202160.4461\\
22.1	478705123.4672\\
22.2	488435615.8222\\
22.3	521134981.0696\\
22.4	525049019.8306\\
22.5	525474981.6925\\
22.6	525181723.6958\\
22.7	526417843.8027\\
22.8	527317635.2691\\
22.9	525803314.7795\\
23	525764500.8714\\
23.1	527336477.7655\\
23.2	529916180.0102\\
23.3	534508518.1266\\
23.4	537383594.8652\\
23.5	537774918.8335\\
23.6	539073024.5557\\
23.7	542718243.4654\\
23.8	544188970.7669\\
23.9	545942054.7526\\
24	548359291.4185\\
24.1	554984302.5747\\
24.2	570380538.2067\\
24.3	574466807.6143\\
24.4	577988845.9245\\
24.5	578011702.5423\\
24.6	578411579.2878\\
24.7	584428220.5503\\
24.8	590404924.9629\\
24.9	592073848.1824\\
25	596574298.5346\\
25.1	598317854.493\\
25.2	599923718.9772\\
25.3	602991148.2013\\
25.4	609029281.9493\\
25.5	610031740.62\\
25.6	611783496.9415\\
25.7	614356030.7938\\
25.8	619762472.5001\\
25.9	621313001.1378\\
26	626824961.0616\\
26.1	631028428.1747\\
26.2	632975536.4658\\
26.3	637317503.4272\\
26.4	636857881.1961\\
26.5	640885312.7721\\
26.6	655547155.7677\\
26.7	656108625.5827\\
26.8	660550256.6189\\
26.9	670945302.9953\\
27	672082068.7472\\
27.1	672337396.2293\\
27.2	671588972.3386\\
27.3	674636400.8144\\
27.4	689615127.006\\
27.5	690628514.6705\\
27.6	691288078.0049\\
27.7	694490135.2191\\
27.8	696457350.5842\\
27.9	698949610.3248\\
28	701081983.0886\\
28.1	701110476.7211\\
28.2	701327930.5326\\
28.3	706913668.0548\\
28.4	712393100.5709\\
28.5	713844809.6817\\
28.6	716711771.8375\\
28.7	733503045.282\\
28.8	772833419.3031\\
28.9	771801807.2403\\
29	773249530.5513\\
29.1	775772936.9942\\
29.2	775875972.0667\\
29.3	776131906.6816\\
29.4	776892489.8246\\
29.5	778718693.8248\\
29.6	782384381.4187\\
29.7	783736132.0822\\
29.8	784526552.9683\\
29.9	797055720.9875\\
30	828863502.7653\\
30.1	832206763.1577\\
30.2	836010140.576\\
30.3	839251587.7727\\
30.4	840305339.1744\\
30.5	840682339.2226\\
30.6	843107040.3349\\
30.7	843385574.657\\
30.8	850170535.9814\\
30.9	860510453.4279\\
31	864409603.9424\\
31.1	865876117.8453\\
31.2	867386709.6148\\
31.3	869203327.748\\
31.4	871660151.9729\\
31.5	872530483.7138\\
31.6	871793126.0492\\
31.7	872529825.1074\\
31.8	877423351.038\\
31.9	879501291.4968\\
32	884453700.0664\\
32.1	890333747.4335\\
32.2	892744598.2947\\
32.3	892568006.1555\\
32.4	892965184.5267\\
32.5	895343971.8646\\
32.6	897992503.3279\\
32.7	908114588.1794\\
32.8	925772853.2226\\
32.9	933854707.6963\\
33	949296104.0365\\
33.1	955990115.2917\\
33.2	980614635.9956\\
33.3	982956699.6002\\
33.4	983319565.2633\\
33.5	986299351.8147\\
33.6	992092337.6188\\
33.7	991302294.2488\\
33.8	991762740.7527\\
33.9	993054676.6995\\
34	993696786.4112\\
34.1	1001540458.5562\\
34.2	1015023931.4902\\
34.3	1015780728.1515\\
34.4	1016751732.9899\\
34.5	1019171647.4592\\
34.6	1022140667.5181\\
34.7	1022564790.1363\\
34.8	1023121233.6162\\
34.9	1024806163.2327\\
35	1028785406.0861\\
35.1	1032874669.3947\\
35.2	1037902989.3754\\
35.3	1046999064.6373\\
35.4	1064123329.8069\\
35.5	1071087316.5087\\
35.6	1075257695.2999\\
35.7	1081634225.8557\\
35.8	1081537325.4367\\
35.9	1081490625.2059\\
36	1082733629.2658\\
36.1	1085170346.9647\\
36.2	1085906372.6815\\
36.3	1086549615.8606\\
36.4	1088138541.4471\\
36.5	1088233421.2993\\
36.6	1088943423.4032\\
36.7	1089245190.3235\\
36.8	1089039622.9145\\
36.9	1091880696.7371\\
37	1095410000.5298\\
37.1	1098402579.7476\\
37.2	1104354511.817\\
37.3	1117948195.279\\
37.4	1116979530.3162\\
37.5	1123491614.956\\
37.6	1129826667.4575\\
37.7	1127242691.2529\\
37.8	1126277576.0817\\
37.9	1127917017.3167\\
38	1129778620.1831\\
38.1	1132234809.5226\\
38.2	1130428043.9069\\
38.3	1130496227.3817\\
38.4	1131174434.0119\\
38.5	1131264677.5964\\
38.6	1130486941.4946\\
38.7	1127698748.6321\\
38.8	1128346731.9458\\
38.9	1131207661.5838\\
39	1141213078.9854\\
39.1	1144948798.7074\\
39.2	1148742653.1097\\
39.3	1156374927.7691\\
39.4	1167012101.9759\\
39.5	1169185376.6521\\
39.6	1168393917.0665\\
39.7	1171038776.8607\\
39.8	1181978944.4663\\
39.9	1186827991.0393\\
40	1185727429.7074\\
40.1	1189182735.0074\\
40.2	1195278690.2982\\
40.3	1196886524.4249\\
40.4	1198648024.67\\
40.5	1192796864.775\\
40.6	1194533946.9225\\
40.7	1197305763.4108\\
40.8	1199186418.4767\\
40.9	1198366068.7345\\
41	1197631024.4088\\
41.1	1198619882.67\\
41.2	1200638965.4908\\
41.3	1200729070.0593\\
41.4	1201487189.6697\\
41.5	1203977473.9469\\
41.6	1206406568.1025\\
41.7	1210202362.2478\\
41.8	1213299184.6702\\
41.9	1212344590.2116\\
42	1215113236.2315\\
42.1	1215191825.4085\\
42.2	1216120725.3947\\
42.3	1218305612.3715\\
42.4	1219545917.3518\\
42.5	1219974041.1196\\
42.6	1225819202.712\\
42.7	1238988956.5514\\
42.8	1241912797.8186\\
42.9	1244002905.1901\\
43	1243897355.8998\\
43.1	1243362306.2283\\
43.2	1242351055.3611\\
43.3	1244611985.3436\\
43.4	1248910405.6713\\
43.5	1251826949.749\\
43.6	1256472603.5017\\
43.7	1262531021.5702\\
43.8	1266729073.6775\\
43.9	1269107209.485\\
44	1271268898.1432\\
44.1	1274033405.3749\\
44.2	1275725984.3999\\
44.3	1274935965.3726\\
44.4	1277750788.7505\\
44.5	1279049902.8241\\
44.6	1279731645.9033\\
44.7	1282409681.6605\\
44.8	1288971626.2074\\
44.9	1296112957.6328\\
45	1306311802.2639\\
45.1	1306897043.9519\\
45.2	1308305152.5801\\
45.3	1308825768.5881\\
45.4	1309664897.3785\\
45.5	1310650990.2206\\
45.6	1312144672.6092\\
45.7	1311907227.3521\\
45.8	1311878248.857\\
45.9	1312294749.7377\\
46	1314506323.6148\\
46.1	1319556658.1277\\
46.2	1316967424.7274\\
46.3	1316987108.8967\\
46.4	1315940881.7247\\
46.5	1314444362.637\\
46.6	1315499997.7259\\
46.7	1315409004.103\\
46.8	1314508259.6215\\
46.9	1312090102.5915\\
47	1311962730.6328\\
47.1	1313805624.704\\
47.2	1316683052.7597\\
47.3	1312551715.4708\\
47.4	1313508016.3639\\
47.5	1311186891.996\\
47.6	1311092266.0061\\
47.7	1311549870.5431\\
47.8	1310248024.0236\\
47.9	1312142260.9733\\
48	1314713527.7055\\
48.1	1313999488.5716\\
48.2	1314652622.6918\\
48.3	1319787940.6103\\
48.4	1321506241.1643\\
48.5	1322547862.3009\\
48.6	1325398673.2799\\
48.7	1325871192.1294\\
48.8	1326012716.1465\\
48.9	1327923115.2179\\
49	1328454457.0001\\
49.1	1331951614.497\\
49.2	1337515103.1719\\
49.3	1337928896.5122\\
49.4	1340476160.5973\\
49.5	1344285019.8106\\
49.6	1346677004.0363\\
49.7	1350677681.9634\\
49.8	1352534760.33\\
49.9	1355737396.0907\\
50	1358506912.3453\\
50.1	1361089515.5233\\
50.2	1367286016.499\\
50.3	1370310724.5288\\
50.4	1370444119.9211\\
50.5	1372131811.6979\\
50.6	1374299860.6575\\
50.7	1376566877.0535\\
50.8	1376057202.834\\
50.9	1374983351.6992\\
51	1372916186.0439\\
51.1	1372282905.2523\\
51.2	1373979585.2589\\
51.3	1376785609.1131\\
51.4	1380074831.4843\\
51.5	1381372875.1168\\
51.6	1379776062.2553\\
51.7	1380415047.6136\\
51.8	1388955414.5982\\
51.9	1393510674.645\\
52	1397852387.5043\\
52.1	1396184032.1842\\
52.2	1396140507.7996\\
52.3	1400861719.1963\\
52.4	1402922048.3966\\
52.5	1405428127.3913\\
52.6	1408371775.7299\\
52.7	1411840610.0701\\
52.8	1413871343.9289\\
52.9	1414359854.588\\
53	1414450117.1929\\
53.1	1417937604.4933\\
53.2	1420662194.9537\\
53.3	1421263226.0935\\
53.4	1423513029.7296\\
53.5	1428353715.775\\
53.6	1434627586.5977\\
53.7	1437403939.1629\\
53.8	1439344967.0444\\
53.9	1438601600.2968\\
54	1437946555.9644\\
54.1	1438516371.9432\\
54.2	1444208033.5085\\
54.3	1450451803.7732\\
54.4	1450900280.3903\\
54.5	1456747364.0809\\
54.6	1469253633.6896\\
54.7	1475079729.293\\
54.8	1476689827.7538\\
54.9	1479544382.5388\\
55	1477743310.4021\\
55.1	1476858897.7627\\
55.2	1478991578.2506\\
55.3	1480889509.7309\\
55.4	1483293000.1857\\
55.5	1486630151.5665\\
55.6	1487906523.5372\\
55.7	1490092851.8804\\
55.8	1492800231.5284\\
55.9	1496897967.4567\\
56	1503447022.234\\
56.1	1510886324.5957\\
56.2	1515183534.6352\\
56.3	1515684323.2082\\
56.4	1530421868.5792\\
56.5	1526182007.037\\
56.6	1534658709.8211\\
56.7	1535475889.8426\\
56.8	1541542492.7064\\
56.9	1547079000.1211\\
57	1548295552.814\\
57.1	1548977612.2583\\
57.2	1550517853.3667\\
57.3	1551567178.0123\\
57.4	1544415794.2953\\
57.5	1537995008.0477\\
57.6	1536248653.3148\\
57.7	1537554615.6633\\
57.8	1541162988.5994\\
57.9	1541257172.7403\\
58	1539860591.511\\
58.1	1541696881.3501\\
58.2	1540714835.7386\\
58.3	1543763790.9764\\
58.4	1546414356.4385\\
58.5	1548996991.9442\\
58.6	1548506763.7937\\
58.7	1548671008.6344\\
58.8	1551001122.2898\\
58.9	1554247429.9202\\
59	1556755475.2099\\
59.1	1558813759.8915\\
59.2	1558719716.0389\\
59.3	1557569666.0419\\
59.4	1560291380.2985\\
59.5	1563815177.649\\
59.6	1561257314.3897\\
59.7	1561727147.6682\\
59.8	1564100843.9501\\
59.9	1564587019.8964\\
60	1567084133.0583\\
60.1	1570006783.2394\\
60.2	1571853985.0856\\
60.3	1574024327.6358\\
60.4	1579612594.9361\\
60.5	1584022132.4276\\
60.6	1583149544.9859\\
60.7	1591504915.7737\\
60.8	1603202057.7673\\
60.9	1602037555.3889\\
61	1604509829.757\\
61.1	1605798508.587\\
61.2	1604865004.826\\
61.3	1604001667.8196\\
61.4	1604406063.5145\\
61.5	1606526755.6253\\
61.6	1599173580.0047\\
61.7	1608972379.3194\\
61.8	1607949752.9882\\
61.9	1609001590.2422\\
62	1611049024.0355\\
62.1	1610769991.5533\\
62.2	1618388204.7204\\
62.3	1630058140.5457\\
62.4	1634027790.8576\\
62.5	1637746198.1754\\
62.6	1640729532.5749\\
62.7	1643588402.1267\\
62.8	1648205186.1575\\
62.9	1650716427.2161\\
63	1649484764.3946\\
63.1	1647460595.6766\\
63.2	1649475669.4122\\
63.3	1650836844.2628\\
63.4	1656676435.5714\\
63.5	1652085303.9649\\
63.6	1645068372.3408\\
63.7	1641750946.1122\\
63.8	1642898536.6467\\
63.9	1644777481.4389\\
64	1643831061.1389\\
64.1	1637509100.3363\\
64.2	1639954745.7417\\
64.3	1642545817.3164\\
64.4	1643777305.6185\\
64.5	1649052047.8071\\
64.6	1648758501.8872\\
64.7	1647202533.0142\\
64.8	1646086494.2279\\
64.9	1645301065.1324\\
65	1643518394.7243\\
65.1	1646983044.4075\\
65.2	1644522953.7125\\
65.3	1640736757.7926\\
65.4	1640789416.2161\\
65.5	1651678603.7729\\
65.6	1665133849.2297\\
65.7	1665875072.9505\\
65.8	1664834175.6125\\
65.9	1666196728.8044\\
66	1673395631.6711\\
66.1	1682481219.9649\\
66.2	1685683257.3731\\
66.3	1687475995.58\\
66.4	1689294625.5502\\
66.5	1690480936.1731\\
66.6	1689785241.8723\\
66.7	1692005246.3614\\
66.8	1697801921.4298\\
66.9	1700409752.3069\\
67	1707822413.3719\\
67.1	1720961738.8101\\
67.2	1725543641.7828\\
67.3	1728367246.4533\\
67.4	1729825154.5918\\
67.5	1731921791.8515\\
67.6	1739397315.4896\\
67.7	1746004680.2339\\
67.8	1753909409.6571\\
67.9	1767317968.1157\\
68	1774404754.5251\\
68.1	1773425377.3414\\
68.2	1774115264.6113\\
68.3	1777219017.164\\
68.4	1784591530.6052\\
68.5	1792052895.1885\\
68.6	1793513549.8149\\
68.7	1795373826.1081\\
68.8	1798869428.2401\\
68.9	1804217881.6912\\
69	1805114086.6037\\
69.1	1808517665.3892\\
69.2	1810823114.0657\\
69.3	1810891588.445\\
69.4	1809285962.3625\\
69.5	1810562232.5757\\
69.6	1814092611.5244\\
69.7	1814147450.4025\\
69.8	1818905920.875\\
69.9	1823922463.6926\\
70	1830286157.5843\\
70.1	1834876575.1751\\
70.2	1833906903.7975\\
70.3	1836045605.9891\\
70.4	1840951420.4946\\
70.5	1843905307.5169\\
70.6	1841454580.6111\\
70.7	1840562427.6949\\
70.8	1841658293.4687\\
70.9	1846466521.8144\\
71	1845780592.5896\\
71.1	1844106467.8787\\
71.2	1844117895.7581\\
71.3	1846042458.6235\\
71.4	1846226978.1287\\
71.5	1845858767.6191\\
71.6	1848015224.1116\\
71.7	1848937448.1653\\
71.8	1850183424.5598\\
71.9	1850945907.3649\\
72	1852761174.5429\\
72.1	1856033024.1775\\
72.2	1854944088.7283\\
72.3	1855192689.4453\\
72.4	1855415407.363\\
72.5	1855520608.5626\\
72.6	1855334487.5862\\
72.7	1857856031.4503\\
72.8	1861959653.0525\\
72.9	1865181942.2578\\
73	1863292849.768\\
73.1	1870472196.9098\\
73.2	1885442582.0068\\
73.3	1886427694.4611\\
73.4	1889659075.0368\\
73.5	1892850018.1758\\
73.6	1888945659.4234\\
73.7	1888740997.5047\\
73.8	1891715634.3958\\
73.9	1893194068.7043\\
74	1893856538.8091\\
74.1	1892293977.8561\\
74.2	1889609832.9644\\
74.3	1890922919.3732\\
74.4	1892544500.9575\\
74.5	1897331996.7435\\
74.6	1906119244.2767\\
74.7	1908247083.6038\\
74.8	1908458736.6136\\
74.9	1910545419.3166\\
75	1917496251.0133\\
75.1	1926696380.9182\\
75.2	1930920933.6591\\
75.3	1936547014.0836\\
75.4	1940007644.0649\\
75.5	1944059548.8681\\
75.6	1943986203.3649\\
75.7	1944134788.2432\\
75.8	1946803298.5569\\
75.9	1945479553.0443\\
76	1942811686.9705\\
76.1	1943213454.2021\\
76.2	1942576773.5806\\
76.3	1942816517.6378\\
76.4	1945954979.8581\\
76.5	1950048285.5243\\
76.6	1958580937.2508\\
76.7	1966087290.7549\\
76.8	1969489753.8198\\
76.9	1969848494.7049\\
77	1966988521.8671\\
77.1	1965679292.747\\
77.2	1968865294.9065\\
77.3	1970871744.9747\\
77.4	1977831886.0646\\
77.5	1982025659.3659\\
77.6	1981616171.3528\\
77.7	1979169189.5795\\
77.8	1980806881.2203\\
77.9	1981896094.4707\\
78	1981150322.4186\\
78.1	1981800765.7552\\
78.2	1982735524.4444\\
78.3	1982754407.2451\\
78.4	1982470816.9742\\
78.5	1985451673.7237\\
78.6	1988495947.7035\\
78.7	1989918424.7671\\
78.8	1989188477.7972\\
78.9	1987948173.6701\\
79	1992454027.4466\\
79.1	1999294664.9269\\
79.2	2006035052.8997\\
79.3	2020081690.9604\\
79.4	2021551875.7604\\
79.5	2022935809.5872\\
79.6	2024699096.5097\\
79.7	2025001472.889\\
79.8	2023076182.1583\\
79.9	2030568785.8913\\
80	2041752520.8657\\
80.1	2040351509.5251\\
80.2	2042983086.797\\
80.3	2046898440.9229\\
80.4	2044500054.9103\\
80.5	2041141109.99\\
80.6	2041698806.0314\\
80.7	2044850644.2062\\
80.8	2042698908.355\\
80.9	2039896243.5281\\
81	2037658153.6513\\
81.1	2036465289.0906\\
81.2	2039541252.3019\\
81.3	2043696710.0186\\
81.4	2043276699.1607\\
81.5	2044725575.592\\
81.6	2048606675.2547\\
81.7	2049071006.3786\\
81.8	2048137872.9761\\
81.9	2043923412.1441\\
82	2042831194.9546\\
82.1	2041065895.3898\\
82.2	2040774504.834\\
82.3	2043756921.8946\\
82.4	2042570979.5684\\
82.5	2043948713.9059\\
82.6	2045755375.8652\\
82.7	2045923198.0926\\
82.8	2047161665.9772\\
82.9	2045211874.0124\\
83	2040081213.5504\\
83.1	2040636768.0139\\
83.2	2043099401.639\\
83.3	2045478915.5034\\
83.4	2048461416.1856\\
83.5	2050879071.4479\\
83.6	2052658484.967\\
83.7	2054476910.3055\\
83.8	2056008560.416\\
83.9	2058044639.0081\\
84	2057283279.7193\\
84.1	2055288367.9109\\
84.2	2055251597.2935\\
84.3	2056893600.1729\\
84.4	2058031736.4433\\
84.5	2060305776.9766\\
84.6	2060992945.7848\\
84.7	2059191712.7549\\
84.8	2060091308.7848\\
84.9	2064847179.9934\\
85	2065164733.1657\\
85.1	2066931059.1557\\
85.2	2073450170.0216\\
85.3	2075757910.9707\\
85.4	2078991518.614\\
85.5	2086538643.3015\\
85.6	2088810276.1318\\
85.7	2090695152.2586\\
85.8	2088377410.6648\\
85.9	2090870684.1265\\
86	2096158853.9069\\
86.1	2102326796.9347\\
86.2	2104127203.6353\\
86.3	2099208391.0837\\
86.4	2103703327.422\\
86.5	2105398137.6703\\
86.6	2107216620.3532\\
86.7	2108018305.8731\\
86.8	2104903671.5649\\
86.9	2103605989.7349\\
87	2108213898.1034\\
87.1	2113475070.1552\\
87.2	2114108843.513\\
87.3	2120425823.7671\\
87.4	2126059141.5792\\
87.5	2122526207.9867\\
87.6	2121213543.0533\\
87.7	2122162819.5219\\
87.8	2124868208.0462\\
87.9	2135504152.2251\\
88	2140714743.45\\
88.1	2138161348.9777\\
88.2	2138860851.7837\\
88.3	2142290040.7915\\
88.4	2138548140.5887\\
88.5	2137930170.4841\\
88.6	2142263610.7945\\
88.7	2148706922.2026\\
88.8	2153717290.4605\\
88.9	2155688957.3963\\
89	2186426103.3062\\
89.1	2255848590.8376\\
89.2	2256569837.8611\\
89.3	2248656774.0002\\
89.4	2242207301.4437\\
89.5	2244830384.9439\\
89.6	2243653057.9268\\
89.7	2239417211.3556\\
89.8	2237876498.0279\\
89.9	2237668600.4328\\
90	2238172071.8273\\
90.1	2239568659.5222\\
90.2	2241671616.4708\\
90.3	2239866302.7541\\
90.4	2239084892.589\\
90.5	2238944186.2675\\
90.6	2237865891.6053\\
90.7	2240624645.0122\\
90.8	2245822016.0347\\
90.9	2247445402.8937\\
91	2247046090.1738\\
91.1	2251381843.5307\\
91.2	2264213109.6546\\
91.3	2268958912.215\\
91.4	2274240909.022\\
91.5	2277585308.4409\\
91.6	2281812282.3187\\
91.7	2286357363.4844\\
91.8	2288588337.7288\\
91.9	2290228116.8962\\
92	2301631154.6009\\
92.1	2337272629.9255\\
92.2	2332599650.7904\\
92.3	2330639672.5383\\
92.4	2330546795.6919\\
92.5	2330820718.5566\\
92.6	2327525609.9252\\
92.7	2326339155.2578\\
92.8	2334933482.7147\\
92.9	2355444458.8729\\
93	2359491849.2238\\
93.1	2359366681.9672\\
93.2	2362181059.0864\\
93.3	2365303004.539\\
93.4	2372819520.6125\\
93.5	2379927810.3876\\
93.6	2381780271.9991\\
93.7	2385715110.9694\\
93.8	2390900603.8042\\
93.9	2392892988.2466\\
94	2389090656.8036\\
94.1	2387275756.6748\\
94.2	2385726900.281\\
94.3	2382616713.4412\\
94.4	2379724076.2682\\
94.5	2380589573.0291\\
94.6	2384809351.4148\\
94.7	2386513804.8236\\
94.8	2387316574.8983\\
94.9	2392097186.4756\\
95	2394342158.0036\\
95.1	2395661921.6573\\
95.2	2396195773.3245\\
95.3	2396488469.5997\\
95.4	2396079182.0188\\
95.5	2398295100.6444\\
95.6	2403261442.12\\
95.7	2406278970.0452\\
95.8	2410223745.2658\\
95.9	2413997949.8463\\
96	2417192988.0392\\
96.1	2420331432.9727\\
96.2	2429484546.4172\\
96.3	2438413698.6497\\
96.4	2439547707.6858\\
96.5	2438692207.4484\\
96.6	2439318171.1559\\
96.7	2442481475.1183\\
96.8	2443917127.7291\\
96.9	2447237096.3054\\
97	2450379055.586\\
97.1	2455752880.7046\\
97.2	2463421725.768\\
97.3	2468716584.0427\\
97.4	2473394218.3513\\
97.5	2475644416.1858\\
97.6	2475421301.0923\\
97.7	2479615930.9289\\
97.8	2480707742.5339\\
97.9	2479824636.866\\
98	2479124407.4117\\
98.1	2476014830.6727\\
98.2	2474358998.3288\\
98.3	2476628489.6727\\
98.4	2479423235.1316\\
98.5	2483319638.3091\\
98.6	2484557073.1462\\
98.7	2494763505.7914\\
98.8	2512202235.9907\\
98.9	2511083925.2665\\
99	2512385001.1453\\
99.1	2515175978.7071\\
99.2	2519896516.6484\\
99.3	2526145894.8657\\
99.4	2528827832.1583\\
99.5	2531273789.3124\\
99.6	2531410654.3204\\
99.7	2535016865.2887\\
99.8	2522580921.2148\\
99.9	2592106061.8661\\
};
\addlegendentry{Total energy};

\addplot [color=mycolor2,solid]
  table[row sep=crcr]{%
0	-857.0284\\
0.1	-133.2825\\
0.2	2576.5563\\
0.3	2281.3926\\
0.4	2752.7477\\
0.5	2594.4902\\
0.6	1989.902\\
0.7	1856.2758\\
0.8	2670.3686\\
0.9	2521.1948\\
1	2711.3269\\
1.1	2702.3887\\
1.2	2222.4094\\
1.3	1895.6478\\
1.4	2760.4033\\
1.5	1977.6586\\
1.6	2936.1026\\
1.7	2636.902\\
1.8	2542.5431\\
1.9	2568.4542\\
2	2642.5013\\
2.1	2101.5411\\
2.2	2482.1399\\
2.3	2690.3318\\
2.4	1887.1138\\
2.5	2160.667\\
2.6	3435.6702\\
2.7	2517.3828\\
2.8	2840.8781\\
2.9	2508.4154\\
3	2549.2923\\
3.1	1954.4631\\
3.2	1976.2074\\
3.3	2058.327\\
3.4	2419.3428\\
3.5	2583.7818\\
3.6	2497.2806\\
3.7	2727.5677\\
3.8	2018.0071\\
3.9	2338.7138\\
4	2590.1965\\
4.1	2669.299\\
4.2	2603.2727\\
4.3	2228.7105\\
4.4	2925.0473\\
4.5	2226.8022\\
4.6	2720.5287\\
4.7	2734.0307\\
4.8	2344.0141\\
4.9	2498.6158\\
5	2319.0283\\
5.1	2757.7532\\
5.2	2436.3019\\
5.3	2440.45\\
5.4	2153.1431\\
5.5	2449.4314\\
5.6	2750.0143\\
5.7	3178.0237\\
5.8	2619.5416\\
5.9	2409.1643\\
6	2168.5\\
6.1	2218.7701\\
6.2	2344.9023\\
6.3	1823.1828\\
6.4	2506.5739\\
6.5	2334.5655\\
6.6	2802.2436\\
6.7	2597.145\\
6.8	2055.2902\\
6.9	2583.3201\\
7	2877.0557\\
7.1	2302.1299\\
7.2	2337.6708\\
7.3	2812.7387\\
7.4	2670.388\\
7.5	2575.486\\
7.6	2385.416\\
7.7	2773.7285\\
7.8	1827.87\\
7.9	3008.7087\\
8	2348.938\\
8.1	2104.5198\\
8.2	2508.0973\\
8.3	2638.2275\\
8.4	2333.7847\\
8.5	2618.1319\\
8.6	2150.948\\
8.7	2612.5651\\
8.8	2398.8566\\
8.9	2567.571\\
9	2633.1612\\
9.1	2641.1032\\
9.2	2636.8076\\
9.3	3004.0463\\
9.4	2585.6883\\
9.5	2474.264\\
9.6	2527.927\\
9.7	2216.9186\\
9.8	2293.8317\\
9.9	2469.3886\\
10	2224.5888\\
10.1	2931.4828\\
10.2	2617.3429\\
10.3	2959.4956\\
10.4	2759.5183\\
10.5	2514.7843\\
10.6	2706.6276\\
10.7	2249.1643\\
10.8	2573.3541\\
10.9	2226.9087\\
11	2287.5868\\
11.1	2395.121\\
11.2	2039.3587\\
11.3	1970.8763\\
11.4	2262.551\\
11.5	1723.1027\\
11.6	2240.1882\\
11.7	2446.0537\\
11.8	1934.57\\
11.9	2481.1682\\
12	2614.7286\\
12.1	2648.0994\\
12.2	1999.6695\\
12.3	2629.0705\\
12.4	2487.4821\\
12.5	2107.9636\\
12.6	3975.0432\\
12.7	2371.9913\\
12.8	2297.2355\\
12.9	2865.0719\\
13	2540.3668\\
13.1	2292.1583\\
13.2	1926.4242\\
13.3	2061.3916\\
13.4	3025.6736\\
13.5	2762.3162\\
13.6	3001.3668\\
13.7	2578.3007\\
13.8	2296.453\\
13.9	2657.2112\\
14	2617.4945\\
14.1	2738.1555\\
14.2	2041.1622\\
14.3	2584.7228\\
14.4	3036.1019\\
14.5	2533.7327\\
14.6	2603.1436\\
14.7	2620.3555\\
14.8	2614.9338\\
14.9	2163.5346\\
15	2542.981\\
15.1	2327.7708\\
15.2	2778.9173\\
15.3	1919.5084\\
15.4	2422.4056\\
15.5	2734.8021\\
15.6	2609.2352\\
15.7	2299.46\\
15.8	2579.9673\\
15.9	2056.6533\\
16	1995.6017\\
16.1	2359.5028\\
16.2	2024.2216\\
16.3	3366.2327\\
16.4	2049.6949\\
16.5	2549.3654\\
16.6	3193.9206\\
16.7	2709.5838\\
16.8	3113.7107\\
16.9	2450.9856\\
17	2758.3732\\
17.1	3199.8095\\
17.2	2328.7952\\
17.3	2211.8868\\
17.4	2424.046\\
17.5	2641.6388\\
17.6	2420.9898\\
17.7	2622.9501\\
17.8	2370.7863\\
17.9	2708.5073\\
18	2457.8207\\
18.1	3210.1768\\
18.2	2076.5251\\
18.3	2192.1093\\
18.4	2407.7809\\
18.5	2294.0654\\
18.6	2266.4117\\
18.7	2026.6481\\
18.8	2835.6632\\
18.9	2624.0071\\
19	2551.6331\\
19.1	3050.4186\\
19.2	2393.8602\\
19.3	2655.1456\\
19.4	2300.8517\\
19.5	2314.1108\\
19.6	3012.8755\\
19.7	2285.4557\\
19.8	2460.8719\\
19.9	2307.8748\\
20	2944.2872\\
20.1	2576.067\\
20.2	2807.78\\
20.3	2473.284\\
20.4	2052.1989\\
20.5	2537.0949\\
20.6	3453.8635\\
20.7	2461.376\\
20.8	2167.3255\\
20.9	2554.4274\\
21	2610.5935\\
21.1	2131.2183\\
21.2	2325.2927\\
21.3	2434.2203\\
21.4	2548.448\\
21.5	1881.9144\\
21.6	2209.9735\\
21.7	2854.3342\\
21.8	2753.0658\\
21.9	3331.822\\
22	2530.408\\
22.1	2443.7413\\
22.2	3843.7529\\
22.3	3218.5336\\
22.4	2630.0758\\
22.5	2178.9362\\
22.6	2152.6628\\
22.7	2381.9493\\
22.8	2283.0295\\
22.9	2096.9607\\
23	2250.1247\\
23.1	2641.9042\\
23.2	3058.6302\\
23.3	2238.1462\\
23.4	2261.9824\\
23.5	2913.7582\\
23.6	2644.5575\\
23.7	2577.4434\\
23.8	2670.6911\\
23.9	2890.5759\\
24	1699.9347\\
24.1	3758.5186\\
24.2	2196.5972\\
24.3	2868.4904\\
24.4	2366.3985\\
24.5	1896.7284\\
24.6	2539.7214\\
24.7	3179.7968\\
24.8	2009.2106\\
24.9	3098.8898\\
25	2650.1076\\
25.1	2098.6224\\
25.2	2579.5121\\
25.3	2807.9432\\
25.4	2168.9133\\
25.5	2169.2697\\
25.6	2980.5486\\
25.7	2787.5689\\
25.8	1780.5103\\
25.9	3281.5761\\
26	2539.7689\\
26.1	2000.9159\\
26.2	3101.9406\\
26.3	2367.3129\\
26.4	2309.3584\\
26.5	3582.597\\
26.6	2507.676\\
26.7	2826.9261\\
26.8	3398.602\\
26.9	2220.7881\\
27	2213.8377\\
27.1	2413.2348\\
27.2	2362.8716\\
27.3	3315.7763\\
27.4	2979.1235\\
27.5	2525.5892\\
27.6	3160.6176\\
27.7	2396.0558\\
27.8	2614.7922\\
27.9	2531.9669\\
28	1854.8989\\
28.1	2166.7784\\
28.2	2079.8877\\
28.3	2781.1753\\
28.4	2543.1752\\
28.5	3097.6558\\
28.6	2195.7115\\
28.7	3668.0379\\
28.8	2645.0743\\
28.9	2330.5885\\
29	2661.8528\\
29.1	2280.7531\\
29.2	2595.5264\\
29.3	2300.8238\\
29.4	2081.2465\\
29.5	2562.2499\\
29.6	2050.1701\\
29.7	2491.0785\\
29.8	1786.5721\\
29.9	3202.259\\
30	2331.1036\\
30.1	2528.889\\
30.2	2637.9541\\
30.3	2328.2885\\
30.4	1832.7392\\
30.5	2313.6308\\
30.6	2107.2776\\
30.7	2733.0278\\
30.8	2861.8578\\
30.9	2277.4522\\
31	2510.3305\\
31.1	1996.3267\\
31.2	2331.3744\\
31.3	2464.748\\
31.4	2507.8015\\
31.5	2266.9371\\
31.6	3334.3624\\
31.7	3038.6189\\
31.8	2645.3265\\
31.9	2535.3271\\
32	2908.9916\\
32.1	2462.1889\\
32.2	2460.4297\\
32.3	1980.8614\\
32.4	2441.9125\\
32.5	2181.2269\\
32.6	2921.7252\\
32.7	3242.7542\\
32.8	1994.8382\\
32.9	3262.1558\\
33	2739.3475\\
33.1	3629.5389\\
33.2	2174.4726\\
33.3	2341.1536\\
33.4	2666.8943\\
33.5	2715.6804\\
33.6	2512.9207\\
33.7	2900.6259\\
33.8	2259.3263\\
33.9	2327.6115\\
34	2283.1563\\
34.1	2789.5721\\
34.2	2662.1887\\
34.3	2637.6767\\
34.4	2449.0674\\
34.5	2055.0811\\
34.6	2087.8138\\
34.7	3030.1046\\
34.8	2966.0863\\
34.9	2886.8971\\
35	2456.1212\\
35.1	2094.3165\\
35.2	3143.9017\\
35.3	3166.5932\\
35.4	2751.3242\\
35.5	2277.958\\
35.6	2700.7382\\
35.7	2009.9651\\
35.8	2623.8659\\
35.9	2355.1408\\
36	2104.0688\\
36.1	2817.4705\\
36.2	2211.3884\\
36.3	2519.7951\\
36.4	2522.0382\\
36.5	2138.5962\\
36.6	2348.6256\\
36.7	2200.5045\\
36.8	2630.558\\
36.9	1973.4607\\
37	2678.2087\\
37.1	2467.9751\\
37.2	2768.6769\\
37.3	2415.257\\
37.4	2780.4158\\
37.5	3243.9821\\
37.6	2437.0262\\
37.7	2387.0437\\
37.8	2810.8749\\
37.9	2020.057\\
38	2578.655\\
38.1	2125.5747\\
38.2	2495.2278\\
38.3	2413.9006\\
38.4	2607.4702\\
38.5	2287.5415\\
38.6	1975.506\\
38.7	2680.2338\\
38.8	2263.0992\\
38.9	2655.0753\\
39	2279.6457\\
39.1	2789.2465\\
39.2	3421.2422\\
39.3	3074.3496\\
39.4	2430.182\\
39.5	2592.6556\\
39.6	2429.713\\
39.7	2895.8417\\
39.8	2737.2421\\
39.9	1589.2373\\
40	2482.5476\\
40.1	2895.2259\\
40.2	2464.0945\\
40.3	2201.7066\\
40.4	2150.9709\\
40.5	3537.7584\\
40.6	2815.6817\\
40.7	2103.5072\\
40.8	2400.5215\\
40.9	2465.6406\\
41	2416.4203\\
41.1	2344.7888\\
41.2	2382.7903\\
41.3	2215.7876\\
41.4	2765.9183\\
41.5	2688.8625\\
41.6	2413.697\\
41.7	2355.7802\\
41.8	2678.673\\
41.9	2200.7516\\
42	2554.1944\\
42.1	2523.5657\\
42.2	2020.9564\\
42.3	2592.522\\
42.4	2251.1181\\
42.5	2612.1299\\
42.6	3105.3396\\
42.7	2611.3589\\
42.8	2052.1451\\
42.9	2540.1117\\
43	2763.7475\\
43.1	2288.9629\\
43.2	2851.5808\\
43.3	2520.7487\\
43.4	1961.1845\\
43.5	2380.1866\\
43.6	2214.5901\\
43.7	2325.7981\\
43.8	2367.0751\\
43.9	2231.1557\\
44	2376.0235\\
44.1	2671.7417\\
44.2	2518.1735\\
44.3	2926.4697\\
44.4	2576.2257\\
44.5	1993.9101\\
44.6	1968.2661\\
44.7	2708.8203\\
44.8	3138.2336\\
44.9	3941.3331\\
45	2465.771\\
45.1	2687.167\\
45.2	2602.7883\\
45.3	2679.0511\\
45.4	2487.8144\\
45.5	1817.3732\\
45.6	2052.4777\\
45.7	2491.3613\\
45.8	2505.8286\\
45.9	2191.8782\\
46	2897.2656\\
46.1	2300.7183\\
46.2	3183.8157\\
46.3	2468.6303\\
46.4	2148.2166\\
46.5	1817.7402\\
46.6	3294.4993\\
46.7	2372.8293\\
46.8	2440.6957\\
46.9	2935.7665\\
47	2310.9111\\
47.1	2534.6578\\
47.2	1949.2985\\
47.3	3240.4871\\
47.4	3031.8422\\
47.5	2170.4483\\
47.6	2930.5387\\
47.7	2420.3813\\
47.8	2830.6622\\
47.9	2385.4327\\
48	2821.2394\\
48.1	2383.333\\
48.2	2591.54\\
48.3	2342.5742\\
48.4	2985.714\\
48.5	3058.1146\\
48.6	2294.2754\\
48.7	1810.996\\
48.8	2210.2184\\
48.9	2121.4795\\
49	2192.8395\\
49.1	2584.4996\\
49.2	2558.0822\\
49.3	2387.7422\\
49.4	2454.5525\\
49.5	2631.102\\
49.6	2543.2736\\
49.7	2389.8003\\
49.8	2655.0379\\
49.9	2612.024\\
50	2478.518\\
50.1	2462.9177\\
50.2	2445.3504\\
50.3	2364.4666\\
50.4	2377.2133\\
50.5	2410.1174\\
50.6	2357.038\\
50.7	2723.1809\\
50.8	2893.4324\\
50.9	2722.3699\\
51	2146.3349\\
51.1	2435.4897\\
51.2	2604.2993\\
51.3	2878.3523\\
51.4	2325.4999\\
51.5	2334.7038\\
51.6	2260.2874\\
51.7	3202.3521\\
51.8	1785.9158\\
51.9	2524.8514\\
52	2291.307\\
52.1	2218.8778\\
52.2	2877.3565\\
52.3	2490.7283\\
52.4	2957.1496\\
52.5	2502.509\\
52.6	2149.8101\\
52.7	2386.0009\\
52.8	2935.464\\
52.9	2559.3515\\
53	2456.1696\\
53.1	2828.6588\\
53.2	2551.8906\\
53.3	2405.4075\\
53.4	2652.5786\\
53.5	2590.2757\\
53.6	2487.7755\\
53.7	2679.4275\\
53.8	2350.6985\\
53.9	2110.741\\
54	2257.5828\\
54.1	2080.9061\\
54.2	3297.3998\\
54.3	2691.7067\\
54.4	2388.9615\\
54.5	3174.2475\\
54.6	2731.3686\\
54.7	2516.5259\\
54.8	3289.4767\\
54.9	1924.9695\\
55	2766.1375\\
55.1	2131.2849\\
55.2	2254.7825\\
55.3	2547.8567\\
55.4	2563.9244\\
55.5	2761.6474\\
55.6	2205.034\\
55.7	2019.7878\\
55.8	2448.9277\\
55.9	2410.4524\\
56	2930.6713\\
56.1	2402.1295\\
56.2	2603.007\\
56.3	3683.3472\\
56.4	2742.6557\\
56.5	3293.4504\\
56.6	2107.8229\\
56.7	2418.8238\\
56.8	2568.8438\\
56.9	2389.0724\\
57	2699.8523\\
57.1	2194.8103\\
57.2	2258.3374\\
57.3	1757.7264\\
57.4	2649.1758\\
57.5	2197.9996\\
57.6	2466.6136\\
57.7	2402.3114\\
57.8	2554.692\\
57.9	2399.8411\\
58	2570.0922\\
58.1	2043.914\\
58.2	2631.2293\\
58.3	2437.7992\\
58.4	3088.4263\\
58.5	2197.4703\\
58.6	1928.0488\\
58.7	2975.3599\\
58.8	2825.9231\\
58.9	3053.4451\\
59	3021.8612\\
59.1	2180.2691\\
59.2	2155.0538\\
59.3	2595.9099\\
59.4	2782.1033\\
59.5	2481.921\\
59.6	2827.9401\\
59.7	3059.0398\\
59.8	2318.9975\\
59.9	2049.8757\\
60	2724.8635\\
60.1	2438.8575\\
60.2	2510.2091\\
60.3	1828.3558\\
60.4	2640.5143\\
60.5	2444.0667\\
60.6	2382.4147\\
60.7	3321.8875\\
60.8	2306.2345\\
60.9	2628.4167\\
61	2388.0738\\
61.1	2394.0151\\
61.2	2600.5308\\
61.3	2080.2166\\
61.4	2965.4227\\
61.5	1697.5533\\
61.6	3374.9548\\
61.7	2456.3509\\
61.8	2448.2461\\
61.9	2051.8876\\
62	2608.3455\\
62.1	2721.5791\\
62.2	2695.3866\\
62.3	2180.7499\\
62.4	2224.5091\\
62.5	2401.6036\\
62.6	2028.4444\\
62.7	2582.7977\\
62.8	2278.8929\\
62.9	2406.5087\\
63	2291.317\\
63.1	2353.7265\\
63.2	2764.4142\\
63.3	3137.0303\\
63.4	2826.8054\\
63.5	2811.7778\\
63.6	2148.0678\\
63.7	1991.1776\\
63.8	1988.2709\\
63.9	2690.0668\\
64	2763.9714\\
64.1	3238.8348\\
64.2	2321.2339\\
64.3	2410.4853\\
64.4	3462.1643\\
64.5	2670.5681\\
64.6	3061.072\\
64.7	2321.4765\\
64.8	1889.9755\\
64.9	2736.8043\\
65	3093.3643\\
65.1	2310.7775\\
65.2	2622.0662\\
65.3	2847.3149\\
65.4	1963.753\\
65.5	2602.6132\\
65.6	1995.036\\
65.7	2582.7904\\
65.8	2724.8618\\
65.9	2190.0175\\
66	2546.007\\
66.1	2203.8672\\
66.2	1951.2329\\
66.3	2873.1848\\
66.4	2303.3274\\
66.5	2249.6535\\
66.6	2893.7605\\
66.7	2518.1425\\
66.8	2535.134\\
66.9	2607.0622\\
67	2502.1677\\
67.1	2571.0766\\
67.2	2167.2162\\
67.3	2350.862\\
67.4	2683.8918\\
67.5	2352.9227\\
67.6	3083.3793\\
67.7	2635.6641\\
67.8	2948.4259\\
67.9	3395.0968\\
68	2602.7572\\
68.1	2772.983\\
68.2	2634.0047\\
68.3	2344.0692\\
68.4	2517.6485\\
68.5	2300.7152\\
68.6	2222.8429\\
68.7	2525.0562\\
68.8	2372.2091\\
68.9	2411.2377\\
69	2120.7751\\
69.1	2660.858\\
69.2	2776.3099\\
69.3	2295.9831\\
69.4	2255.7594\\
69.5	2583.6528\\
69.6	2194.2958\\
69.7	3174.8569\\
69.8	2648.5834\\
69.9	3066.6867\\
70	3153.5028\\
70.1	2478.3617\\
70.2	2808.953\\
70.3	2604.3209\\
70.4	2519.4357\\
70.5	2733.6514\\
70.6	2707.3863\\
70.7	2452.7867\\
70.8	2912.3457\\
70.9	2727.0136\\
71	2784.6839\\
71.1	2103.6782\\
71.2	2759.893\\
71.3	2350.2806\\
71.4	2452.8131\\
71.5	2355.8326\\
71.6	2653.1457\\
71.7	2475.7623\\
71.8	2281.2045\\
71.9	2247.8534\\
72	2120.7296\\
72.1	2230.4123\\
72.2	2771.8493\\
72.3	2564.6939\\
72.4	2610.1254\\
72.5	2355.8365\\
72.6	2596.6458\\
72.7	2196.026\\
72.8	3309.9729\\
72.9	2544.3677\\
73	2569.7705\\
73.1	3500.1095\\
73.2	2449.279\\
73.3	2374.4726\\
73.4	3180.3319\\
73.5	1952.5744\\
73.6	2240.5548\\
73.7	2465.6403\\
73.8	2494.9027\\
73.9	2992.8309\\
74	2459.5598\\
74.1	2673.7318\\
74.2	2591.1902\\
74.3	2231.9743\\
74.4	2742.8586\\
74.5	2664.2062\\
74.6	2556.705\\
74.7	2000.2335\\
74.8	2388.2466\\
74.9	2123.7475\\
75	2860.0508\\
75.1	2523.5709\\
75.2	2462.1591\\
75.3	2261.5001\\
75.4	2311.9977\\
75.5	2530.8879\\
75.6	2465.1132\\
75.7	1960.6278\\
75.8	2274.555\\
75.9	2843.467\\
76	2296.0021\\
76.1	2511.6882\\
76.2	2971.1393\\
76.3	2397.3642\\
76.4	2796.5365\\
76.5	2745.287\\
76.6	2646.0156\\
76.7	2389.9862\\
76.8	2396.617\\
76.9	2666.6877\\
77	2127.6916\\
77.1	2374.2071\\
77.2	2383.4823\\
77.3	3060.4473\\
77.4	2491.1263\\
77.5	2511.5227\\
77.6	1885.2009\\
77.7	2350.8787\\
77.8	2131.5917\\
77.9	2227.6084\\
78	2305.0244\\
78.1	2222.7597\\
78.2	2083.1019\\
78.3	2150.4954\\
78.4	2431.0946\\
78.5	2207.3195\\
78.6	2167.1117\\
78.7	2141.3549\\
78.8	2269.1265\\
78.9	2079.5591\\
79	2795.3816\\
79.1	2015.3326\\
79.2	3037.252\\
79.3	2500.6269\\
79.4	2327.2603\\
79.5	2057.9611\\
79.6	1904.4214\\
79.7	2720.8584\\
79.8	2628.9312\\
79.9	2843.3942\\
80	2175.5364\\
80.1	2285.1624\\
80.2	2836.9847\\
80.3	2556.2064\\
80.4	2454.2943\\
80.5	2235.4612\\
80.6	2799.2866\\
80.7	2230.3975\\
80.8	2561.04\\
80.9	2009.2339\\
81	2400.7348\\
81.1	2612.7831\\
81.2	2465.8285\\
81.3	2820.7192\\
81.4	2163.9828\\
81.5	2504.4184\\
81.6	2684.1497\\
81.7	2335.6844\\
81.8	2138.8463\\
81.9	3099.7564\\
82	2331.3671\\
82.1	2642.9306\\
82.2	2403.6135\\
82.3	2436.0978\\
82.4	2579.0581\\
82.5	2368.2935\\
82.6	2617.8532\\
82.7	1961.6099\\
82.8	2604.2963\\
82.9	2319.5\\
83	2403.643\\
83.1	2278.7727\\
83.2	2284.5214\\
83.3	2536.8937\\
83.4	2873.678\\
83.5	2207.1463\\
83.6	2640.7316\\
83.7	2067.7051\\
83.8	2582.9467\\
83.9	2608.5348\\
84	2532.0562\\
84.1	2654.8971\\
84.2	2788.7378\\
84.3	2544.2547\\
84.4	2295.7431\\
84.5	2793.7958\\
84.6	2560.8413\\
84.7	2582.3933\\
84.8	2934.3363\\
84.9	2469.2885\\
85	2858.9707\\
85.1	2611.7695\\
85.2	2090.1464\\
85.3	2186.5731\\
85.4	2687.1572\\
85.5	2364.6639\\
85.6	2338.8672\\
85.7	2346.337\\
85.8	2890.1017\\
85.9	2387.4615\\
86	2532.2254\\
86.1	2476.4176\\
86.2	2737.0368\\
86.3	3255.4977\\
86.4	2388.8475\\
86.5	2203.3506\\
86.6	2397.7799\\
86.7	2769.6747\\
86.8	2406.4247\\
86.9	2284.4584\\
87	2538.0418\\
87.1	2586.4214\\
87.2	2739.8083\\
87.3	3152.2386\\
87.4	3407.5024\\
87.5	2957.0868\\
87.6	2576.6446\\
87.7	2387.4997\\
87.8	3171.5802\\
87.9	2557.0033\\
88	2023.8203\\
88.1	2113.5004\\
88.2	2009.4529\\
88.3	2358.6432\\
88.4	2352.8598\\
88.5	2383.0657\\
88.6	2001.8521\\
88.7	2291.4812\\
88.8	2389.3471\\
88.9	2444.6248\\
89	3959.8108\\
89.1	2409.1892\\
89.2	2571.8174\\
89.3	2728.402\\
89.4	1892.2453\\
89.5	2702.3309\\
89.6	2261.1952\\
89.7	2498.0326\\
89.8	2140.732\\
89.9	2122.5929\\
90	2204.8992\\
90.1	2821.4051\\
90.2	2526.152\\
90.3	2894.44\\
90.4	2105.0313\\
90.5	2670.7248\\
90.6	1965.7262\\
90.7	2238.0228\\
90.8	2398.5499\\
90.9	2180.0784\\
91	2126.4253\\
91.1	3285.8705\\
91.2	2562.5308\\
91.3	2472.9794\\
91.4	2199.2799\\
91.5	2526.8236\\
91.6	2375.1838\\
91.7	2350.1536\\
91.8	2556.9812\\
91.9	1980.2704\\
92	4719.1901\\
92.1	2933.4546\\
92.2	2165.3249\\
92.3	2788.6201\\
92.4	2704.2438\\
92.5	2280.663\\
92.6	1884.1424\\
92.7	2191.6387\\
92.8	3039.0155\\
92.9	2258.0906\\
93	1946.6326\\
93.1	3049.0742\\
93.2	2505.2489\\
93.3	3072.9934\\
93.4	2741.8917\\
93.5	2350.8586\\
93.6	2315.5244\\
93.7	2129.332\\
93.8	3099.5152\\
93.9	2505.4613\\
94	2358.0177\\
94.1	2331.7999\\
94.2	2608.5162\\
94.3	2631.5408\\
94.4	2340.998\\
94.5	2243.2255\\
94.6	2442.521\\
94.7	2240.8592\\
94.8	2562.0628\\
94.9	2362.1148\\
95	2185.9575\\
95.1	2273.7198\\
95.2	2248.013\\
95.3	1989.6915\\
95.4	2666.9476\\
95.5	2942.4125\\
95.6	2401.8932\\
95.7	2218.7525\\
95.8	2131.2067\\
95.9	2705.0061\\
96	1675.5385\\
96.1	2143.0542\\
96.2	2477.2473\\
96.3	2230.6587\\
96.4	2611.1197\\
96.5	2234.3516\\
96.6	2578.2167\\
96.7	2560.2412\\
96.8	2582.6303\\
96.9	2509.2163\\
97	2391.6994\\
97.1	2574.0616\\
97.2	2078.7788\\
97.3	2740.162\\
97.4	2387.4407\\
97.5	1916.9423\\
97.6	2312.0046\\
97.7	2604.3082\\
97.8	2548.0345\\
97.9	2606.3608\\
98	2369.6807\\
98.1	2548.9752\\
98.2	2101.6394\\
98.3	2181.5208\\
98.4	2434.3797\\
98.5	2169.6727\\
98.6	2211.3396\\
98.7	3196.6253\\
98.8	2836.2252\\
98.9	2623.4341\\
99	2377.8663\\
99.1	2576.0013\\
99.2	2881.9668\\
99.3	2475.5186\\
99.4	2484.8205\\
99.5	2424.5316\\
99.6	3089.3291\\
99.7	2418.2766\\
99.8	4694.522\\
99.9	3373.073\\
};
\addlegendentry{Potential energy};

\addplot [color=mycolor3,solid]
  table[row sep=crcr]{%
0	0\\
0.1	24228.8778\\
0.2	365906.2625\\
0.3	1342856.6825\\
0.4	2287956.9957\\
0.5	4025500.2065\\
0.6	5096148.2047\\
0.7	5826424.9326\\
0.8	6564537.3184\\
0.9	7855869.6138\\
1	9839559.5993\\
1.1	12586209.6901\\
1.2	13904298.8711\\
1.3	14361753.9943\\
1.4	15124819.708\\
1.5	16946588.7993\\
1.6	17906707.8844\\
1.7	20364831.2118\\
1.8	22866874.9912\\
1.9	25191310.5318\\
2	28340183.9514\\
2.1	28837348.8524\\
2.2	30328648.1848\\
2.3	32525181.3971\\
2.4	34632812.9216\\
2.5	35310842.9995\\
2.6	38609758.7291\\
2.7	48635493.1658\\
2.8	50148797.2767\\
2.9	52592438.9313\\
3	54227847.8584\\
3.1	55549409.5762\\
3.2	56351907.3023\\
3.3	56768414.87\\
3.4	57218082.3805\\
3.5	58581811.5823\\
3.6	60566597.478\\
3.7	62689339.9583\\
3.8	64322190.0818\\
3.9	65760383.8159\\
4	67436173.4977\\
4.1	69549540.4543\\
4.2	70427719.1358\\
4.3	71844947.5792\\
4.4	73322503.1577\\
4.5	75171214.0888\\
4.6	76400465.9044\\
4.7	78040001.9136\\
4.8	80433965.9633\\
4.9	81017833.6631\\
5	81311190.7536\\
5.1	82970537.0358\\
5.2	85799838.7095\\
5.3	88122368.6511\\
5.4	89399512.3212\\
5.5	91375847.2047\\
5.6	94521414.0926\\
5.7	96457642.7607\\
5.8	99446281.0953\\
5.9	101520339.2352\\
6	102527559.3694\\
6.1	103487094.5507\\
6.2	104607315.1944\\
6.3	105363259.2624\\
6.4	105529579.9489\\
6.5	106506609.307\\
6.6	108341686.1367\\
6.7	110119112.6963\\
6.8	111761128.0872\\
6.9	113231181.7155\\
7	116166196.0492\\
7.1	118102188.5092\\
7.2	119353293.9519\\
7.3	121434080.4074\\
7.4	125411927.0314\\
7.5	126662258.2939\\
7.6	127858517.9445\\
7.7	128491585.4633\\
7.8	131507839.2002\\
7.9	132865406.3836\\
8	135950727.5754\\
8.1	137420561.7564\\
8.2	139359371.0558\\
8.3	141301087.5987\\
8.4	142344208.3193\\
8.5	142515015.572\\
8.6	144267957.3389\\
8.7	145167989.4873\\
8.8	145449228.2695\\
8.9	147641324.093\\
9	151445116.1181\\
9.1	152872899.0435\\
9.2	153356499.6101\\
9.3	156233428.0221\\
9.4	162085555.6639\\
9.5	163919327.8789\\
9.6	165652438.1144\\
9.7	169056758.3817\\
9.8	171537809.9555\\
9.9	172043545.6621\\
10	172734582.2359\\
10.1	173682376.0815\\
10.2	175730097.0913\\
10.3	183292210.7012\\
10.4	200795080.8461\\
10.5	201754100.1017\\
10.6	202523019.9416\\
10.7	203420327.4016\\
10.8	203622239.0628\\
10.9	205180533.2132\\
11	207084965.3494\\
11.1	208868790.1098\\
11.2	210004326.7871\\
11.3	210253366.4404\\
11.4	210889963.2477\\
11.5	211805059.1671\\
11.6	212369070.6444\\
11.7	211841059.3276\\
11.8	211414708.8393\\
11.9	212024166.9979\\
12	214090886.5318\\
12.1	216841644.665\\
12.2	217813811.2685\\
12.3	221436404.758\\
12.4	225260162.4719\\
12.5	225907421.3843\\
12.6	236857180.3235\\
12.7	275087365.4212\\
12.8	275931922.0107\\
12.9	277001919.969\\
13	279089778.0909\\
13.1	280788901.5785\\
13.2	282400345.8431\\
13.3	283469754.1117\\
13.4	283844209.8243\\
13.5	284673987.4045\\
13.6	286696414.0213\\
13.7	289318620.8237\\
13.8	289392056.7101\\
13.9	288545607.2721\\
14	288330418.5209\\
14.1	290493178.4847\\
14.2	293605615.359\\
14.3	293671747.4839\\
14.4	293408822.9581\\
14.5	294027529.5417\\
14.6	295969501.8906\\
14.7	298371802.1315\\
14.8	298512235.472\\
14.9	299807611.984\\
15	301293002.6021\\
15.1	301677509.7709\\
15.2	301857958.5836\\
15.3	303142847.021\\
15.4	302823132.5308\\
15.5	304033889.6515\\
15.6	306716181.9383\\
15.7	308353495.8964\\
15.8	309574358.3735\\
15.9	310343173.5788\\
16	311014057.5361\\
16.1	312129688.3478\\
16.2	312625716.8151\\
16.3	311752191.7801\\
16.4	320521056.1134\\
16.5	320507569.0132\\
16.6	323308767.8114\\
16.7	330596782.1237\\
16.8	332633353.1657\\
16.9	339346080.9998\\
17	341510730.4742\\
17.1	348056483.6119\\
17.2	358401423.6464\\
17.3	359868511.3256\\
17.4	360712465.0612\\
17.5	361478186.3577\\
17.6	363362518.7171\\
17.7	363757729.882\\
17.8	364058531.8713\\
17.9	369028723.4776\\
18	381225987.5301\\
18.1	383348596.4386\\
18.2	388210147.845\\
18.3	388986147.2939\\
18.4	391769056.7567\\
18.5	394359771.1239\\
18.6	395644247.4187\\
18.7	397943048.424\\
18.8	398550423.8677\\
18.9	398318469.9205\\
19	400932640.4868\\
19.1	402943820.7859\\
19.2	403763811.2713\\
19.3	405451052.2953\\
19.4	407299424.0336\\
19.5	409718503.1111\\
19.6	416565704.6529\\
19.7	424103494.2875\\
19.8	425673780.0816\\
19.9	427652428.9082\\
20	429353735.5977\\
20.1	431712970.7686\\
20.2	434803092.2359\\
20.3	437404172.5933\\
20.4	437917223.8661\\
20.5	438372487.7607\\
20.6	438837788.8058\\
20.7	442553808.0372\\
20.8	443975389.9598\\
20.9	446015307.7061\\
21	451842862.6821\\
21.1	456737221.8376\\
21.2	456656063.5152\\
21.3	457837931.1455\\
21.4	459055902.1653\\
21.5	458421682.119\\
21.6	457390860.6111\\
21.7	456939996.3874\\
21.8	461460933.9518\\
21.9	471058008.9154\\
22	477199630.0381\\
22.1	478702679.7259\\
22.2	488431772.0693\\
22.3	521131762.536\\
22.4	525046389.7548\\
22.5	525472802.7563\\
22.6	525179571.033\\
22.7	526415461.8534\\
22.8	527315352.2396\\
22.9	525801217.8188\\
23	525762250.7467\\
23.1	527333835.8613\\
23.2	529913121.38\\
23.3	534506279.9804\\
23.4	537381332.8828\\
23.5	537772005.0753\\
23.6	539070379.9982\\
23.7	542715666.022\\
23.8	544186300.0758\\
23.9	545939164.1767\\
24	548357591.4838\\
24.1	554980544.0561\\
24.2	570378341.6095\\
24.3	574463939.1239\\
24.4	577986479.526\\
24.5	578009805.8139\\
24.6	578409039.5664\\
24.7	584425040.7535\\
24.8	590402915.7523\\
24.9	592070749.2926\\
25	596571648.427\\
25.1	598315755.8706\\
25.2	599921139.4651\\
25.3	602988340.2581\\
25.4	609027113.036\\
25.5	610029571.3503\\
25.6	611780516.3929\\
25.7	614353243.2249\\
25.8	619760691.9898\\
25.9	621309719.5617\\
26	626822421.2927\\
26.1	631026427.2588\\
26.2	632972434.5252\\
26.3	637315136.1143\\
26.4	636855571.8377\\
26.5	640881730.1751\\
26.6	655544648.0917\\
26.7	656105798.6566\\
26.8	660546858.0169\\
26.9	670943082.2072\\
27	672079854.9095\\
27.1	672334982.9945\\
27.2	671586609.467\\
27.3	674633085.0381\\
27.4	689612147.8825\\
27.5	690625989.0813\\
27.6	691284917.3873\\
27.7	694487739.1633\\
27.8	696454735.792\\
27.9	698947078.3579\\
28	701080128.1897\\
28.1	701108309.9427\\
28.2	701325850.6449\\
28.3	706910886.8795\\
28.4	712390557.3957\\
28.5	713841712.0259\\
28.6	716709576.126\\
28.7	733499377.2441\\
28.8	772830774.2288\\
28.9	771799476.6518\\
29	773246868.6985\\
29.1	775770656.2411\\
29.2	775873376.5403\\
29.3	776129605.8578\\
29.4	776890408.5781\\
29.5	778716131.5749\\
29.6	782382331.2486\\
29.7	783733641.0037\\
29.8	784524766.3962\\
29.9	797052518.7285\\
30	828861171.6617\\
30.1	832204234.2687\\
30.2	836007502.6219\\
30.3	839249259.4842\\
30.4	840303506.4352\\
30.5	840680025.5918\\
30.6	843104933.0573\\
30.7	843382841.6292\\
30.8	850167674.1236\\
30.9	860508175.9757\\
31	864407093.6119\\
31.1	865874121.5186\\
31.2	867384378.2404\\
31.3	869200863\\
31.4	871657644.1714\\
31.5	872528216.7767\\
31.6	871789791.6868\\
31.7	872526786.4885\\
31.8	877420705.7115\\
31.9	879498756.1697\\
32	884450791.0748\\
32.1	890331285.2446\\
32.2	892742137.865\\
32.3	892566025.2941\\
32.4	892962742.6142\\
32.5	895341790.6377\\
32.6	897989581.6027\\
32.7	908111345.4252\\
32.8	925770858.3844\\
32.9	933851445.5405\\
33	949293364.689\\
33.1	955986485.7528\\
33.2	980612461.523\\
33.3	982954358.4466\\
33.4	983316898.369\\
33.5	986296636.1343\\
33.6	992089824.6981\\
33.7	991299393.6229\\
33.8	991760481.4264\\
33.9	993052349.088\\
34	993694503.2549\\
34.1	1001537668.9841\\
34.2	1015021269.3015\\
34.3	1015778090.4748\\
34.4	1016749283.9225\\
34.5	1019169592.3781\\
34.6	1022138579.7043\\
34.7	1022561760.0317\\
34.8	1023118267.5299\\
34.9	1024803276.3356\\
35	1028782949.9649\\
35.1	1032872575.0782\\
35.2	1037899845.4737\\
35.3	1046995898.0441\\
35.4	1064120578.4827\\
35.5	1071085038.5507\\
35.6	1075254994.5617\\
35.7	1081632215.8906\\
35.8	1081534701.5708\\
35.9	1081488270.0651\\
36	1082731525.197\\
36.1	1085167529.4942\\
36.2	1085904161.2931\\
36.3	1086547096.0655\\
36.4	1088136019.4089\\
36.5	1088231282.7031\\
36.6	1088941074.7776\\
36.7	1089242989.819\\
36.8	1089036992.3565\\
36.9	1091878723.2764\\
37	1095407322.3211\\
37.1	1098400111.7725\\
37.2	1104351743.1401\\
37.3	1117945780.022\\
37.4	1116976749.9004\\
37.5	1123488370.9739\\
37.6	1129824230.4313\\
37.7	1127240304.2092\\
37.8	1126274765.2068\\
37.9	1127914997.2597\\
38	1129776041.5281\\
38.1	1132232683.9479\\
38.2	1130425548.6791\\
38.3	1130493813.4811\\
38.4	1131171826.5417\\
38.5	1131262390.0549\\
38.6	1130484965.9886\\
38.7	1127696068.3983\\
38.8	1128344468.8466\\
38.9	1131205006.5085\\
39	1141210799.3397\\
39.1	1144946009.4609\\
39.2	1148739231.8675\\
39.3	1156371853.4195\\
39.4	1167009671.7939\\
39.5	1169182783.9965\\
39.6	1168391487.3535\\
39.7	1171035881.019\\
39.8	1181976207.2242\\
39.9	1186826401.802\\
40	1185724947.1598\\
40.1	1189179839.7815\\
40.2	1195276226.2037\\
40.3	1196884322.7183\\
40.4	1198645873.6991\\
40.5	1192793327.0166\\
40.6	1194531131.2408\\
40.7	1197303659.9036\\
40.8	1199184017.9552\\
40.9	1198363603.0939\\
41	1197628607.9885\\
41.1	1198617537.8812\\
41.2	1200636582.7005\\
41.3	1200726854.2717\\
41.4	1201484423.7514\\
41.5	1203974785.0844\\
41.6	1206404154.4055\\
41.7	1210200006.4676\\
41.8	1213296505.9972\\
41.9	1212342389.46\\
42	1215110682.0371\\
42.1	1215189301.8428\\
42.2	1216118704.4383\\
42.3	1218303019.8495\\
42.4	1219543666.2337\\
42.5	1219971428.9897\\
42.6	1225816097.3724\\
42.7	1238986345.1925\\
42.8	1241910745.6735\\
42.9	1244000365.0784\\
43	1243894592.1523\\
43.1	1243360017.2654\\
43.2	1242348203.7803\\
43.3	1244609464.5949\\
43.4	1248908444.4868\\
43.5	1251824569.5624\\
43.6	1256470388.9116\\
43.7	1262528695.7721\\
43.8	1266726706.6024\\
43.9	1269104978.3293\\
44	1271266522.1197\\
44.1	1274030733.6332\\
44.2	1275723466.2264\\
44.3	1274933038.9029\\
44.4	1277748212.5248\\
44.5	1279047908.914\\
44.6	1279729677.6372\\
44.7	1282406972.8402\\
44.8	1288968487.9738\\
44.9	1296109016.2997\\
45	1306309336.4929\\
45.1	1306894356.7849\\
45.2	1308302549.7918\\
45.3	1308823089.537\\
45.4	1309662409.5641\\
45.5	1310649172.8474\\
45.6	1312142620.1315\\
45.7	1311904735.9908\\
45.8	1311875743.0284\\
45.9	1312292557.8595\\
46	1314503426.3492\\
46.1	1319554357.4094\\
46.2	1316964240.9117\\
46.3	1316984640.2664\\
46.4	1315938733.5081\\
46.5	1314442544.8968\\
46.6	1315496703.2266\\
46.7	1315406631.2737\\
46.8	1314505818.9258\\
46.9	1312087166.825\\
47	1311960419.7217\\
47.1	1313803090.0462\\
47.2	1316681103.4612\\
47.3	1312548474.9837\\
47.4	1313504984.5217\\
47.5	1311184721.5477\\
47.6	1311089335.4674\\
47.7	1311547450.1618\\
47.8	1310245193.3614\\
47.9	1312139875.5406\\
48	1314710706.4661\\
48.1	1313997105.2386\\
48.2	1314650031.1518\\
48.3	1319785598.0361\\
48.4	1321503255.4503\\
48.5	1322544804.1863\\
48.6	1325396379.0045\\
48.7	1325869381.1334\\
48.8	1326010505.9281\\
48.9	1327920993.7384\\
49	1328452264.1606\\
49.1	1331949029.9974\\
49.2	1337512545.0897\\
49.3	1337926508.77\\
49.4	1340473706.0448\\
49.5	1344282388.7086\\
49.6	1346674460.7627\\
49.7	1350675292.1631\\
49.8	1352532105.2921\\
49.9	1355734784.0667\\
50	1358504433.8273\\
50.1	1361087052.6056\\
50.2	1367283571.1486\\
50.3	1370308360.0622\\
50.4	1370441742.7078\\
50.5	1372129401.5805\\
50.6	1374297503.6195\\
50.7	1376564153.8726\\
50.8	1376054309.4016\\
50.9	1374980629.3293\\
51	1372914039.709\\
51.1	1372280469.7626\\
51.2	1373976980.9596\\
51.3	1376782730.7608\\
51.4	1380072505.9844\\
51.5	1381370540.413\\
51.6	1379773801.9679\\
51.7	1380411845.2615\\
51.8	1388953628.6824\\
51.9	1393508149.7936\\
52	1397850096.1973\\
52.1	1396181813.3064\\
52.2	1396137630.4431\\
52.3	1400859228.468\\
52.4	1402919091.247\\
52.5	1405425624.8823\\
52.6	1408369625.9198\\
52.7	1411838224.0692\\
52.8	1413868408.4649\\
52.9	1414357295.2365\\
53	1414447661.0233\\
53.1	1417934775.8345\\
53.2	1420659643.0631\\
53.3	1421260820.686\\
53.4	1423510377.151\\
53.5	1428351125.4993\\
53.6	1434625098.8222\\
53.7	1437401259.7354\\
53.8	1439342616.3459\\
53.9	1438599489.5558\\
54	1437944298.3816\\
54.1	1438514291.0371\\
54.2	1444204736.1087\\
54.3	1450449112.0665\\
54.4	1450897891.4288\\
54.5	1456744189.8334\\
54.6	1469250902.321\\
54.7	1475077212.7671\\
54.8	1476686538.2771\\
54.9	1479542457.5693\\
55	1477740544.2646\\
55.1	1476856766.4778\\
55.2	1478989323.4681\\
55.3	1480886961.8742\\
55.4	1483290436.2613\\
55.5	1486627389.9191\\
55.6	1487904318.5032\\
55.7	1490090832.0926\\
55.8	1492797782.6007\\
55.9	1496895557.0043\\
56	1503444091.5627\\
56.1	1510883922.4662\\
56.2	1515180931.6282\\
56.3	1515680639.861\\
56.4	1530419125.9235\\
56.5	1526178713.5866\\
56.6	1534656601.9982\\
56.7	1535473471.0188\\
56.8	1541539923.8626\\
56.9	1547076611.0487\\
57	1548292852.9617\\
57.1	1548975417.448\\
57.2	1550515595.0293\\
57.3	1551565420.2859\\
57.4	1544413145.1195\\
57.5	1537992810.0481\\
57.6	1536246186.7012\\
57.7	1537552213.3519\\
57.8	1541160433.9074\\
57.9	1541254772.8992\\
58	1539858021.4188\\
58.1	1541694837.4361\\
58.2	1540712204.5093\\
58.3	1543761353.1772\\
58.4	1546411268.0122\\
58.5	1548994794.4739\\
58.6	1548504835.7449\\
58.7	1548668033.2745\\
58.8	1550998296.3667\\
58.9	1554244376.4751\\
59	1556752453.3487\\
59.1	1558811579.6224\\
59.2	1558717560.9851\\
59.3	1557567070.132\\
59.4	1560288598.1952\\
59.5	1563812695.728\\
59.6	1561254486.4496\\
59.7	1561724088.6284\\
59.8	1564098524.9526\\
59.9	1564584970.0207\\
60	1567081408.1948\\
60.1	1570004344.3819\\
60.2	1571851474.8765\\
60.3	1574022499.28\\
60.4	1579609954.4218\\
60.5	1584019688.3609\\
60.6	1583147162.5712\\
60.7	1591501593.8862\\
60.8	1603199751.5328\\
60.9	1602034926.9722\\
61	1604507441.6832\\
61.1	1605796114.5719\\
61.2	1604862404.2952\\
61.3	1603999587.603\\
61.4	1604403098.0918\\
61.5	1606525058.072\\
61.6	1599170205.0499\\
61.7	1608969922.9685\\
61.8	1607947304.7421\\
61.9	1608999538.3546\\
62	1611046415.69\\
62.1	1610767269.9742\\
62.2	1618385509.3338\\
62.3	1630055959.7958\\
62.4	1634025566.3485\\
62.5	1637743796.5718\\
62.6	1640727504.1305\\
62.7	1643585819.329\\
62.8	1648202907.2646\\
62.9	1650714020.7074\\
63	1649482473.0776\\
63.1	1647458241.9501\\
63.2	1649472904.998\\
63.3	1650833707.2325\\
63.4	1656673608.766\\
63.5	1652082492.1871\\
63.6	1645066224.273\\
63.7	1641748954.9346\\
63.8	1642896548.3758\\
63.9	1644774791.3721\\
64	1643828297.1675\\
64.1	1637505861.5015\\
64.2	1639952424.5078\\
64.3	1642543406.8311\\
64.4	1643773843.4542\\
64.5	1649049377.239\\
64.6	1648755440.8152\\
64.7	1647200211.5377\\
64.8	1646084604.2524\\
64.9	1645298328.3281\\
65	1643515301.36\\
65.1	1646980733.63\\
65.2	1644520331.6463\\
65.3	1640733910.4777\\
65.4	1640787452.4631\\
65.5	1651676001.1597\\
65.6	1665131854.1937\\
65.7	1665872490.1601\\
65.8	1664831450.7507\\
65.9	1666194538.7869\\
66	1673393085.6641\\
66.1	1682479016.0977\\
66.2	1685681306.1402\\
66.3	1687473122.3952\\
66.4	1689292322.2228\\
66.5	1690478686.5196\\
66.6	1689782348.1118\\
66.7	1692002728.2189\\
66.8	1697799386.2958\\
66.9	1700407145.2447\\
67	1707819911.2042\\
67.1	1720959167.7335\\
67.2	1725541474.5666\\
67.3	1728364895.5913\\
67.4	1729822470.7\\
67.5	1731919438.9288\\
67.6	1739394232.1103\\
67.7	1746002044.5698\\
67.8	1753906461.2312\\
67.9	1767314573.0189\\
68	1774402151.7679\\
68.1	1773422604.3584\\
68.2	1774112630.6066\\
68.3	1777216673.0948\\
68.4	1784589012.9567\\
68.5	1792050594.4733\\
68.6	1793511326.972\\
68.7	1795371301.0519\\
68.8	1798867056.031\\
68.9	1804215470.4535\\
69	1805111965.8286\\
69.1	1808515004.5312\\
69.2	1810820337.7558\\
69.3	1810889292.4619\\
69.4	1809283706.6031\\
69.5	1810559648.9229\\
69.6	1814090417.2286\\
69.7	1814144275.5456\\
69.8	1818903272.2916\\
69.9	1823919397.0059\\
70	1830283004.0815\\
70.1	1834874096.8134\\
70.2	1833904094.8445\\
70.3	1836043001.6682\\
70.4	1840948901.0589\\
70.5	1843902573.8655\\
70.6	1841451873.2248\\
70.7	1840559974.9082\\
70.8	1841655381.123\\
70.9	1846463794.8008\\
71	1845777807.9057\\
71.1	1844104364.2005\\
71.2	1844115135.8651\\
71.3	1846040108.3429\\
71.4	1846224525.3156\\
71.5	1845856411.7865\\
71.6	1848012570.9659\\
71.7	1848934972.403\\
71.8	1850181143.3553\\
71.9	1850943659.5115\\
72	1852759053.8133\\
72.1	1856030793.7652\\
72.2	1854941316.879\\
72.3	1855190124.7514\\
72.4	1855412797.2376\\
72.5	1855518252.7261\\
72.6	1855331890.9404\\
72.7	1857853835.4243\\
72.8	1861956343.0796\\
72.9	1865179397.8901\\
73	1863290279.9975\\
73.1	1870468696.8003\\
73.2	1885440132.7278\\
73.3	1886425319.9885\\
73.4	1889655894.7049\\
73.5	1892848065.6014\\
73.6	1888943418.8686\\
73.7	1888738531.8644\\
73.8	1891713139.4931\\
73.9	1893191075.8734\\
74	1893854079.2493\\
74.1	1892291304.1243\\
74.2	1889607241.7742\\
74.3	1890920687.3989\\
74.4	1892541758.0989\\
74.5	1897329332.5373\\
74.6	1906116687.5717\\
74.7	1908245083.3703\\
74.8	1908456348.367\\
74.9	1910543295.5691\\
75	1917493390.9625\\
75.1	1926693857.3473\\
75.2	1930918471.5\\
75.3	1936544752.5835\\
75.4	1940005332.0672\\
75.5	1944057017.9802\\
75.6	1943983738.2517\\
75.7	1944132827.6154\\
75.8	1946801024.0019\\
75.9	1945476709.5773\\
76	1942809390.9684\\
76.1	1943210942.5139\\
76.2	1942573802.4413\\
76.3	1942814120.2736\\
76.4	1945952183.3216\\
76.5	1950045540.2373\\
76.6	1958578291.2352\\
76.7	1966084900.7687\\
76.8	1969487357.2028\\
76.9	1969845828.0172\\
77	1966986394.1755\\
77.1	1965676918.5399\\
77.2	1968862911.4242\\
77.3	1970868684.5274\\
77.4	1977829394.9383\\
77.5	1982023147.8432\\
77.6	1981614286.1519\\
77.7	1979166838.7008\\
77.8	1980804749.6286\\
77.9	1981893866.8623\\
78	1981148017.3942\\
78.1	1981798542.9955\\
78.2	1982733441.3425\\
78.3	1982752256.7497\\
78.4	1982468385.8796\\
78.5	1985449466.4042\\
78.6	1988493780.5918\\
78.7	1989916283.4122\\
78.8	1989186208.6707\\
78.9	1987946094.111\\
79	1992451232.065\\
79.1	1999292649.5943\\
79.2	2006032015.6477\\
79.3	2020079190.3335\\
79.4	2021549548.5001\\
79.5	2022933751.6261\\
79.6	2024697192.0883\\
79.7	2024998752.0306\\
79.8	2023073553.2271\\
79.9	2030565942.4971\\
80	2041750345.3293\\
80.1	2040349224.3627\\
80.2	2042980249.8123\\
80.3	2046895884.7165\\
80.4	2044497600.616\\
80.5	2041138874.5288\\
80.6	2041696006.7448\\
80.7	2044848413.8087\\
80.8	2042696347.315\\
80.9	2039894234.2942\\
81	2037655752.9165\\
81.1	2036462676.3075\\
81.2	2039538786.4734\\
81.3	2043693889.2994\\
81.4	2043274535.1779\\
81.5	2044723071.1736\\
81.6	2048603991.105\\
81.7	2049068670.6942\\
81.8	2048135734.1298\\
81.9	2043920312.3877\\
82	2042828863.5875\\
82.1	2041063252.4592\\
82.2	2040772101.2205\\
82.3	2043754485.7968\\
82.4	2042568400.5103\\
82.5	2043946345.6124\\
82.6	2045752758.012\\
82.7	2045921236.4827\\
82.8	2047159061.6809\\
82.9	2045209554.5124\\
83	2040078809.9074\\
83.1	2040634489.2412\\
83.2	2043097117.1176\\
83.3	2045476378.6097\\
83.4	2048458542.5076\\
83.5	2050876864.3016\\
83.6	2052655844.2354\\
83.7	2054474842.6004\\
83.8	2056005977.4693\\
83.9	2058042030.4733\\
84	2057280747.6631\\
84.1	2055285713.0138\\
84.2	2055248808.5557\\
84.3	2056891055.9182\\
84.4	2058029440.7002\\
84.5	2060302983.1808\\
84.6	2060990384.9435\\
84.7	2059189130.3616\\
84.8	2060088374.4485\\
84.9	2064844710.7049\\
85	2065161874.195\\
85.1	2066928447.3862\\
85.2	2073448079.8752\\
85.3	2075755724.3976\\
85.4	2078988831.4568\\
85.5	2086536278.6376\\
85.6	2088807937.2646\\
85.7	2090692805.9216\\
85.8	2088374520.5631\\
85.9	2090868296.665\\
86	2096156321.6815\\
86.1	2102324320.5171\\
86.2	2104124466.5985\\
86.3	2099205135.586\\
86.4	2103700938.5745\\
86.5	2105395934.3197\\
86.6	2107214222.5733\\
86.7	2108015536.1984\\
86.8	2104901265.1402\\
86.9	2103603705.2765\\
87	2108211360.0616\\
87.1	2113472483.7338\\
87.2	2114106103.7047\\
87.3	2120422671.5285\\
87.4	2126055734.0768\\
87.5	2122523250.8999\\
87.6	2121210966.4087\\
87.7	2122160432.0222\\
87.8	2124865036.466\\
87.9	2135501595.2218\\
88	2140712719.6297\\
88.1	2138159235.4773\\
88.2	2138858842.3308\\
88.3	2142287682.1483\\
88.4	2138545787.7289\\
88.5	2137927787.4184\\
88.6	2142261608.9424\\
88.7	2148704630.7214\\
88.8	2153714901.1134\\
88.9	2155686512.7715\\
89	2186422143.4954\\
89.1	2255846181.6484\\
89.2	2256567266.0437\\
89.3	2248654045.5982\\
89.4	2242205409.1984\\
89.5	2244827682.613\\
89.6	2243650796.7316\\
89.7	2239414713.323\\
89.8	2237874357.2959\\
89.9	2237666477.8399\\
90	2238169866.9281\\
90.1	2239565838.1171\\
90.2	2241669090.3188\\
90.3	2239863408.3141\\
90.4	2239082787.5577\\
90.5	2238941515.5427\\
90.6	2237863925.8791\\
90.7	2240622406.9894\\
90.8	2245819617.4848\\
90.9	2247443222.8153\\
91	2247043963.7485\\
91.1	2251378557.6602\\
91.2	2264210547.1238\\
91.3	2268956439.2356\\
91.4	2274238709.7421\\
91.5	2277582781.6173\\
91.6	2281809907.1349\\
91.7	2286355013.3308\\
91.8	2288585780.7476\\
91.9	2290226136.6258\\
92	2301626435.4108\\
92.1	2337269696.4709\\
92.2	2332597485.4655\\
92.3	2330636883.9182\\
92.4	2330544091.4481\\
92.5	2330818437.8936\\
92.6	2327523725.7828\\
92.7	2326336963.6191\\
92.8	2334930443.6992\\
92.9	2355442200.7823\\
93	2359489902.5912\\
93.1	2359363632.893\\
93.2	2362178553.8375\\
93.3	2365299931.5456\\
93.4	2372816778.7208\\
93.5	2379925459.529\\
93.6	2381777956.4747\\
93.7	2385712981.6374\\
93.8	2390897504.289\\
93.9	2392890482.7853\\
94	2389088298.7859\\
94.1	2387273424.8749\\
94.2	2385724291.7648\\
94.3	2382614081.9004\\
94.4	2379721735.2702\\
94.5	2380587329.8036\\
94.6	2384806908.8938\\
94.7	2386511563.9644\\
94.8	2387314012.8355\\
94.9	2392094824.3608\\
95	2394339972.0461\\
95.1	2395659647.9375\\
95.2	2396193525.3115\\
95.3	2396486479.9082\\
95.4	2396076515.0712\\
95.5	2398292158.2319\\
95.6	2403259040.2268\\
95.7	2406276751.2927\\
95.8	2410221614.0591\\
95.9	2413995244.8402\\
96	2417191312.5007\\
96.1	2420329289.9185\\
96.2	2429482069.1699\\
96.3	2438411467.991\\
96.4	2439545096.5661\\
96.5	2438689973.0968\\
96.6	2439315592.9392\\
96.7	2442478914.8771\\
96.8	2443914545.0988\\
96.9	2447234587.0891\\
97	2450376663.8866\\
97.1	2455750306.643\\
97.2	2463419646.9892\\
97.3	2468713843.8807\\
97.4	2473391830.9106\\
97.5	2475642499.2435\\
97.6	2475418989.0877\\
97.7	2479613326.6207\\
97.8	2480705194.4994\\
97.9	2479822030.5052\\
98	2479122037.731\\
98.1	2476012281.6975\\
98.2	2474356896.6894\\
98.3	2476626308.1519\\
98.4	2479420800.7519\\
98.5	2483317468.6364\\
98.6	2484554861.8066\\
98.7	2494760309.1661\\
98.8	2512199399.7655\\
98.9	2511081301.8324\\
99	2512382623.279\\
99.1	2515173402.7058\\
99.2	2519893634.6816\\
99.3	2526143419.3471\\
99.4	2528825347.3378\\
99.5	2531271364.7808\\
99.6	2531407564.9913\\
99.7	2535014447.0121\\
99.8	2522576226.6928\\
99.9	2592102688.7931\\
};
\addlegendentry{Kinetic energy};

\end{axis}
\end{tikzpicture}%}
        \caption{$\Delta t =\unit[0.1]{ASU}$}
        \label{fig:timestep_a}
    \end{subfigure}
    \begin{subfigure}[b]{0.40\textwidth}
        \centering
        \resizebox{\columnwidth}{!}{% This file was created by matlab2tikz.
%
%The latest updates can be retrieved from
%  http://www.mathworks.com/matlabcentral/fileexchange/22022-matlab2tikz-matlab2tikz
%where you can also make suggestions and rate matlab2tikz.
%
\definecolor{mycolor1}{rgb}{0.00000,0.44700,0.74100}%
\definecolor{mycolor2}{rgb}{0.85000,0.32500,0.09800}%
\definecolor{mycolor3}{rgb}{0.92900,0.69400,0.12500}%
%
\begin{tikzpicture}

\begin{axis}[%
width=4.521in,
height=3.566in,
at={(0.758in,0.481in)},
scale only axis,
xmin=0,
xmax=10,
xlabel={Time / [$\unit{Å}$]},
ymin=-900,
ymax=100,
ylabel={Energy / [$\unit{eV}$]},
label style ={font=\Large},
axis background/.style={fill=white},
title style={font=\bfseries\Huge},
title={Time evolution of energy},
legend style={draw=white!15!black},
legend pos=north west
]
\addplot [color=mycolor1,solid]
  table[row sep=crcr]{%
0	-857.2121\\
0.01	-857.2514\\
0.02	-857.3192\\
0.03	-857.336\\
0.04	-857.2966\\
0.05	-857.2526\\
0.06	-857.243\\
0.07	-857.2734\\
0.08	-857.3111\\
0.09	-857.3104\\
0.1	-857.2777\\
0.11	-857.2559\\
0.12	-857.2698\\
0.13	-857.3015\\
0.14	-857.3088\\
0.15	-857.2852\\
0.16	-857.263\\
0.17	-857.267\\
0.18	-857.2876\\
0.19	-857.2991\\
0.2	-857.2927\\
0.21	-857.2783\\
0.22	-857.271\\
0.23	-857.2789\\
0.24	-857.2916\\
0.25	-857.2945\\
0.26	-857.2843\\
0.27	-857.274\\
0.28	-857.2753\\
0.29	-857.2865\\
0.3	-857.292\\
0.31	-857.2859\\
0.32	-857.2739\\
0.33	-857.274\\
0.34	-857.2841\\
0.35	-857.2936\\
0.36	-857.289\\
0.37	-857.2825\\
0.38	-857.277\\
0.39	-857.2784\\
0.4	-857.2809\\
0.41	-857.2826\\
0.42	-857.2851\\
0.43	-857.2852\\
0.44	-857.2854\\
0.45	-857.2793\\
0.46	-857.2754\\
0.47	-857.2766\\
0.48	-857.2819\\
0.49	-857.2867\\
0.5	-857.2896\\
0.51	-857.2861\\
0.52	-857.28\\
0.53	-857.2761\\
0.54	-857.2777\\
0.55	-857.2858\\
0.56	-857.2898\\
0.57	-857.2865\\
0.58	-857.2798\\
0.59	-857.276\\
0.6	-857.28\\
0.61	-857.2853\\
0.62	-857.2876\\
0.63	-857.2835\\
0.64	-857.2792\\
0.65	-857.2819\\
0.66	-857.2861\\
0.67	-857.2877\\
0.68	-857.2816\\
0.69	-857.2758\\
0.7	-857.2781\\
0.71	-857.2851\\
0.72	-857.2866\\
0.73	-857.2819\\
0.74	-857.2783\\
0.75	-857.2805\\
0.76	-857.2863\\
0.77	-857.2914\\
0.78	-857.2877\\
0.79	-857.2767\\
0.8	-857.2718\\
0.81	-857.2775\\
0.82	-857.2892\\
0.83	-857.2923\\
0.84	-857.2863\\
0.85	-857.2793\\
0.86	-857.2772\\
0.87	-857.2805\\
0.88	-857.2808\\
0.89	-857.2807\\
0.9	-857.2845\\
0.91	-857.2887\\
0.92	-857.2863\\
0.93	-857.2802\\
0.94	-857.2753\\
0.95	-857.2773\\
0.96	-857.284\\
0.97	-857.2911\\
0.98	-857.2912\\
0.99	-857.2829\\
1	-857.2717\\
1.01	-857.2715\\
1.02	-857.2819\\
1.03	-857.2925\\
1.04	-857.2924\\
1.05	-857.283\\
1.06	-857.2721\\
1.07	-857.2696\\
1.08	-857.2772\\
1.09	-857.2903\\
1.1	-857.2934\\
1.11	-857.2856\\
1.12	-857.2777\\
1.13	-857.2759\\
1.14	-857.2806\\
1.15	-857.2859\\
1.16	-857.2862\\
1.17	-857.2831\\
1.18	-857.2799\\
1.19	-857.2784\\
1.2	-857.2794\\
1.21	-857.2807\\
1.22	-857.284\\
1.23	-857.2892\\
1.24	-857.2866\\
1.25	-857.2757\\
1.26	-857.2702\\
1.27	-857.2764\\
1.28	-857.2893\\
1.29	-857.297\\
1.3	-857.2868\\
1.31	-857.2729\\
1.32	-857.2674\\
1.33	-857.2752\\
1.34	-857.2887\\
1.35	-857.2943\\
1.36	-857.2873\\
1.37	-857.2744\\
1.38	-857.2721\\
1.39	-857.2803\\
1.4	-857.2867\\
1.41	-857.2863\\
1.42	-857.2794\\
1.43	-857.2759\\
1.44	-857.2812\\
1.45	-857.2876\\
1.46	-857.2866\\
1.47	-857.2779\\
1.48	-857.2725\\
1.49	-857.2788\\
1.5	-857.2884\\
1.51	-857.2931\\
1.52	-857.2826\\
1.53	-857.2716\\
1.54	-857.2734\\
1.55	-857.2861\\
1.56	-857.2963\\
1.57	-857.2899\\
1.58	-857.2757\\
1.59	-857.2666\\
1.6	-857.2742\\
1.61	-857.2899\\
1.62	-857.2962\\
1.63	-857.2904\\
1.64	-857.2785\\
1.65	-857.2707\\
1.66	-857.2723\\
1.67	-857.2805\\
1.68	-857.287\\
1.69	-857.2875\\
1.7	-857.2834\\
1.71	-857.2815\\
1.72	-857.284\\
1.73	-857.2851\\
1.74	-857.279\\
1.75	-857.274\\
1.76	-857.2775\\
1.77	-857.2873\\
1.78	-857.2912\\
1.79	-857.2865\\
1.8	-857.2776\\
1.81	-857.2734\\
1.82	-857.2809\\
1.83	-857.288\\
1.84	-857.289\\
1.85	-857.2835\\
1.86	-857.2778\\
1.87	-857.2783\\
1.88	-857.2825\\
1.89	-857.2846\\
1.9	-857.2814\\
1.91	-857.2769\\
1.92	-857.2764\\
1.93	-857.2835\\
1.94	-857.2887\\
1.95	-857.2892\\
1.96	-857.2809\\
1.97	-857.2748\\
1.98	-857.2768\\
1.99	-857.2852\\
2	-857.2874\\
2.01	-857.2836\\
2.02	-857.2781\\
2.03	-857.2797\\
2.04	-857.2828\\
2.05	-857.2831\\
2.06	-857.2768\\
2.07	-857.2743\\
2.08	-857.2804\\
2.09	-857.2897\\
2.1	-857.2931\\
2.11	-857.2865\\
2.12	-857.273\\
2.13	-857.2689\\
2.14	-857.2752\\
2.15	-857.287\\
2.16	-857.2929\\
2.17	-857.2911\\
2.18	-857.2819\\
2.19	-857.2739\\
2.2	-857.2748\\
2.21	-857.2806\\
2.22	-857.2875\\
2.23	-857.2897\\
2.24	-857.2867\\
2.25	-857.2786\\
2.26	-857.2723\\
2.27	-857.274\\
2.28	-857.2828\\
2.29	-857.289\\
2.3	-857.2868\\
2.31	-857.2798\\
2.32	-857.2744\\
2.33	-857.2756\\
2.34	-857.2843\\
2.35	-857.2921\\
2.36	-857.2869\\
2.37	-857.2754\\
2.38	-857.2711\\
2.39	-857.278\\
2.4	-857.2888\\
2.41	-857.2935\\
2.42	-857.2877\\
2.43	-857.2785\\
2.44	-857.2744\\
2.45	-857.2786\\
2.46	-857.2861\\
2.47	-857.2873\\
2.48	-857.2832\\
2.49	-857.2783\\
2.5	-857.2804\\
2.51	-857.2845\\
2.52	-857.2849\\
2.53	-857.2815\\
2.54	-857.2766\\
2.55	-857.279\\
2.56	-857.2855\\
2.57	-857.2901\\
2.58	-857.286\\
2.59	-857.2762\\
2.6	-857.2722\\
2.61	-857.2789\\
2.62	-857.2894\\
2.63	-857.2921\\
2.64	-857.2855\\
2.65	-857.2763\\
2.66	-857.2738\\
2.67	-857.2778\\
2.68	-857.2871\\
2.69	-857.29\\
2.7	-857.2864\\
2.71	-857.2812\\
2.72	-857.2789\\
2.73	-857.2799\\
2.74	-857.2827\\
2.75	-857.2831\\
2.76	-857.284\\
2.77	-857.2822\\
2.78	-857.2787\\
2.79	-857.2799\\
2.8	-857.2862\\
2.81	-857.2879\\
2.82	-857.2844\\
2.83	-857.278\\
2.84	-857.2759\\
2.85	-857.2796\\
2.86	-857.285\\
2.87	-857.2871\\
2.88	-857.285\\
2.89	-857.2788\\
2.9	-857.2754\\
2.91	-857.2785\\
2.92	-857.2863\\
2.93	-857.2871\\
2.94	-857.2865\\
2.95	-857.2811\\
2.96	-857.2793\\
2.97	-857.2804\\
2.98	-857.2835\\
2.99	-857.2825\\
3	-857.2817\\
3.01	-857.2809\\
3.02	-857.2823\\
3.03	-857.2844\\
3.04	-857.2836\\
3.05	-857.2799\\
3.06	-857.2755\\
3.07	-857.2738\\
3.08	-857.2816\\
3.09	-857.2914\\
3.1	-857.2944\\
3.11	-857.286\\
3.12	-857.2723\\
3.13	-857.2689\\
3.14	-857.2764\\
3.15	-857.2863\\
3.16	-857.293\\
3.17	-857.2904\\
3.18	-857.2801\\
3.19	-857.27\\
3.2	-857.2723\\
3.21	-857.2841\\
3.22	-857.2906\\
3.23	-857.2871\\
3.24	-857.279\\
3.25	-857.2753\\
3.26	-857.2792\\
3.27	-857.2841\\
3.28	-857.2867\\
3.29	-857.2862\\
3.3	-857.2817\\
3.31	-857.2786\\
3.32	-857.2775\\
3.33	-857.2805\\
3.34	-857.2835\\
3.35	-857.2847\\
3.36	-857.2846\\
3.37	-857.2816\\
3.38	-857.2776\\
3.39	-857.2782\\
3.4	-857.2838\\
3.41	-857.2899\\
3.42	-857.287\\
3.43	-857.2815\\
3.44	-857.2797\\
3.45	-857.2816\\
3.46	-857.2828\\
3.47	-857.2816\\
3.48	-857.2817\\
3.49	-857.2834\\
3.5	-857.2857\\
3.51	-857.2833\\
3.52	-857.2784\\
3.53	-857.2773\\
3.54	-857.2804\\
3.55	-857.2838\\
3.56	-857.2864\\
3.57	-857.2825\\
3.58	-857.2792\\
3.59	-857.2787\\
3.6	-857.2828\\
3.61	-857.2884\\
3.62	-857.2869\\
3.63	-857.2804\\
3.64	-857.2771\\
3.65	-857.2779\\
3.66	-857.2827\\
3.67	-857.2859\\
3.68	-857.2851\\
3.69	-857.2808\\
3.7	-857.2768\\
3.71	-857.2774\\
3.72	-857.2811\\
3.73	-857.2834\\
3.74	-857.2826\\
3.75	-857.2802\\
3.76	-857.2807\\
3.77	-857.2834\\
3.78	-857.284\\
3.79	-857.281\\
3.8	-857.2792\\
3.81	-857.2803\\
3.82	-857.2821\\
3.83	-857.2858\\
3.84	-857.2846\\
3.85	-857.2806\\
3.86	-857.278\\
3.87	-857.2783\\
3.88	-857.2823\\
3.89	-857.2859\\
3.9	-857.285\\
3.91	-857.2797\\
3.92	-857.2754\\
3.93	-857.2767\\
3.94	-857.2831\\
3.95	-857.2851\\
3.96	-857.285\\
3.97	-857.283\\
3.98	-857.2821\\
3.99	-857.2832\\
4	-857.2805\\
4.01	-857.2793\\
4.02	-857.2815\\
4.03	-857.283\\
4.04	-857.2835\\
4.05	-857.2836\\
4.06	-857.2844\\
4.07	-857.2825\\
4.08	-857.279\\
4.09	-857.2807\\
4.1	-857.2847\\
4.11	-857.2843\\
4.12	-857.2794\\
4.13	-857.2769\\
4.14	-857.2786\\
4.15	-857.2853\\
4.16	-857.2877\\
4.17	-857.2843\\
4.18	-857.2787\\
4.19	-857.2737\\
4.2	-857.2767\\
4.21	-857.2836\\
4.22	-857.2893\\
4.23	-857.2875\\
4.24	-857.2822\\
4.25	-857.2763\\
4.26	-857.2784\\
4.27	-857.2816\\
4.28	-857.2855\\
4.29	-857.2857\\
4.3	-857.2834\\
4.31	-857.2805\\
4.32	-857.2809\\
4.33	-857.2797\\
4.34	-857.2773\\
4.35	-857.28\\
4.36	-857.285\\
4.37	-857.2855\\
4.38	-857.2832\\
4.39	-857.28\\
4.4	-857.2786\\
4.41	-857.2803\\
4.42	-857.2831\\
4.43	-857.286\\
4.44	-857.2829\\
4.45	-857.2774\\
4.46	-857.2764\\
4.47	-857.281\\
4.48	-857.2881\\
4.49	-857.2888\\
4.5	-857.281\\
4.51	-857.2739\\
4.52	-857.2748\\
4.53	-857.2816\\
4.54	-857.2893\\
4.55	-857.2882\\
4.56	-857.2798\\
4.57	-857.2746\\
4.58	-857.2761\\
4.59	-857.2846\\
4.6	-857.291\\
4.61	-857.2881\\
4.62	-857.2784\\
4.63	-857.2735\\
4.64	-857.2747\\
4.65	-857.2809\\
4.66	-857.2868\\
4.67	-857.2873\\
4.68	-857.284\\
4.69	-857.2778\\
4.7	-857.2766\\
4.71	-857.2795\\
4.72	-857.2826\\
4.73	-857.2847\\
4.74	-857.2848\\
4.75	-857.2835\\
4.76	-857.2813\\
4.77	-857.2806\\
4.78	-857.281\\
4.79	-857.2773\\
4.8	-857.2761\\
4.81	-857.2805\\
4.82	-857.2858\\
4.83	-857.2888\\
4.84	-857.2863\\
4.85	-857.2796\\
4.86	-857.2783\\
4.87	-857.2793\\
4.88	-857.2846\\
4.89	-857.2865\\
4.9	-857.2825\\
4.91	-857.2788\\
4.92	-857.2783\\
4.93	-857.2811\\
4.94	-857.2836\\
4.95	-857.2863\\
4.96	-857.2816\\
4.97	-857.278\\
4.98	-857.2787\\
4.99	-857.2843\\
5	-857.2884\\
5.01	-857.284\\
5.02	-857.2765\\
5.03	-857.2772\\
5.04	-857.2829\\
5.05	-857.2875\\
5.06	-857.2842\\
5.07	-857.2808\\
5.08	-857.2809\\
5.09	-857.2826\\
5.1	-857.2822\\
5.11	-857.2816\\
5.12	-857.2813\\
5.13	-857.2831\\
5.14	-857.2857\\
5.15	-857.2866\\
5.16	-857.2845\\
5.17	-857.2824\\
5.18	-857.2802\\
5.19	-857.2808\\
5.2	-857.2825\\
5.21	-857.2844\\
5.22	-857.283\\
5.23	-857.2844\\
5.24	-857.2832\\
5.25	-857.2818\\
5.26	-857.2805\\
5.27	-857.2805\\
5.28	-857.2799\\
5.29	-857.2841\\
5.3	-857.2858\\
5.31	-857.2822\\
5.32	-857.2794\\
5.33	-857.2788\\
5.34	-857.2812\\
5.35	-857.2833\\
5.36	-857.2841\\
5.37	-857.2842\\
5.38	-857.2804\\
5.39	-857.2796\\
5.4	-857.2823\\
5.41	-857.2854\\
5.42	-857.2835\\
5.43	-857.2806\\
5.44	-857.277\\
5.45	-857.2788\\
5.46	-857.2834\\
5.47	-857.2884\\
5.48	-857.2867\\
5.49	-857.2808\\
5.5	-857.278\\
5.51	-857.2775\\
5.52	-857.2823\\
5.53	-857.2882\\
5.54	-857.2895\\
5.55	-857.2862\\
5.56	-857.2802\\
5.57	-857.2768\\
5.58	-857.2783\\
5.59	-857.2813\\
5.6	-857.2827\\
5.61	-857.2841\\
5.62	-857.2836\\
5.63	-857.2818\\
5.64	-857.2825\\
5.65	-857.2827\\
5.66	-857.2834\\
5.67	-857.2814\\
5.68	-857.2832\\
5.69	-857.2835\\
5.7	-857.2841\\
5.71	-857.2825\\
5.72	-857.2816\\
5.73	-857.2822\\
5.74	-857.2809\\
5.75	-857.2816\\
5.76	-857.2813\\
5.77	-857.2819\\
5.78	-857.2816\\
5.79	-857.2818\\
5.8	-857.2843\\
5.81	-857.284\\
5.82	-857.2853\\
5.83	-857.2825\\
5.84	-857.2807\\
5.85	-857.2788\\
5.86	-857.2818\\
5.87	-857.2859\\
5.88	-857.2877\\
5.89	-857.2862\\
5.9	-857.2797\\
5.91	-857.2757\\
5.92	-857.2773\\
5.93	-857.2829\\
5.94	-857.2862\\
5.95	-857.2842\\
5.96	-857.2802\\
5.97	-857.279\\
5.98	-857.2812\\
5.99	-857.2855\\
6	-857.2858\\
6.01	-857.2803\\
6.02	-857.2767\\
6.03	-857.2791\\
6.04	-857.283\\
6.05	-857.2864\\
6.06	-857.2847\\
6.07	-857.2803\\
6.08	-857.2769\\
6.09	-857.2796\\
6.1	-857.2852\\
6.11	-857.2876\\
6.12	-857.2856\\
6.13	-857.2808\\
6.14	-857.2796\\
6.15	-857.2817\\
6.16	-857.2834\\
6.17	-857.2822\\
6.18	-857.2812\\
6.19	-857.2812\\
6.2	-857.2811\\
6.21	-857.2813\\
6.22	-857.2822\\
6.23	-857.2821\\
6.24	-857.2814\\
6.25	-857.2823\\
6.26	-857.285\\
6.27	-857.2865\\
6.28	-857.2836\\
6.29	-857.2789\\
6.3	-857.2785\\
6.31	-857.2796\\
6.32	-857.2828\\
6.33	-857.2842\\
6.34	-857.2841\\
6.35	-857.2845\\
6.36	-857.2844\\
6.37	-857.2794\\
6.38	-857.275\\
6.39	-857.2754\\
6.4	-857.2815\\
6.41	-857.2872\\
6.42	-857.2853\\
6.43	-857.2803\\
6.44	-857.2773\\
6.45	-857.2807\\
6.46	-857.2846\\
6.47	-857.2846\\
6.48	-857.2803\\
6.49	-857.279\\
6.5	-857.283\\
6.51	-857.2862\\
6.52	-857.2842\\
6.53	-857.2823\\
6.54	-857.2785\\
6.55	-857.2798\\
6.56	-857.2852\\
6.57	-857.2868\\
6.58	-857.2819\\
6.59	-857.2783\\
6.6	-857.2798\\
6.61	-857.286\\
6.62	-857.287\\
6.63	-857.2824\\
6.64	-857.2807\\
6.65	-857.2818\\
6.66	-857.2842\\
6.67	-857.2817\\
6.68	-857.2828\\
6.69	-857.2825\\
6.7	-857.283\\
6.71	-857.2827\\
6.72	-857.2832\\
6.73	-857.2845\\
6.74	-857.2833\\
6.75	-857.2818\\
6.76	-857.2802\\
6.77	-857.2795\\
6.78	-857.2811\\
6.79	-857.2852\\
6.8	-857.2907\\
6.81	-857.2867\\
6.82	-857.2813\\
6.83	-857.2779\\
6.84	-857.2784\\
6.85	-857.283\\
6.86	-857.2866\\
6.87	-857.2868\\
6.88	-857.283\\
6.89	-857.2822\\
6.9	-857.2838\\
6.91	-857.2821\\
6.92	-857.2795\\
6.93	-857.2797\\
6.94	-857.2823\\
6.95	-857.2839\\
6.96	-857.2827\\
6.97	-857.281\\
6.98	-857.2818\\
6.99	-857.2831\\
7	-857.2822\\
7.01	-857.2799\\
7.02	-857.2804\\
7.03	-857.2828\\
7.04	-857.2863\\
7.05	-857.2845\\
7.06	-857.2814\\
7.07	-857.2802\\
7.08	-857.2796\\
7.09	-857.2791\\
7.1	-857.2806\\
7.11	-857.2816\\
7.12	-857.2832\\
7.13	-857.2856\\
7.14	-857.2835\\
7.15	-857.2789\\
7.16	-857.2766\\
7.17	-857.2784\\
7.18	-857.2834\\
7.19	-857.2864\\
7.2	-857.2873\\
7.21	-857.282\\
7.22	-857.2759\\
7.23	-857.2756\\
7.24	-857.2817\\
7.25	-857.2878\\
7.26	-857.2872\\
7.27	-857.2831\\
7.28	-857.2784\\
7.29	-857.2823\\
7.3	-857.2854\\
7.31	-857.2846\\
7.32	-857.2791\\
7.33	-857.2781\\
7.34	-857.2802\\
7.35	-857.2828\\
7.36	-857.2853\\
7.37	-857.2843\\
7.38	-857.2787\\
7.39	-857.2759\\
7.4	-857.2771\\
7.41	-857.2841\\
7.42	-857.2878\\
7.43	-857.287\\
7.44	-857.2821\\
7.45	-857.277\\
7.46	-857.2787\\
7.47	-857.2838\\
7.48	-857.2852\\
7.49	-857.2852\\
7.5	-857.2823\\
7.51	-857.2824\\
7.52	-857.2848\\
7.53	-857.2863\\
7.54	-857.2817\\
7.55	-857.2755\\
7.56	-857.2768\\
7.57	-857.2817\\
7.58	-857.2864\\
7.59	-857.2858\\
7.6	-857.282\\
7.61	-857.2803\\
7.62	-857.2816\\
7.63	-857.2828\\
7.64	-857.2824\\
7.65	-857.2822\\
7.66	-857.2845\\
7.67	-857.2857\\
7.68	-857.2826\\
7.69	-857.2794\\
7.7	-857.2802\\
7.71	-857.2804\\
7.72	-857.2819\\
7.73	-857.2828\\
7.74	-857.2828\\
7.75	-857.2818\\
7.76	-857.2815\\
7.77	-857.2795\\
7.78	-857.2806\\
7.79	-857.2835\\
7.8	-857.2867\\
7.81	-857.2838\\
7.82	-857.2806\\
7.83	-857.2813\\
7.84	-857.2824\\
7.85	-857.2838\\
7.86	-857.2833\\
7.87	-857.2818\\
7.88	-857.2808\\
7.89	-857.2829\\
7.9	-857.2856\\
7.91	-857.285\\
7.92	-857.2833\\
7.93	-857.2825\\
7.94	-857.2815\\
7.95	-857.2804\\
7.96	-857.2792\\
7.97	-857.2793\\
7.98	-857.2827\\
7.99	-857.286\\
8	-857.2857\\
8.01	-857.2836\\
8.02	-857.2778\\
8.03	-857.2758\\
8.04	-857.278\\
8.05	-857.2839\\
8.06	-857.2882\\
8.07	-857.2858\\
8.08	-857.2803\\
8.09	-857.277\\
8.1	-857.2789\\
8.11	-857.2844\\
8.12	-857.2862\\
8.13	-857.2842\\
8.14	-857.2808\\
8.15	-857.2773\\
8.16	-857.2782\\
8.17	-857.2839\\
8.18	-857.2849\\
8.19	-857.2849\\
8.2	-857.2812\\
8.21	-857.2787\\
8.22	-857.28\\
8.23	-857.2818\\
8.24	-857.2841\\
8.25	-857.2819\\
8.26	-857.2824\\
8.27	-857.2807\\
8.28	-857.2794\\
8.29	-857.2815\\
8.3	-857.2849\\
8.31	-857.2856\\
8.32	-857.2825\\
8.33	-857.2806\\
8.34	-857.278\\
8.35	-857.2784\\
8.36	-857.2821\\
8.37	-857.2857\\
8.38	-857.286\\
8.39	-857.2806\\
8.4	-857.2768\\
8.41	-857.2783\\
8.42	-857.2841\\
8.43	-857.2848\\
8.44	-857.282\\
8.45	-857.2796\\
8.46	-857.281\\
8.47	-857.2831\\
8.48	-857.2828\\
8.49	-857.2814\\
8.5	-857.2814\\
8.51	-857.2822\\
8.52	-857.2839\\
8.53	-857.2843\\
8.54	-857.2836\\
8.55	-857.2817\\
8.56	-857.2788\\
8.57	-857.2775\\
8.58	-857.2808\\
8.59	-857.2839\\
8.6	-857.2837\\
8.61	-857.2826\\
8.62	-857.2817\\
8.63	-857.2815\\
8.64	-857.2837\\
8.65	-857.2833\\
8.66	-857.2824\\
8.67	-857.2795\\
8.68	-857.2785\\
8.69	-857.2818\\
8.7	-857.2863\\
8.71	-857.2856\\
8.72	-857.2809\\
8.73	-857.2784\\
8.74	-857.279\\
8.75	-857.2842\\
8.76	-857.2859\\
8.77	-857.2843\\
8.78	-857.2812\\
8.79	-857.2796\\
8.8	-857.2815\\
8.81	-857.2819\\
8.82	-857.2821\\
8.83	-857.2816\\
8.84	-857.2825\\
8.85	-857.2818\\
8.86	-857.2837\\
8.87	-857.2813\\
8.88	-857.2825\\
8.89	-857.2803\\
8.9	-857.2772\\
8.91	-857.2794\\
8.92	-857.2847\\
8.93	-857.2883\\
8.94	-857.2861\\
8.95	-857.2792\\
8.96	-857.2744\\
8.97	-857.2757\\
8.98	-857.2815\\
8.99	-857.2866\\
9	-857.2879\\
9.01	-857.2837\\
9.02	-857.2792\\
9.03	-857.2787\\
9.04	-857.2811\\
9.05	-857.2836\\
9.06	-857.2867\\
9.07	-857.2871\\
9.08	-857.2833\\
9.09	-857.2793\\
9.1	-857.2796\\
9.11	-857.2838\\
9.12	-857.2837\\
9.13	-857.2808\\
9.14	-857.2758\\
9.15	-857.2784\\
9.16	-857.2849\\
9.17	-857.2875\\
9.18	-857.2846\\
9.19	-857.2798\\
9.2	-857.2745\\
9.21	-857.2769\\
9.22	-857.2837\\
9.23	-857.2883\\
9.24	-857.287\\
9.25	-857.2834\\
9.26	-857.2794\\
9.27	-857.2793\\
9.28	-857.2812\\
9.29	-857.2807\\
9.3	-857.2809\\
9.31	-857.2826\\
9.32	-857.2856\\
9.33	-857.2878\\
9.34	-857.2835\\
9.35	-857.2778\\
9.36	-857.2767\\
9.37	-857.2805\\
9.38	-857.2855\\
9.39	-857.2869\\
9.4	-857.2838\\
9.41	-857.28\\
9.42	-857.279\\
9.43	-857.2812\\
9.44	-857.2865\\
9.45	-857.2861\\
9.46	-857.2814\\
9.47	-857.2778\\
9.48	-857.2795\\
9.49	-857.2827\\
9.5	-857.2857\\
9.51	-857.2858\\
9.52	-857.2829\\
9.53	-857.2804\\
9.54	-857.2785\\
9.55	-857.2819\\
9.56	-857.2874\\
9.57	-857.2868\\
9.58	-857.2827\\
9.59	-857.2778\\
9.6	-857.2796\\
9.61	-857.2795\\
9.62	-857.2825\\
9.63	-857.2852\\
9.64	-857.2858\\
9.65	-857.2836\\
9.66	-857.2801\\
9.67	-857.2775\\
9.68	-857.2799\\
9.69	-857.2832\\
9.7	-857.2845\\
9.71	-857.2838\\
9.72	-857.2835\\
9.73	-857.2836\\
9.74	-857.2844\\
9.75	-857.2815\\
9.76	-857.2775\\
9.77	-857.2796\\
9.78	-857.2852\\
9.79	-857.2864\\
9.8	-857.2832\\
9.81	-857.2809\\
9.82	-857.2797\\
9.83	-857.2807\\
9.84	-857.2807\\
9.85	-857.2792\\
9.86	-857.2827\\
9.87	-857.2851\\
9.88	-857.2831\\
9.89	-857.2809\\
9.9	-857.2817\\
9.91	-857.2829\\
9.92	-857.2836\\
9.93	-857.283\\
9.94	-857.2821\\
9.95	-857.2834\\
9.96	-857.283\\
9.97	-857.283\\
9.98	-857.2844\\
9.99	-857.2827\\
};
\addlegendentry{Total energy};

\addplot [color=mycolor2,solid]
  table[row sep=crcr]{%
0	-857.2121\\
0.01	-857.8086\\
0.02	-858.9914\\
0.03	-859.6548\\
0.04	-859.3671\\
0.05	-858.6332\\
0.06	-858.2333\\
0.07	-858.44\\
0.08	-858.8268\\
0.09	-858.8425\\
0.1	-858.4806\\
0.11	-858.2232\\
0.12	-858.4106\\
0.13	-858.827\\
0.14	-858.988\\
0.15	-858.754\\
0.16	-858.4568\\
0.17	-858.4451\\
0.18	-858.6845\\
0.19	-858.8633\\
0.2	-858.7848\\
0.21	-858.5525\\
0.22	-858.4176\\
0.23	-858.5034\\
0.24	-858.6858\\
0.25	-858.7577\\
0.26	-858.6673\\
0.27	-858.5632\\
0.28	-858.5946\\
0.29	-858.7264\\
0.3	-858.787\\
0.31	-858.6943\\
0.32	-858.5506\\
0.33	-858.5204\\
0.34	-858.622\\
0.35	-858.7234\\
0.36	-858.6965\\
0.37	-858.5725\\
0.38	-858.4712\\
0.39	-858.4841\\
0.4	-858.5956\\
0.41	-858.7382\\
0.42	-858.8465\\
0.43	-858.8648\\
0.44	-858.7735\\
0.45	-858.6097\\
0.46	-858.4774\\
0.47	-858.454\\
0.48	-858.5352\\
0.49	-858.6448\\
0.5	-858.6986\\
0.51	-858.6582\\
0.52	-858.5743\\
0.53	-858.5494\\
0.54	-858.6414\\
0.55	-858.7917\\
0.56	-858.8609\\
0.57	-858.776\\
0.58	-858.6027\\
0.59	-858.4779\\
0.6	-858.4787\\
0.61	-858.5562\\
0.62	-858.6163\\
0.63	-858.6272\\
0.64	-858.6442\\
0.65	-858.7099\\
0.66	-858.7753\\
0.67	-858.7641\\
0.68	-858.6663\\
0.69	-858.5682\\
0.7	-858.5472\\
0.71	-858.5819\\
0.72	-858.5978\\
0.73	-858.5801\\
0.74	-858.5916\\
0.75	-858.6807\\
0.76	-858.7994\\
0.77	-858.8317\\
0.78	-858.7076\\
0.79	-858.5091\\
0.8	-858.416\\
0.81	-858.5132\\
0.82	-858.7038\\
0.83	-858.8094\\
0.84	-858.7593\\
0.85	-858.6306\\
0.86	-858.5344\\
0.87	-858.515\\
0.88	-858.558\\
0.89	-858.6526\\
0.9	-858.7669\\
0.91	-858.8217\\
0.92	-858.7551\\
0.93	-858.6107\\
0.94	-858.5071\\
0.95	-858.5283\\
0.96	-858.6412\\
0.97	-858.7287\\
0.98	-858.6865\\
0.99	-858.5288\\
1	-858.4025\\
1.01	-858.4711\\
1.02	-858.7254\\
1.03	-858.9673\\
1.04	-858.9874\\
1.05	-858.7634\\
1.06	-858.4774\\
1.07	-858.3531\\
1.08	-858.4567\\
1.09	-858.6553\\
1.1	-858.7537\\
1.11	-858.6983\\
1.12	-858.5988\\
1.13	-858.5783\\
1.14	-858.6521\\
1.15	-858.7357\\
1.16	-858.744\\
1.17	-858.67\\
1.18	-858.5754\\
1.19	-858.5299\\
1.2	-858.5623\\
1.21	-858.6527\\
1.22	-858.7532\\
1.23	-858.7876\\
1.24	-858.693\\
1.25	-858.5261\\
1.26	-858.4562\\
1.27	-858.5785\\
1.28	-858.7828\\
1.29	-858.8547\\
1.3	-858.7035\\
1.31	-858.4778\\
1.32	-858.3982\\
1.33	-858.5434\\
1.34	-858.7735\\
1.35	-858.8673\\
1.36	-858.7454\\
1.37	-858.5448\\
1.38	-858.471\\
1.39	-858.5708\\
1.4	-858.7141\\
1.41	-858.7701\\
1.42	-858.7304\\
1.43	-858.6829\\
1.44	-858.6786\\
1.45	-858.6644\\
1.46	-858.5707\\
1.47	-858.4387\\
1.48	-858.4137\\
1.49	-858.5717\\
1.5	-858.7898\\
1.51	-858.8685\\
1.52	-858.7477\\
1.53	-858.594\\
1.54	-858.5902\\
1.55	-858.7168\\
1.56	-858.7785\\
1.57	-858.6395\\
1.58	-858.406\\
1.59	-858.322\\
1.6	-858.517\\
1.61	-858.8468\\
1.62	-859.0326\\
1.63	-858.9342\\
1.64	-858.6455\\
1.65	-858.3893\\
1.66	-858.3315\\
1.67	-858.4666\\
1.68	-858.6551\\
1.69	-858.7697\\
1.7	-858.793\\
1.71	-858.777\\
1.72	-858.7377\\
1.73	-858.6469\\
1.74	-858.5218\\
1.75	-858.4615\\
1.76	-858.5391\\
1.77	-858.7005\\
1.78	-858.7988\\
1.79	-858.7484\\
1.8	-858.6126\\
1.81	-858.5394\\
1.82	-858.5993\\
1.83	-858.703\\
1.84	-858.7284\\
1.85	-858.6512\\
1.86	-858.5586\\
1.87	-858.5411\\
1.88	-858.5988\\
1.89	-858.6626\\
1.9	-858.6845\\
1.91	-858.6847\\
1.92	-858.707\\
1.93	-858.7539\\
1.94	-858.7608\\
1.95	-858.6709\\
1.96	-858.5085\\
1.97	-858.4022\\
1.98	-858.4521\\
1.99	-858.6212\\
2	-858.7718\\
2.01	-858.8215\\
2.02	-858.7919\\
2.03	-858.7428\\
2.04	-858.6735\\
2.05	-858.5599\\
2.06	-858.4376\\
2.07	-858.421\\
2.08	-858.5682\\
2.09	-858.7805\\
2.1	-858.8746\\
2.11	-858.7645\\
2.12	-858.5541\\
2.13	-858.4531\\
2.14	-858.5633\\
2.15	-858.7858\\
2.16	-858.9111\\
2.17	-858.8208\\
2.18	-858.5769\\
2.19	-858.3691\\
2.2	-858.354\\
2.21	-858.529\\
2.22	-858.7583\\
2.23	-858.8832\\
2.24	-858.8328\\
2.25	-858.6614\\
2.26	-858.5153\\
2.27	-858.5091\\
2.28	-858.6185\\
2.29	-858.7127\\
2.3	-858.6968\\
2.31	-858.6036\\
2.32	-858.5522\\
2.33	-858.6218\\
2.34	-858.7573\\
2.35	-858.8089\\
2.36	-858.6918\\
2.37	-858.5059\\
2.38	-858.4403\\
2.39	-858.5708\\
2.4	-858.7784\\
2.41	-858.8723\\
2.42	-858.7733\\
2.43	-858.5765\\
2.44	-858.4457\\
2.45	-858.4649\\
2.46	-858.5779\\
2.47	-858.6681\\
2.48	-858.6949\\
2.49	-858.708\\
2.5	-858.7539\\
2.51	-858.79\\
2.52	-858.745\\
2.53	-858.6248\\
2.54	-858.5227\\
2.55	-858.5271\\
2.56	-858.611\\
2.57	-858.6674\\
2.58	-858.6256\\
2.59	-858.5414\\
2.6	-858.5459\\
2.61	-858.6896\\
2.62	-858.8629\\
2.63	-858.8954\\
2.64	-858.7341\\
2.65	-858.5015\\
2.66	-858.3849\\
2.67	-858.4599\\
2.68	-858.6397\\
2.69	-858.7657\\
2.7	-858.7713\\
2.71	-858.7045\\
2.72	-858.6445\\
2.73	-858.6293\\
2.74	-858.6494\\
2.75	-858.6657\\
2.76	-858.6453\\
2.77	-858.5894\\
2.78	-858.5549\\
2.79	-858.595\\
2.8	-858.6815\\
2.81	-858.7228\\
2.82	-858.6769\\
2.83	-858.5938\\
2.84	-858.5636\\
2.85	-858.6212\\
2.86	-858.7192\\
2.87	-858.7711\\
2.88	-858.7266\\
2.89	-858.6192\\
2.9	-858.5471\\
2.91	-858.5735\\
2.92	-858.6632\\
2.93	-858.7142\\
2.94	-858.681\\
2.95	-858.5962\\
2.96	-858.5453\\
2.97	-858.5671\\
2.98	-858.6352\\
2.99	-858.6923\\
3	-858.717\\
3.01	-858.7193\\
3.02	-858.7144\\
3.03	-858.6903\\
3.04	-858.6258\\
3.05	-858.535\\
3.06	-858.4819\\
3.07	-858.5395\\
3.08	-858.7133\\
3.09	-858.8806\\
3.1	-858.8826\\
3.11	-858.6831\\
3.12	-858.4314\\
3.13	-858.3402\\
3.14	-858.482\\
3.15	-858.7346\\
3.16	-858.8992\\
3.17	-858.8531\\
3.18	-858.648\\
3.19	-858.4704\\
3.2	-858.4773\\
3.21	-858.6339\\
3.22	-858.7626\\
3.23	-858.7475\\
3.24	-858.6363\\
3.25	-858.5597\\
3.26	-858.5869\\
3.27	-858.674\\
3.28	-858.7357\\
3.29	-858.7193\\
3.3	-858.6348\\
3.31	-858.5486\\
3.32	-858.5251\\
3.33	-858.5819\\
3.34	-858.672\\
3.35	-858.7309\\
3.36	-858.7202\\
3.37	-858.6498\\
3.38	-858.5857\\
3.39	-858.601\\
3.4	-858.6911\\
3.41	-858.7651\\
3.42	-858.7352\\
3.43	-858.6234\\
3.44	-858.5204\\
3.45	-858.4903\\
3.46	-858.5341\\
3.47	-858.6222\\
3.48	-858.7224\\
3.49	-858.7916\\
3.5	-858.7899\\
3.51	-858.7089\\
3.52	-858.6015\\
3.53	-858.5445\\
3.54	-858.5653\\
3.55	-858.6221\\
3.56	-858.6542\\
3.57	-858.6359\\
3.58	-858.6102\\
3.59	-858.6284\\
3.6	-858.6953\\
3.61	-858.749\\
3.62	-858.7169\\
3.63	-858.614\\
3.64	-858.5374\\
3.65	-858.5576\\
3.66	-858.6528\\
3.67	-858.7347\\
3.68	-858.7385\\
3.69	-858.6691\\
3.7	-858.5879\\
3.71	-858.5564\\
3.72	-858.5812\\
3.73	-858.6214\\
3.74	-858.6442\\
3.75	-858.6575\\
3.76	-858.685\\
3.77	-858.7154\\
3.78	-858.7123\\
3.79	-858.6703\\
3.8	-858.6294\\
3.81	-858.6193\\
3.82	-858.6286\\
3.83	-858.6312\\
3.84	-858.607\\
3.85	-858.576\\
3.86	-858.583\\
3.87	-858.6555\\
3.88	-858.7606\\
3.89	-858.8106\\
3.9	-858.7437\\
3.91	-858.5947\\
3.92	-858.4731\\
3.93	-858.4635\\
3.94	-858.5548\\
3.95	-858.6635\\
3.96	-858.7323\\
3.97	-858.7539\\
3.98	-858.7487\\
3.99	-858.7248\\
4	-858.6795\\
4.01	-858.6358\\
4.02	-858.61\\
4.03	-858.5943\\
4.04	-858.5871\\
4.05	-858.595\\
4.06	-858.608\\
4.07	-858.6067\\
4.08	-858.6097\\
4.09	-858.6476\\
4.1	-858.6931\\
4.11	-858.6955\\
4.12	-858.6625\\
4.13	-858.6554\\
4.14	-858.7009\\
4.15	-858.7573\\
4.16	-858.7455\\
4.17	-858.6399\\
4.18	-858.5039\\
4.19	-858.4445\\
4.2	-858.5257\\
4.21	-858.6899\\
4.22	-858.8103\\
4.23	-858.7958\\
4.24	-858.6703\\
4.25	-858.5411\\
4.26	-858.512\\
4.27	-858.5823\\
4.28	-858.68\\
4.29	-858.7312\\
4.3	-858.7201\\
4.31	-858.6714\\
4.32	-858.6184\\
4.33	-858.5816\\
4.34	-858.589\\
4.35	-858.6556\\
4.36	-858.7361\\
4.37	-858.7573\\
4.38	-858.6977\\
4.39	-858.603\\
4.4	-858.545\\
4.41	-858.5597\\
4.42	-858.6166\\
4.43	-858.6503\\
4.44	-858.6212\\
4.45	-858.5761\\
4.46	-858.6053\\
4.47	-858.7283\\
4.48	-858.846\\
4.49	-858.8316\\
4.5	-858.6782\\
4.51	-858.5157\\
4.52	-858.476\\
4.53	-858.5659\\
4.54	-858.6758\\
4.55	-858.6898\\
4.56	-858.6096\\
4.57	-858.5509\\
4.58	-858.6107\\
4.59	-858.7557\\
4.6	-858.8393\\
4.61	-858.7632\\
4.62	-858.5847\\
4.63	-858.4555\\
4.64	-858.471\\
4.65	-858.611\\
4.66	-858.7675\\
4.67	-858.8292\\
4.68	-858.7622\\
4.69	-858.6277\\
4.7	-858.5356\\
4.71	-858.5378\\
4.72	-858.6032\\
4.73	-858.6693\\
4.74	-858.6961\\
4.75	-858.6881\\
4.76	-858.6686\\
4.77	-858.6431\\
4.78	-858.6006\\
4.79	-858.5571\\
4.8	-858.5693\\
4.81	-858.6531\\
4.82	-858.7459\\
4.83	-858.7675\\
4.84	-858.6964\\
4.85	-858.5925\\
4.86	-858.5464\\
4.87	-858.5844\\
4.88	-858.66\\
4.89	-858.6926\\
4.9	-858.6605\\
4.91	-858.6179\\
4.92	-858.6213\\
4.93	-858.6718\\
4.94	-858.7198\\
4.95	-858.7191\\
4.96	-858.6584\\
4.97	-858.5969\\
4.98	-858.5936\\
4.99	-858.6367\\
5	-858.6457\\
5.01	-858.5824\\
5.02	-858.5217\\
5.03	-858.5576\\
5.04	-858.6703\\
5.05	-858.7594\\
5.06	-858.762\\
5.07	-858.7148\\
5.08	-858.6737\\
5.09	-858.6501\\
5.1	-858.625\\
5.11	-858.5991\\
5.12	-858.5933\\
5.13	-858.6241\\
5.14	-858.6738\\
5.15	-858.7003\\
5.16	-858.6758\\
5.17	-858.6169\\
5.18	-858.5673\\
5.19	-858.5644\\
5.2	-858.6042\\
5.21	-858.656\\
5.22	-858.6885\\
5.23	-858.697\\
5.24	-858.6772\\
5.25	-858.6415\\
5.26	-858.6126\\
5.27	-858.6199\\
5.28	-858.6697\\
5.29	-858.734\\
5.3	-858.75\\
5.31	-858.692\\
5.32	-858.6042\\
5.33	-858.5482\\
5.34	-858.5504\\
5.35	-858.5886\\
5.36	-858.6245\\
5.37	-858.6363\\
5.38	-858.6316\\
5.39	-858.6459\\
5.4	-858.6854\\
5.41	-858.7174\\
5.42	-858.706\\
5.43	-858.6619\\
5.44	-858.6288\\
5.45	-858.6499\\
5.46	-858.7053\\
5.47	-858.7271\\
5.48	-858.6622\\
5.49	-858.5423\\
5.5	-858.4598\\
5.51	-858.4836\\
5.52	-858.6131\\
5.53	-858.7628\\
5.54	-858.8279\\
5.55	-858.7699\\
5.56	-858.6422\\
5.57	-858.5413\\
5.58	-858.5265\\
5.59	-858.5859\\
5.6	-858.6654\\
5.61	-858.714\\
5.62	-858.7127\\
5.63	-858.6858\\
5.64	-858.6663\\
5.65	-858.6534\\
5.66	-858.6348\\
5.67	-858.6128\\
5.68	-858.6122\\
5.69	-858.6234\\
5.7	-858.63\\
5.71	-858.628\\
5.72	-858.6317\\
5.73	-858.6432\\
5.74	-858.6508\\
5.75	-858.6543\\
5.76	-858.6516\\
5.77	-858.6473\\
5.78	-858.6478\\
5.79	-858.6636\\
5.8	-858.6901\\
5.81	-858.6972\\
5.82	-858.667\\
5.83	-858.5999\\
5.84	-858.5426\\
5.85	-858.5408\\
5.86	-858.6108\\
5.87	-858.7095\\
5.88	-858.7644\\
5.89	-858.7303\\
5.9	-858.6324\\
5.91	-858.5561\\
5.92	-858.5621\\
5.93	-858.6344\\
5.94	-858.6979\\
5.95	-858.6969\\
5.96	-858.6514\\
5.97	-858.625\\
5.98	-858.6456\\
5.99	-858.6761\\
6	-858.6595\\
6.01	-858.5973\\
6.02	-858.5595\\
6.03	-858.6018\\
6.04	-858.6956\\
6.05	-858.7557\\
6.06	-858.7213\\
6.07	-858.6261\\
6.08	-858.5603\\
6.09	-858.5806\\
6.1	-858.6489\\
6.11	-858.6822\\
6.12	-858.6474\\
6.13	-858.5894\\
6.14	-858.5788\\
6.15	-858.6293\\
6.16	-858.695\\
6.17	-858.7263\\
6.18	-858.7124\\
6.19	-858.6724\\
6.2	-858.6322\\
6.21	-858.6075\\
6.22	-858.6026\\
6.23	-858.616\\
6.24	-858.6428\\
6.25	-858.6736\\
6.26	-858.6923\\
6.27	-858.6776\\
6.28	-858.6212\\
6.29	-858.5518\\
6.3	-858.5245\\
6.31	-858.565\\
6.32	-858.6514\\
6.33	-858.7343\\
6.34	-858.7798\\
6.35	-858.7663\\
6.36	-858.6825\\
6.37	-858.5591\\
6.38	-858.4853\\
6.39	-858.5312\\
6.4	-858.671\\
6.41	-858.7895\\
6.42	-858.7934\\
6.43	-858.7062\\
6.44	-858.6213\\
6.45	-858.5964\\
6.46	-858.5979\\
6.47	-858.5729\\
6.48	-858.5314\\
6.49	-858.5383\\
6.5	-858.6162\\
6.51	-858.7107\\
6.52	-858.7554\\
6.53	-858.7395\\
6.54	-858.6987\\
6.55	-858.6829\\
6.56	-858.6818\\
6.57	-858.6419\\
6.58	-858.5619\\
6.59	-858.5197\\
6.6	-858.5687\\
6.61	-858.6646\\
6.62	-858.7137\\
6.63	-858.6905\\
6.64	-858.6452\\
6.65	-858.617\\
6.66	-858.6096\\
6.67	-858.6135\\
6.68	-858.6396\\
6.69	-858.6736\\
6.7	-858.6987\\
6.71	-858.7044\\
6.72	-858.6945\\
6.73	-858.6685\\
6.74	-858.6242\\
6.75	-858.5785\\
6.76	-858.5549\\
6.77	-858.5738\\
6.78	-858.6406\\
6.79	-858.7253\\
6.8	-858.7621\\
6.81	-858.6975\\
6.82	-858.5796\\
6.83	-858.5061\\
6.84	-858.5358\\
6.85	-858.6386\\
6.86	-858.7292\\
6.87	-858.7512\\
6.88	-858.7177\\
6.89	-858.6807\\
6.9	-858.6561\\
6.91	-858.6254\\
6.92	-858.5932\\
6.93	-858.5907\\
6.94	-858.627\\
6.95	-858.6691\\
6.96	-858.6811\\
6.97	-858.6605\\
6.98	-858.6304\\
6.99	-858.6056\\
7	-858.5887\\
7.01	-858.5904\\
7.02	-858.6263\\
7.03	-858.6827\\
7.04	-858.7201\\
7.05	-858.7032\\
7.06	-858.6485\\
7.07	-858.5976\\
7.08	-858.5766\\
7.09	-858.5913\\
7.1	-858.6425\\
7.11	-858.7142\\
7.12	-858.7701\\
7.13	-858.7625\\
7.14	-858.6713\\
7.15	-858.5488\\
7.16	-858.4852\\
7.17	-858.5264\\
7.18	-858.6392\\
7.19	-858.7413\\
7.2	-858.7636\\
7.21	-858.6911\\
7.22	-858.5926\\
7.23	-858.56\\
7.24	-858.6148\\
7.25	-858.6854\\
7.26	-858.6929\\
7.27	-858.646\\
7.28	-858.6116\\
7.29	-858.6304\\
7.3	-858.6592\\
7.31	-858.6577\\
7.32	-858.638\\
7.33	-858.6475\\
7.34	-858.6918\\
7.35	-858.7323\\
7.36	-858.7214\\
7.37	-858.6378\\
7.38	-858.5204\\
7.39	-858.4616\\
7.4	-858.5215\\
7.41	-858.6674\\
7.42	-858.7829\\
7.43	-858.7865\\
7.44	-858.6962\\
7.45	-858.6005\\
7.46	-858.5709\\
7.47	-858.6009\\
7.48	-858.6405\\
7.49	-858.6657\\
7.5	-858.6822\\
7.51	-858.7061\\
7.52	-858.7165\\
7.53	-858.6755\\
7.54	-858.5831\\
7.55	-858.5116\\
7.56	-858.537\\
7.57	-858.6417\\
7.58	-858.7364\\
7.59	-858.7496\\
7.6	-858.6971\\
7.61	-858.6419\\
7.62	-858.6194\\
7.63	-858.6215\\
7.64	-858.632\\
7.65	-858.6447\\
7.66	-858.6533\\
7.67	-858.6446\\
7.68	-858.6185\\
7.69	-858.5975\\
7.7	-858.6065\\
7.71	-858.6441\\
7.72	-858.6923\\
7.73	-858.7226\\
7.74	-858.7178\\
7.75	-858.6755\\
7.76	-858.6147\\
7.77	-858.5686\\
7.78	-858.5731\\
7.79	-858.6218\\
7.8	-858.6732\\
7.81	-858.6881\\
7.82	-858.6798\\
7.83	-858.6806\\
7.84	-858.6918\\
7.85	-858.6875\\
7.86	-858.6461\\
7.87	-858.5864\\
7.88	-858.5524\\
7.89	-858.5724\\
7.9	-858.6241\\
7.91	-858.6604\\
7.92	-858.6671\\
7.93	-858.6609\\
7.94	-858.6521\\
7.95	-858.6415\\
7.96	-858.6392\\
7.97	-858.6633\\
7.98	-858.7083\\
7.99	-858.7325\\
8	-858.696\\
8.01	-858.605\\
8.02	-858.5115\\
8.03	-858.4939\\
8.04	-858.5827\\
8.05	-858.7251\\
8.06	-858.8116\\
8.07	-858.7773\\
8.08	-858.6661\\
8.09	-858.5774\\
8.1	-858.5696\\
8.11	-858.6164\\
8.12	-858.6453\\
8.13	-858.6211\\
8.14	-858.5726\\
8.15	-858.5581\\
8.16	-858.6115\\
8.17	-858.7026\\
8.18	-858.761\\
8.19	-858.7529\\
8.2	-858.691\\
8.21	-858.6301\\
8.22	-858.6107\\
8.23	-858.6273\\
8.24	-858.6509\\
8.25	-858.6513\\
8.26	-858.6325\\
8.27	-858.6077\\
8.28	-858.6074\\
8.29	-858.6451\\
8.3	-858.6906\\
8.31	-858.6943\\
8.32	-858.6402\\
8.33	-858.5697\\
8.34	-858.5397\\
8.35	-858.5857\\
8.36	-858.6825\\
8.37	-858.7563\\
8.38	-858.7431\\
8.39	-858.6534\\
8.4	-858.5786\\
8.41	-858.5885\\
8.42	-858.659\\
8.43	-858.7083\\
8.44	-858.7027\\
8.45	-858.6683\\
8.46	-858.6407\\
8.47	-858.6186\\
8.48	-858.5874\\
8.49	-858.5605\\
8.5	-858.5703\\
8.51	-858.6258\\
8.52	-858.7015\\
8.53	-858.7524\\
8.54	-858.7446\\
8.55	-858.6767\\
8.56	-858.5905\\
8.57	-858.5484\\
8.58	-858.5777\\
8.59	-858.6371\\
8.6	-858.6695\\
8.61	-858.6643\\
8.62	-858.6526\\
8.63	-858.6585\\
8.64	-858.6691\\
8.65	-858.654\\
8.66	-858.6194\\
8.67	-858.5994\\
8.68	-858.631\\
8.69	-858.705\\
8.7	-858.7545\\
8.71	-858.7172\\
8.72	-858.6155\\
8.73	-858.5366\\
8.74	-858.5408\\
8.75	-858.6153\\
8.76	-858.6872\\
8.77	-858.7092\\
8.78	-858.6879\\
8.79	-858.6543\\
8.8	-858.6262\\
8.81	-858.6055\\
8.82	-858.6027\\
8.83	-858.6205\\
8.84	-858.6509\\
8.85	-858.6748\\
8.86	-858.6851\\
8.87	-858.6689\\
8.88	-858.6342\\
8.89	-858.5918\\
8.9	-858.5824\\
8.91	-858.6379\\
8.92	-858.7254\\
8.93	-858.7657\\
8.94	-858.7097\\
8.95	-858.5929\\
8.96	-858.5089\\
8.97	-858.5303\\
8.98	-858.645\\
8.99	-858.7602\\
9	-858.7823\\
9.01	-858.6967\\
9.02	-858.578\\
9.03	-858.5158\\
9.04	-858.544\\
9.05	-858.6314\\
9.06	-858.713\\
9.07	-858.7279\\
9.08	-858.6716\\
9.09	-858.6059\\
9.1	-858.5934\\
9.11	-858.6258\\
9.12	-858.6411\\
9.13	-858.626\\
9.14	-858.6291\\
9.15	-858.6972\\
9.16	-858.7862\\
9.17	-858.8024\\
9.18	-858.7049\\
9.19	-858.5539\\
9.2	-858.458\\
9.21	-858.489\\
9.22	-858.6101\\
9.23	-858.7189\\
9.24	-858.741\\
9.25	-858.685\\
9.26	-858.6105\\
9.27	-858.573\\
9.28	-858.5823\\
9.29	-858.6196\\
9.3	-858.6742\\
9.31	-858.7343\\
9.32	-858.7712\\
9.33	-858.7446\\
9.34	-858.6454\\
9.35	-858.5403\\
9.36	-858.5168\\
9.37	-858.5895\\
9.38	-858.6873\\
9.39	-858.7285\\
9.4	-858.6909\\
9.41	-858.623\\
9.42	-858.5951\\
9.43	-858.6273\\
9.44	-858.6698\\
9.45	-858.6568\\
9.46	-858.5987\\
9.47	-858.5648\\
9.48	-858.6006\\
9.49	-858.6791\\
9.5	-858.7419\\
9.51	-858.7442\\
9.52	-858.6845\\
9.53	-858.6108\\
9.54	-858.5831\\
9.55	-858.6229\\
9.56	-858.6793\\
9.57	-858.6833\\
9.58	-858.6279\\
9.59	-858.5627\\
9.6	-858.5485\\
9.61	-858.596\\
9.62	-858.6897\\
9.63	-858.7734\\
9.64	-858.7889\\
9.65	-858.7199\\
9.66	-858.6114\\
9.67	-858.5356\\
9.68	-858.5382\\
9.69	-858.5967\\
9.7	-858.6554\\
9.71	-858.6844\\
9.72	-858.6926\\
9.73	-858.685\\
9.74	-858.6524\\
9.75	-858.5997\\
9.76	-858.5746\\
9.77	-858.6182\\
9.78	-858.698\\
9.79	-858.7343\\
9.8	-858.6954\\
9.81	-858.6269\\
9.82	-858.5848\\
9.83	-858.5838\\
9.84	-858.6062\\
9.85	-858.644\\
9.86	-858.6925\\
9.87	-858.7143\\
9.88	-858.6882\\
9.89	-858.6454\\
9.9	-858.6269\\
9.91	-858.6331\\
9.92	-858.6396\\
9.93	-858.633\\
9.94	-858.6228\\
9.95	-858.6258\\
9.96	-858.6413\\
9.97	-858.6578\\
9.98	-858.6533\\
9.99	-858.6194\\
};
\addlegendentry{Potential energy};

\addplot [color=mycolor3,solid]
  table[row sep=crcr]{%
0	0\\
0.01	0.5572\\
0.02	1.6722\\
0.03	2.3188\\
0.04	2.0705\\
0.05	1.3806\\
0.06	0.9903\\
0.07	1.1666\\
0.08	1.5157\\
0.09	1.5321\\
0.1	1.2029\\
0.11	0.9673\\
0.12	1.1408\\
0.13	1.5255\\
0.14	1.6792\\
0.15	1.4688\\
0.16	1.1938\\
0.17	1.1781\\
0.18	1.3969\\
0.19	1.5642\\
0.2	1.4921\\
0.21	1.2742\\
0.22	1.1466\\
0.23	1.2245\\
0.24	1.3942\\
0.25	1.4632\\
0.26	1.383\\
0.27	1.2892\\
0.28	1.3193\\
0.29	1.4399\\
0.3	1.495\\
0.31	1.4084\\
0.32	1.2767\\
0.33	1.2464\\
0.34	1.3379\\
0.35	1.4298\\
0.36	1.4075\\
0.37	1.29\\
0.38	1.1942\\
0.39	1.2057\\
0.4	1.3147\\
0.41	1.4556\\
0.42	1.5614\\
0.43	1.5796\\
0.44	1.4881\\
0.45	1.3304\\
0.46	1.202\\
0.47	1.1774\\
0.48	1.2533\\
0.49	1.3581\\
0.5	1.409\\
0.51	1.3721\\
0.52	1.2943\\
0.53	1.2733\\
0.54	1.3637\\
0.55	1.5059\\
0.56	1.5711\\
0.57	1.4895\\
0.58	1.3229\\
0.59	1.2019\\
0.6	1.1987\\
0.61	1.2709\\
0.62	1.3287\\
0.63	1.3437\\
0.64	1.365\\
0.65	1.428\\
0.66	1.4892\\
0.67	1.4764\\
0.68	1.3847\\
0.69	1.2924\\
0.7	1.2691\\
0.71	1.2968\\
0.72	1.3112\\
0.73	1.2982\\
0.74	1.3133\\
0.75	1.4002\\
0.76	1.5131\\
0.77	1.5403\\
0.78	1.4199\\
0.79	1.2324\\
0.8	1.1442\\
0.81	1.2357\\
0.82	1.4146\\
0.83	1.5171\\
0.84	1.473\\
0.85	1.3513\\
0.86	1.2572\\
0.87	1.2345\\
0.88	1.2772\\
0.89	1.3719\\
0.9	1.4824\\
0.91	1.533\\
0.92	1.4688\\
0.93	1.3305\\
0.94	1.2318\\
0.95	1.251\\
0.96	1.3572\\
0.97	1.4376\\
0.98	1.3953\\
0.99	1.2459\\
1	1.1308\\
1.01	1.1996\\
1.02	1.4435\\
1.03	1.6748\\
1.04	1.695\\
1.05	1.4804\\
1.06	1.2053\\
1.07	1.0835\\
1.08	1.1795\\
1.09	1.365\\
1.1	1.4603\\
1.11	1.4127\\
1.12	1.3211\\
1.13	1.3024\\
1.14	1.3715\\
1.15	1.4498\\
1.16	1.4578\\
1.17	1.3869\\
1.18	1.2955\\
1.19	1.2515\\
1.2	1.2829\\
1.21	1.372\\
1.22	1.4692\\
1.23	1.4984\\
1.24	1.4064\\
1.25	1.2504\\
1.26	1.186\\
1.27	1.3021\\
1.28	1.4935\\
1.29	1.5577\\
1.3	1.4167\\
1.31	1.2049\\
1.32	1.1308\\
1.33	1.2682\\
1.34	1.4848\\
1.35	1.573\\
1.36	1.4581\\
1.37	1.2704\\
1.38	1.1989\\
1.39	1.2905\\
1.4	1.4274\\
1.41	1.4838\\
1.42	1.451\\
1.43	1.407\\
1.44	1.3974\\
1.45	1.3768\\
1.46	1.2841\\
1.47	1.1608\\
1.48	1.1412\\
1.49	1.2929\\
1.5	1.5014\\
1.51	1.5754\\
1.52	1.4651\\
1.53	1.3224\\
1.54	1.3168\\
1.55	1.4307\\
1.56	1.4822\\
1.57	1.3496\\
1.58	1.1303\\
1.59	1.0554\\
1.6	1.2428\\
1.61	1.5569\\
1.62	1.7364\\
1.63	1.6438\\
1.64	1.367\\
1.65	1.1186\\
1.66	1.0592\\
1.67	1.1861\\
1.68	1.3681\\
1.69	1.4822\\
1.7	1.5096\\
1.71	1.4955\\
1.72	1.4537\\
1.73	1.3618\\
1.74	1.2428\\
1.75	1.1875\\
1.76	1.2616\\
1.77	1.4132\\
1.78	1.5076\\
1.79	1.4619\\
1.8	1.335\\
1.81	1.266\\
1.82	1.3184\\
1.83	1.415\\
1.84	1.4394\\
1.85	1.3677\\
1.86	1.2808\\
1.87	1.2628\\
1.88	1.3163\\
1.89	1.378\\
1.9	1.4031\\
1.91	1.4078\\
1.92	1.4306\\
1.93	1.4704\\
1.94	1.4721\\
1.95	1.3817\\
1.96	1.2276\\
1.97	1.1274\\
1.98	1.1753\\
1.99	1.336\\
2	1.4844\\
2.01	1.5379\\
2.02	1.5138\\
2.03	1.4631\\
2.04	1.3907\\
2.05	1.2768\\
2.06	1.1608\\
2.07	1.1467\\
2.08	1.2878\\
2.09	1.4908\\
2.1	1.5815\\
2.11	1.478\\
2.12	1.2811\\
2.13	1.1842\\
2.14	1.2881\\
2.15	1.4988\\
2.16	1.6182\\
2.17	1.5297\\
2.18	1.295\\
2.19	1.0952\\
2.2	1.0792\\
2.21	1.2484\\
2.22	1.4708\\
2.23	1.5935\\
2.24	1.5461\\
2.25	1.3828\\
2.26	1.243\\
2.27	1.2351\\
2.28	1.3357\\
2.29	1.4237\\
2.3	1.41\\
2.31	1.3238\\
2.32	1.2778\\
2.33	1.3462\\
2.34	1.473\\
2.35	1.5168\\
2.36	1.4049\\
2.37	1.2305\\
2.38	1.1692\\
2.39	1.2928\\
2.4	1.4896\\
2.41	1.5788\\
2.42	1.4856\\
2.43	1.298\\
2.44	1.1713\\
2.45	1.1863\\
2.46	1.2918\\
2.47	1.3808\\
2.48	1.4117\\
2.49	1.4297\\
2.5	1.4735\\
2.51	1.5055\\
2.52	1.4601\\
2.53	1.3433\\
2.54	1.2461\\
2.55	1.2481\\
2.56	1.3255\\
2.57	1.3773\\
2.58	1.3396\\
2.59	1.2652\\
2.6	1.2737\\
2.61	1.4107\\
2.62	1.5735\\
2.63	1.6033\\
2.64	1.4486\\
2.65	1.2252\\
2.66	1.1111\\
2.67	1.1821\\
2.68	1.3526\\
2.69	1.4757\\
2.7	1.4849\\
2.71	1.4233\\
2.72	1.3656\\
2.73	1.3494\\
2.74	1.3667\\
2.75	1.3826\\
2.76	1.3613\\
2.77	1.3072\\
2.78	1.2762\\
2.79	1.3151\\
2.8	1.3953\\
2.81	1.4349\\
2.82	1.3925\\
2.83	1.3158\\
2.84	1.2877\\
2.85	1.3416\\
2.86	1.4342\\
2.87	1.484\\
2.88	1.4416\\
2.89	1.3404\\
2.9	1.2717\\
2.91	1.295\\
2.92	1.3769\\
2.93	1.4271\\
2.94	1.3945\\
2.95	1.3151\\
2.96	1.266\\
2.97	1.2867\\
2.98	1.3517\\
2.99	1.4098\\
3	1.4353\\
3.01	1.4384\\
3.02	1.4321\\
3.03	1.4059\\
3.04	1.3422\\
3.05	1.2551\\
3.06	1.2064\\
3.07	1.2657\\
3.08	1.4317\\
3.09	1.5892\\
3.1	1.5882\\
3.11	1.3971\\
3.12	1.1591\\
3.13	1.0713\\
3.14	1.2056\\
3.15	1.4483\\
3.16	1.6062\\
3.17	1.5627\\
3.18	1.3679\\
3.19	1.2004\\
3.2	1.205\\
3.21	1.3498\\
3.22	1.472\\
3.23	1.4604\\
3.24	1.3573\\
3.25	1.2844\\
3.26	1.3077\\
3.27	1.3899\\
3.28	1.449\\
3.29	1.4331\\
3.3	1.3531\\
3.31	1.27\\
3.32	1.2476\\
3.33	1.3014\\
3.34	1.3885\\
3.35	1.4462\\
3.36	1.4356\\
3.37	1.3682\\
3.38	1.3081\\
3.39	1.3228\\
3.4	1.4073\\
3.41	1.4752\\
3.42	1.4482\\
3.43	1.3419\\
3.44	1.2407\\
3.45	1.2087\\
3.46	1.2513\\
3.47	1.3406\\
3.48	1.4407\\
3.49	1.5082\\
3.5	1.5042\\
3.51	1.4256\\
3.52	1.3231\\
3.53	1.2672\\
3.54	1.2849\\
3.55	1.3383\\
3.56	1.3678\\
3.57	1.3534\\
3.58	1.331\\
3.59	1.3497\\
3.6	1.4125\\
3.61	1.4606\\
3.62	1.43\\
3.63	1.3336\\
3.64	1.2603\\
3.65	1.2797\\
3.66	1.3701\\
3.67	1.4488\\
3.68	1.4534\\
3.69	1.3883\\
3.7	1.3111\\
3.71	1.279\\
3.72	1.3001\\
3.73	1.338\\
3.74	1.3616\\
3.75	1.3773\\
3.76	1.4043\\
3.77	1.432\\
3.78	1.4283\\
3.79	1.3893\\
3.8	1.3502\\
3.81	1.339\\
3.82	1.3465\\
3.83	1.3454\\
3.84	1.3224\\
3.85	1.2954\\
3.86	1.305\\
3.87	1.3772\\
3.88	1.4783\\
3.89	1.5247\\
3.9	1.4587\\
3.91	1.315\\
3.92	1.1977\\
3.93	1.1868\\
3.94	1.2717\\
3.95	1.3784\\
3.96	1.4473\\
3.97	1.4709\\
3.98	1.4666\\
3.99	1.4416\\
4	1.399\\
4.01	1.3565\\
4.02	1.3285\\
4.03	1.3113\\
4.04	1.3036\\
4.05	1.3114\\
4.06	1.3236\\
4.07	1.3242\\
4.08	1.3307\\
4.09	1.3669\\
4.1	1.4084\\
4.11	1.4112\\
4.12	1.3831\\
4.13	1.3785\\
4.14	1.4223\\
4.15	1.472\\
4.16	1.4578\\
4.17	1.3556\\
4.18	1.2252\\
4.19	1.1708\\
4.2	1.249\\
4.21	1.4063\\
4.22	1.521\\
4.23	1.5083\\
4.24	1.3881\\
4.25	1.2648\\
4.26	1.2336\\
4.27	1.3007\\
4.28	1.3945\\
4.29	1.4455\\
4.3	1.4367\\
4.31	1.3909\\
4.32	1.3375\\
4.33	1.3019\\
4.34	1.3117\\
4.35	1.3756\\
4.36	1.4511\\
4.37	1.4718\\
4.38	1.4145\\
4.39	1.323\\
4.4	1.2664\\
4.41	1.2794\\
4.42	1.3335\\
4.43	1.3643\\
4.44	1.3383\\
4.45	1.2987\\
4.46	1.3289\\
4.47	1.4473\\
4.48	1.5579\\
4.49	1.5428\\
4.5	1.3972\\
4.51	1.2418\\
4.52	1.2012\\
4.53	1.2843\\
4.54	1.3865\\
4.55	1.4016\\
4.56	1.3298\\
4.57	1.2763\\
4.58	1.3346\\
4.59	1.4711\\
4.6	1.5483\\
4.61	1.4751\\
4.62	1.3063\\
4.63	1.182\\
4.64	1.1963\\
4.65	1.3301\\
4.66	1.4807\\
4.67	1.5419\\
4.68	1.4782\\
4.69	1.3499\\
4.7	1.259\\
4.71	1.2583\\
4.72	1.3206\\
4.73	1.3846\\
4.74	1.4113\\
4.75	1.4046\\
4.76	1.3873\\
4.77	1.3625\\
4.78	1.3196\\
4.79	1.2798\\
4.8	1.2932\\
4.81	1.3726\\
4.82	1.4601\\
4.83	1.4787\\
4.84	1.4101\\
4.85	1.3129\\
4.86	1.2681\\
4.87	1.3051\\
4.88	1.3754\\
4.89	1.4061\\
4.9	1.378\\
4.91	1.3391\\
4.92	1.343\\
4.93	1.3907\\
4.94	1.4362\\
4.95	1.4328\\
4.96	1.3768\\
4.97	1.3189\\
4.98	1.3149\\
4.99	1.3524\\
5	1.3573\\
5.01	1.2984\\
5.02	1.2452\\
5.03	1.2804\\
5.04	1.3874\\
5.05	1.4719\\
5.06	1.4778\\
5.07	1.434\\
5.08	1.3928\\
5.09	1.3675\\
5.1	1.3428\\
5.11	1.3175\\
5.12	1.312\\
5.13	1.341\\
5.14	1.3881\\
5.15	1.4137\\
5.16	1.3913\\
5.17	1.3345\\
5.18	1.2871\\
5.19	1.2836\\
5.2	1.3217\\
5.21	1.3716\\
5.22	1.4055\\
5.23	1.4126\\
5.24	1.394\\
5.25	1.3597\\
5.26	1.3321\\
5.27	1.3394\\
5.28	1.3898\\
5.29	1.4499\\
5.3	1.4642\\
5.31	1.4098\\
5.32	1.3248\\
5.33	1.2694\\
5.34	1.2692\\
5.35	1.3053\\
5.36	1.3404\\
5.37	1.3521\\
5.38	1.3512\\
5.39	1.3663\\
5.4	1.4031\\
5.41	1.432\\
5.42	1.4225\\
5.43	1.3813\\
5.44	1.3518\\
5.45	1.3711\\
5.46	1.4219\\
5.47	1.4387\\
5.48	1.3755\\
5.49	1.2615\\
5.5	1.1818\\
5.51	1.2061\\
5.52	1.3308\\
5.53	1.4746\\
5.54	1.5384\\
5.55	1.4837\\
5.56	1.362\\
5.57	1.2645\\
5.58	1.2482\\
5.59	1.3046\\
5.6	1.3827\\
5.61	1.4299\\
5.62	1.4291\\
5.63	1.404\\
5.64	1.3838\\
5.65	1.3707\\
5.66	1.3514\\
5.67	1.3314\\
5.68	1.329\\
5.69	1.3399\\
5.7	1.3459\\
5.71	1.3455\\
5.72	1.3501\\
5.73	1.361\\
5.74	1.3699\\
5.75	1.3727\\
5.76	1.3703\\
5.77	1.3654\\
5.78	1.3662\\
5.79	1.3818\\
5.8	1.4058\\
5.81	1.4132\\
5.82	1.3817\\
5.83	1.3174\\
5.84	1.2619\\
5.85	1.262\\
5.86	1.329\\
5.87	1.4236\\
5.88	1.4767\\
5.89	1.4441\\
5.9	1.3527\\
5.91	1.2804\\
5.92	1.2848\\
5.93	1.3515\\
5.94	1.4117\\
5.95	1.4127\\
5.96	1.3712\\
5.97	1.346\\
5.98	1.3644\\
5.99	1.3906\\
6	1.3737\\
6.01	1.317\\
6.02	1.2828\\
6.03	1.3227\\
6.04	1.4126\\
6.05	1.4693\\
6.06	1.4366\\
6.07	1.3458\\
6.08	1.2834\\
6.09	1.301\\
6.1	1.3637\\
6.11	1.3946\\
6.12	1.3618\\
6.13	1.3086\\
6.14	1.2992\\
6.15	1.3476\\
6.16	1.4116\\
6.17	1.4441\\
6.18	1.4312\\
6.19	1.3912\\
6.2	1.3511\\
6.21	1.3262\\
6.22	1.3204\\
6.23	1.3339\\
6.24	1.3614\\
6.25	1.3913\\
6.26	1.4073\\
6.27	1.3911\\
6.28	1.3376\\
6.29	1.2729\\
6.3	1.246\\
6.31	1.2854\\
6.32	1.3686\\
6.33	1.4501\\
6.34	1.4957\\
6.35	1.4818\\
6.36	1.3981\\
6.37	1.2797\\
6.38	1.2103\\
6.39	1.2558\\
6.4	1.3895\\
6.41	1.5023\\
6.42	1.5081\\
6.43	1.4259\\
6.44	1.344\\
6.45	1.3157\\
6.46	1.3133\\
6.47	1.2883\\
6.48	1.2511\\
6.49	1.2593\\
6.5	1.3332\\
6.51	1.4245\\
6.52	1.4712\\
6.53	1.4572\\
6.54	1.4202\\
6.55	1.4031\\
6.56	1.3966\\
6.57	1.3551\\
6.58	1.28\\
6.59	1.2414\\
6.6	1.2889\\
6.61	1.3786\\
6.62	1.4267\\
6.63	1.4081\\
6.64	1.3645\\
6.65	1.3352\\
6.66	1.3254\\
6.67	1.3318\\
6.68	1.3568\\
6.69	1.3911\\
6.7	1.4157\\
6.71	1.4217\\
6.72	1.4113\\
6.73	1.384\\
6.74	1.3409\\
6.75	1.2967\\
6.76	1.2747\\
6.77	1.2943\\
6.78	1.3595\\
6.79	1.4401\\
6.8	1.4714\\
6.81	1.4108\\
6.82	1.2983\\
6.83	1.2282\\
6.84	1.2574\\
6.85	1.3556\\
6.86	1.4426\\
6.87	1.4644\\
6.88	1.4347\\
6.89	1.3985\\
6.9	1.3723\\
6.91	1.3433\\
6.92	1.3137\\
6.93	1.311\\
6.94	1.3447\\
6.95	1.3852\\
6.96	1.3984\\
6.97	1.3795\\
6.98	1.3486\\
6.99	1.3225\\
7	1.3065\\
7.01	1.3105\\
7.02	1.3459\\
7.03	1.3999\\
7.04	1.4338\\
7.05	1.4187\\
7.06	1.3671\\
7.07	1.3174\\
7.08	1.297\\
7.09	1.3122\\
7.1	1.3619\\
7.11	1.4326\\
7.12	1.4869\\
7.13	1.4769\\
7.14	1.3878\\
7.15	1.2699\\
7.16	1.2086\\
7.17	1.248\\
7.18	1.3558\\
7.19	1.4549\\
7.2	1.4763\\
7.21	1.4091\\
7.22	1.3167\\
7.23	1.2844\\
7.24	1.3331\\
7.25	1.3976\\
7.26	1.4057\\
7.27	1.3629\\
7.28	1.3332\\
7.29	1.3481\\
7.3	1.3738\\
7.31	1.3731\\
7.32	1.3589\\
7.33	1.3694\\
7.34	1.4116\\
7.35	1.4495\\
7.36	1.4361\\
7.37	1.3535\\
7.38	1.2417\\
7.39	1.1857\\
7.4	1.2444\\
7.41	1.3833\\
7.42	1.4951\\
7.43	1.4995\\
7.44	1.4141\\
7.45	1.3235\\
7.46	1.2922\\
7.47	1.3171\\
7.48	1.3553\\
7.49	1.3805\\
7.5	1.3999\\
7.51	1.4237\\
7.52	1.4317\\
7.53	1.3892\\
7.54	1.3014\\
7.55	1.2361\\
7.56	1.2602\\
7.57	1.36\\
7.58	1.45\\
7.59	1.4638\\
7.6	1.4151\\
7.61	1.3616\\
7.62	1.3378\\
7.63	1.3387\\
7.64	1.3496\\
7.65	1.3625\\
7.66	1.3688\\
7.67	1.3589\\
7.68	1.3359\\
7.69	1.3181\\
7.7	1.3263\\
7.71	1.3637\\
7.72	1.4104\\
7.73	1.4398\\
7.74	1.435\\
7.75	1.3937\\
7.76	1.3332\\
7.77	1.2891\\
7.78	1.2925\\
7.79	1.3383\\
7.8	1.3865\\
7.81	1.4043\\
7.82	1.3992\\
7.83	1.3993\\
7.84	1.4094\\
7.85	1.4037\\
7.86	1.3628\\
7.87	1.3046\\
7.88	1.2716\\
7.89	1.2895\\
7.9	1.3385\\
7.91	1.3754\\
7.92	1.3838\\
7.93	1.3784\\
7.94	1.3706\\
7.95	1.3611\\
7.96	1.36\\
7.97	1.384\\
7.98	1.4256\\
7.99	1.4465\\
8	1.4103\\
8.01	1.3214\\
8.02	1.2337\\
8.03	1.2181\\
8.04	1.3047\\
8.05	1.4412\\
8.06	1.5234\\
8.07	1.4915\\
8.08	1.3858\\
8.09	1.3004\\
8.1	1.2907\\
8.11	1.332\\
8.12	1.3591\\
8.13	1.3369\\
8.14	1.2918\\
8.15	1.2808\\
8.16	1.3333\\
8.17	1.4187\\
8.18	1.4761\\
8.19	1.468\\
8.2	1.4098\\
8.21	1.3514\\
8.22	1.3307\\
8.23	1.3455\\
8.24	1.3668\\
8.25	1.3694\\
8.26	1.3501\\
8.27	1.327\\
8.28	1.328\\
8.29	1.3636\\
8.3	1.4057\\
8.31	1.4087\\
8.32	1.3577\\
8.33	1.2891\\
8.34	1.2617\\
8.35	1.3073\\
8.36	1.4004\\
8.37	1.4706\\
8.38	1.4571\\
8.39	1.3728\\
8.4	1.3018\\
8.41	1.3102\\
8.42	1.3749\\
8.43	1.4235\\
8.44	1.4207\\
8.45	1.3887\\
8.46	1.3597\\
8.47	1.3355\\
8.48	1.3046\\
8.49	1.2791\\
8.5	1.2889\\
8.51	1.3436\\
8.52	1.4176\\
8.53	1.4681\\
8.54	1.461\\
8.55	1.395\\
8.56	1.3117\\
8.57	1.2709\\
8.58	1.2969\\
8.59	1.3532\\
8.6	1.3858\\
8.61	1.3817\\
8.62	1.3709\\
8.63	1.377\\
8.64	1.3854\\
8.65	1.3707\\
8.66	1.337\\
8.67	1.3199\\
8.68	1.3525\\
8.69	1.4232\\
8.7	1.4682\\
8.71	1.4316\\
8.72	1.3346\\
8.73	1.2582\\
8.74	1.2618\\
8.75	1.3311\\
8.76	1.4013\\
8.77	1.4249\\
8.78	1.4067\\
8.79	1.3747\\
8.8	1.3447\\
8.81	1.3236\\
8.82	1.3206\\
8.83	1.3389\\
8.84	1.3684\\
8.85	1.393\\
8.86	1.4014\\
8.87	1.3876\\
8.88	1.3517\\
8.89	1.3115\\
8.9	1.3052\\
8.91	1.3585\\
8.92	1.4407\\
8.93	1.4774\\
8.94	1.4236\\
8.95	1.3137\\
8.96	1.2345\\
8.97	1.2546\\
8.98	1.3635\\
8.99	1.4736\\
9	1.4944\\
9.01	1.413\\
9.02	1.2988\\
9.03	1.2371\\
9.04	1.2629\\
9.05	1.3478\\
9.06	1.4263\\
9.07	1.4408\\
9.08	1.3883\\
9.09	1.3266\\
9.1	1.3138\\
9.11	1.342\\
9.12	1.3574\\
9.13	1.3452\\
9.14	1.3533\\
9.15	1.4188\\
9.16	1.5013\\
9.17	1.5149\\
9.18	1.4203\\
9.19	1.2741\\
9.2	1.1835\\
9.21	1.2121\\
9.22	1.3264\\
9.23	1.4306\\
9.24	1.454\\
9.25	1.4016\\
9.26	1.3311\\
9.27	1.2937\\
9.28	1.3011\\
9.29	1.3389\\
9.3	1.3933\\
9.31	1.4517\\
9.32	1.4856\\
9.33	1.4568\\
9.34	1.3619\\
9.35	1.2625\\
9.36	1.2401\\
9.37	1.309\\
9.38	1.4018\\
9.39	1.4416\\
9.4	1.4071\\
9.41	1.343\\
9.42	1.3161\\
9.43	1.3461\\
9.44	1.3833\\
9.45	1.3707\\
9.46	1.3173\\
9.47	1.287\\
9.48	1.3211\\
9.49	1.3964\\
9.5	1.4562\\
9.51	1.4584\\
9.52	1.4016\\
9.53	1.3304\\
9.54	1.3046\\
9.55	1.341\\
9.56	1.3919\\
9.57	1.3965\\
9.58	1.3452\\
9.59	1.2849\\
9.6	1.2689\\
9.61	1.3165\\
9.62	1.4072\\
9.63	1.4882\\
9.64	1.5031\\
9.65	1.4363\\
9.66	1.3313\\
9.67	1.2581\\
9.68	1.2583\\
9.69	1.3135\\
9.7	1.3709\\
9.71	1.4006\\
9.72	1.4091\\
9.73	1.4014\\
9.74	1.368\\
9.75	1.3182\\
9.76	1.2971\\
9.77	1.3386\\
9.78	1.4128\\
9.79	1.4479\\
9.8	1.4122\\
9.81	1.346\\
9.82	1.3051\\
9.83	1.3031\\
9.84	1.3255\\
9.85	1.3648\\
9.86	1.4098\\
9.87	1.4292\\
9.88	1.4051\\
9.89	1.3645\\
9.9	1.3452\\
9.91	1.3502\\
9.92	1.356\\
9.93	1.35\\
9.94	1.3407\\
9.95	1.3424\\
9.96	1.3583\\
9.97	1.3748\\
9.98	1.3689\\
9.99	1.3367\\
};
\addlegendentry{Kinetic energy};

\end{axis}
\end{tikzpicture}%}
        \caption{$\Delta t =\unit[0.01]{ASU}$}
        \label{fig:timestep_b}
    \end{subfigure}
    \caption{For the different energy simulations, the same number of timesteps was used but the lengths of the different timesteps makes them evolve over different times. As can be seen in \ref{fig:timestep_a}, the energy explodes due to insufficient resolution of the time, something which is not present in \ref{fig:timestep_b}.}
    \label{fig:timestep}
\end{figure}

As we can see in figure \ref{fig:timestep} the required timestep is between $\Delta t = 0.1 \sim \unit[0.01]{ASU}$, so for the rest of the assignment a timestep of $\Delta t = \unit[0.01]{ASU}$ will be used.

\section*{Problem 3}

\begin{figure}[H]
    \centering
    \captionsetup[subfigure]{justification=centering}
    \begin{subfigure}[b]{0.40\textwidth}
        \centering
        \resizebox{\columnwidth}{!}{% This file was created by matlab2tikz.
%
%The latest updates can be retrieved from
%  http://www.mathworks.com/matlabcentral/fileexchange/22022-matlab2tikz-matlab2tikz
%where you can also make suggestions and rate matlab2tikz.
%
\definecolor{mycolor1}{rgb}{0.00000,0.44700,0.74100}%
%
\begin{tikzpicture}

\begin{axis}[%
width=4.521in,
height=3.566in,
at={(0.758in,0.481in)},
scale only axis,
xmin=0,
xmax=0.2,
xlabel={Time / [$\unit{ps}$]},
ymin=0,
ymax=350,
ylabel={Average pressure / [$\unit{eV/A}$]},
label style ={font=\Large},
axis background/.style={fill=white},
title style={font=\bfseries\Huge},
title={Average pressure}
]
\addplot [color=mycolor1,solid,forget plot]
  table[row sep=crcr]{%
0	49.54005\\
0	311.7985\\
0	233.8703\\
0	207.9031\\
0	194.9256\\
0.001	187.1452\\
0.001	181.9628\\
0.001	178.2651\\
0.001	175.4948\\
0.001	173.3433\\
0.001	171.6246\\
0.001	170.221\\
0.001	169.0536\\
0.001	168.0681\\
0.001	167.2257\\
0.002	166.4976\\
0.002	165.8621\\
0.002	165.3032\\
0.002	164.808\\
0.002	164.3664\\
0.002	163.9706\\
0.002	163.6138\\
0.002	163.2909\\
0.002	162.9973\\
0.002	162.7295\\
0.003	162.4843\\
0.003	162.259\\
0.003	162.0515\\
0.003	161.8597\\
0.003	161.6821\\
0.003	161.5173\\
0.003	161.364\\
0.003	161.2211\\
0.003	161.0876\\
0.003	160.9629\\
0.004	160.846\\
0.004	160.7363\\
0.004	160.6333\\
0.004	160.5364\\
0.004	160.4451\\
0.004	160.359\\
0.004	160.2778\\
0.004	160.201\\
0.004	160.1284\\
0.004	160.0597\\
0.005	159.9945\\
0.005	159.9326\\
0.005	159.8738\\
0.005	159.818\\
0.005	159.7648\\
0.005	159.7143\\
0.005	159.6663\\
0.005	159.6205\\
0.005	159.5768\\
0.005	159.5351\\
0.006	159.4953\\
0.006	159.4572\\
0.006	159.4209\\
0.006	159.3862\\
0.006	159.3529\\
0.006	159.3211\\
0.006	159.2907\\
0.006	159.2615\\
0.006	159.2334\\
0.006	159.2065\\
0.007	159.1806\\
0.007	159.1557\\
0.007	159.1319\\
0.007	159.1089\\
0.007	159.0868\\
0.007	159.0656\\
0.007	159.0451\\
0.007	159.0255\\
0.007	159.0066\\
0.007	158.9884\\
0.008	158.9708\\
0.008	158.9539\\
0.008	158.9376\\
0.008	158.9219\\
0.008	158.9067\\
0.008	158.892\\
0.008	158.8777\\
0.008	158.8641\\
0.008	158.8508\\
0.008	158.838\\
0.009	158.8257\\
0.009	158.8137\\
0.009	158.8021\\
0.009	158.7908\\
0.009	158.7799\\
0.009	158.7694\\
0.009	158.7591\\
0.009	158.7491\\
0.009	158.7394\\
0.009	158.73\\
0.009	158.7208\\
0.01	158.7119\\
0.01	158.7031\\
0.01	158.6946\\
0.01	158.6862\\
0.01	158.6781\\
0.01	158.6702\\
0.01	158.6624\\
0.01	158.6549\\
0.01	158.6475\\
0.011	158.6402\\
0.011	158.6331\\
0.011	158.6261\\
0.011	158.6192\\
0.011	158.6124\\
0.011	158.6058\\
0.011	158.5993\\
0.011	158.5928\\
0.011	158.5865\\
0.011	158.5803\\
0.011	158.5742\\
0.012	158.5682\\
0.012	158.5622\\
0.012	158.5564\\
0.012	158.5506\\
0.012	158.5449\\
0.012	158.5393\\
0.012	158.5337\\
0.012	158.5282\\
0.012	158.5228\\
0.013	158.5174\\
0.013	158.5121\\
0.013	158.5068\\
0.013	158.5016\\
0.013	158.4965\\
0.013	158.4914\\
0.013	158.4863\\
0.013	158.4813\\
0.013	158.4763\\
0.013	158.4713\\
0.013	158.4663\\
0.014	158.4614\\
0.014	158.4565\\
0.014	158.4516\\
0.014	158.4467\\
0.014	158.4418\\
0.014	158.4369\\
0.014	158.432\\
0.014	158.4272\\
0.014	158.4223\\
0.015	158.4175\\
0.015	158.4127\\
0.015	158.4079\\
0.015	158.403\\
0.015	158.3982\\
0.015	158.3934\\
0.015	158.3886\\
0.015	158.3838\\
0.015	158.379\\
0.015	158.3742\\
0.015	158.3694\\
0.016	158.3646\\
0.016	158.3598\\
0.016	158.355\\
0.016	158.3502\\
0.016	158.3455\\
0.016	158.3407\\
0.016	158.3359\\
0.016	158.3311\\
0.016	158.3262\\
0.017	158.3214\\
0.017	158.3166\\
0.017	158.3117\\
0.017	158.3069\\
0.017	158.302\\
0.017	158.2972\\
0.017	158.2923\\
0.017	158.2875\\
0.017	158.2826\\
0.017	158.2777\\
0.018	158.2729\\
0.018	158.268\\
0.018	158.2632\\
0.018	158.2583\\
0.018	158.2535\\
0.018	158.2486\\
0.018	158.2437\\
0.018	158.2389\\
0.018	158.234\\
0.018	158.2291\\
0.019	158.2243\\
0.019	158.2194\\
0.019	158.2145\\
0.019	158.2096\\
0.019	158.2047\\
0.019	158.1998\\
0.019	158.1949\\
0.019	158.19\\
0.019	158.1851\\
0.019	158.1803\\
0.019	158.1754\\
0.02	158.1705\\
0.02	158.1656\\
0.02	158.1607\\
0.02	158.1558\\
0.02	158.1509\\
0.02	158.1459\\
0.02	158.141\\
0.02	158.1361\\
0.02	158.1311\\
0.021	158.1262\\
0.021	158.1212\\
0.021	158.1162\\
0.021	158.1113\\
0.021	158.1063\\
0.021	158.1013\\
0.021	158.0963\\
0.021	158.0913\\
0.021	158.0863\\
0.021	158.0813\\
0.022	158.0763\\
0.022	158.0713\\
0.022	158.0663\\
0.022	158.0612\\
0.022	158.0561\\
0.022	158.0511\\
0.022	158.0459\\
0.022	158.0408\\
0.022	158.0357\\
0.022	158.0306\\
0.023	158.0254\\
0.023	158.0202\\
0.023	158.015\\
0.023	158.0098\\
0.023	158.0046\\
0.023	157.9993\\
0.023	157.994\\
0.023	157.9887\\
0.023	157.9833\\
0.023	157.9779\\
0.024	157.9726\\
0.024	157.9672\\
0.024	157.9618\\
0.024	157.9563\\
0.024	157.9509\\
0.024	157.9454\\
0.024	157.9398\\
0.024	157.9343\\
0.024	157.9287\\
0.024	157.9231\\
0.025	157.9174\\
0.025	157.9118\\
0.025	157.9061\\
0.025	157.9004\\
0.025	157.8947\\
0.025	157.8889\\
0.025	157.8832\\
0.025	157.8774\\
0.025	157.8716\\
0.025	157.8658\\
0.026	157.8599\\
0.026	157.854\\
0.026	157.8481\\
0.026	157.8422\\
0.026	157.8363\\
0.026	157.8303\\
0.026	157.8244\\
0.026	157.8184\\
0.026	157.8124\\
0.026	157.8063\\
0.027	157.8003\\
0.027	157.7942\\
0.027	157.7882\\
0.027	157.7821\\
0.027	157.776\\
0.027	157.7699\\
0.027	157.7638\\
0.027	157.7576\\
0.027	157.7515\\
0.027	157.7453\\
0.028	157.7391\\
0.028	157.7329\\
0.028	157.7267\\
0.028	157.7204\\
0.028	157.7142\\
0.028	157.7079\\
0.028	157.7016\\
0.028	157.6952\\
0.028	157.6889\\
0.028	157.6826\\
0.029	157.6762\\
0.029	157.6698\\
0.029	157.6634\\
0.029	157.657\\
0.029	157.6505\\
0.029	157.6441\\
0.029	157.6376\\
0.029	157.6311\\
0.029	157.6245\\
0.029	157.618\\
0.03	157.6114\\
0.03	157.6048\\
0.03	157.5982\\
0.03	157.5916\\
0.03	157.5849\\
0.03	157.5782\\
0.03	157.5715\\
0.03	157.5648\\
0.03	157.5581\\
0.03	157.5514\\
0.031	157.5446\\
0.031	157.5378\\
0.031	157.531\\
0.031	157.5242\\
0.031	157.5174\\
0.031	157.5105\\
0.031	157.5037\\
0.031	157.4968\\
0.031	157.4899\\
0.031	157.483\\
0.032	157.476\\
0.032	157.4691\\
0.032	157.4621\\
0.032	157.4552\\
0.032	157.4482\\
0.032	157.4412\\
0.032	157.4341\\
0.032	157.4271\\
0.032	157.42\\
0.032	157.4129\\
0.033	157.4058\\
0.033	157.3987\\
0.033	157.3915\\
0.033	157.3843\\
0.033	157.3771\\
0.033	157.3699\\
0.033	157.3626\\
0.033	157.3553\\
0.033	157.348\\
0.033	157.3407\\
0.034	157.3333\\
0.034	157.3259\\
0.034	157.3185\\
0.034	157.311\\
0.034	157.3036\\
0.034	157.2961\\
0.034	157.2886\\
0.034	157.2811\\
0.034	157.2735\\
0.034	157.266\\
0.035	157.2584\\
0.035	157.2507\\
0.035	157.2431\\
0.035	157.2354\\
0.035	157.2277\\
0.035	157.22\\
0.035	157.2123\\
0.035	157.2045\\
0.035	157.1967\\
0.035	157.1889\\
0.036	157.1811\\
0.036	157.1732\\
0.036	157.1653\\
0.036	157.1574\\
0.036	157.1495\\
0.036	157.1415\\
0.036	157.1336\\
0.036	157.1256\\
0.036	157.1175\\
0.036	157.1095\\
0.037	157.1014\\
0.037	157.0934\\
0.037	157.0852\\
0.037	157.0771\\
0.037	157.0689\\
0.037	157.0608\\
0.037	157.0525\\
0.037	157.0443\\
0.037	157.036\\
0.037	157.0277\\
0.037	157.0194\\
0.038	157.0111\\
0.038	157.0028\\
0.038	156.9944\\
0.038	156.986\\
0.038	156.9776\\
0.038	156.9691\\
0.038	156.9607\\
0.038	156.9522\\
0.038	156.9437\\
0.038	156.9352\\
0.039	156.9266\\
0.039	156.918\\
0.039	156.9094\\
0.039	156.9007\\
0.039	156.8921\\
0.039	156.8834\\
0.039	156.8746\\
0.039	156.8659\\
0.039	156.8571\\
0.04	156.8482\\
0.04	156.8394\\
0.04	156.8305\\
0.04	156.8216\\
0.04	156.8126\\
0.04	156.8036\\
0.04	156.7946\\
0.04	156.7856\\
0.04	156.7765\\
0.04	156.7674\\
0.041	156.7582\\
0.041	156.749\\
0.041	156.7398\\
0.041	156.7305\\
0.041	156.7213\\
0.041	156.7119\\
0.041	156.7026\\
0.041	156.6932\\
0.041	156.6839\\
0.041	156.6744\\
0.042	156.665\\
0.042	156.6555\\
0.042	156.646\\
0.042	156.6364\\
0.042	156.6268\\
0.042	156.6172\\
0.042	156.6076\\
0.042	156.5979\\
0.042	156.5882\\
0.042	156.5784\\
0.043	156.5687\\
0.043	156.5589\\
0.043	156.549\\
0.043	156.5392\\
0.043	156.5293\\
0.043	156.5194\\
0.043	156.5094\\
0.043	156.4994\\
0.043	156.4894\\
0.043	156.4793\\
0.044	156.4692\\
0.044	156.4591\\
0.044	156.449\\
0.044	156.4388\\
0.044	156.4286\\
0.044	156.4183\\
0.044	156.4081\\
0.044	156.3978\\
0.044	156.3874\\
0.044	156.377\\
0.045	156.3666\\
0.045	156.3562\\
0.045	156.3457\\
0.045	156.3352\\
0.045	156.3247\\
0.045	156.3142\\
0.045	156.3036\\
0.045	156.2929\\
0.045	156.2823\\
0.045	156.2716\\
0.045	156.2609\\
0.046	156.2501\\
0.046	156.2393\\
0.046	156.2285\\
0.046	156.2176\\
0.046	156.2067\\
0.046	156.1958\\
0.046	156.1849\\
0.046	156.1739\\
0.046	156.1628\\
0.046	156.1518\\
0.047	156.1406\\
0.047	156.1295\\
0.047	156.1183\\
0.047	156.1071\\
0.047	156.0959\\
0.047	156.0846\\
0.047	156.0733\\
0.047	156.0619\\
0.047	156.0506\\
0.048	156.0392\\
0.048	156.0277\\
0.048	156.0162\\
0.048	156.0047\\
0.048	155.9932\\
0.048	155.9816\\
0.048	155.9699\\
0.048	155.9583\\
0.048	155.9466\\
0.048	155.9349\\
0.049	155.9231\\
0.049	155.9113\\
0.049	155.8994\\
0.049	155.8875\\
0.049	155.8756\\
0.049	155.8636\\
0.049	155.8516\\
0.049	155.8396\\
0.049	155.8275\\
0.049	155.8154\\
0.05	155.8032\\
0.05	155.7911\\
0.05	155.7788\\
0.05	155.7665\\
0.05	155.7542\\
0.05	155.7418\\
0.05	155.7294\\
0.05	155.717\\
0.05	155.7045\\
0.05	155.692\\
0.051	155.6794\\
0.051	155.6668\\
0.051	155.6542\\
0.051	155.6415\\
0.051	155.6287\\
0.051	155.6159\\
0.051	155.6031\\
0.051	155.5902\\
0.051	155.5773\\
0.051	155.5643\\
0.052	155.5513\\
0.052	155.5383\\
0.052	155.5252\\
0.052	155.512\\
0.052	155.4989\\
0.052	155.4856\\
0.052	155.4724\\
0.052	155.4591\\
0.052	155.4457\\
0.052	155.4323\\
0.053	155.4188\\
0.053	155.4053\\
0.053	155.3918\\
0.053	155.3782\\
0.053	155.3646\\
0.053	155.3509\\
0.053	155.3372\\
0.053	155.3234\\
0.053	155.3096\\
0.053	155.2957\\
0.054	155.2818\\
0.054	155.2679\\
0.054	155.2539\\
0.054	155.2399\\
0.054	155.2258\\
0.054	155.2117\\
0.054	155.1975\\
0.054	155.1833\\
0.054	155.1691\\
0.054	155.1548\\
0.054	155.1405\\
0.055	155.1261\\
0.055	155.1117\\
0.055	155.0973\\
0.055	155.0828\\
0.055	155.0683\\
0.055	155.0537\\
0.055	155.0391\\
0.055	155.0245\\
0.055	155.0098\\
0.056	154.9951\\
0.056	154.9803\\
0.056	154.9655\\
0.056	154.9507\\
0.056	154.9358\\
0.056	154.9209\\
0.056	154.906\\
0.056	154.891\\
0.056	154.876\\
0.056	154.861\\
0.057	154.846\\
0.057	154.8309\\
0.057	154.8158\\
0.057	154.8006\\
0.057	154.7855\\
0.057	154.7703\\
0.057	154.7551\\
0.057	154.7399\\
0.057	154.7246\\
0.057	154.7093\\
0.058	154.6941\\
0.058	154.6787\\
0.058	154.6634\\
0.058	154.6481\\
0.058	154.6327\\
0.058	154.6173\\
0.058	154.6019\\
0.058	154.5865\\
0.058	154.571\\
0.058	154.5556\\
0.059	154.5401\\
0.059	154.5246\\
0.059	154.5091\\
0.059	154.4936\\
0.059	154.478\\
0.059	154.4624\\
0.059	154.4469\\
0.059	154.4313\\
0.059	154.4157\\
0.059	154.4\\
0.06	154.3844\\
0.06	154.3688\\
0.06	154.3531\\
0.06	154.3374\\
0.06	154.3218\\
0.06	154.3061\\
0.06	154.2904\\
0.06	154.2747\\
0.06	154.259\\
0.06	154.2433\\
0.061	154.2276\\
0.061	154.2119\\
0.061	154.1962\\
0.061	154.1805\\
0.061	154.1648\\
0.061	154.1491\\
0.061	154.1334\\
0.061	154.1177\\
0.061	154.102\\
0.061	154.0864\\
0.062	154.0707\\
0.062	154.055\\
0.062	154.0394\\
0.062	154.0237\\
0.062	154.0081\\
0.062	153.9925\\
0.062	153.9769\\
0.062	153.9613\\
0.062	153.9457\\
0.062	153.9302\\
0.062	153.9146\\
0.063	153.8991\\
0.063	153.8836\\
0.063	153.8681\\
0.063	153.8527\\
0.063	153.8373\\
0.063	153.8219\\
0.063	153.8065\\
0.063	153.7911\\
0.063	153.7758\\
0.064	153.7605\\
0.064	153.7452\\
0.064	153.73\\
0.064	153.7148\\
0.064	153.6996\\
0.064	153.6845\\
0.064	153.6694\\
0.064	153.6543\\
0.064	153.6392\\
0.064	153.6242\\
0.065	153.6093\\
0.065	153.5943\\
0.065	153.5794\\
0.065	153.5646\\
0.065	153.5498\\
0.065	153.535\\
0.065	153.5203\\
0.065	153.5056\\
0.065	153.491\\
0.065	153.4764\\
0.066	153.4618\\
0.066	153.4473\\
0.066	153.4328\\
0.066	153.4184\\
0.066	153.404\\
0.066	153.3897\\
0.066	153.3754\\
0.066	153.3612\\
0.066	153.347\\
0.066	153.3328\\
0.067	153.3187\\
0.067	153.3047\\
0.067	153.2907\\
0.067	153.2768\\
0.067	153.2629\\
0.067	153.2491\\
0.067	153.2353\\
0.067	153.2216\\
0.067	153.2079\\
0.067	153.1943\\
0.068	153.1807\\
0.068	153.1672\\
0.068	153.1537\\
0.068	153.1403\\
0.068	153.127\\
0.068	153.1137\\
0.068	153.1004\\
0.068	153.0873\\
0.068	153.0742\\
0.068	153.0611\\
0.069	153.0481\\
0.069	153.0352\\
0.069	153.0223\\
0.069	153.0095\\
0.069	152.9968\\
0.069	152.9841\\
0.069	152.9715\\
0.069	152.9589\\
0.069	152.9464\\
0.069	152.934\\
0.07	152.9216\\
0.07	152.9094\\
0.07	152.8971\\
0.07	152.885\\
0.07	152.8729\\
0.07	152.8609\\
0.07	152.849\\
0.07	152.8371\\
0.07	152.8253\\
0.07	152.8136\\
0.071	152.802\\
0.071	152.7904\\
0.071	152.7789\\
0.071	152.7675\\
0.071	152.7562\\
0.071	152.745\\
0.071	152.7338\\
0.071	152.7227\\
0.071	152.7117\\
0.071	152.7008\\
0.072	152.69\\
0.072	152.6793\\
0.072	152.6687\\
0.072	152.6581\\
0.072	152.6477\\
0.072	152.6373\\
0.072	152.627\\
0.072	152.6169\\
0.072	152.6068\\
0.072	152.5968\\
0.073	152.5869\\
0.073	152.5771\\
0.073	152.5675\\
0.073	152.5579\\
0.073	152.5484\\
0.073	152.539\\
0.073	152.5298\\
0.073	152.5206\\
0.073	152.5116\\
0.073	152.5027\\
0.074	152.4938\\
0.074	152.4851\\
0.074	152.4765\\
0.074	152.4681\\
0.074	152.4597\\
0.074	152.4515\\
0.074	152.4433\\
0.074	152.4353\\
0.074	152.4275\\
0.074	152.4197\\
0.074	152.4121\\
0.075	152.4046\\
0.075	152.3972\\
0.075	152.3899\\
0.075	152.3828\\
0.075	152.3758\\
0.075	152.3689\\
0.075	152.3622\\
0.075	152.3555\\
0.075	152.349\\
0.075	152.3427\\
0.076	152.3364\\
0.076	152.3303\\
0.076	152.3244\\
0.076	152.3185\\
0.076	152.3128\\
0.076	152.3073\\
0.076	152.3018\\
0.076	152.2965\\
0.076	152.2914\\
0.076	152.2863\\
0.077	152.2814\\
0.077	152.2767\\
0.077	152.272\\
0.077	152.2675\\
0.077	152.2632\\
0.077	152.259\\
0.077	152.2549\\
0.077	152.251\\
0.077	152.2472\\
0.077	152.2435\\
0.078	152.24\\
0.078	152.2366\\
0.078	152.2333\\
0.078	152.2302\\
0.078	152.2273\\
0.078	152.2245\\
0.078	152.2218\\
0.078	152.2192\\
0.078	152.2168\\
0.079	152.2146\\
0.079	152.2124\\
0.079	152.2105\\
0.079	152.2086\\
0.079	152.2069\\
0.079	152.2054\\
0.079	152.204\\
0.079	152.2027\\
0.079	152.2016\\
0.079	152.2006\\
0.08	152.1997\\
0.08	152.199\\
0.08	152.1984\\
0.08	152.198\\
0.08	152.1977\\
0.08	152.1975\\
0.08	152.1975\\
0.08	152.1976\\
0.08	152.1978\\
0.08	152.1982\\
0.081	152.1988\\
0.081	152.1994\\
0.081	152.2003\\
0.081	152.2012\\
0.081	152.2023\\
0.081	152.2035\\
0.081	152.2049\\
0.081	152.2064\\
0.081	152.208\\
0.081	152.2098\\
0.082	152.2117\\
0.082	152.2137\\
0.082	152.2159\\
0.082	152.2182\\
0.082	152.2206\\
0.082	152.2232\\
0.082	152.2259\\
0.082	152.2287\\
0.082	152.2317\\
0.082	152.2348\\
0.083	152.2381\\
0.083	152.2414\\
0.083	152.2449\\
0.083	152.2486\\
0.083	152.2523\\
0.083	152.2562\\
0.083	152.2602\\
0.083	152.2644\\
0.083	152.2686\\
0.083	152.273\\
0.084	152.2775\\
0.084	152.2822\\
0.084	152.2869\\
0.084	152.2918\\
0.084	152.2968\\
0.084	152.3019\\
0.084	152.3072\\
0.084	152.3125\\
0.084	152.318\\
0.084	152.3236\\
0.085	152.3293\\
0.085	152.3352\\
0.085	152.3411\\
0.085	152.3472\\
0.085	152.3534\\
0.085	152.3597\\
0.085	152.3661\\
0.085	152.3726\\
0.085	152.3792\\
0.085	152.386\\
0.086	152.3928\\
0.086	152.3998\\
0.086	152.4069\\
0.086	152.414\\
0.086	152.4213\\
0.086	152.4287\\
0.086	152.4361\\
0.086	152.4437\\
0.086	152.4514\\
0.086	152.4592\\
0.087	152.4671\\
0.087	152.475\\
0.087	152.4831\\
0.087	152.4913\\
0.087	152.4995\\
0.087	152.5078\\
0.087	152.5163\\
0.087	152.5248\\
0.087	152.5334\\
0.087	152.5421\\
0.088	152.5509\\
0.088	152.5597\\
0.088	152.5687\\
0.088	152.5777\\
0.088	152.5868\\
0.088	152.5959\\
0.088	152.6052\\
0.088	152.6145\\
0.088	152.6239\\
0.088	152.6334\\
0.089	152.6429\\
0.089	152.6525\\
0.089	152.6622\\
0.089	152.672\\
0.089	152.6818\\
0.089	152.6917\\
0.089	152.7016\\
0.089	152.7116\\
0.089	152.7217\\
0.089	152.7318\\
0.09	152.742\\
0.09	152.7522\\
0.09	152.7625\\
0.09	152.7729\\
0.09	152.7833\\
0.09	152.7938\\
0.09	152.8043\\
0.09	152.8148\\
0.09	152.8254\\
0.09	152.836\\
0.091	152.8467\\
0.091	152.8575\\
0.091	152.8682\\
0.091	152.879\\
0.091	152.8899\\
0.091	152.9008\\
0.091	152.9117\\
0.091	152.9226\\
0.091	152.9336\\
0.091	152.9446\\
0.091	152.9557\\
0.092	152.9668\\
0.092	152.9779\\
0.092	152.989\\
0.092	153.0001\\
0.092	153.0113\\
0.092	153.0225\\
0.092	153.0337\\
0.092	153.045\\
0.092	153.0562\\
0.092	153.0675\\
0.093	153.0788\\
0.093	153.0901\\
0.093	153.1014\\
0.093	153.1127\\
0.093	153.124\\
0.093	153.1353\\
0.093	153.1467\\
0.093	153.158\\
0.093	153.1694\\
0.093	153.1807\\
0.094	153.1921\\
0.094	153.2035\\
0.094	153.2148\\
0.094	153.2262\\
0.094	153.2375\\
0.094	153.2488\\
0.094	153.2602\\
0.094	153.2715\\
0.094	153.2828\\
0.095	153.2941\\
0.095	153.3054\\
0.095	153.3167\\
0.095	153.328\\
0.095	153.3392\\
0.095	153.3504\\
0.095	153.3616\\
0.095	153.3728\\
0.095	153.384\\
0.095	153.3951\\
0.096	153.4062\\
0.096	153.4173\\
0.096	153.4284\\
0.096	153.4394\\
0.096	153.4504\\
0.096	153.4613\\
0.096	153.4723\\
0.096	153.4832\\
0.096	153.494\\
0.096	153.5049\\
0.097	153.5157\\
0.097	153.5264\\
0.097	153.5371\\
0.097	153.5478\\
0.097	153.5584\\
0.097	153.569\\
0.097	153.5795\\
0.097	153.59\\
0.097	153.6004\\
0.097	153.6108\\
0.098	153.6212\\
0.098	153.6315\\
0.098	153.6417\\
0.098	153.6519\\
0.098	153.662\\
0.098	153.6721\\
0.098	153.6822\\
0.098	153.6921\\
0.098	153.702\\
0.098	153.7119\\
0.099	153.7217\\
0.099	153.7314\\
0.099	153.7411\\
0.099	153.7507\\
0.099	153.7603\\
0.099	153.7698\\
0.099	153.7792\\
0.099	153.7886\\
0.099	153.7979\\
0.099	153.8071\\
0.1	153.8162\\
0.1	153.8253\\
0.1	153.8343\\
0.1	153.8433\\
0.1	153.8521\\
0.1	153.8609\\
0.1	153.8696\\
0.1	153.8783\\
0.1	153.8869\\
0.1	153.8954\\
0.101	153.9038\\
0.101	153.9121\\
0.101	153.9204\\
0.101	153.9286\\
0.101	153.9367\\
0.101	153.9447\\
0.101	153.9527\\
0.101	153.9606\\
0.101	153.9684\\
0.101	153.9761\\
0.102	153.9837\\
0.102	153.9913\\
0.102	153.9988\\
0.102	154.0061\\
0.102	154.0135\\
0.102	154.0207\\
0.102	154.0278\\
0.102	154.0349\\
0.102	154.0419\\
0.102	154.0488\\
0.103	154.0556\\
0.103	154.0623\\
0.103	154.0689\\
0.103	154.0755\\
0.103	154.0819\\
0.103	154.0883\\
0.103	154.0946\\
0.103	154.1008\\
0.103	154.1069\\
0.103	154.1129\\
0.104	154.1189\\
0.104	154.1247\\
0.104	154.1305\\
0.104	154.1362\\
0.104	154.1417\\
0.104	154.1472\\
0.104	154.1526\\
0.104	154.1579\\
0.104	154.1631\\
0.104	154.1683\\
0.105	154.1733\\
0.105	154.1782\\
0.105	154.1831\\
0.105	154.1878\\
0.105	154.1925\\
0.105	154.197\\
0.105	154.2015\\
0.105	154.2058\\
0.105	154.2101\\
0.105	154.2142\\
0.106	154.2183\\
0.106	154.2223\\
0.106	154.2261\\
0.106	154.2299\\
0.106	154.2336\\
0.106	154.2372\\
0.106	154.2407\\
0.106	154.2441\\
0.106	154.2473\\
0.106	154.2505\\
0.107	154.2536\\
0.107	154.2566\\
0.107	154.2595\\
0.107	154.2623\\
0.107	154.265\\
0.107	154.2676\\
0.107	154.2701\\
0.107	154.2725\\
0.107	154.2748\\
0.107	154.277\\
0.107	154.2791\\
0.108	154.2811\\
0.108	154.283\\
0.108	154.2848\\
0.108	154.2865\\
0.108	154.2881\\
0.108	154.2896\\
0.108	154.291\\
0.108	154.2923\\
0.108	154.2935\\
0.108	154.2946\\
0.109	154.2956\\
0.109	154.2965\\
0.109	154.2974\\
0.109	154.2981\\
0.109	154.2987\\
0.109	154.2993\\
0.109	154.2997\\
0.109	154.3\\
0.109	154.3003\\
0.11	154.3004\\
0.11	154.3005\\
0.11	154.3004\\
0.11	154.3003\\
0.11	154.3\\
0.11	154.2997\\
0.11	154.2993\\
0.11	154.2988\\
0.11	154.2981\\
0.11	154.2974\\
0.111	154.2966\\
0.111	154.2957\\
0.111	154.2947\\
0.111	154.2936\\
0.111	154.2924\\
0.111	154.2911\\
0.111	154.2898\\
0.111	154.2883\\
0.111	154.2867\\
0.111	154.2851\\
0.112	154.2834\\
0.112	154.2815\\
0.112	154.2796\\
0.112	154.2776\\
0.112	154.2755\\
0.112	154.2733\\
0.112	154.271\\
0.112	154.2686\\
0.112	154.2661\\
0.112	154.2635\\
0.113	154.2609\\
0.113	154.2581\\
0.113	154.2553\\
0.113	154.2524\\
0.113	154.2494\\
0.113	154.2463\\
0.113	154.2431\\
0.113	154.2398\\
0.113	154.2364\\
0.113	154.233\\
0.114	154.2294\\
0.114	154.2258\\
0.114	154.2221\\
0.114	154.2183\\
0.114	154.2144\\
0.114	154.2104\\
0.114	154.2063\\
0.114	154.2021\\
0.114	154.1979\\
0.114	154.1936\\
0.115	154.1892\\
0.115	154.1847\\
0.115	154.1801\\
0.115	154.1755\\
0.115	154.1708\\
0.115	154.166\\
0.115	154.1611\\
0.115	154.1561\\
0.115	154.1511\\
0.115	154.146\\
0.116	154.1408\\
0.116	154.1355\\
0.116	154.1302\\
0.116	154.1247\\
0.116	154.1192\\
0.116	154.1137\\
0.116	154.108\\
0.116	154.1023\\
0.116	154.0965\\
0.116	154.0907\\
0.117	154.0847\\
0.117	154.0787\\
0.117	154.0727\\
0.117	154.0666\\
0.117	154.0604\\
0.117	154.0541\\
0.117	154.0478\\
0.117	154.0414\\
0.117	154.0349\\
0.117	154.0284\\
0.118	154.0218\\
0.118	154.0152\\
0.118	154.0084\\
0.118	154.0017\\
0.118	153.9948\\
0.118	153.9879\\
0.118	153.981\\
0.118	153.974\\
0.118	153.9669\\
0.118	153.9598\\
0.119	153.9526\\
0.119	153.9453\\
0.119	153.938\\
0.119	153.9307\\
0.119	153.9232\\
0.119	153.9158\\
0.119	153.9082\\
0.119	153.9007\\
0.119	153.893\\
0.119	153.8854\\
0.12	153.8776\\
0.12	153.8698\\
0.12	153.862\\
0.12	153.8541\\
0.12	153.8462\\
0.12	153.8382\\
0.12	153.8302\\
0.12	153.8221\\
0.12	153.814\\
0.12	153.8058\\
0.121	153.7976\\
0.121	153.7894\\
0.121	153.7811\\
0.121	153.7727\\
0.121	153.7644\\
0.121	153.7559\\
0.121	153.7475\\
0.121	153.739\\
0.121	153.7304\\
0.121	153.7218\\
0.122	153.7132\\
0.122	153.7045\\
0.122	153.6958\\
0.122	153.687\\
0.122	153.6783\\
0.122	153.6694\\
0.122	153.6606\\
0.122	153.6517\\
0.122	153.6427\\
0.122	153.6338\\
0.123	153.6248\\
0.123	153.6157\\
0.123	153.6066\\
0.123	153.5975\\
0.123	153.5884\\
0.123	153.5792\\
0.123	153.57\\
0.123	153.5607\\
0.123	153.5514\\
0.123	153.5421\\
0.124	153.5328\\
0.124	153.5234\\
0.124	153.514\\
0.124	153.5046\\
0.124	153.4951\\
0.124	153.4857\\
0.124	153.4761\\
0.124	153.4666\\
0.124	153.457\\
0.124	153.4474\\
0.124	153.4378\\
0.125	153.4282\\
0.125	153.4185\\
0.125	153.4088\\
0.125	153.3991\\
0.125	153.3893\\
0.125	153.3796\\
0.125	153.3698\\
0.125	153.36\\
0.125	153.3501\\
0.126	153.3403\\
0.126	153.3304\\
0.126	153.3205\\
0.126	153.3105\\
0.126	153.3006\\
0.126	153.2906\\
0.126	153.2807\\
0.126	153.2707\\
0.126	153.2607\\
0.126	153.2506\\
0.127	153.2406\\
0.127	153.2305\\
0.127	153.2204\\
0.127	153.2103\\
0.127	153.2002\\
0.127	153.1901\\
0.127	153.1799\\
0.127	153.1698\\
0.127	153.1596\\
0.127	153.1494\\
0.128	153.1392\\
0.128	153.129\\
0.128	153.1188\\
0.128	153.1085\\
0.128	153.0983\\
0.128	153.088\\
0.128	153.0777\\
0.128	153.0675\\
0.128	153.0572\\
0.128	153.0468\\
0.129	153.0365\\
0.129	153.0262\\
0.129	153.0159\\
0.129	153.0055\\
0.129	152.9952\\
0.129	152.9848\\
0.129	152.9744\\
0.129	152.964\\
0.129	152.9537\\
0.129	152.9433\\
0.13	152.9329\\
0.13	152.9225\\
0.13	152.9121\\
0.13	152.9016\\
0.13	152.8912\\
0.13	152.8808\\
0.13	152.8703\\
0.13	152.8599\\
0.13	152.8495\\
0.13	152.839\\
0.131	152.8286\\
0.131	152.8182\\
0.131	152.8077\\
0.131	152.7973\\
0.131	152.7868\\
0.131	152.7764\\
0.131	152.7659\\
0.131	152.7555\\
0.131	152.745\\
0.131	152.7346\\
0.132	152.7241\\
0.132	152.7137\\
0.132	152.7032\\
0.132	152.6928\\
0.132	152.6823\\
0.132	152.6719\\
0.132	152.6615\\
0.132	152.651\\
0.132	152.6406\\
0.132	152.6302\\
0.133	152.6198\\
0.133	152.6094\\
0.133	152.599\\
0.133	152.5886\\
0.133	152.5782\\
0.133	152.5678\\
0.133	152.5575\\
0.133	152.5471\\
0.133	152.5367\\
0.133	152.5264\\
0.134	152.5161\\
0.134	152.5058\\
0.134	152.4954\\
0.134	152.4852\\
0.134	152.4749\\
0.134	152.4646\\
0.134	152.4544\\
0.134	152.4441\\
0.134	152.4339\\
0.134	152.4237\\
0.135	152.4135\\
0.135	152.4033\\
0.135	152.3931\\
0.135	152.383\\
0.135	152.3729\\
0.135	152.3628\\
0.135	152.3527\\
0.135	152.3426\\
0.135	152.3325\\
0.135	152.3225\\
0.136	152.3125\\
0.136	152.3025\\
0.136	152.2925\\
0.136	152.2825\\
0.136	152.2726\\
0.136	152.2627\\
0.136	152.2528\\
0.136	152.2429\\
0.136	152.2331\\
0.136	152.2232\\
0.137	152.2134\\
0.137	152.2037\\
0.137	152.1939\\
0.137	152.1842\\
0.137	152.1744\\
0.137	152.1648\\
0.137	152.1551\\
0.137	152.1454\\
0.137	152.1358\\
0.137	152.1262\\
0.138	152.1167\\
0.138	152.1071\\
0.138	152.0976\\
0.138	152.0881\\
0.138	152.0786\\
0.138	152.0692\\
0.138	152.0597\\
0.138	152.0503\\
0.138	152.041\\
0.138	152.0316\\
0.139	152.0223\\
0.139	152.013\\
0.139	152.0038\\
0.139	151.9946\\
0.139	151.9854\\
0.139	151.9762\\
0.139	151.967\\
0.139	151.9579\\
0.139	151.9488\\
0.139	151.9398\\
0.14	151.9308\\
0.14	151.9218\\
0.14	151.9128\\
0.14	151.9039\\
0.14	151.895\\
0.14	151.8862\\
0.14	151.8773\\
0.14	151.8685\\
0.14	151.8598\\
0.14	151.8511\\
0.141	151.8424\\
0.141	151.8337\\
0.141	151.8251\\
0.141	151.8165\\
0.141	151.8079\\
0.141	151.7994\\
0.141	151.7909\\
0.141	151.7825\\
0.141	151.7741\\
0.141	151.7657\\
0.142	151.7574\\
0.142	151.7491\\
0.142	151.7408\\
0.142	151.7326\\
0.142	151.7245\\
0.142	151.7163\\
0.142	151.7082\\
0.142	151.7002\\
0.142	151.6921\\
0.142	151.6842\\
0.143	151.6762\\
0.143	151.6683\\
0.143	151.6605\\
0.143	151.6527\\
0.143	151.6449\\
0.143	151.6372\\
0.143	151.6295\\
0.143	151.6219\\
0.143	151.6143\\
0.143	151.6067\\
0.144	151.5992\\
0.144	151.5917\\
0.144	151.5843\\
0.144	151.5769\\
0.144	151.5695\\
0.144	151.5622\\
0.144	151.5549\\
0.144	151.5477\\
0.144	151.5405\\
0.144	151.5333\\
0.145	151.5262\\
0.145	151.5192\\
0.145	151.5121\\
0.145	151.5051\\
0.145	151.4982\\
0.145	151.4913\\
0.145	151.4844\\
0.145	151.4776\\
0.145	151.4708\\
0.145	151.4641\\
0.146	151.4574\\
0.146	151.4507\\
0.146	151.4441\\
0.146	151.4375\\
0.146	151.431\\
0.146	151.4245\\
0.146	151.418\\
0.146	151.4116\\
0.146	151.4053\\
0.146	151.3989\\
0.147	151.3927\\
0.147	151.3864\\
0.147	151.3803\\
0.147	151.3741\\
0.147	151.368\\
0.147	151.362\\
0.147	151.356\\
0.147	151.35\\
0.147	151.3441\\
0.147	151.3382\\
0.148	151.3324\\
0.148	151.3266\\
0.148	151.3209\\
0.148	151.3152\\
0.148	151.3095\\
0.148	151.3039\\
0.148	151.2984\\
0.148	151.2929\\
0.148	151.2874\\
0.148	151.282\\
0.149	151.2767\\
0.149	151.2714\\
0.149	151.2661\\
0.149	151.2609\\
0.149	151.2557\\
0.149	151.2506\\
0.149	151.2456\\
0.149	151.2406\\
0.149	151.2356\\
0.149	151.2307\\
0.149	151.2258\\
0.15	151.221\\
0.15	151.2163\\
0.15	151.2116\\
0.15	151.2069\\
0.15	151.2023\\
0.15	151.1978\\
0.15	151.1933\\
0.15	151.1888\\
0.15	151.1845\\
0.15	151.1801\\
0.151	151.1758\\
0.151	151.1716\\
0.151	151.1674\\
0.151	151.1633\\
0.151	151.1592\\
0.151	151.1552\\
0.151	151.1513\\
0.151	151.1473\\
0.151	151.1435\\
0.151	151.1397\\
0.152	151.136\\
0.152	151.1323\\
0.152	151.1286\\
0.152	151.1251\\
0.152	151.1215\\
0.152	151.1181\\
0.152	151.1147\\
0.152	151.1113\\
0.152	151.108\\
0.152	151.1048\\
0.153	151.1016\\
0.153	151.0985\\
0.153	151.0954\\
0.153	151.0924\\
0.153	151.0894\\
0.153	151.0865\\
0.153	151.0837\\
0.153	151.0809\\
0.153	151.0781\\
0.153	151.0755\\
0.154	151.0728\\
0.154	151.0703\\
0.154	151.0677\\
0.154	151.0653\\
0.154	151.0629\\
0.154	151.0605\\
0.154	151.0582\\
0.154	151.056\\
0.154	151.0538\\
0.154	151.0517\\
0.155	151.0496\\
0.155	151.0476\\
0.155	151.0456\\
0.155	151.0437\\
0.155	151.0418\\
0.155	151.04\\
0.155	151.0382\\
0.155	151.0365\\
0.155	151.0349\\
0.155	151.0332\\
0.156	151.0317\\
0.156	151.0302\\
0.156	151.0287\\
0.156	151.0273\\
0.156	151.026\\
0.156	151.0247\\
0.156	151.0234\\
0.156	151.0222\\
0.156	151.0211\\
0.157	151.0199\\
0.157	151.0189\\
0.157	151.0179\\
0.157	151.0169\\
0.157	151.0159\\
0.157	151.0151\\
0.157	151.0142\\
0.157	151.0134\\
0.157	151.0127\\
0.157	151.012\\
0.158	151.0113\\
0.158	151.0107\\
0.158	151.0101\\
0.158	151.0096\\
0.158	151.0091\\
0.158	151.0086\\
0.158	151.0082\\
0.158	151.0079\\
0.158	151.0075\\
0.158	151.0072\\
0.159	151.007\\
0.159	151.0068\\
0.159	151.0066\\
0.159	151.0065\\
0.159	151.0064\\
0.159	151.0063\\
0.159	151.0063\\
0.159	151.0063\\
0.159	151.0063\\
0.159	151.0064\\
0.16	151.0066\\
0.16	151.0067\\
0.16	151.0069\\
0.16	151.0071\\
0.16	151.0074\\
0.16	151.0077\\
0.16	151.008\\
0.16	151.0084\\
0.16	151.0088\\
0.16	151.0093\\
0.161	151.0097\\
0.161	151.0102\\
0.161	151.0108\\
0.161	151.0113\\
0.161	151.0119\\
0.161	151.0126\\
0.161	151.0132\\
0.161	151.0139\\
0.161	151.0146\\
0.161	151.0154\\
0.162	151.0162\\
0.162	151.017\\
0.162	151.0178\\
0.162	151.0187\\
0.162	151.0196\\
0.162	151.0205\\
0.162	151.0215\\
0.162	151.0225\\
0.162	151.0235\\
0.162	151.0246\\
0.163	151.0256\\
0.163	151.0267\\
0.163	151.0279\\
0.163	151.029\\
0.163	151.0302\\
0.163	151.0315\\
0.163	151.0327\\
0.163	151.034\\
0.163	151.0353\\
0.163	151.0367\\
0.164	151.038\\
0.164	151.0394\\
0.164	151.0408\\
0.164	151.0423\\
0.164	151.0437\\
0.164	151.0452\\
0.164	151.0468\\
0.164	151.0483\\
0.164	151.0499\\
0.164	151.0515\\
0.165	151.0531\\
0.165	151.0548\\
0.165	151.0565\\
0.165	151.0582\\
0.165	151.06\\
0.165	151.0618\\
0.165	151.0636\\
0.165	151.0654\\
0.165	151.0673\\
0.165	151.0692\\
0.166	151.0711\\
0.166	151.0731\\
0.166	151.0751\\
0.166	151.0771\\
0.166	151.0791\\
0.166	151.0812\\
0.166	151.0833\\
0.166	151.0855\\
0.166	151.0876\\
0.166	151.0898\\
0.167	151.0921\\
0.167	151.0943\\
0.167	151.0966\\
0.167	151.099\\
0.167	151.1013\\
0.167	151.1037\\
0.167	151.1061\\
0.167	151.1086\\
0.167	151.1111\\
0.167	151.1136\\
0.168	151.1162\\
0.168	151.1187\\
0.168	151.1214\\
0.168	151.124\\
0.168	151.1267\\
0.168	151.1294\\
0.168	151.1321\\
0.168	151.1349\\
0.168	151.1377\\
0.168	151.1405\\
0.169	151.1434\\
0.169	151.1463\\
0.169	151.1492\\
0.169	151.1522\\
0.169	151.1552\\
0.169	151.1582\\
0.169	151.1613\\
0.169	151.1643\\
0.169	151.1674\\
0.169	151.1706\\
0.17	151.1737\\
0.17	151.1769\\
0.17	151.1802\\
0.17	151.1834\\
0.17	151.1867\\
0.17	151.19\\
0.17	151.1933\\
0.17	151.1967\\
0.17	151.2001\\
0.17	151.2035\\
0.171	151.2069\\
0.171	151.2104\\
0.171	151.2139\\
0.171	151.2174\\
0.171	151.221\\
0.171	151.2246\\
0.171	151.2281\\
0.171	151.2318\\
0.171	151.2354\\
0.171	151.2391\\
0.172	151.2427\\
0.172	151.2464\\
0.172	151.2502\\
0.172	151.2539\\
0.172	151.2577\\
0.172	151.2615\\
0.172	151.2653\\
0.172	151.2691\\
0.172	151.2729\\
0.172	151.2768\\
0.173	151.2807\\
0.173	151.2845\\
0.173	151.2884\\
0.173	151.2924\\
0.173	151.2963\\
0.173	151.3002\\
0.173	151.3042\\
0.173	151.3082\\
0.173	151.3122\\
0.173	151.3162\\
0.174	151.3202\\
0.174	151.3242\\
0.174	151.3282\\
0.174	151.3323\\
0.174	151.3363\\
0.174	151.3404\\
0.174	151.3444\\
0.174	151.3485\\
0.174	151.3526\\
0.174	151.3566\\
0.175	151.3607\\
0.175	151.3648\\
0.175	151.3689\\
0.175	151.373\\
0.175	151.3771\\
0.175	151.3812\\
0.175	151.3853\\
0.175	151.3893\\
0.175	151.3934\\
0.175	151.3975\\
0.176	151.4016\\
0.176	151.4057\\
0.176	151.4098\\
0.176	151.4139\\
0.176	151.4179\\
0.176	151.422\\
0.176	151.4261\\
0.176	151.4301\\
0.176	151.4341\\
0.176	151.4382\\
0.177	151.4422\\
0.177	151.4462\\
0.177	151.4502\\
0.177	151.4542\\
0.177	151.4582\\
0.177	151.4622\\
0.177	151.4662\\
0.177	151.4701\\
0.177	151.4741\\
0.177	151.478\\
0.178	151.4819\\
0.178	151.4858\\
0.178	151.4897\\
0.178	151.4936\\
0.178	151.4974\\
0.178	151.5013\\
0.178	151.5051\\
0.178	151.5089\\
0.178	151.5127\\
0.178	151.5165\\
0.179	151.5202\\
0.179	151.5239\\
0.179	151.5277\\
0.179	151.5313\\
0.179	151.535\\
0.179	151.5387\\
0.179	151.5423\\
0.179	151.5459\\
0.179	151.5495\\
0.179	151.553\\
0.18	151.5566\\
0.18	151.5601\\
0.18	151.5635\\
0.18	151.567\\
0.18	151.5704\\
0.18	151.5738\\
0.18	151.5772\\
0.18	151.5805\\
0.18	151.5839\\
0.18	151.5872\\
0.181	151.5904\\
0.181	151.5937\\
0.181	151.5969\\
0.181	151.6\\
0.181	151.6032\\
0.181	151.6063\\
0.181	151.6094\\
0.181	151.6125\\
0.181	151.6155\\
0.181	151.6185\\
0.182	151.6214\\
0.182	151.6243\\
0.182	151.6272\\
0.182	151.6301\\
0.182	151.6329\\
0.182	151.6357\\
0.182	151.6384\\
0.182	151.6412\\
0.182	151.6438\\
0.182	151.6465\\
0.182	151.6491\\
0.183	151.6516\\
0.183	151.6542\\
0.183	151.6566\\
0.183	151.6591\\
0.183	151.6615\\
0.183	151.6639\\
0.183	151.6662\\
0.183	151.6685\\
0.183	151.6708\\
0.183	151.673\\
0.184	151.6752\\
0.184	151.6773\\
0.184	151.6794\\
0.184	151.6814\\
0.184	151.6835\\
0.184	151.6854\\
0.184	151.6873\\
0.184	151.6892\\
0.184	151.6911\\
0.184	151.6929\\
0.185	151.6946\\
0.185	151.6963\\
0.185	151.698\\
0.185	151.6996\\
0.185	151.7012\\
0.185	151.7028\\
0.185	151.7043\\
0.185	151.7057\\
0.185	151.7071\\
0.185	151.7085\\
0.186	151.7098\\
0.186	151.7111\\
0.186	151.7124\\
0.186	151.7136\\
0.186	151.7147\\
0.186	151.7158\\
0.186	151.7169\\
0.186	151.7179\\
0.186	151.7189\\
0.186	151.7199\\
0.187	151.7207\\
0.187	151.7216\\
0.187	151.7224\\
0.187	151.7232\\
0.187	151.7239\\
0.187	151.7246\\
0.187	151.7253\\
0.187	151.7259\\
0.187	151.7264\\
0.188	151.727\\
0.188	151.7274\\
0.188	151.7279\\
0.188	151.7283\\
0.188	151.7286\\
0.188	151.729\\
0.188	151.7293\\
0.188	151.7295\\
0.188	151.7297\\
0.188	151.7299\\
0.189	151.73\\
0.189	151.7301\\
0.189	151.7302\\
0.189	151.7302\\
0.189	151.7302\\
0.189	151.7301\\
0.189	151.73\\
0.189	151.7299\\
0.189	151.7297\\
0.189	151.7295\\
0.19	151.7292\\
0.19	151.7289\\
0.19	151.7286\\
0.19	151.7283\\
0.19	151.7279\\
0.19	151.7274\\
0.19	151.727\\
0.19	151.7265\\
0.19	151.7259\\
0.19	151.7253\\
0.191	151.7247\\
0.191	151.7241\\
0.191	151.7234\\
0.191	151.7227\\
0.191	151.7219\\
0.191	151.7211\\
0.191	151.7203\\
0.191	151.7194\\
0.191	151.7185\\
0.191	151.7176\\
0.192	151.7166\\
0.192	151.7156\\
0.192	151.7146\\
0.192	151.7135\\
0.192	151.7124\\
0.192	151.7113\\
0.192	151.7101\\
0.192	151.7089\\
0.192	151.7077\\
0.192	151.7064\\
0.193	151.7051\\
0.193	151.7038\\
0.193	151.7024\\
0.193	151.701\\
0.193	151.6996\\
0.193	151.6981\\
0.193	151.6966\\
0.193	151.6951\\
0.193	151.6935\\
0.193	151.6919\\
0.194	151.6903\\
0.194	151.6886\\
0.194	151.6869\\
0.194	151.6852\\
0.194	151.6835\\
0.194	151.6817\\
0.194	151.6799\\
0.194	151.678\\
0.194	151.6762\\
0.194	151.6743\\
0.195	151.6723\\
0.195	151.6704\\
0.195	151.6684\\
0.195	151.6663\\
0.195	151.6643\\
0.195	151.6622\\
0.195	151.6601\\
0.195	151.6579\\
0.195	151.6558\\
0.195	151.6536\\
0.196	151.6513\\
0.196	151.6491\\
0.196	151.6468\\
0.196	151.6445\\
0.196	151.6421\\
0.196	151.6397\\
0.196	151.6373\\
0.196	151.6349\\
0.196	151.6324\\
0.196	151.63\\
0.197	151.6274\\
0.197	151.6249\\
0.197	151.6223\\
0.197	151.6197\\
0.197	151.6171\\
0.197	151.6144\\
0.197	151.6118\\
0.197	151.609\\
0.197	151.6063\\
0.197	151.6035\\
0.198	151.6008\\
0.198	151.5979\\
0.198	151.5951\\
0.198	151.5922\\
0.198	151.5893\\
0.198	151.5864\\
0.198	151.5835\\
0.198	151.5805\\
0.198	151.5775\\
0.198	151.5745\\
0.199	151.5714\\
0.199	151.5684\\
0.199	151.5653\\
0.199	151.5621\\
0.199	151.559\\
0.199	151.5558\\
0.199	151.5526\\
0.199	151.5494\\
0.199	151.5462\\
0.199	151.5429\\
0.2	151.5397\\
0.2	151.5364\\
0.2	151.533\\
0.2	151.5297\\
0.2	151.5263\\
};
\end{axis}
\end{tikzpicture}%}
        \caption{Pressure}
        \label{fig:otherImg_a}
    \end{subfigure}
    \begin{subfigure}[b]{0.40\textwidth}
        \centering
        \resizebox{\columnwidth}{!}{% This file was created by matlab2tikz.
%
%The latest updates can be retrieved from
%  http://www.mathworks.com/matlabcentral/fileexchange/22022-matlab2tikz-matlab2tikz
%where you can also make suggestions and rate matlab2tikz.
%
\definecolor{mycolor1}{rgb}{0.00000,0.44700,0.74100}%
%
\begin{tikzpicture}

\begin{axis}[%
width=4.521in,
height=3.566in,
at={(0.758in,0.481in)},
scale only axis,
xmin=0,
xmax=0.2,
xlabel={Time / [$\unit{ps}$]},
ymin=700,
ymax=1500,
ylabel={Average temperature / [$\unit{\bar{K}}$]},
label style ={font=\Large},
axis background/.style={fill=white},
title style={font=\bfseries\Huge},
title={Average temperature}
]
\addplot [color=mycolor1,solid,forget plot]
  table[row sep=crcr]{%
0	749.9018\\
0	1499.483\\
0	1124.371\\
0	999.2261\\
0	936.5729\\
0.001	898.9161\\
0.001	873.7575\\
0.001	855.7405\\
0.001	842.187\\
0.001	831.609\\
0.001	823.1139\\
0.001	816.1336\\
0.001	810.2892\\
0.001	805.3187\\
0.001	801.0348\\
0.002	797.3\\
0.002	794.0115\\
0.002	791.0905\\
0.002	788.4756\\
0.002	786.1186\\
0.002	783.9808\\
0.002	782.0308\\
0.002	780.243\\
0.002	778.5963\\
0.002	777.0731\\
0.003	775.6585\\
0.003	774.34\\
0.003	773.107\\
0.003	771.9503\\
0.003	770.8619\\
0.003	769.8352\\
0.003	768.8641\\
0.003	767.9434\\
0.003	767.0686\\
0.003	766.2357\\
0.004	765.441\\
0.004	764.6814\\
0.004	763.9541\\
0.004	763.2565\\
0.004	762.5864\\
0.004	761.9418\\
0.004	761.3208\\
0.004	760.7217\\
0.004	760.143\\
0.004	759.5834\\
0.005	759.0417\\
0.005	758.5166\\
0.005	758.0072\\
0.005	757.5125\\
0.005	757.0316\\
0.005	756.5638\\
0.005	756.1083\\
0.005	755.6643\\
0.005	755.2314\\
0.005	754.8089\\
0.006	754.3962\\
0.006	753.993\\
0.006	753.5986\\
0.006	753.2128\\
0.006	752.835\\
0.006	752.465\\
0.006	752.1023\\
0.006	751.7467\\
0.006	751.3978\\
0.006	751.0555\\
0.007	750.7193\\
0.007	750.389\\
0.007	750.0645\\
0.007	749.7455\\
0.007	749.4318\\
0.007	749.1232\\
0.007	748.8195\\
0.007	748.5206\\
0.007	748.2263\\
0.007	747.9364\\
0.008	747.6508\\
0.008	747.3694\\
0.008	747.0921\\
0.008	746.8186\\
0.008	746.549\\
0.008	746.2831\\
0.008	746.0208\\
0.008	745.7619\\
0.008	745.5065\\
0.008	745.2544\\
0.009	745.0056\\
0.009	744.7599\\
0.009	744.5173\\
0.009	744.2778\\
0.009	744.0412\\
0.009	743.8075\\
0.009	743.5766\\
0.009	743.3485\\
0.009	743.1232\\
0.009	742.9005\\
0.009	742.6804\\
0.01	742.4629\\
0.01	742.2479\\
0.01	742.0355\\
0.01	741.8254\\
0.01	741.6178\\
0.01	741.4126\\
0.01	741.2097\\
0.01	741.0091\\
0.01	740.8108\\
0.011	740.6148\\
0.011	740.4209\\
0.011	740.2292\\
0.011	740.0398\\
0.011	739.8524\\
0.011	739.6672\\
0.011	739.484\\
0.011	739.3029\\
0.011	739.1239\\
0.011	738.9469\\
0.011	738.7719\\
0.012	738.5989\\
0.012	738.4278\\
0.012	738.2588\\
0.012	738.0916\\
0.012	737.9264\\
0.012	737.7631\\
0.012	737.6016\\
0.012	737.4421\\
0.012	737.2844\\
0.013	737.1286\\
0.013	736.9746\\
0.013	736.8224\\
0.013	736.6721\\
0.013	736.5235\\
0.013	736.3767\\
0.013	736.2318\\
0.013	736.0886\\
0.013	735.9471\\
0.013	735.8074\\
0.013	735.6695\\
0.014	735.5333\\
0.014	735.3988\\
0.014	735.266\\
0.014	735.1349\\
0.014	735.0056\\
0.014	734.8779\\
0.014	734.7519\\
0.014	734.6276\\
0.014	734.505\\
0.015	734.384\\
0.015	734.2647\\
0.015	734.1471\\
0.015	734.0311\\
0.015	733.9167\\
0.015	733.804\\
0.015	733.6929\\
0.015	733.5834\\
0.015	733.4755\\
0.015	733.3692\\
0.015	733.2646\\
0.016	733.1615\\
0.016	733.06\\
0.016	732.9601\\
0.016	732.8618\\
0.016	732.765\\
0.016	732.6698\\
0.016	732.5762\\
0.016	732.4841\\
0.016	732.3936\\
0.017	732.3046\\
0.017	732.2172\\
0.017	732.1313\\
0.017	732.0469\\
0.017	731.964\\
0.017	731.8827\\
0.017	731.8028\\
0.017	731.7245\\
0.017	731.6477\\
0.017	731.5723\\
0.018	731.4985\\
0.018	731.4261\\
0.018	731.3552\\
0.018	731.2858\\
0.018	731.2178\\
0.018	731.1513\\
0.018	731.0862\\
0.018	731.0226\\
0.018	730.9605\\
0.018	730.8997\\
0.019	730.8404\\
0.019	730.7825\\
0.019	730.7261\\
0.019	730.671\\
0.019	730.6174\\
0.019	730.5651\\
0.019	730.5143\\
0.019	730.4648\\
0.019	730.4168\\
0.019	730.37\\
0.019	730.3247\\
0.02	730.2807\\
0.02	730.2381\\
0.02	730.1969\\
0.02	730.1569\\
0.02	730.1183\\
0.02	730.0811\\
0.02	730.0452\\
0.02	730.0106\\
0.02	729.9773\\
0.021	729.9453\\
0.021	729.9146\\
0.021	729.8852\\
0.021	729.8571\\
0.021	729.8303\\
0.021	729.8047\\
0.021	729.7804\\
0.021	729.7574\\
0.021	729.7356\\
0.021	729.7151\\
0.022	729.6958\\
0.022	729.6777\\
0.022	729.6609\\
0.022	729.6453\\
0.022	729.6309\\
0.022	729.6177\\
0.022	729.6057\\
0.022	729.5949\\
0.022	729.5853\\
0.022	729.5769\\
0.023	729.5696\\
0.023	729.5635\\
0.023	729.5586\\
0.023	729.5548\\
0.023	729.5522\\
0.023	729.5506\\
0.023	729.5503\\
0.023	729.551\\
0.023	729.5528\\
0.023	729.5558\\
0.024	729.5598\\
0.024	729.565\\
0.024	729.5712\\
0.024	729.5785\\
0.024	729.5869\\
0.024	729.5963\\
0.024	729.6068\\
0.024	729.6184\\
0.024	729.6309\\
0.024	729.6446\\
0.025	729.6592\\
0.025	729.6748\\
0.025	729.6915\\
0.025	729.7092\\
0.025	729.7279\\
0.025	729.7475\\
0.025	729.7681\\
0.025	729.7898\\
0.025	729.8123\\
0.025	729.8359\\
0.026	729.8604\\
0.026	729.8858\\
0.026	729.9122\\
0.026	729.9395\\
0.026	729.9677\\
0.026	729.9969\\
0.026	730.0269\\
0.026	730.0579\\
0.026	730.0897\\
0.026	730.1224\\
0.027	730.1561\\
0.027	730.1905\\
0.027	730.2259\\
0.027	730.2621\\
0.027	730.2991\\
0.027	730.337\\
0.027	730.3757\\
0.027	730.4153\\
0.027	730.4557\\
0.027	730.4968\\
0.028	730.5388\\
0.028	730.5816\\
0.028	730.6252\\
0.028	730.6695\\
0.028	730.7146\\
0.028	730.7605\\
0.028	730.8071\\
0.028	730.8545\\
0.028	730.9027\\
0.028	730.9515\\
0.029	731.0011\\
0.029	731.0514\\
0.029	731.1025\\
0.029	731.1542\\
0.029	731.2066\\
0.029	731.2598\\
0.029	731.3136\\
0.029	731.368\\
0.029	731.4232\\
0.029	731.479\\
0.03	731.5354\\
0.03	731.5925\\
0.03	731.6502\\
0.03	731.7086\\
0.03	731.7675\\
0.03	731.8271\\
0.03	731.8873\\
0.03	731.9481\\
0.03	732.0094\\
0.03	732.0714\\
0.031	732.1339\\
0.031	732.197\\
0.031	732.2606\\
0.031	732.3248\\
0.031	732.3896\\
0.031	732.4548\\
0.031	732.5206\\
0.031	732.587\\
0.031	732.6538\\
0.031	732.7211\\
0.032	732.789\\
0.032	732.8573\\
0.032	732.9261\\
0.032	732.9954\\
0.032	733.0651\\
0.032	733.1353\\
0.032	733.2059\\
0.032	733.277\\
0.032	733.3486\\
0.032	733.4205\\
0.033	733.4929\\
0.033	733.5657\\
0.033	733.6389\\
0.033	733.7125\\
0.033	733.7865\\
0.033	733.8609\\
0.033	733.9356\\
0.033	734.0107\\
0.033	734.0862\\
0.033	734.1621\\
0.034	734.2383\\
0.034	734.3148\\
0.034	734.3917\\
0.034	734.4689\\
0.034	734.5464\\
0.034	734.6242\\
0.034	734.7023\\
0.034	734.7808\\
0.034	734.8595\\
0.034	734.9385\\
0.035	735.0178\\
0.035	735.0974\\
0.035	735.1772\\
0.035	735.2573\\
0.035	735.3377\\
0.035	735.4183\\
0.035	735.4991\\
0.035	735.5802\\
0.035	735.6615\\
0.035	735.743\\
0.036	735.8248\\
0.036	735.9068\\
0.036	735.9889\\
0.036	736.0713\\
0.036	736.1539\\
0.036	736.2366\\
0.036	736.3196\\
0.036	736.4027\\
0.036	736.486\\
0.036	736.5694\\
0.037	736.653\\
0.037	736.7368\\
0.037	736.8208\\
0.037	736.9048\\
0.037	736.9891\\
0.037	737.0734\\
0.037	737.1579\\
0.037	737.2425\\
0.037	737.3273\\
0.037	737.4121\\
0.037	737.4971\\
0.038	737.5822\\
0.038	737.6674\\
0.038	737.7527\\
0.038	737.8381\\
0.038	737.9236\\
0.038	738.0091\\
0.038	738.0948\\
0.038	738.1805\\
0.038	738.2664\\
0.038	738.3523\\
0.039	738.4382\\
0.039	738.5243\\
0.039	738.6104\\
0.039	738.6965\\
0.039	738.7827\\
0.039	738.869\\
0.039	738.9553\\
0.039	739.0417\\
0.039	739.1281\\
0.04	739.2145\\
0.04	739.301\\
0.04	739.3876\\
0.04	739.4741\\
0.04	739.5608\\
0.04	739.6474\\
0.04	739.734\\
0.04	739.8207\\
0.04	739.9075\\
0.04	739.9942\\
0.041	740.0809\\
0.041	740.1677\\
0.041	740.2545\\
0.041	740.3413\\
0.041	740.4282\\
0.041	740.515\\
0.041	740.6018\\
0.041	740.6887\\
0.041	740.7756\\
0.041	740.8625\\
0.042	740.9494\\
0.042	741.0362\\
0.042	741.1231\\
0.042	741.2101\\
0.042	741.297\\
0.042	741.3839\\
0.042	741.4708\\
0.042	741.5577\\
0.042	741.6446\\
0.042	741.7315\\
0.043	741.8185\\
0.043	741.9054\\
0.043	741.9923\\
0.043	742.0793\\
0.043	742.1662\\
0.043	742.2531\\
0.043	742.3401\\
0.043	742.427\\
0.043	742.5139\\
0.043	742.6009\\
0.044	742.6878\\
0.044	742.7748\\
0.044	742.8617\\
0.044	742.9487\\
0.044	743.0357\\
0.044	743.1227\\
0.044	743.2097\\
0.044	743.2967\\
0.044	743.3837\\
0.044	743.4707\\
0.045	743.5578\\
0.045	743.6448\\
0.045	743.7319\\
0.045	743.819\\
0.045	743.9061\\
0.045	743.9933\\
0.045	744.0804\\
0.045	744.1676\\
0.045	744.2549\\
0.045	744.3421\\
0.045	744.4294\\
0.046	744.5167\\
0.046	744.6041\\
0.046	744.6915\\
0.046	744.7789\\
0.046	744.8664\\
0.046	744.9539\\
0.046	745.0415\\
0.046	745.1292\\
0.046	745.2169\\
0.046	745.3046\\
0.047	745.3924\\
0.047	745.4803\\
0.047	745.5682\\
0.047	745.6562\\
0.047	745.7443\\
0.047	745.8324\\
0.047	745.9206\\
0.047	746.0089\\
0.047	746.0973\\
0.048	746.1857\\
0.048	746.2742\\
0.048	746.3629\\
0.048	746.4516\\
0.048	746.5404\\
0.048	746.6293\\
0.048	746.7183\\
0.048	746.8074\\
0.048	746.8967\\
0.048	746.986\\
0.049	747.0755\\
0.049	747.165\\
0.049	747.2547\\
0.049	747.3445\\
0.049	747.4344\\
0.049	747.5245\\
0.049	747.6147\\
0.049	747.705\\
0.049	747.7955\\
0.049	747.8861\\
0.05	747.9768\\
0.05	748.0677\\
0.05	748.1588\\
0.05	748.25\\
0.05	748.3413\\
0.05	748.4328\\
0.05	748.5245\\
0.05	748.6164\\
0.05	748.7084\\
0.05	748.8005\\
0.051	748.8929\\
0.051	748.9854\\
0.051	749.0781\\
0.051	749.1709\\
0.051	749.264\\
0.051	749.3572\\
0.051	749.4506\\
0.051	749.5443\\
0.051	749.638\\
0.051	749.732\\
0.052	749.8262\\
0.052	749.9206\\
0.052	750.0152\\
0.052	750.11\\
0.052	750.2049\\
0.052	750.3001\\
0.052	750.3955\\
0.052	750.4911\\
0.052	750.5869\\
0.052	750.6829\\
0.053	750.7792\\
0.053	750.8756\\
0.053	750.9722\\
0.053	751.0691\\
0.053	751.1662\\
0.053	751.2635\\
0.053	751.361\\
0.053	751.4587\\
0.053	751.5567\\
0.053	751.6548\\
0.054	751.7532\\
0.054	751.8518\\
0.054	751.9506\\
0.054	752.0497\\
0.054	752.1489\\
0.054	752.2484\\
0.054	752.3481\\
0.054	752.448\\
0.054	752.5481\\
0.054	752.6485\\
0.054	752.749\\
0.055	752.8498\\
0.055	752.9508\\
0.055	753.052\\
0.055	753.1534\\
0.055	753.2551\\
0.055	753.3569\\
0.055	753.4589\\
0.055	753.5612\\
0.055	753.6637\\
0.056	753.7663\\
0.056	753.8692\\
0.056	753.9722\\
0.056	754.0755\\
0.056	754.1789\\
0.056	754.2826\\
0.056	754.3864\\
0.056	754.4904\\
0.056	754.5946\\
0.056	754.699\\
0.057	754.8035\\
0.057	754.9083\\
0.057	755.0132\\
0.057	755.1182\\
0.057	755.2235\\
0.057	755.3289\\
0.057	755.4344\\
0.057	755.5401\\
0.057	755.646\\
0.057	755.752\\
0.058	755.8581\\
0.058	755.9644\\
0.058	756.0708\\
0.058	756.1773\\
0.058	756.284\\
0.058	756.3908\\
0.058	756.4976\\
0.058	756.6046\\
0.058	756.7117\\
0.058	756.8189\\
0.059	756.9262\\
0.059	757.0336\\
0.059	757.1411\\
0.059	757.2486\\
0.059	757.3562\\
0.059	757.4639\\
0.059	757.5716\\
0.059	757.6794\\
0.059	757.7873\\
0.059	757.8951\\
0.06	758.0031\\
0.06	758.111\\
0.06	758.219\\
0.06	758.327\\
0.06	758.4349\\
0.06	758.5429\\
0.06	758.6509\\
0.06	758.7589\\
0.06	758.8669\\
0.06	758.9748\\
0.061	759.0828\\
0.061	759.1906\\
0.061	759.2985\\
0.061	759.4063\\
0.061	759.514\\
0.061	759.6217\\
0.061	759.7293\\
0.061	759.8368\\
0.061	759.9442\\
0.061	760.0516\\
0.062	760.1588\\
0.062	760.266\\
0.062	760.373\\
0.062	760.4799\\
0.062	760.5867\\
0.062	760.6933\\
0.062	760.7998\\
0.062	760.9062\\
0.062	761.0124\\
0.062	761.1184\\
0.062	761.2242\\
0.063	761.3299\\
0.063	761.4354\\
0.063	761.5407\\
0.063	761.6458\\
0.063	761.7506\\
0.063	761.8553\\
0.063	761.9597\\
0.063	762.0639\\
0.063	762.1679\\
0.064	762.2716\\
0.064	762.375\\
0.064	762.4782\\
0.064	762.5811\\
0.064	762.6838\\
0.064	762.7861\\
0.064	762.8882\\
0.064	762.99\\
0.064	763.0914\\
0.064	763.1925\\
0.065	763.2933\\
0.065	763.3938\\
0.065	763.4939\\
0.065	763.5937\\
0.065	763.6932\\
0.065	763.7922\\
0.065	763.8909\\
0.065	763.9892\\
0.065	764.0872\\
0.065	764.1847\\
0.066	764.2818\\
0.066	764.3786\\
0.066	764.4749\\
0.066	764.5708\\
0.066	764.6662\\
0.066	764.7613\\
0.066	764.8558\\
0.066	764.95\\
0.066	765.0436\\
0.066	765.1368\\
0.067	765.2296\\
0.067	765.3218\\
0.067	765.4136\\
0.067	765.5048\\
0.067	765.5956\\
0.067	765.6858\\
0.067	765.7756\\
0.067	765.8648\\
0.067	765.9535\\
0.067	766.0416\\
0.068	766.1292\\
0.068	766.2163\\
0.068	766.3028\\
0.068	766.3887\\
0.068	766.4741\\
0.068	766.5589\\
0.068	766.6431\\
0.068	766.7267\\
0.068	766.8097\\
0.068	766.8921\\
0.069	766.974\\
0.069	767.0552\\
0.069	767.1358\\
0.069	767.2157\\
0.069	767.2951\\
0.069	767.3738\\
0.069	767.4518\\
0.069	767.5292\\
0.069	767.606\\
0.069	767.6821\\
0.07	767.7576\\
0.07	767.8323\\
0.07	767.9064\\
0.07	767.9799\\
0.07	768.0526\\
0.07	768.1247\\
0.07	768.1961\\
0.07	768.2667\\
0.07	768.3367\\
0.07	768.406\\
0.071	768.4746\\
0.071	768.5424\\
0.071	768.6095\\
0.071	768.676\\
0.071	768.7417\\
0.071	768.8066\\
0.071	768.8708\\
0.071	768.9343\\
0.071	768.9971\\
0.071	769.0591\\
0.072	769.1203\\
0.072	769.1809\\
0.072	769.2406\\
0.072	769.2996\\
0.072	769.3578\\
0.072	769.4153\\
0.072	769.472\\
0.072	769.5279\\
0.072	769.5831\\
0.072	769.6375\\
0.073	769.6911\\
0.073	769.7439\\
0.073	769.7959\\
0.073	769.8472\\
0.073	769.8976\\
0.073	769.9473\\
0.073	769.9962\\
0.073	770.0442\\
0.073	770.0915\\
0.073	770.138\\
0.074	770.1837\\
0.074	770.2285\\
0.074	770.2726\\
0.074	770.3158\\
0.074	770.3583\\
0.074	770.3999\\
0.074	770.4407\\
0.074	770.4807\\
0.074	770.5198\\
0.074	770.5582\\
0.074	770.5957\\
0.075	770.6324\\
0.075	770.6683\\
0.075	770.7034\\
0.075	770.7376\\
0.075	770.771\\
0.075	770.8036\\
0.075	770.8354\\
0.075	770.8663\\
0.075	770.8964\\
0.075	770.9256\\
0.076	770.9541\\
0.076	770.9817\\
0.076	771.0085\\
0.076	771.0344\\
0.076	771.0595\\
0.076	771.0838\\
0.076	771.1072\\
0.076	771.1298\\
0.076	771.1516\\
0.076	771.1726\\
0.077	771.1927\\
0.077	771.212\\
0.077	771.2304\\
0.077	771.248\\
0.077	771.2648\\
0.077	771.2808\\
0.077	771.2959\\
0.077	771.3102\\
0.077	771.3237\\
0.077	771.3363\\
0.078	771.3481\\
0.078	771.3591\\
0.078	771.3693\\
0.078	771.3786\\
0.078	771.3871\\
0.078	771.3948\\
0.078	771.4017\\
0.078	771.4078\\
0.078	771.413\\
0.079	771.4174\\
0.079	771.421\\
0.079	771.4238\\
0.079	771.4258\\
0.079	771.427\\
0.079	771.4273\\
0.079	771.4269\\
0.079	771.4256\\
0.079	771.4236\\
0.079	771.4207\\
0.08	771.4171\\
0.08	771.4126\\
0.08	771.4074\\
0.08	771.4013\\
0.08	771.3945\\
0.08	771.3869\\
0.08	771.3784\\
0.08	771.3692\\
0.08	771.3592\\
0.08	771.3485\\
0.081	771.3369\\
0.081	771.3246\\
0.081	771.3115\\
0.081	771.2976\\
0.081	771.283\\
0.081	771.2676\\
0.081	771.2514\\
0.081	771.2345\\
0.081	771.2168\\
0.081	771.1984\\
0.082	771.1792\\
0.082	771.1592\\
0.082	771.1386\\
0.082	771.1171\\
0.082	771.095\\
0.082	771.072\\
0.082	771.0484\\
0.082	771.024\\
0.082	770.9989\\
0.082	770.9731\\
0.083	770.9466\\
0.083	770.9193\\
0.083	770.8913\\
0.083	770.8626\\
0.083	770.8333\\
0.083	770.8032\\
0.083	770.7724\\
0.083	770.7409\\
0.083	770.7087\\
0.083	770.6759\\
0.084	770.6423\\
0.084	770.6081\\
0.084	770.5732\\
0.084	770.5377\\
0.084	770.5014\\
0.084	770.4646\\
0.084	770.427\\
0.084	770.3888\\
0.084	770.35\\
0.084	770.3105\\
0.085	770.2704\\
0.085	770.2296\\
0.085	770.1882\\
0.085	770.1462\\
0.085	770.1036\\
0.085	770.0604\\
0.085	770.0165\\
0.085	769.9721\\
0.085	769.927\\
0.085	769.8814\\
0.086	769.8351\\
0.086	769.7883\\
0.086	769.7409\\
0.086	769.693\\
0.086	769.6444\\
0.086	769.5953\\
0.086	769.5457\\
0.086	769.4955\\
0.086	769.4448\\
0.086	769.3935\\
0.087	769.3417\\
0.087	769.2893\\
0.087	769.2364\\
0.087	769.1831\\
0.087	769.1292\\
0.087	769.0748\\
0.087	769.0199\\
0.087	768.9645\\
0.087	768.9086\\
0.087	768.8523\\
0.088	768.7954\\
0.088	768.7381\\
0.088	768.6804\\
0.088	768.6222\\
0.088	768.5635\\
0.088	768.5044\\
0.088	768.4449\\
0.088	768.3849\\
0.088	768.3245\\
0.088	768.2637\\
0.089	768.2024\\
0.089	768.1408\\
0.089	768.0788\\
0.089	768.0164\\
0.089	767.9536\\
0.089	767.8904\\
0.089	767.8268\\
0.089	767.7629\\
0.089	767.6986\\
0.089	767.634\\
0.09	767.569\\
0.09	767.5037\\
0.09	767.438\\
0.09	767.3721\\
0.09	767.3058\\
0.09	767.2392\\
0.09	767.1723\\
0.09	767.1051\\
0.09	767.0376\\
0.09	766.9698\\
0.091	766.9018\\
0.091	766.8335\\
0.091	766.7649\\
0.091	766.696\\
0.091	766.627\\
0.091	766.5576\\
0.091	766.4881\\
0.091	766.4183\\
0.091	766.3483\\
0.091	766.2781\\
0.091	766.2077\\
0.092	766.1371\\
0.092	766.0663\\
0.092	765.9953\\
0.092	765.9241\\
0.092	765.8528\\
0.092	765.7813\\
0.092	765.7096\\
0.092	765.6378\\
0.092	765.5659\\
0.092	765.4938\\
0.093	765.4216\\
0.093	765.3493\\
0.093	765.2769\\
0.093	765.2044\\
0.093	765.1318\\
0.093	765.059\\
0.093	764.9862\\
0.093	764.9134\\
0.093	764.8404\\
0.093	764.7675\\
0.094	764.6944\\
0.094	764.6213\\
0.094	764.5482\\
0.094	764.475\\
0.094	764.4018\\
0.094	764.3286\\
0.094	764.2554\\
0.094	764.1822\\
0.094	764.109\\
0.095	764.0358\\
0.095	763.9626\\
0.095	763.8894\\
0.095	763.8163\\
0.095	763.7432\\
0.095	763.6702\\
0.095	763.5972\\
0.095	763.5243\\
0.095	763.4514\\
0.095	763.3786\\
0.096	763.3059\\
0.096	763.2333\\
0.096	763.1608\\
0.096	763.0884\\
0.096	763.016\\
0.096	762.9438\\
0.096	762.8718\\
0.096	762.7998\\
0.096	762.728\\
0.096	762.6563\\
0.097	762.5848\\
0.097	762.5134\\
0.097	762.4422\\
0.097	762.3711\\
0.097	762.3002\\
0.097	762.2295\\
0.097	762.159\\
0.097	762.0887\\
0.097	762.0185\\
0.097	761.9486\\
0.098	761.8789\\
0.098	761.8093\\
0.098	761.7401\\
0.098	761.671\\
0.098	761.6022\\
0.098	761.5336\\
0.098	761.4652\\
0.098	761.3971\\
0.098	761.3293\\
0.098	761.2617\\
0.099	761.1944\\
0.099	761.1273\\
0.099	761.0605\\
0.099	760.994\\
0.099	760.9278\\
0.099	760.8619\\
0.099	760.7963\\
0.099	760.731\\
0.099	760.666\\
0.099	760.6013\\
0.1	760.5369\\
0.1	760.4729\\
0.1	760.4092\\
0.1	760.3458\\
0.1	760.2827\\
0.1	760.22\\
0.1	760.1577\\
0.1	760.0957\\
0.1	760.034\\
0.1	759.9727\\
0.101	759.9118\\
0.101	759.8512\\
0.101	759.791\\
0.101	759.7312\\
0.101	759.6718\\
0.101	759.6127\\
0.101	759.5541\\
0.101	759.4958\\
0.101	759.438\\
0.101	759.3805\\
0.102	759.3234\\
0.102	759.2668\\
0.102	759.2105\\
0.102	759.1547\\
0.102	759.0993\\
0.102	759.0443\\
0.102	758.9897\\
0.102	758.9356\\
0.102	758.8819\\
0.102	758.8286\\
0.103	758.7757\\
0.103	758.7233\\
0.103	758.6714\\
0.103	758.6198\\
0.103	758.5688\\
0.103	758.5182\\
0.103	758.468\\
0.103	758.4183\\
0.103	758.369\\
0.103	758.3202\\
0.104	758.2719\\
0.104	758.224\\
0.104	758.1766\\
0.104	758.1297\\
0.104	758.0832\\
0.104	758.0372\\
0.104	757.9917\\
0.104	757.9466\\
0.104	757.9021\\
0.104	757.858\\
0.105	757.8143\\
0.105	757.7712\\
0.105	757.7286\\
0.105	757.6864\\
0.105	757.6447\\
0.105	757.6035\\
0.105	757.5628\\
0.105	757.5226\\
0.105	757.4828\\
0.105	757.4436\\
0.106	757.4049\\
0.106	757.3666\\
0.106	757.3288\\
0.106	757.2915\\
0.106	757.2548\\
0.106	757.2185\\
0.106	757.1827\\
0.106	757.1474\\
0.106	757.1125\\
0.106	757.0782\\
0.107	757.0444\\
0.107	757.0111\\
0.107	756.9782\\
0.107	756.9459\\
0.107	756.914\\
0.107	756.8827\\
0.107	756.8518\\
0.107	756.8214\\
0.107	756.7915\\
0.107	756.7621\\
0.107	756.7332\\
0.108	756.7048\\
0.108	756.6768\\
0.108	756.6494\\
0.108	756.6224\\
0.108	756.5959\\
0.108	756.5699\\
0.108	756.5444\\
0.108	756.5193\\
0.108	756.4948\\
0.108	756.4707\\
0.109	756.4471\\
0.109	756.4239\\
0.109	756.4013\\
0.109	756.3791\\
0.109	756.3573\\
0.109	756.3361\\
0.109	756.3153\\
0.109	756.2949\\
0.109	756.2751\\
0.11	756.2557\\
0.11	756.2367\\
0.11	756.2182\\
0.11	756.2001\\
0.11	756.1825\\
0.11	756.1654\\
0.11	756.1487\\
0.11	756.1324\\
0.11	756.1166\\
0.11	756.1012\\
0.111	756.0863\\
0.111	756.0717\\
0.111	756.0577\\
0.111	756.044\\
0.111	756.0307\\
0.111	756.0179\\
0.111	756.0055\\
0.111	755.9935\\
0.111	755.982\\
0.111	755.9708\\
0.112	755.9601\\
0.112	755.9497\\
0.112	755.9398\\
0.112	755.9302\\
0.112	755.921\\
0.112	755.9123\\
0.112	755.9039\\
0.112	755.8959\\
0.112	755.8882\\
0.112	755.881\\
0.113	755.8741\\
0.113	755.8676\\
0.113	755.8615\\
0.113	755.8557\\
0.113	755.8503\\
0.113	755.8452\\
0.113	755.8405\\
0.113	755.8361\\
0.113	755.8321\\
0.113	755.8284\\
0.114	755.825\\
0.114	755.822\\
0.114	755.8193\\
0.114	755.817\\
0.114	755.8149\\
0.114	755.8132\\
0.114	755.8118\\
0.114	755.8106\\
0.114	755.8098\\
0.114	755.8093\\
0.115	755.8091\\
0.115	755.8092\\
0.115	755.8096\\
0.115	755.8102\\
0.115	755.8112\\
0.115	755.8124\\
0.115	755.8138\\
0.115	755.8156\\
0.115	755.8176\\
0.115	755.8199\\
0.116	755.8224\\
0.116	755.8252\\
0.116	755.8282\\
0.116	755.8315\\
0.116	755.835\\
0.116	755.8387\\
0.116	755.8427\\
0.116	755.8469\\
0.116	755.8513\\
0.116	755.8559\\
0.117	755.8608\\
0.117	755.8658\\
0.117	755.8711\\
0.117	755.8766\\
0.117	755.8822\\
0.117	755.8881\\
0.117	755.8942\\
0.117	755.9004\\
0.117	755.9068\\
0.117	755.9134\\
0.118	755.9202\\
0.118	755.9271\\
0.118	755.9342\\
0.118	755.9415\\
0.118	755.9489\\
0.118	755.9565\\
0.118	755.9643\\
0.118	755.9721\\
0.118	755.9802\\
0.118	755.9883\\
0.119	755.9966\\
0.119	756.0051\\
0.119	756.0137\\
0.119	756.0224\\
0.119	756.0312\\
0.119	756.0401\\
0.119	756.0492\\
0.119	756.0583\\
0.119	756.0676\\
0.119	756.077\\
0.12	756.0865\\
0.12	756.0961\\
0.12	756.1058\\
0.12	756.1155\\
0.12	756.1254\\
0.12	756.1353\\
0.12	756.1454\\
0.12	756.1555\\
0.12	756.1657\\
0.12	756.176\\
0.121	756.1863\\
0.121	756.1967\\
0.121	756.2072\\
0.121	756.2178\\
0.121	756.2284\\
0.121	756.2391\\
0.121	756.2498\\
0.121	756.2606\\
0.121	756.2714\\
0.121	756.2823\\
0.122	756.2932\\
0.122	756.3042\\
0.122	756.3152\\
0.122	756.3263\\
0.122	756.3374\\
0.122	756.3485\\
0.122	756.3597\\
0.122	756.3709\\
0.122	756.3821\\
0.122	756.3934\\
0.123	756.4046\\
0.123	756.4159\\
0.123	756.4273\\
0.123	756.4386\\
0.123	756.45\\
0.123	756.4614\\
0.123	756.4728\\
0.123	756.4842\\
0.123	756.4956\\
0.123	756.507\\
0.124	756.5185\\
0.124	756.5299\\
0.124	756.5414\\
0.124	756.5528\\
0.124	756.5643\\
0.124	756.5758\\
0.124	756.5872\\
0.124	756.5987\\
0.124	756.6101\\
0.124	756.6216\\
0.124	756.6331\\
0.125	756.6445\\
0.125	756.656\\
0.125	756.6674\\
0.125	756.6788\\
0.125	756.6902\\
0.125	756.7017\\
0.125	756.7131\\
0.125	756.7244\\
0.125	756.7358\\
0.126	756.7472\\
0.126	756.7585\\
0.126	756.7699\\
0.126	756.7812\\
0.126	756.7925\\
0.126	756.8038\\
0.126	756.8151\\
0.126	756.8263\\
0.126	756.8376\\
0.126	756.8488\\
0.127	756.86\\
0.127	756.8712\\
0.127	756.8824\\
0.127	756.8935\\
0.127	756.9047\\
0.127	756.9158\\
0.127	756.9269\\
0.127	756.938\\
0.127	756.949\\
0.127	756.9601\\
0.128	756.9711\\
0.128	756.9821\\
0.128	756.993\\
0.128	757.004\\
0.128	757.0149\\
0.128	757.0259\\
0.128	757.0367\\
0.128	757.0476\\
0.128	757.0585\\
0.128	757.0693\\
0.129	757.0801\\
0.129	757.0909\\
0.129	757.1017\\
0.129	757.1125\\
0.129	757.1232\\
0.129	757.1339\\
0.129	757.1446\\
0.129	757.1553\\
0.129	757.166\\
0.129	757.1766\\
0.13	757.1873\\
0.13	757.1979\\
0.13	757.2085\\
0.13	757.219\\
0.13	757.2296\\
0.13	757.2402\\
0.13	757.2507\\
0.13	757.2612\\
0.13	757.2717\\
0.13	757.2822\\
0.131	757.2926\\
0.131	757.3031\\
0.131	757.3135\\
0.131	757.3239\\
0.131	757.3344\\
0.131	757.3448\\
0.131	757.3551\\
0.131	757.3655\\
0.131	757.3759\\
0.131	757.3862\\
0.132	757.3965\\
0.132	757.4069\\
0.132	757.4172\\
0.132	757.4275\\
0.132	757.4378\\
0.132	757.4481\\
0.132	757.4583\\
0.132	757.4686\\
0.132	757.4788\\
0.132	757.4891\\
0.133	757.4993\\
0.133	757.5095\\
0.133	757.5198\\
0.133	757.53\\
0.133	757.5402\\
0.133	757.5504\\
0.133	757.5606\\
0.133	757.5707\\
0.133	757.5809\\
0.133	757.5911\\
0.134	757.6012\\
0.134	757.6114\\
0.134	757.6215\\
0.134	757.6317\\
0.134	757.6418\\
0.134	757.652\\
0.134	757.6621\\
0.134	757.6722\\
0.134	757.6823\\
0.134	757.6925\\
0.135	757.7026\\
0.135	757.7127\\
0.135	757.7228\\
0.135	757.7329\\
0.135	757.743\\
0.135	757.7531\\
0.135	757.7631\\
0.135	757.7732\\
0.135	757.7833\\
0.135	757.7934\\
0.136	757.8034\\
0.136	757.8135\\
0.136	757.8236\\
0.136	757.8336\\
0.136	757.8437\\
0.136	757.8537\\
0.136	757.8638\\
0.136	757.8738\\
0.136	757.8839\\
0.136	757.8939\\
0.137	757.9039\\
0.137	757.914\\
0.137	757.924\\
0.137	757.934\\
0.137	757.944\\
0.137	757.954\\
0.137	757.964\\
0.137	757.974\\
0.137	757.984\\
0.137	757.994\\
0.138	758.0039\\
0.138	758.0139\\
0.138	758.0239\\
0.138	758.0338\\
0.138	758.0438\\
0.138	758.0537\\
0.138	758.0637\\
0.138	758.0736\\
0.138	758.0835\\
0.138	758.0934\\
0.139	758.1033\\
0.139	758.1132\\
0.139	758.1231\\
0.139	758.1329\\
0.139	758.1428\\
0.139	758.1526\\
0.139	758.1625\\
0.139	758.1723\\
0.139	758.1821\\
0.139	758.1919\\
0.14	758.2017\\
0.14	758.2114\\
0.14	758.2212\\
0.14	758.2309\\
0.14	758.2406\\
0.14	758.2503\\
0.14	758.26\\
0.14	758.2697\\
0.14	758.2793\\
0.14	758.2889\\
0.141	758.2985\\
0.141	758.3081\\
0.141	758.3177\\
0.141	758.3272\\
0.141	758.3367\\
0.141	758.3462\\
0.141	758.3557\\
0.141	758.3651\\
0.141	758.3746\\
0.141	758.3839\\
0.142	758.3933\\
0.142	758.4026\\
0.142	758.4119\\
0.142	758.4212\\
0.142	758.4305\\
0.142	758.4397\\
0.142	758.4489\\
0.142	758.458\\
0.142	758.4671\\
0.142	758.4762\\
0.143	758.4853\\
0.143	758.4943\\
0.143	758.5032\\
0.143	758.5122\\
0.143	758.5211\\
0.143	758.5299\\
0.143	758.5387\\
0.143	758.5475\\
0.143	758.5562\\
0.143	758.5649\\
0.144	758.5736\\
0.144	758.5822\\
0.144	758.5907\\
0.144	758.5992\\
0.144	758.6077\\
0.144	758.6161\\
0.144	758.6244\\
0.144	758.6327\\
0.144	758.641\\
0.144	758.6492\\
0.145	758.6574\\
0.145	758.6655\\
0.145	758.6735\\
0.145	758.6815\\
0.145	758.6894\\
0.145	758.6973\\
0.145	758.7051\\
0.145	758.7129\\
0.145	758.7206\\
0.145	758.7282\\
0.146	758.7358\\
0.146	758.7433\\
0.146	758.7508\\
0.146	758.7582\\
0.146	758.7655\\
0.146	758.7728\\
0.146	758.78\\
0.146	758.7871\\
0.146	758.7942\\
0.146	758.8012\\
0.147	758.8081\\
0.147	758.815\\
0.147	758.8218\\
0.147	758.8285\\
0.147	758.8351\\
0.147	758.8417\\
0.147	758.8482\\
0.147	758.8547\\
0.147	758.861\\
0.147	758.8673\\
0.148	758.8735\\
0.148	758.8796\\
0.148	758.8857\\
0.148	758.8916\\
0.148	758.8975\\
0.148	758.9033\\
0.148	758.909\\
0.148	758.9147\\
0.148	758.9203\\
0.148	758.9257\\
0.149	758.9311\\
0.149	758.9364\\
0.149	758.9417\\
0.149	758.9468\\
0.149	758.9519\\
0.149	758.9569\\
0.149	758.9617\\
0.149	758.9665\\
0.149	758.9712\\
0.149	758.9759\\
0.149	758.9804\\
0.15	758.9848\\
0.15	758.9892\\
0.15	758.9935\\
0.15	758.9976\\
0.15	759.0017\\
0.15	759.0057\\
0.15	759.0096\\
0.15	759.0134\\
0.15	759.0171\\
0.15	759.0207\\
0.151	759.0242\\
0.151	759.0276\\
0.151	759.0309\\
0.151	759.0341\\
0.151	759.0372\\
0.151	759.0403\\
0.151	759.0432\\
0.151	759.046\\
0.151	759.0487\\
0.151	759.0514\\
0.152	759.0539\\
0.152	759.0563\\
0.152	759.0586\\
0.152	759.0608\\
0.152	759.063\\
0.152	759.065\\
0.152	759.0669\\
0.152	759.0687\\
0.152	759.0704\\
0.152	759.072\\
0.153	759.0735\\
0.153	759.0749\\
0.153	759.0762\\
0.153	759.0774\\
0.153	759.0785\\
0.153	759.0794\\
0.153	759.0803\\
0.153	759.0811\\
0.153	759.0817\\
0.153	759.0823\\
0.154	759.0827\\
0.154	759.0831\\
0.154	759.0833\\
0.154	759.0834\\
0.154	759.0834\\
0.154	759.0833\\
0.154	759.0831\\
0.154	759.0828\\
0.154	759.0824\\
0.154	759.0819\\
0.155	759.0813\\
0.155	759.0805\\
0.155	759.0797\\
0.155	759.0787\\
0.155	759.0776\\
0.155	759.0765\\
0.155	759.0752\\
0.155	759.0738\\
0.155	759.0723\\
0.155	759.0707\\
0.156	759.069\\
0.156	759.0672\\
0.156	759.0653\\
0.156	759.0632\\
0.156	759.0611\\
0.156	759.0588\\
0.156	759.0565\\
0.156	759.054\\
0.156	759.0514\\
0.157	759.0488\\
0.157	759.046\\
0.157	759.0431\\
0.157	759.0401\\
0.157	759.037\\
0.157	759.0338\\
0.157	759.0305\\
0.157	759.0271\\
0.157	759.0236\\
0.157	759.02\\
0.158	759.0162\\
0.158	759.0124\\
0.158	759.0085\\
0.158	759.0044\\
0.158	759.0003\\
0.158	758.9961\\
0.158	758.9917\\
0.158	758.9873\\
0.158	758.9828\\
0.158	758.9781\\
0.159	758.9734\\
0.159	758.9685\\
0.159	758.9636\\
0.159	758.9586\\
0.159	758.9534\\
0.159	758.9482\\
0.159	758.9429\\
0.159	758.9374\\
0.159	758.9319\\
0.159	758.9263\\
0.16	758.9206\\
0.16	758.9148\\
0.16	758.9089\\
0.16	758.9029\\
0.16	758.8968\\
0.16	758.8906\\
0.16	758.8844\\
0.16	758.878\\
0.16	758.8716\\
0.16	758.8651\\
0.161	758.8584\\
0.161	758.8517\\
0.161	758.8449\\
0.161	758.8381\\
0.161	758.8311\\
0.161	758.824\\
0.161	758.8169\\
0.161	758.8097\\
0.161	758.8024\\
0.161	758.795\\
0.162	758.7876\\
0.162	758.78\\
0.162	758.7724\\
0.162	758.7647\\
0.162	758.757\\
0.162	758.7491\\
0.162	758.7412\\
0.162	758.7332\\
0.162	758.7251\\
0.162	758.717\\
0.163	758.7088\\
0.163	758.7005\\
0.163	758.6922\\
0.163	758.6838\\
0.163	758.6753\\
0.163	758.6667\\
0.163	758.6581\\
0.163	758.6494\\
0.163	758.6407\\
0.163	758.6319\\
0.164	758.623\\
0.164	758.6141\\
0.164	758.6051\\
0.164	758.5961\\
0.164	758.587\\
0.164	758.5778\\
0.164	758.5686\\
0.164	758.5593\\
0.164	758.55\\
0.164	758.5406\\
0.165	758.5312\\
0.165	758.5217\\
0.165	758.5122\\
0.165	758.5027\\
0.165	758.4931\\
0.165	758.4834\\
0.165	758.4737\\
0.165	758.464\\
0.165	758.4542\\
0.165	758.4444\\
0.166	758.4345\\
0.166	758.4246\\
0.166	758.4147\\
0.166	758.4047\\
0.166	758.3947\\
0.166	758.3846\\
0.166	758.3746\\
0.166	758.3645\\
0.166	758.3543\\
0.166	758.3442\\
0.167	758.334\\
0.167	758.3238\\
0.167	758.3135\\
0.167	758.3033\\
0.167	758.293\\
0.167	758.2827\\
0.167	758.2724\\
0.167	758.262\\
0.167	758.2516\\
0.167	758.2413\\
0.168	758.2309\\
0.168	758.2205\\
0.168	758.2101\\
0.168	758.1996\\
0.168	758.1892\\
0.168	758.1787\\
0.168	758.1683\\
0.168	758.1578\\
0.168	758.1474\\
0.168	758.1369\\
0.169	758.1264\\
0.169	758.1159\\
0.169	758.1055\\
0.169	758.095\\
0.169	758.0845\\
0.169	758.0741\\
0.169	758.0636\\
0.169	758.0532\\
0.169	758.0427\\
0.169	758.0323\\
0.17	758.0219\\
0.17	758.0115\\
0.17	758.0011\\
0.17	757.9907\\
0.17	757.9804\\
0.17	757.97\\
0.17	757.9597\\
0.17	757.9494\\
0.17	757.9392\\
0.17	757.9289\\
0.171	757.9187\\
0.171	757.9085\\
0.171	757.8983\\
0.171	757.8882\\
0.171	757.8781\\
0.171	757.868\\
0.171	757.858\\
0.171	757.848\\
0.171	757.838\\
0.171	757.8281\\
0.172	757.8182\\
0.172	757.8084\\
0.172	757.7986\\
0.172	757.7889\\
0.172	757.7792\\
0.172	757.7695\\
0.172	757.7599\\
0.172	757.7504\\
0.172	757.7409\\
0.172	757.7314\\
0.173	757.722\\
0.173	757.7127\\
0.173	757.7034\\
0.173	757.6942\\
0.173	757.6851\\
0.173	757.676\\
0.173	757.667\\
0.173	757.658\\
0.173	757.6492\\
0.173	757.6404\\
0.174	757.6316\\
0.174	757.6229\\
0.174	757.6144\\
0.174	757.6058\\
0.174	757.5974\\
0.174	757.589\\
0.174	757.5808\\
0.174	757.5726\\
0.174	757.5644\\
0.174	757.5564\\
0.175	757.5485\\
0.175	757.5406\\
0.175	757.5329\\
0.175	757.5252\\
0.175	757.5176\\
0.175	757.5101\\
0.175	757.5027\\
0.175	757.4954\\
0.175	757.4882\\
0.175	757.4811\\
0.176	757.4742\\
0.176	757.4673\\
0.176	757.4605\\
0.176	757.4538\\
0.176	757.4472\\
0.176	757.4408\\
0.176	757.4344\\
0.176	757.4282\\
0.176	757.422\\
0.176	757.416\\
0.177	757.4101\\
0.177	757.4043\\
0.177	757.3987\\
0.177	757.3931\\
0.177	757.3877\\
0.177	757.3824\\
0.177	757.3772\\
0.177	757.3722\\
0.177	757.3672\\
0.177	757.3624\\
0.178	757.3578\\
0.178	757.3532\\
0.178	757.3488\\
0.178	757.3445\\
0.178	757.3404\\
0.178	757.3364\\
0.178	757.3325\\
0.178	757.3288\\
0.178	757.3252\\
0.178	757.3217\\
0.179	757.3184\\
0.179	757.3152\\
0.179	757.3122\\
0.179	757.3093\\
0.179	757.3065\\
0.179	757.3039\\
0.179	757.3015\\
0.179	757.2992\\
0.179	757.297\\
0.179	757.295\\
0.18	757.2932\\
0.18	757.2915\\
0.18	757.2899\\
0.18	757.2885\\
0.18	757.2873\\
0.18	757.2862\\
0.18	757.2853\\
0.18	757.2845\\
0.18	757.2839\\
0.18	757.2834\\
0.181	757.2831\\
0.181	757.283\\
0.181	757.283\\
0.181	757.2832\\
0.181	757.2836\\
0.181	757.2841\\
0.181	757.2847\\
0.181	757.2856\\
0.181	757.2866\\
0.181	757.2878\\
0.182	757.2891\\
0.182	757.2906\\
0.182	757.2923\\
0.182	757.2941\\
0.182	757.2961\\
0.182	757.2983\\
0.182	757.3006\\
0.182	757.3031\\
0.182	757.3058\\
0.182	757.3086\\
0.182	757.3117\\
0.183	757.3148\\
0.183	757.3182\\
0.183	757.3217\\
0.183	757.3254\\
0.183	757.3293\\
0.183	757.3333\\
0.183	757.3376\\
0.183	757.3419\\
0.183	757.3465\\
0.183	757.3512\\
0.184	757.3561\\
0.184	757.3612\\
0.184	757.3664\\
0.184	757.3718\\
0.184	757.3774\\
0.184	757.3832\\
0.184	757.3891\\
0.184	757.3952\\
0.184	757.4015\\
0.184	757.4079\\
0.185	757.4145\\
0.185	757.4213\\
0.185	757.4283\\
0.185	757.4354\\
0.185	757.4427\\
0.185	757.4502\\
0.185	757.4578\\
0.185	757.4656\\
0.185	757.4736\\
0.185	757.4817\\
0.186	757.49\\
0.186	757.4985\\
0.186	757.5072\\
0.186	757.516\\
0.186	757.525\\
0.186	757.5341\\
0.186	757.5434\\
0.186	757.5529\\
0.186	757.5625\\
0.186	757.5724\\
0.187	757.5823\\
0.187	757.5925\\
0.187	757.6028\\
0.187	757.6132\\
0.187	757.6239\\
0.187	757.6347\\
0.187	757.6456\\
0.187	757.6567\\
0.187	757.668\\
0.188	757.6794\\
0.188	757.691\\
0.188	757.7028\\
0.188	757.7147\\
0.188	757.7267\\
0.188	757.739\\
0.188	757.7513\\
0.188	757.7639\\
0.188	757.7766\\
0.188	757.7894\\
0.189	757.8024\\
0.189	757.8155\\
0.189	757.8288\\
0.189	757.8423\\
0.189	757.8559\\
0.189	757.8697\\
0.189	757.8836\\
0.189	757.8976\\
0.189	757.9118\\
0.189	757.9261\\
0.19	757.9406\\
0.19	757.9553\\
0.19	757.97\\
0.19	757.985\\
0.19	758\\
0.19	758.0152\\
0.19	758.0306\\
0.19	758.0461\\
0.19	758.0617\\
0.19	758.0775\\
0.191	758.0934\\
0.191	758.1094\\
0.191	758.1256\\
0.191	758.1419\\
0.191	758.1584\\
0.191	758.175\\
0.191	758.1917\\
0.191	758.2085\\
0.191	758.2255\\
0.191	758.2426\\
0.192	758.2599\\
0.192	758.2772\\
0.192	758.2947\\
0.192	758.3123\\
0.192	758.3301\\
0.192	758.348\\
0.192	758.366\\
0.192	758.3841\\
0.192	758.4023\\
0.192	758.4207\\
0.193	758.4392\\
0.193	758.4578\\
0.193	758.4765\\
0.193	758.4954\\
0.193	758.5143\\
0.193	758.5334\\
0.193	758.5526\\
0.193	758.5719\\
0.193	758.5913\\
0.193	758.6109\\
0.194	758.6305\\
0.194	758.6503\\
0.194	758.6702\\
0.194	758.6901\\
0.194	758.7102\\
0.194	758.7304\\
0.194	758.7507\\
0.194	758.7712\\
0.194	758.7917\\
0.194	758.8123\\
0.195	758.833\\
0.195	758.8538\\
0.195	758.8748\\
0.195	758.8958\\
0.195	758.9169\\
0.195	758.9382\\
0.195	758.9595\\
0.195	758.9809\\
0.195	759.0024\\
0.195	759.024\\
0.196	759.0457\\
0.196	759.0676\\
0.196	759.0894\\
0.196	759.1114\\
0.196	759.1335\\
0.196	759.1557\\
0.196	759.1779\\
0.196	759.2003\\
0.196	759.2227\\
0.196	759.2452\\
0.197	759.2678\\
0.197	759.2905\\
0.197	759.3133\\
0.197	759.3361\\
0.197	759.359\\
0.197	759.382\\
0.197	759.4051\\
0.197	759.4283\\
0.197	759.4515\\
0.197	759.4748\\
0.198	759.4982\\
0.198	759.5217\\
0.198	759.5452\\
0.198	759.5688\\
0.198	759.5925\\
0.198	759.6162\\
0.198	759.64\\
0.198	759.6639\\
0.198	759.6878\\
0.198	759.7118\\
0.199	759.7359\\
0.199	759.76\\
0.199	759.7842\\
0.199	759.8085\\
0.199	759.8328\\
0.199	759.8571\\
0.199	759.8816\\
0.199	759.906\\
0.199	759.9306\\
0.199	759.9551\\
0.2	759.9798\\
0.2	760.0045\\
0.2	760.0292\\
0.2	760.054\\
0.2	760.0788\\
};
\end{axis}
\end{tikzpicture}%}
        \caption{Temperature}
        \label{fig:otherImg_b}
    \end{subfigure}
    \caption{}
\end{figure}

$\unit[1]{\frac{j}{mol K}}=\unit[1.0366e-05]{\frac{eV}{K}}$\\

$C_v\left[AL\right]:\unit[24.20]{\frac{j}{mol K}}=\unit[1.0366e-05]{\frac{eV}{K}}$


\section*{Problem 4}

As a starting point we first look at scattering from a hard-sphere
potential. We also consider the Lennard--Jones potential, which is depicted
in Figure~\ref{fig1}. (\emph{Always refer to Figures in the text.})

\begin{figure}[!ht]
\begin{center}
  \includegraphics[width=0.7\textwidth]{template_files/LJ} 
  \caption{The Lennard--Jones potential.
  Make sure you label and have units on all axes! Also make sure that labels etc.\
  are legible and that, if you print in black and white, that you use different line
  styles when required to differentiate between curves. In \textsc{matlab}
  you can export any figure to an .eps file from File $\rightarrow$
  Export\ldots\ in the Figure window.}
  \label{fig1}
\end{center}
\end{figure}
  
\section*{Problem 2}

In the following we give an example of how to produce a table.
Use the code for Table~\ref{tab1} as a template.

\begin{table}[!ht]
  \begin{center}
    \caption{A dummy table}
    \begin{tabular}{l|c|c}\hline\hline
      \textbf{Col.~1} & \textbf{Col.~2} & \textbf{Col.~3} \\ \hline
      the & quick & brown \\ 
      fox & jumps & over \\ 
      the & lazy  & dog \\ 
      \hline\hline
    \end{tabular}
    \label{tab1}
  \end{center}
\end{table}

\section*{Problem 3}

If you find some part of the code particularly interesting you may 
include it in the text, otherwise it should be included in the appedix.
If you do want to include code the following commands will print
the text directly, with no \LaTeX~commands executed:

\begin{lstlisting}[language=matlab]
% Hello world ten times in MATLAB
for i = 1 : 10
  fprintf('Hello world %d!\n',i);
end
\end{lstlisting}

\begin{lstlisting}[language=python]
# Hello world ten times in Python
for i in range(10):
  print 'Hello world %d!' % i
\end{lstlisting}

\section*{Problem 4}
At some point it may be appropriate to include equations. It is done in the
following way:

\begin{equation}
  V(r) = 4\epsilon \left[ \left( \frac{\sigma}{r} \right)^{12} - 
    \left(\frac{\sigma}{r} \right)^{6} \right]
\end{equation}

Do number and reference all your equations.

\section*{Concluding discussion}

Use your favourite flavor of \LaTeX{} to compile the file:
\begin{verbatim}
xelatex template.tex
pdflatex template.tex
latex template.tex
\end{verbatim}
should all work.
If you use \verb+pdflatex+ or \verb+xelatex+, included figures need to be in
\verb+pdf+, \verb+jpg+, or \verb+png+ format. If you want to include eps
figures, you can easily convert them to \verb+pdf+ using the command
\begin{verbatim}
ps2pdf -dEPSCrop figure.eps figure.pdf
\end{verbatim}

\begin{thebibliography}{69}
\bibitem{lamport94} Leslie Lamport, \emph{\LaTeX: A Document Preparation
System}. Addison Wesley, Massachusetts, 2nd Edition, 1994.
\end{thebibliography}

\newpage

\appendix

\section{Source Code}

Include all source code here in the appendix. Keep the code formatting clean,
use indentation, and comment your code to make it easy to understand. Also,
break lines that are too long. (Keep them under 80 characters!)

\subsection{Calculating pi using matlab: \texttt{pi.m}}
\lstinputlisting[language=matlab,numbers=left]{template_files/pi.m}

\subsection{Calculating pi using python: \texttt{pi.py}}
\lstinputlisting[language=python,numbers=left]{template_files/pi.py}

\subsection{Calculating pi using C: \texttt{pi.c}}
\lstinputlisting[language=c,numbers=left]{template_files/pi.c}

\end{document}
